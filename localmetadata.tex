\title{Representation and parsing of multiword expressions}  %look no further, you can change those things right here.
\subtitle{Current trends}
\BackTitle{Representation and parsing of multiword expressions} % Change if BackTitle != Title
\BackBody{Deep parsing is the fundamental process aiming at the representation of the syntactic structure of phrases and sentences. In the traditional methodology this process is based on lexicons and grammars representing roughly properties of words and interactions of words and structures in sentences. Several linguistic frameworks, such as Head-driven Phrase Structure Grammar (HPSG), Lexical Functional Grammar (LFG), Tree Adjoining Grammar (TAG), Combinatory Categorial Grammar (CCG), etc., offer different structures and combining operations for building grammar rules. These already contain mechanisms for expressing properties of Multiword Expressions (MWE), which, however, need improvement in how they account for idiosyncrasies of MWEs on the one hand and their similarities to regular structures on the other hand. This collaborative book constitutes a survey on various attempts at representing and parsing MWEs in the context of linguistic theories and applications.}
%\dedication{Change dedication in localmetadata.tex}
\typesetter{Felix Kopecky}
%\proofreader{Change proofreaders in localmetadata.tex}
\author{Yannick Parmentier \lastand Jakub Waszczuk}
\BookDOI{10.5281/zenodo.2579017}%ask coordinator for DOI
\renewcommand{\lsISBNdigital}{978-3-96110-145-0}
\renewcommand{\lsISBNhardcover}{978-3-96110-146-7}
\renewcommand{\lsSeries}{pmwe} % use lowercase acronym, e.g. cfls, sidl, eotms, tgdi
\renewcommand{\lsSeriesNumber}{3} %will be assigned when the book enters the proofreading stage
\renewcommand{\lsID}{202} % contact the coordinator for the right number

\SpineAuthor{Parmentier, Waszczuk}
%<*coverdimen>
% \setlength{\csspine}{25.0559784mm} % Please calculate: Total Page Number (excluding cover, usually (Total Page - 3)) * 0.0572008 mm
% \setlength{\bodspine}{20mm} % Please use BoD's algorithm: http://www.bod.de/hilfe/coverberechnung.html (German only, please contact LangSci staff for help)
%</coverdimen>
