%%%%%%%%%%%%%%%%%%%%%%%%%%%%%%%%%%%%%%%%%%%%%%%%%%%%
%%%                                              %%%
%%%     Language Science Press Master File       %%%
%%%         follow the instructions below        %%%
%%%                                              %%%
%%%%%%%%%%%%%%%%%%%%%%%%%%%%%%%%%%%%%%%%%%%%%%%%%%%%
 
% Everything following a % is ignored
% Some lines start with %. Remove the % to include them

\documentclass[output=book
                ,leqno
		,modfonts
		,nonflat
	        ,collection
	        ,collectionchapter
		,colorlinks,citecolor=brown
		,arseneau
		,arabicfont
		,chinesefont
 	      % ,draft
%  	      ,draftmode
	      % ,showindex
	      % ,multiauthors
	      % ,nobabel
		  ]{langsci/langscibook}                              
%%%%%%%%%%%%%%%%%%%%%%%%%%%%%%%%%%%%%%%%%%%%%%%%%%%%

% put all additional commands you need in the 
% following files. If you do not know what this might 
% mean, you can safely ignore this section

\title{Representation and parsing of multiword expressions}  %look no further, you can change those things right here.
\subtitle{Current trends}
\BackTitle{Representation and parsing of multiword expressions} % Change if BackTitle != Title
\BackBody{Deep parsing is the fundamental process aiming at the representation of the syntactic structure of phrases and sentences. In the traditional methodology this process is based on lexicons and grammars representing roughly properties of words and interactions of words and structures in sentences. Several linguistic frameworks, such as Head-driven Phrase Structure Grammar (HPSG), Lexical Functional Grammar (LFG), Tree Adjoining Grammar (TAG), Combinatory Categorial Grammar (CCG), etc., offer different structures and combining operations for building grammar rules. These already contain mechanisms for expressing properties of Multiword Expressions (MWE), which, however, need improvement in how they account for idiosyncrasies of MWEs on the one hand and their similarities to regular structures on the other hand. This collaborative book constitutes a survey on various attempts at representing and parsing MWEs in the context of linguistic theories and applications.}
%\dedication{Change dedication in localmetadata.tex}
\typesetter{Felix Kopecky, Jakub Waszczuk, Yannick Parmentier }
\proofreader{Alexandr Rosen,
Amir Ghorbanpour,
Aniefon Daniel,
Brett Reynolds,
Carlos Ramisch,
Daniela Schroeder,
Ikmi Nur Oktavianti,
Jakub Waszczuk,
Jeroen van de Weijer,
Jean Nitzke,
Lachlan Mackenzie,
Phil Duncan,
Timm Lichte,
Valentin Vydrin,
Valeria Quochi,
Vasiliki Foufi}

\author{Yannick Parmentier \lastand Jakub Waszczuk}
\BookDOI{10.5281/zenodo.2579017}%ask coordinator for DOI
\renewcommand{\lsISBNdigital}{978-3-96110-145-0}
\renewcommand{\lsISBNhardcover}{978-3-96110-146-7}
\renewcommand{\lsSeries}{pmwe} % use lowercase acronym, e.g. cfls, sidl, eotms, tgdi
\renewcommand{\lsSeriesNumber}{3} %will be assigned when the book enters the proofreading stage
\renewcommand{\lsID}{202} % contact the coordinator for the right number

\SpineAuthor{Parmentier, Waszczuk}
%<*coverdimen>
% \setlength{\csspine}{25.0559784mm} % Please calculate: Total Page Number (excluding cover, usually (Total Page - 3)) * 0.0572008 mm
% \setlength{\bodspine}{20mm} % Please use BoD's algorithm: http://www.bod.de/hilfe/coverberechnung.html (German only, please contact LangSci staff for help)
%</coverdimen>

% add all extra packages you need to load to this file 
\usepackage{graphicx}
\usepackage{tabularx}
\usepackage{amsmath} 
\usepackage{multicol}
\usepackage{lipsum}
\usepackage{caption}
\usepackage{enumitem}
%%%%%%%%%%%%%%%%%%%%%%%%%%%%%%%%%%%%%%%%%%%%%%%%%%%%
%%%                                              %%%
%%%           Examples                           %%%
%%%                                              %%%
%%%%%%%%%%%%%%%%%%%%%%%%%%%%%%%%%%%%%%%%%%%%%%%%%%%%
% remove the percentage signs in the following lines
% if your book makes use of linguistic examples
\usepackage{langsci/styles/avm} 
\usepackage{langsci/styles/jambox}
\usepackage{langsci/styles/langsci-optional} 
\usepackage{langsci/styles/langsci-lgr}
\usepackage{langsci/styles/langsci-linguex}
\usepackage{morewrites} 

\makeatletter
\let\pgfmathModX=\pgfmathMod@
\usepackage{pgfplots,pgfplotstable}%
\let\pgfmathMod@=\pgfmathModX
\makeatother
\usepgfplotslibrary{colorbrewer,groupplots}

\usepackage{siunitx}
\sisetup{output-decimal-marker={.},detect-weight=true, detect-family=true, detect-all, input-symbols={\%}, uncertainty-separator={\,},group-digits=false,detect-inline-weight=math}
\DeclareSIUnit[number-unit-product={}]{\percent}{\%}
\makeatletter \def\new@fontshape{} \makeatother
\robustify\bfseries % For detect weight to work

%%%%%%%%%%%%%%%%%%%%%%%%%%%%%%%%%%%%%%%%%%%%%%%%%%%%
%%%                                              %%%
%%%      Optimality Theory                       %%%
%%%                                              %%%
%%%%%%%%%%%%%%%%%%%%%%%%%%%%%%%%%%%%%%%%%%%%%%%%%%%%
% If you are using OT, uncomment the following lines      
% % OT pointing hand
% \usepackage{pifont}
% \newcommand{\hand}{\ding{43}}
% % OT tableaux                                                
% \usepackage{pstricks,colortab}    

%%%%%%%%%%%%%%%%%%%%%%%%%%%%%%%%%%%%%%%%%%%%%%%%%%%%
%%%                                              %%%
%%%       Attribute Value Matrices               %%%
%%%                                              %%%
%%%%%%%%%%%%%%%%%%%%%%%%%%%%%%%%%%%%%%%%%%%%%%%%%%%%
%If you are using Attribute-Value-Matrices, uncomment the following lines 
% \usepackage{lsp-avm}
% \usepackage{avm}
% \avmfont{\sc} 
% \avmvalfont{\it} 
% % command to fontify the type values of an avm 
% \newcommand{\tpv}[1]{{\avmjvalfont #1}} 
% % command to fontify the type of an avm and avmspan it
% \newcommand{\tp}[1]{\avmspan{\tpv{#1}}}

%%%%%%%%%%%%%%%%%%%%%%%%%%%%%%%%%%%%%%%%%%%%%%%%%%%%
%%%                                              %%%
%%%     Discourse Representation Structures      %%%
%%%                                              %%%
%%%%%%%%%%%%%%%%%%%%%%%%%%%%%%%%%%%%%%%%%%%%%%%%%%%%
% DRS package by Alexis Dimitriadis
% \usepackage{drs}

%%%%%%%%%%%%%%%%%%%%%%%%%%%%%%%%%%%%%%%%%%%%%%%%%%%%
%%%                                              %%%
%%%            Chinese Japanese Korean           %%%
%%%                                              %%%
%%%%%%%%%%%%%%%%%%%%%%%%%%%%%%%%%%%%%%%%%%%%%%%%%%%%

% For Chinese characters, uncomment the following lines
% \usepackage[indentfirst=false]{xeCJK}
% \setCJKmainfont{SimSun}

%%%%%%%%%%%%%%%%%%%%%%%%%%%%%%%%%%%%%%%%%%%%%%%%%%%%
%%%                                              %%%
%%%               Arabic / Persian               %%%
%%%                                              %%%
%%%%%%%%%%%%%%%%%%%%%%%%%%%%%%%%%%%%%%%%%%%%%%%%%%%%

% for bidirectional text and support for Arabic/Persian, uncomment the following lines
%% \usepackage{fontspec}
% \newfontfamily\Parsifont[Script=Arabic]{XB Niloofar}
% %\usepackage{bidi}
% \usepackage{lsp-bidi}
% \newcommand{\PRL}[1]{\RL{\Parsifont #1}}
% %\TeXXeTOff
 

%%%%%%%%%%%%%%%%%%%%%%%%%%%%%%%%%%%%%%%%%%%%%%%%%%%%
%%%                                              %%%
%%%          Trees                               %%%
%%%                                              %%%
%%%%%%%%%%%%%%%%%%%%%%%%%%%%%%%%%%%%%%%%%%%%%%%%%%%%

% For trees, uncomment the following lines
\usepackage{tikz}
\usepackage{tikz-qtree}
\usepackage{tikz-qtree-compat}
\usepackage{tikz-dependency}
% has strange side effects
\tikzset{every tree node/.style={align=left, anchor=north}}
\tikzset{every roof node/.append style={inner sep=0.1pt,text height=2ex,text depth=0.3ex}}

\usepackage{algorithm}
\usepackage{algpseudocode}
\usepackage{amssymb}
\usepackage{listings}
\usepackage{float}

%%%%%%%%%%%%%%%%%%%%%%%%
%Constant
\usepackage{color}
\usepackage{tikz-dependency}
\usepackage{pifont}
\usepackage{rotating}

%%%%%%%%%%%%%%%%%%%%%%%%
%Fonseca
%\usepackage{subcaption}
%\usetikzlibrary{arrows, automata}

%%%%%%%%%%%%%%%%%%%%%%%%
%Markantonatou
\usepackage{forest}

%%%%%%%%%%%%%%%%%%%%%%%%
%Lichte
%\usepackage{linguex}
\usepackage{soul} % for text highlighting
\lstset{basicstyle=\ttfamily,tabsize=2,breaklines=true}

%%%%%%%%%%%%%%%%%%%%%%%%
%Semmar
\usepackage{multirow}

%%%%%%%%%%%%%%%%%%%%%%%%
%Jacquet
%\usepackage{booktabs}

%%%%% TIPA
% % \usepackage{tipa}
% % \usepackage{tipx}

\usepackage{langsci/styles/langsci-gb4e} 

%% hyphenation points for line breaks
%% Normally, automatic hyphenation in LaTeX is very good
%% If a word is mis-hyphenated, add it to this file
%%
%% add information to TeX file before \begin{document} with:
%% %% hyphenation points for line breaks
%% Normally, automatic hyphenation in LaTeX is very good
%% If a word is mis-hyphenated, add it to this file
%%
%% add information to TeX file before \begin{document} with:
%% %% hyphenation points for line breaks
%% Normally, automatic hyphenation in LaTeX is very good
%% If a word is mis-hyphenated, add it to this file
%%
%% add information to TeX file before \begin{document} with:
%% \include{localhyphenation}
\hyphenation{
affri-ca-te
affri-ca-tes
com-ple-ments
Zwei-gen-baum
data-set
data-sets
}

\hyphenation{
affri-ca-te
affri-ca-tes
com-ple-ments
Zwei-gen-baum
data-set
data-sets
}

\hyphenation{
affri-ca-te
affri-ca-tes
com-ple-ments
Zwei-gen-baum
data-set
data-sets
}

\renewcommand*{\lsChapterFooterSize}{\footnotesize}


%Math Operators in Chapters
\DeclareMathOperator{\tf}{tf}
\DeclareMathOperator{\idf}{idf}
\DeclareMathOperator{\CTok}{Tok}
\DeclareMathOperator{\CombMNZ}{CombMNZ}
\DeclareMathOperator{\InvEdit}{InvEdit}
\DeclareMathOperator{\Lev}{Lev}
\DeclareMathOperator{\TransMod}{TransMod}
\DeclareMathOperator{\TransTok}{TransTok}
\DeclareMathOperator{\Rel}{Rel}
\DeclareMathOperator{\EXP}{EXP}

%add all your local new commands to this file
\counterwithout{equation}{chapter} % remove the chapter number
\newcommand{\sg}{\textsc{sg}{}\xspace}	%singular 
\newcommand{\definite}{\textsc{def}{}\xspace}	%definite

\newcommand{\tobi}[2]}}
\renewcommand{\S}[1]{\tobi{#1}{\textsc{*}}}

%DELHONEUX
\newcommand{\subscript}[1]{\raisebox{-.4ex}{\scriptsize #1}}
\newcommand{\modelB}{\textit{model\subscript{B}}}
\newcommand{\modelA}{\textit{model\subscript{A}}}
\newcommand{\citepos}[1]{\citeauthor{#1}'s (\citeyear{#1})}

%CONSTANT
\newcommand{\asu}{UAS}
\newcommand{\asl}{LAS}
\newcommand{\asuo}{UAS\textsubscript{OA}}
\newcommand{\aslo}{LAS\textsubscript{OA}}
\newcommand{\asus}{UAS\textsubscript{surr}}
\newcommand{\asls}{LAS\textsubscript{surr}}
\newcommand{\ww}{$WW_U$}

%WEHRLI
\newcommand{\cat}[2]{[\nolinebreak[4]\textsubscript{\texttt{{}#1}} #2]}
\newcommand{\cati}[3]{\texttt{[\nolinebreak[4]}\textsubscript{\texttt{{}#1}} #2\texttt{]$_{#3}$}}
\newcommand{\ik}[1]{$_{#1}$}
%% \newcounter{example}
%% \newcounter{examplebis}
%% \newcounter{exampleref}
%% \newcommand{\refex}[1]{\setcounter{exampleref}{\value{example}}\addtocounter{exampleref}{#1}\arabic{exampleref}}
%% \newcommand{\emptyex}{\stepcounter{example}\noindent \\
%%   \makebox[0.9cm][r]{(\theexample)}}
%% \newcommand{\debex}[1]{\setcounter{examplebis}{1}
%%   \begin{list}{}{\vspace{-0.4cm}
%%       \setlength{\labelwidth}{1cm}}
%%     \addex{\alph{examplebis}.}{#1}}
%% \newcommand{\finex}{\vspace{-0.2cm} \noindent\end{list}}
%% \newcommand{\addex}[2]{\stepcounter{example}
%%           \item[(\theexample)#1] #2}
%% \newcommand{\putex}[1]{\stepcounter{examplebis}
%%           \item[\alph{examplebis}.] #1}
%% \newcommand{\makeex}[1]{\begin{list}{}{\vspace{-0.2cm}}
%%           \stepcounter{example}
%%           \item[(\theexample)] #1 \finex}
%% \newcommand{\textel}[1]{{#1}}

%SHEINFUX
\renewcommand{\emph}{\textit}
%For hebrew transliteration with and without glosses:
\newcommand{\textgl}[1]{`#1'}
\newcommand{\heb}[1]{\textit{#1}}
%\newcommand{\hebgloss}[2]{\textit{#1} (`\textit{#2}')}
\newcommand{\hebgloss}[2]{\textit{#1} `#2'}
\newcommand{\idgloss}[3]{\textit{\textit{#1}} \textgl{#2}~→ \textgl{#3}}
\newcommand{\reff}[1]{(\ref{#1})}
\newcommand{\quotecite}[1]{\citeauthor{#1}'s (\citeyear{#1})}
\newcommand{\undbf}[1]{\underline{\textbf{#1}}}
%For Hebrew transcription
\newcommand{\alef}{ʔ}
\newcommand{\alefB}{\textbf{ʔ}}
\newcommand{\ayin}{ʕ}
\newcommand{\ayinB}{\textbf{ʕ}}
\newcommand{\shin}{ʃ}
\newcommand{\shinB}{\textbf{ʃ}}
\newcommand{\het}{ħ}
\newcommand{\hetB}{\textbf{ħ}}
\newcommand{\tet}{\d{t}}
\newcommand{\tetB}{\d{t}}
\newcommand{\spacebr}[1]{\hphantom{\{}}

%LICHTE
%%%%%%%%%%%%%%%%%%%%%%
%   AVM SETTINGS     % 
%%%%%%%%%%%%%%%%%%%%%%

\avmoptions{center}	
\avmfont{\sc}
\avmvalfont{\rm}
\avmsortfont{\it}

\newenvironment{topbot}{   	% more flexible than /newcommand ?
	\avmvskip{0.2ex} 
	\hspace{-1.5em}
	\begin{avm}
	\avml
	}
	%%%
	{
	\avmr
    \end{avm}
    \hspace{-0.5em}
}


%%%%%%%%%%%%%%%%%%%%%%
%   TIKZ SETTINGS    % 
%%%%%%%%%%%%%%%%%%%%%%

\tikzset{every tree node/.style={align=center,anchor=north}}	% to allow linebreaks
\usetikzlibrary{calc} % for positioning arrows with ($(t.center)-(1,0)$)
\usetikzlibrary{shapes,snakes}
\usetikzlibrary{backgrounds,fit}
\usetikzlibrary{arrows}
\usetikzlibrary{matrix}
\usetikzlibrary{positioning}

% Define box and box title style (see http://www.texample.net/tikz/examples/boxes-with-text-and-math/)
\tikzstyle{mybox} = [draw=gray, very thick,
    rectangle, rounded corners, inner sep=10pt, inner ysep=17pt,yshift=3pt]
\tikzstyle{fancytitle} =[draw=gray, very thick, fill=white,
    rectangle, rounded corners, inner sep=5pt, inner ysep=5pt]
    
\tikzset{
    %Define standard arrow tip
    >=stealth',
    %Define style for boxes
    box/.style={
           rectangle,
           rounded corners,
           draw=black, very thick,
           text width=10em,
           minimum height=2em,
           text centered},
    % Define arrow style
    arrow/.style={
           ->,
           thick,
           	shorten <=2pt,
           shorten >=2pt,}
}

\newcommand\centertikz[1]{\tikz[baseline=(current bounding box.center)]{#1}}

    

%%%%%%%%%%%%%%%%%%%%%%
%   MISCELLANEOUS    % 
%%%%%%%%%%%%%%%%%%%%%%

\newcommand*\circled[1]{\tikz[baseline=(char.base)]{
    \node[shape=circle,draw,inner sep=.15ex] (char) {#1};}}
\newcommand{\svar}[1]
   {\setbox2=\hbox{$\scriptstyle #1$}\lower.2ex\vbox{\hrule
     \hbox{\vrule\kern1.25pt 
     \vbox{\kern1.25pt\box2\kern1.25pt}\kern1.25pt\vrule}\hrule}}
\newcommand{\ssvar}[1]
   {\setbox2=\hbox{\scalebox{.5}{$#1$}}\lower.2ex\vbox{\hrule
     \hbox{\vrule\kern1.25pt 
     \vbox{\kern1.25pt\box2\kern1.25pt}\kern1.25pt\vrule}\hrule}}
\newcommand{\trace}[0]{\underline{$~~~$}}

\newcommand{\prule}[3]{
      $\begin{array}{c} #1\\ \hline
         #2\end{array} ~~ #3$}
     
%% already defined in langsci.cls 
% \newlength{\stmueTmp}
% \newcommand*{\hspaceThis}[1]{\settowidth{\stmueTmp}{#1}\hspace*{\stmueTmp}}

\newcommand{\minitab}[2][c]{\begin{tabular}{@{}#1@{}}#2\end{tabular}}

\newenvironment{changemargin}[2]{%
  \begin{list}{}{%
    \setlength{\topsep}{0pt}%
    \setlength{\leftmargin}{#1}%
    \setlength{\rightmargin}{#2}%
    \setlength{\listparindent}{\parindent}%
    \setlength{\itemindent}{\parindent}%
    \setlength{\parsep}{\parskip}%
  }%
  \item[]}{\end{list}}
  
% \newtheorem{definition}{Definition}
% \newtheorem{corollary}{Corollary}
% \newtheorem{theorem}{Theorem}

%% \newcounter{sitem-zaehler}
%% \newenvironment{sitem}{\begin{list}{$\bullet$}
%%     {\usecounter{sitem-zaehler}
%%       \setlength{\itemindent}{0pt}
%%       \setlength{\labelwidth}{3ex}
%%       \setlength{\leftmargin}{3ex}
%%       \setlength{\topsep}{0ex plus0ex minus0ex}
%%       \setlength{\parsep}{0.3ex plus0ex minus0ex}
%%       \setlength{\itemsep}{0ex plus0ex minus0ex}
%%     }}{\end{list}}


\newcommand{\tl}[1]{\todo[color=red!30]{TL:\enspace#1}}
\newcommand{\simon}[1]{\todo[color=blue!30]{SP:\enspace#1}}
\newcommand{\jw}[1]{\todo[color=orange!70]{JW:\enspace#1}}
\newcommand{\as}[1]{\todo[color=green!30]{AS:\enspace#1}}

%In-line examples with translations, e.g. \textit{einjagen} lit. 'inchase'=>'chase in’                                                                         
\newcommand{\textex}{\textit}
\newcommand{\textlit}[1]{(lit. \textit{#1})} %literal meaning                   
\newcommand{\texttr}[1]{`#1'} %translation (idiomatic meaning)                  
\newcommand{\ile}[1]{\textex{#1}} %inline example, e.g. einjagen                
\newcommand{\ilet}[2]{\textex{#1} \texttr{#2}} %inline example with translation, e.g. einjagen 'chase in'                                                      
\newcommand{\ilel}[2]{\textex{#1} \textlit{#2}} %inline example with literal meaning, e.g. einjagen (lit. in-chase)                                            
\newcommand{\ilelt}[3]{\textex{#1} \textlit{#2}\,\texttr{#3}} %inline example with literal meaning and tran

\definecolor{lightgray}{gray}{0.7}

\lstnewenvironment{xmg}{%                                                       
  \lstset{language=,
    numbers=left,numbersep=8pt,numberstyle=\color{lightgray},
        % frame=l,                                                              
        basicstyle=\small\ttfamily,%                                            
    xleftmargin=0.7cm,framexleftmargin=12pt,%                                   
    framerule=0.5mm,rulecolor=\color{lightgray},%                               
    escapeinside={|\%}{\%|},%                                                   
    commentstyle=\color{lightgray},
    literate={->}{{{\textbf{->}}}}1 {\{}{{{\textbf{\{}}}}1 {\}}{{{\textbf{\}}}}}1 {\;}{{{\textbf{;}}}}1 {|}{{{\textbf{|}}}}1 {=}{{{\textbf{=}}}}1 {[}{{{\textbf{[}}}}1 {]}{{{\textbf{]}}}}1 {<}{{{\textbf{<}}}}1 {>}{{{\textbf{>}}}}1 {!}{{{\textbf{!}}}}1 {?}{{{\textbf{?}}}}1 {*=}{{{\textbf{*=}}}}1,%                     
    morekeywords={node,type,feature,include,class,import,export,declare,syn,sem,frame,morph,value, use, with, dims}}}{}

\lstnewenvironment{duelme}{%                                                    
  \lstset{language=,
    numbers=left,numbersep=8pt,numberstyle=\color{lightgray},
    % frame=l,                                                                  
    basicstyle=\small\ttfamily,%                                                
    xleftmargin=0.7cm,framexleftmargin=12pt,%                                   
    framerule=0.5mm,rulecolor=\color{lightgray},%                               
    escapeinside={|\%}{\%|},%                                                   
    commentstyle=\color{lightgray},
    morekeywords={PATERN,NAME,POS,PATTERN,MAPPING,EXAMPLE,MWE,SENTENCE,DESCRIPTION,COMMENT,LISTA,LISTB,SUBJECT,OBJECT,MODIFIER,RPRON,CONJUGATION,POLARITY,EXPRESSION,CL}}}{}

\lstnewenvironment{patr-listing}{%                                              
  \lstset{language=,
    numbers=left,numbersep=8pt,numberstyle=\color{lightgray},
    % frame=l,                                                                  
    basicstyle=\small\ttfamily,%                                                
    xleftmargin=0.7cm,framexleftmargin=12pt,%                                   
    framerule=0.5mm,rulecolor=\color{lightgray},%                               
    escapeinside={|\%}{\%|},%                                                   
    commentstyle=\color{lightgray},
    literate={:}{{{\textbf{:}}}}1 {\{}{{{\textbf{\{}}}}1 {\}}{{{\textbf{\}}}}}1 {=}{{{\textbf{=}}}}1 {[}{{{\textbf{[}}}}1 {]}{{{\textbf{]}}}}1 {<}{{{\textbf{<}}}}1 {>}{{{\textbf{>}}}}1 {!}{{{\textbf{!}}}}1,
    morekeywords={Define,as,Word}}}{}

\newcommand{\ixmg}{%                                                            
  \lstinline[language=,keepspaces,%                                             
  literate={->}{{{\textbf{->}}}}1 {\{}{{{\textbf{\{}}}}1 {\}}{{{\textbf{\}}}}}1 {\;}{{{\textbf{;}}}}1 {|}{{{\textbf{|}}}}1 {=}{{{\textbf{=}}}}1 {[}{{{\textbf{[}}}}1 {]}{{{\textbf{]}}}}1 {<}{{{\textbf{<}}}}1 {>}{{{\textbf{>}}}}1 {!}{{{\textbf{!}}}}1 {?}{{{\textbf{?}}}}1 {*=}{{{\textbf{*=}}}}1,%                       
  morekeywords={node,type,feature,include,class,import,export,declare,syn,sem,frame,morph,value, use, with, dims}
  ]}

\newcommand{\needsrevision}[1]{{\color{gray}#1}}

%% JACQUET
\newcommand{\ra}[1]{\renewcommand{\arraystretch}{#1}}
\definecolor{orange}{rgb}{0.64,0.16,0.16}
\newcommand{\remJakub}[1]{\textcolor{orange}{JP : #1}}

\definecolor{bleu}{rgb}{0.16,0.16,0.64}
\newcommand{\remGuillaume}[1]{\textcolor{bleu}{GJ : #1}}

\definecolor{vert}{rgb}{0.16,0.64,0.16}
\newcommand{\remMaud}[1]{\textcolor{vert}{ME : #1}}
 
\bibliography{localbibliography}


%%%%%%%%%%%%%%%%%%%%%%%%%%%%%%%%%%%%%%%%%%%%%%%%%%%%
%%%                                              %%%
%%%              Temporary                       %%%
%%%                                              %%%
%%%%%%%%%%%%%%%%%%%%%%%%%%%%%%%%%%%%%%%%%%%%%%%%%%%%


\definecolor{mygreen}{RGB}{75,200,75}
\definecolor{MYGREEN}{RGB}{75,200,75}
\newcommand{\jwe}[1]{{\color{red}(#1)}}
% \newcommand{\jwc}[1]{{\color{mygreen}#1}}
\newcommand{\jwc}{{\color{mygreen}(corrected)}}


%%%%%%%%%%%%%%%%%%%%%%%%%%%%%%%%%%%%%%%%%%%%%%%%%%%%
%%%                                              %%%
%%%             Frontmatter                      %%%
%%%                                              %%%
%%%%%%%%%%%%%%%%%%%%%%%%%%%%%%%%%%%%%%%%%%%%%%%%%%%%
\begin{document}         
\maketitle                
\frontmatter

\currentpdfbookmark{Contents}{name} % adds a PDF bookmark
{\sloppy\tableofcontents}

%% \setcounter{tocdepth}{1}
%% \listoffigures

%% uncomment if you have preface and/or acknowledgements
%\addchap{Preface}
\begin{refsection}

While \isi{Multiword Expressions} (MWEs), i.e. sequences of words with some
unpredictable properties such as \textit{to count somebody in} or
\textit{to take a haircut}, have been attracting attention for a long
time because of these idiosyncratic properties which go beyond word
boundaries, they remain a challenge for both linguistic theories and
natural language (NL) applications.

Indeed, most of these theories and applications admit an (explicit or
implicit) division of language phenomena into clear-cut levels:
%\begin{description}
%\item[
(i) tokens (indivisible text units, roughly words),
%\item[
(ii) morphology (properties of words e.g. number, gender, etc.),
%\item[
(iii) syntax (structural links between words, e.g. number/gender agreement),
%\item[
(iv) semantics (meaning of words and sentences).
%\end{description}
However, human languages frequently show a high degree of ambiguity
and fuzziness with respect to this layer-oriented model. In
particular, MWEs are placed on the frontier between these levels due
to their idiosyncratic properties on the one hand, and their
morphological, syntactic and semantic \isi{variations} on the other
hand. For instance, their meaning is often non-compositional as in \textit{to
take a haircut} (i.e. \textit{to suffer a serious financial loss}), although
they admit some syntactic variation similarly to many other
expressions (\textit{take/takes/have taken/has taken/took a serious/70\%
haircut}). Strictly layer-oriented language models fail to reflect
this specificity, and thus yield erroneous text processing results
(e.g. word-to-word translations of \isi{idioms}). Although the quantitative
importance of MWEs is well known (they cover up to 30\% of all words
in human language utterances, and are much more numerous in lexicons
than single words), the achievements in their formal representation
and automatic processing are still largely unsatisfactory.

In this context, an international and multilingual consortium of
researchers recently took part in the European PARSEME COST
Action\footnote{\url{http://www.cost.eu/COST_Actions/ict/IC1207}}
(2013--2017), which aimed at better understanding the nature of MWEs in
order to improve their support in natural language applications. Two
main challenges were considered: \textsc{linguistic precision} (how to
account for the highly heterogeneous nature of MWEs in linguistic
resources and treatments?) and \textsc{computational efficiency} (how to
deal with MWEs' idiosyncratic properties within reliable applications?).

To contribute to meeting these two challenges, PARSEME was based on four
Working Groups (WGs):
\begin{description}
\item[WG1] focused on the Grammar/Lexicon interface and the design of
  interoperable MWE lexicons,
\item[WG2] aimed at developing parsing techniques for MWEs,
\item[WG3] studied hybrid (e.g. symbolic and/or statistical) NL
  applications dealing with MWEs (e.g. MWE detection, machine
  translation, etc.),
\item[WG4] was concerned with the annotation of MWEs within \isi{treebanks}.
\end{description}

This book has been created within WG2. It consists of contributions
related to the definition, representation and parsing of MWEs. These
contributions were collected via an open call for chapters. Each
Chapter proposal was reviewed by 2 members of the editorial board. Out
of this reviewing, 10 proposals were selected. They reflect current
trends in the representation and processing of MWEs. They cover
various \emph{categories} of MWEs such as verbal, adverbial and
nominal MWEs, various \emph{linguistic frameworks} (e.g. tree-based
and unification-based grammars), various \emph{languages} including
\ili{English}, \ili{French}, \ili{Modern Greek}, \ili{Hebrew}, \ili{Norwegian}), and various
\emph{applications} (namely MWE detection, parsing, automatic
translation) using both symbolic and statistical approaches.

%\section{Outline of the book}

The book is organized as follows. 

\subsection*{Part 1: MWE representations}

The first part of the volume (Chapters 2 to 6) is dedicated to the
study of MWE properties and representations.

In Chapter~2, {Lichte, Petitjean, Savary and Waszczuk}
discuss the representation of MWEs within lexicalised formalisms. In
particular, they show how the eXtensible MetaGrammar (XMG2) formalism
offers a natural encoding of MWEs, which allows us to account for the
fact that irregularities exhibited by MWEs are a matter of scale
rather than binary properties.

In Chapter~3, {Herzig Sheinfux, Arad Greshler, Melnik and
  Wintner} study a specific type of MWEs (namely \isi{verbal MWEs}),
focusing mostly on \ili{Hebrew}, and show that unlike what previous work
suggests, \isi{flexibility} of \isi{verbal MWEs} is not a discrete concept but
rather a continuous property. They propose a new classification of
MWEs which is based on semantic notions.

In Chapter~4, {Dyvik, Losnegaard and Rosén} present the analysis
of MWEs in an \isi{LFG} \isi{grammar} for \ili{Norwegian}, NorGram, which is used in the
construction of NorGramBank, a treebank of parsed sentences. The
chapter describes how classes of MWEs are analysed by means of \isi{LFG}
templates, which capture the lexical and syntactic properties of MWEs
in a succinct way.

In Chapter~5, {Markantonatou, Samaridi and Minos} present a
\isi{grammar} of \ili{Modern Greek} in the \isi{LFG} formalism. Their \isi{grammar} has been
implemented with the Xerox Linguistic Engine (XLE), a \isi{grammar} editor
which also includes a parsing engine. In their Chapter, the authors
pay a particular attention to the use of a pre-processor to detect and
annotate MWEs prior to parsing.

In Chapter~6, {Angelov} presents the Grammatical Framework, a
description language for developing NLP multilingual resources, and
its application to some classes of MWEs. In particular, the author
shows how to define MWE-aware multilingual grammars, which can be used
for instance for in-domain machine translation.

\subsection*{Part 2: MWE parsing}

The second part of the volume (Chapters 7 to 9) focuses on MWE
parsing, that is, on the automatic construction of deep
representations of the syntax of MWEs. Two main approaches to parsing
coexist: the data-driven approach aims at extracting syntactic
information from corpora using Machine Learning techniques and is
discussed in Chapter~7. The knowledge-based approach relies on the
encoding of linguistic properties of MWEs within lexical entries,
which are used by a parsing algorithm to compute the
expected \isi{syntactic structure}. The impact of MWE detection on
such parsing algorithms is discussed in Chapters~8 (for a categorial
parser) and~9 (for an attachment-rule-based parser).

In Chapter~7, {Constant, Eryiğit, Ramisch, Rosner and Schneider}
give a detailed overview of various ways to extend statistical parsing
with MWE identification, either during parsing or as a pre- or
post-processing step. These extensions are compared and their
evaluation discussed.

In Chapter~8, {de Lhoneux, Abend and Steedman} extend a \isi{CCG}
parsing architecture for \ili{English} with a module for detecting MWEs and
pre-process them. The effect of this pre-processing is evaluated in
terms of parsing accuracy when (i)~the parser is trained on
pre-processed data (so-called training effect) and (ii)~the parser
uses information from pre-processed data (so-called parsing effect).

In Chapter~9, {Foufi, Nerima and Wehrli} investigate the
extension of a knowledge-based parser with collocation
identification. They apply this extension to the description of MWEs
for various languages (including \ili{English} and Greek), and show how it
improves parsing efficiency in terms of percentages of complete
analyses.

\subsection*{Part 3: Multilingual NL applications for MWEs}

Finally, in the third part of the volume (Chapters 10 and 11),
multilingual MWE acquisition techniques are presented.

In Chapter~10, {Semmar, Servan, Laib, Bouamor and Marchand}
present three techniques for word alignment between \isi{parallel corpora}
and their application to \isi{MWEs}. The bilingual MWE
lexicons built using these techniques are then evaluated according to
their effect on phrase-based \isi{statistical machine translation}. The
authors empirically show that MWE-aware lexicons improve translation
quality.

Finally, in Chapter~11, {Jacquet, Ehrmann, Piskorski, Tanev and
  Steinberger} present an architecture which allows for the
identification of multiword entities (organizations, medical terms,
etc.) within large collections of texts, together with the linking of
monolingual variants of a given multiword entity, and of groups of
variants accross multiple languages. Their architecture is evaluated
against data from \textit{Wikipedia}.


\printbibliography[heading=subbibliography]
\end{refsection}


%\addchap{Acknowledgments}
\begin{refsection}

We are grateful to the 16 reviewers for their valuable evaluations,
comments and feedback, and to the proofreaders for their thorough
work. Without their help, this book would not exist.

We also would like to thank the COST program of the European Union for
its support for the PARSEME Action, which created a dynamic environment
leading to fruitful discussions around the topics addressed in this
book.

\begin{flushright}
  Yannick Parmentier and Jakub Waszczuk, Aug. 2017
\end{flushright}

\printbibliography[heading=subbibliography]
\end{refsection}



%% Additional prefaces and/or introductions that also have authors
\lsCollectionPaperFrontmatterMode % Enter the Frontmatter Mode. 
%\includepaper{chapters/prefaceEd}
\lsCollectionPaperMainmatterMode % Leave the Frontmatter Mode 

\setcounter{chapter}{0} % Reset the chapter counter so that preceeding prefaces are not counted
%% 
\mainmatter           
%%%%%%%%%%%%%%%%%%%%%%%%%%%%%%%%%%%%%%%%%%%%%%%%%%%%
%%%                                              %%%
%%%             Chapters                         %%%
%%%                                              %%%
%%%%%%%%%%%%%%%%%%%%%%%%%%%%%%%%%%%%%%%%%%%%%%%%%%%%
 
% MWE REPRESENTATION
\includepaper{chapters/lichte}
\includepaper{chapters/sheinfux}
\includepaper{chapters/dyviketal}
\includepaper{chapters/markantonatou}
\includepaper{chapters/angelov}
%% MWE PARSING
\includepaper{chapters/constant}
\includepaper{chapters/delhoneux}
\includepaper{chapters/foufi}
%% MWE IN MULTILINGUAL APPLICATIONS
\includepaper{chapters/semmar}
\includepaper{chapters/jacquet}

%%%%%%%%%%%%%%%%%%%%%%%%%%%%%%%%%%%%%%%%%%%%%%%%%%%%
%%%                                              %%%
%%%             Backmatter                       %%%
%%%                                              %%%
%%%%%%%%%%%%%%%%%%%%%%%%%%%%%%%%%%%%%%%%%%%%%%%%%%%%

% There is normally no need to change the backmatter section
\backmatter
\phantomsection%this allows hyperlink in ToC to work
\printbibliography[heading=references] 
\cleardoublepage

\phantomsection 
\addcontentsline{toc}{chapter}{Index} 
\addcontentsline{toc}{section}{\lsNameIndexTitle}
\ohead{Name \lsNameIndexTitle} 
\printindex 
\cleardoublepage
  
\phantomsection 
\addcontentsline{toc}{section}{\lsLanguageIndexTitle}
\ohead{\lsLanguageIndexTitle} 
\printindex[lan] 
\cleardoublepage
  
\phantomsection 
\addcontentsline{toc}{section}{\lsSubjectIndexTitle}
\ohead{\lsSubjectIndexTitle} 
\printindex[sbj]
\ohead{} 


\end{document} 

%%%%%%%%%%%%%%%%%%%%%%%%%%%%%%%%%%%%%%%%%%%%%%%%%%%%
%%%                                              %%%
%%%                  END                         %%%
%%%                                              %%%
%%%%%%%%%%%%%%%%%%%%%%%%%%%%%%%%%%%%%%%%%%%%%%%%%%%%

% you can create your book by running
% xelatex main.tex
%
% you can also try a simple 
% make
% on the commandline
