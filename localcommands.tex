%add all your local new commands to this file
\counterwithout{equation}{chapter} % remove the chapter number

\newcommand{\tobi}[2]}}
\renewcommand{\S}[1]{\tobi{#1}{\textsc{*}}}

%DELHONEUX
\newcommand{\subscript}[1]{\raisebox{-.4ex}{\scriptsize #1}}
\newcommand{\modelB}{\textit{model\subscript{B}}}
\newcommand{\modelA}{\textit{model\subscript{A}}}
\newcommand{\citepos}[1]{\citeauthor{#1}'s (\citeyear{#1})}

%CONSTANT
\newcommand{\asu}{UAS}
\newcommand{\asl}{LAS}
\newcommand{\asuo}{UAS$_{OA}$}
\newcommand{\aslo}{LAS$_{OA}$}
\newcommand{\asus}{UAS$_{surr}$}
\newcommand{\asls}{LAS$_{surr}$}
\newcommand{\ww}{$WW_U$}
\newcommand{\main}[1]{\textit{#1}}
\newcommand{\expl}[1]{\emph{#1}}
\newcommand{\carlos}[1]{\textcolor{green}{#1}}
% UNCOMMENT FOR FINAL VERSION!
\renewcommand{\todo}[1]{}
\renewcommand{\carlos}[1]{#1}

%WEHRLI
\newcommand{\cat}[2]{ [\nolinebreak[4]\begin{footnotesize}
	\hspace{-2mm}     \raisebox{-2mm}[2.8mm][2.8mm]{\texttt{{}#1}}
	 \end{footnotesize}\/#2 ]}
\newcommand{\cati}[3]{\texttt{[\nolinebreak[4]}\begin{footnotesize}
	\hspace{-1mm}  \raisebox{-2mm}[2.8mm][2.8mm]{\texttt{{}#1}}
	 \end{footnotesize}\/#2\texttt{]$_{#3}$}}
\newcommand{\ik}[1]{$_{#1}$}
%% \newcounter{example}
%% \newcounter{examplebis}
%% \newcounter{exampleref}
%% \newcommand{\refex}[1]{\setcounter{exampleref}{\value{example}}\addtocounter{exampleref}{#1}\arabic{exampleref}}
%% \newcommand{\emptyex}{\stepcounter{example}\noindent \\
%%   \makebox[0.9cm][r]{(\theexample)}}
%% \newcommand{\debex}[1]{\setcounter{examplebis}{1}
%%   \begin{list}{}{\vspace{-0.4cm}
%%       \setlength{\labelwidth}{1cm}}
%%     \addex{\alph{examplebis}.}{#1}}
%% \newcommand{\finex}{\vspace{-0.2cm} \noindent\end{list}}
%% \newcommand{\addex}[2]{\stepcounter{example}
%%           \item[(\theexample)#1] #2}
%% \newcommand{\putex}[1]{\stepcounter{examplebis}
%%           \item[\alph{examplebis}.] #1}
%% \newcommand{\makeex}[1]{\begin{list}{}{\vspace{-0.2cm}}
%%           \stepcounter{example}
%%           \item[(\theexample)] #1 \finex}
%% \newcommand{\textel}[1]{{#1}}

%SHEINFUX
\renewcommand{\emph}{\textit}
%For hebrew transliteration with and without glosses:
\newcommand{\textnl}{}%\textsl}
\newcommand{\textgl}[1]{`#1'}
\newcommand{\heb}[1]{\textnl{#1}}
%\newcommand{\hebgloss}[2]{\textit{#1} (`\textnl{#2}')}
\newcommand{\hebgloss}[2]{\textit{#1} `#2'}
\newcommand{\idgloss}[3]{\textit{\textnl{#1} \textgl{#2}~$\rightarrow$~\textgl{#3}}}
\newcommand{\reff}[1]{(\ref{#1})}
\newcommand{\quotecite}[1]{\citeauthor{#1}'s (\citeyear{#1})}
\newcommand{\undbf}[1]{\underline{\textbf{#1}}}
%For Hebrew transcription
\newcommand{\alef}{ʔ}
\newcommand{\alefB}{\textbf{ʔ}}
\newcommand{\ayin}{ʕ}
\newcommand{\ayinB}{\textbf{ʕ}}
\newcommand{\shin}{ʃ}
\newcommand{\shinB}{\textbf{ʃ}}
\newcommand{\het}{ħ}
\newcommand{\hetB}{\textbf{ħ}}
\newcommand{\tet}{\d{t}}
\newcommand{\tetB}{\d{t}}
\newcommand{\spacebr}[1]{\hphantom{\{}}

%LICHTE
%%%%%%%%%%%%%%%%%%%%%%
%   AVM SETTINGS     % 
%%%%%%%%%%%%%%%%%%%%%%

\avmoptions{center}	
\avmfont{\sc}
\avmvalfont{\rm}
\avmsortfont{\it}

\newenvironment{topbot}{   	% more flexible than /newcommand ?
	\avmvskip{0.2ex} 
	\hspace{-1.5em}
	\begin{avm}
	\avml
	}
	%%%
	{
	\avmr
    \end{avm}
    \hspace{-0.5em}
}


%%%%%%%%%%%%%%%%%%%%%%
%   TIKZ SETTINGS    % 
%%%%%%%%%%%%%%%%%%%%%%

\tikzset{every tree node/.style={align=center,anchor=north}}	% to allow linebreaks
\usetikzlibrary{calc} % for positioning arrows with ($(t.center)-(1,0)$)
\usetikzlibrary{shapes,snakes}
\usetikzlibrary{backgrounds,fit}
\usetikzlibrary{arrows}
\usetikzlibrary{matrix}
\usetikzlibrary{positioning}

% Define box and box title style (see http://www.texample.net/tikz/examples/boxes-with-text-and-math/)
\tikzstyle{mybox} = [draw=gray, very thick,
    rectangle, rounded corners, inner sep=10pt, inner ysep=17pt,yshift=3pt]
\tikzstyle{fancytitle} =[draw=gray, very thick, fill=white,
    rectangle, rounded corners, inner sep=5pt, inner ysep=5pt]
    
\tikzset{
    %Define standard arrow tip
    >=stealth',
    %Define style for boxes
    box/.style={
           rectangle,
           rounded corners,
           draw=black, very thick,
           text width=10em,
           minimum height=2em,
           text centered},
    % Define arrow style
    arrow/.style={
           ->,
           thick,
           	shorten <=2pt,
           shorten >=2pt,}
}

\newcommand\centertikz[1]{\tikz[baseline=(current bounding box.center)]{#1}}

    

%%%%%%%%%%%%%%%%%%%%%%
%   MISCELLANEOUS    % 
%%%%%%%%%%%%%%%%%%%%%%

\newcommand*\circled[1]{\tikz[baseline=(char.base)]{
    \node[shape=circle,draw,inner sep=.15ex] (char) {#1};}}
\newcommand{\svar}[1]
   {\setbox2=\hbox{$\scriptstyle #1$}\lower.2ex\vbox{\hrule
     \hbox{\vrule\kern1.25pt 
     \vbox{\kern1.25pt\box2\kern1.25pt}\kern1.25pt\vrule}\hrule}}
\newcommand{\ssvar}[1]
   {\setbox2=\hbox{\scalebox{.5}{$#1$}}\lower.2ex\vbox{\hrule
     \hbox{\vrule\kern1.25pt 
     \vbox{\kern1.25pt\box2\kern1.25pt}\kern1.25pt\vrule}\hrule}}
\newcommand{\trace}[0]{\underline{$~~~$}}

\newcommand{\prule}[3]{
      $\begin{array}{c} #1\\ \hline
         #2\end{array} ~~ #3$}
     
%% already defined in langsci.cls 
% \newlength{\stmueTmp}
% \newcommand*{\hspaceThis}[1]{\settowidth{\stmueTmp}{#1}\hspace*{\stmueTmp}}

\newcommand{\minitab}[2][c]{\begin{tabular}{@{}#1@{}}#2\end{tabular}}

\newenvironment{changemargin}[2]{%
  \begin{list}{}{%
    \setlength{\topsep}{0pt}%
    \setlength{\leftmargin}{#1}%
    \setlength{\rightmargin}{#2}%
    \setlength{\listparindent}{\parindent}%
    \setlength{\itemindent}{\parindent}%
    \setlength{\parsep}{\parskip}%
  }%
  \item[]}{\end{list}}
  
% \newtheorem{definition}{Definition}
% \newtheorem{corollary}{Corollary}
% \newtheorem{theorem}{Theorem}

%% \newcounter{sitem-zaehler}
%% \newenvironment{sitem}{\begin{list}{$\bullet$}
%%     {\usecounter{sitem-zaehler}
%%       \setlength{\itemindent}{0pt}
%%       \setlength{\labelwidth}{3ex}
%%       \setlength{\leftmargin}{3ex}
%%       \setlength{\topsep}{0ex plus0ex minus0ex}
%%       \setlength{\parsep}{0.3ex plus0ex minus0ex}
%%       \setlength{\itemsep}{0ex plus0ex minus0ex}
%%     }}{\end{list}}


\newcommand{\tl}[1]{\todo[color=red!30]{TL:\enspace#1}}
\newcommand{\simon}[1]{\todo[color=blue!30]{SP:\enspace#1}}
\newcommand{\jw}[1]{\todo[color=orange!70]{JW:\enspace#1}}
\newcommand{\as}[1]{\todo[color=green!30]{AS:\enspace#1}}

%In-line examples with translations, e.g. \textit{einjagen} lit. 'inchase'=>'chase in’                                                                         
\newcommand{\textex}{\textsl}
\newcommand{\textlit}[1]{(lit. \textsl{#1})} %literal meaning                   
\newcommand{\texttr}[1]{`#1'} %translation (idiomatic meaning)                  
\newcommand{\ile}[1]{\textex{#1}} %inline example, e.g. einjagen                
\newcommand{\ilet}[2]{\textex{#1} \texttr{#2}} %inline example with translation, e.g. einjagen 'chase in'                                                      
\newcommand{\ilel}[2]{\textex{#1} \textlit{#2}} %inline example with literal meaning, e.g. einjagen (lit. in-chase)                                            
\newcommand{\ilelt}[3]{\textex{#1} \textlit{#2}\,\texttr{#3}} %inline example with literal meaning and tran

\definecolor{lightgray}{gray}{0.7}

\lstnewenvironment{xmg}{%                                                       
  \lstset{language=,
    numbers=left,numbersep=8pt,numberstyle=\color{lightgray},
        % frame=l,                                                              
        basicstyle=\small\ttfamily,%                                            
    xleftmargin=0.7cm,framexleftmargin=12pt,%                                   
    framerule=0.5mm,rulecolor=\color{lightgray},%                               
    escapeinside={|\%}{\%|},%                                                   
    commentstyle=\color{lightgray},
    literate={->}{{{\textbf{->}}}}1 {\{}{{{\textbf{\{}}}}1 {\}}{{{\textbf{\}}}}}1 {\;}{{{\textbf{;}}}}1 {|}{{{\textbf{|}}}}1 {=}{{{\textbf{=}}}}1 {[}{{{\textbf{[}}}}1 {]}{{{\textbf{]}}}}1 {<}{{{\textbf{<}}}}1 {>}{{{\textbf{>}}}}1 {!}{{{\textbf{!}}}}1 {?}{{{\textbf{?}}}}1 {*=}{{{\textbf{*=}}}}1,%                     
    morekeywords={node,type,feature,include,class,import,export,declare,syn,sem,frame,morph,value, use, with, dims}}}{}

\lstnewenvironment{duelme}{%                                                    
  \lstset{language=,
    numbers=left,numbersep=8pt,numberstyle=\color{lightgray},
    % frame=l,                                                                  
    basicstyle=\small\ttfamily,%                                                
    xleftmargin=0.7cm,framexleftmargin=12pt,%                                   
    framerule=0.5mm,rulecolor=\color{lightgray},%                               
    escapeinside={|\%}{\%|},%                                                   
    commentstyle=\color{lightgray},
    morekeywords={PATERN,NAME,POS,PATTERN,MAPPING,EXAMPLE,MWE,SENTENCE,DESCRIPTION,COMMENT,LISTA,LISTB,SUBJECT,OBJECT,MODIFIER,RPRON,CONJUGATION,POLARITY,EXPRESSION,CL}}}{}

\lstnewenvironment{patr-listing}{%                                              
  \lstset{language=,
    numbers=left,numbersep=8pt,numberstyle=\color{lightgray},
    % frame=l,                                                                  
    basicstyle=\small\ttfamily,%                                                
    xleftmargin=0.7cm,framexleftmargin=12pt,%                                   
    framerule=0.5mm,rulecolor=\color{lightgray},%                               
    escapeinside={|\%}{\%|},%                                                   
    commentstyle=\color{lightgray},
    literate={:}{{{\textbf{:}}}}1 {\{}{{{\textbf{\{}}}}1 {\}}{{{\textbf{\}}}}}1 {=}{{{\textbf{=}}}}1 {[}{{{\textbf{[}}}}1 {]}{{{\textbf{]}}}}1 {<}{{{\textbf{<}}}}1 {>}{{{\textbf{>}}}}1 {!}{{{\textbf{!}}}}1,
    morekeywords={Define,as,Word}}}{}

\newcommand{\ixmg}{%                                                            
  \lstinline[language=,keepspaces,%                                             
  literate={->}{{{\textbf{->}}}}1 {\{}{{{\textbf{\{}}}}1 {\}}{{{\textbf{\}}}}}1 {\;}{{{\textbf{;}}}}1 {|}{{{\textbf{|}}}}1 {=}{{{\textbf{=}}}}1 {[}{{{\textbf{[}}}}1 {]}{{{\textbf{]}}}}1 {<}{{{\textbf{<}}}}1 {>}{{{\textbf{>}}}}1 {!}{{{\textbf{!}}}}1 {?}{{{\textbf{?}}}}1 {*=}{{{\textbf{*=}}}}1,%                       
  morekeywords={node,type,feature,include,class,import,export,declare,syn,sem,frame,morph,value, use, with, dims}
  ]}

\newcommand{\needsrevision}[1]{{\color{gray}#1}}

%% JACQUET
\newcommand{\ra}[1]{\renewcommand{\arraystretch}{#1}}
\definecolor{orange}{rgb}{0.64,0.16,0.16}
\newcommand{\remJakub}[1]{\textcolor{orange}{JP : #1}}

\definecolor{bleu}{rgb}{0.16,0.16,0.64}
\newcommand{\remGuillaume}[1]{\textcolor{bleu}{GJ : #1}}

\definecolor{vert}{rgb}{0.16,0.64,0.16}
\newcommand{\remMaud}[1]{\textcolor{vert}{ME : #1}}
