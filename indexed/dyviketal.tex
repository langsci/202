%!TEX encoding = UTF-8 Unicode
%!TEX TS-program = xelatex
\documentclass[output=paper]{langsci/langscibook}
\usepackage{hyperref} 
\title{Multiword expressions in an {LFG} grammar for {N}orwegian}
\author{Helge Dyvik\affiliation{University of Bergen}
\and
Gyri Smørdal Losnegaard\affiliation{University of Bergen}
\lastand
Victoria Rosén\affiliation{University of Bergen}
}
\newcommand{\definite}{\textsc{def}{}\xspace}	%definite
%% \newcommand{\eabox}[2][-.7\baselineskip]{
%%  \ea
%%    \parbox[t]{.8\textwidth}{
%%      \vspace{#1}
%%      #2
%%     }
%%  \z
%% }

%% \newcommand{\exbox}[2][-.7\baselineskip]{
%%  \ex
%%    \parbox[t]{.8\textwidth}{
%%      \vspace{#1}
%%      #2
%%     }
%% }
\hyphenation{struc-ture hier-ar-chi-cal ADVcmt Christ-church leksikalsk-funk-sjonell Indur-khya English}
% \chapterDOI{} %will be filled in at production
%\bibliography{dyviketal} % in preamble of main file

%\epigram{“You speak an infinite deal of nothing.”
%- William Shakespeare, The Merchant of Venice}
\abstract{This chapter describes the analysis of multiword expressions in NorGram, an LFG grammar of Norwegian.
All multiword expressions need to be accounted for in the lexicon, but in different ways depending on the flexibility of the expression.
Each multiword expression is provided with a lexical entry that has a special predicate name  incorporating the lexical items that the multiword consists of and that specifies the argument structure of the predicate.
In this way analyses are provided for a wide range of multiword types, including fixed expressions, phrasal verbs, verbal idioms, and others.
}
\maketitle
\begin{document}\ili{}
\ili{}
\ili{}\section\ili{}{Introduction}\ili{}\label\ili{}{sec\ili{}:intro}\ili{}
\ili{}
\ili{}%NOTES\ili{}:\ili{}
\ili{}%\ili{}
\ili{}%In\ili{} a\ili{} lexicalized\ili{} grammar\ili{} like\ili{} LFG\ili{},\ili{} all\ili{} multiword\ili{} expressions\ili{} need\ili{} to\ili{} be\ili{} accounted\ili{} for\ili{} in\ili{} the\ili{} lexicon\ili{},\ili{} but\ili{} in\ili{} different\ili{} ways\ili{} depending\ili{} on\ili{} the\ili{} flexibility\ili{} of\ili{} the\ili{} expression\ili{}.\ili{}
\ili{}%As\ili{} a\ili{} matter\ili{} of\ili{} principle\ili{} they\ili{} should\ili{} be\ili{} handled\ili{} in\ili{} the\ili{} simplest\ili{} way\ili{} possible\ili{}.\ili{}
\ili{}%Expressions\ili{} that\ili{} are\ili{} really\ili{} fixed\ili{} may\ili{} be\ili{} treated\ili{} in\ili{} the\ili{} same\ili{} way\ili{} as\ili{} simplex\ili{} words\ili{}.\ili{}
\ili{}%Expressions\ili{} that\ili{} are\ili{} more\ili{} flexible\ili{} need\ili{} to\ili{} allow\ili{} for\ili{} various\ili{} inflections\ili{} and\ili{} need\ili{} to\ili{} have\ili{} specifications\ili{} related\ili{} to\ili{} \ili{}\isi\ili{}{subcategorization}\ili{}.\ili{}
\ili{}%\ili{}
\ili{}
In\ili{} this\ili{} chapter\ili{}\footnote\ili{}{The\ili{} authors\ili{} have\ili{} contributed\ili{} equally\ili{} and\ili{} are\ili{} listed\ili{} in\ili{} alphabetical\ili{} order\ili{}.}\ili{} we\ili{} show\ili{} how\ili{} multiword\ili{} expressions\ili{} \ili{}(MWEs\ili{})\ili{} are\ili{} represented\ili{} in\ili{} NorGram\ili{},\ili{} a\ili{} hand\ili{}-written\ili{} computational\ili{} grammar\ili{} of\ili{} \ili{}\ili\ili{}{Norwegian}\ili{} \ili{}\citep\ili{}{Dyvik00en}\ili{}.\ili{} \ili{}
The\ili{} grammar\ili{} is\ili{} couched\ili{} in\ili{} the\ili{} Lexical\ili{}-Functional\ili{} Grammar\ili{} \ili{}(LFG\ili{})\ili{} formalism\ili{} \ili{}\citep\ili{}{BresnanLFS\ili{},DalrympleLFG}\ili{}.\ili{}
It\ili{} was\ili{} first\ili{} developed\ili{} in\ili{} the\ili{} context\ili{} of\ili{} the\ili{} Parallel\ili{} Grammar\ili{} Project\ili{} \ili{}(ParGram\ili{})\ili{},\ili{} an\ili{} international\ili{} cooperative\ili{} effort\ili{} to\ili{} develop\ili{} parallel\ili{} LFG\ili{} grammars\ili{} for\ili{} a\ili{} number\ili{} of\ili{} languages\ili{} \ili{}\citep\ili{}{Pargram02}\ili{}.\ili{}
The\ili{} Xerox\ili{} Linguistic\ili{} Environment\ili{} \ili{}(XLE\ili{})\ili{} is\ili{} the\ili{} platform\ili{} we\ili{} use\ili{} for\ili{} grammar\ili{} development\ili{} and\ili{} parsing\ili{} \ili{}\citep\ili{}{Maxwell93}\ili{}.\ili{}
\ili{}
NorGram\ili{} contains\ili{} about\ili{} 380\ili{} complex\ili{} syntactic\ili{} rules\ili{},\ili{} corresponding\ili{} to\ili{} a\ili{} transition\ili{} network\ili{} with\ili{} more\ili{} than\ili{} 160\ili{},000\ili{} states\ili{} and\ili{} more\ili{} than\ili{} 4\ili{}.7\ili{} million\ili{} arcs\ili{}.\ili{}
The\ili{} lexicon\ili{} comprises\ili{} approximately\ili{} 180\ili{},000\ili{} lemmas\ili{} for\ili{} \ili{}\ili\ili{}{Norwegian}\ili{} Bokmål\ili{} and\ili{} 110\ili{},000\ili{} lemmas\ili{} for\ili{} \ili{}\ili\ili{}{Norwegian}\ili{} Nynorsk\ili{}.\ili{}
NorGram\ili{} uses\ili{} not\ili{} only\ili{} the\ili{} grammar\ili{} rules\ili{} and\ili{} the\ili{} lexicon\ili{} but\ili{} also\ili{} templates\ili{} to\ili{} efficiently\ili{} encode\ili{} linguistic\ili{} generalizations\ili{}.\ili{}
As\ili{} noted\ili{} in\ili{} \ili{}\citet\ili{}[207\ili{}]\ili{}{Dalrymple04}\ili{},\ili{} templates\ili{} in\ili{} LFG\ili{} grammars\ili{} \ili{}“can\ili{} play\ili{} the\ili{} same\ili{} role\ili{} in\ili{} capturing\ili{} linguistic\ili{} generalizations\ili{} as\ili{} hierarchical\ili{} type\ili{} systems\ili{} in\ili{} theories\ili{} like\ili{} HPSG\ili{}”\ili{}.\ili{}
Templates\ili{} are\ili{} for\ili{} instance\ili{} used\ili{} to\ili{} express\ili{} generalizations\ili{} about\ili{} \ili{}\isi\ili{}{subcategorization}\ili{} frames\ili{} for\ili{} verbs\ili{};\ili{} there\ili{} are\ili{} more\ili{} than\ili{} 200\ili{} such\ili{} verbal\ili{} templates\ili{}.\ili{} \ili{} \ili{}
\ili{}
NorGram\ili{} analyzes\ili{} several\ili{} types\ili{} of\ili{} MWEs\ili{},\ili{} including\ili{} fixed\ili{} and\ili{} flexible\ili{} expressions\ili{}.\ili{}
The\ili{} classification\ili{} of\ili{} MWEs\ili{} according\ili{} to\ili{} their\ili{} relative\ili{} flexibility\ili{} was\ili{} initially\ili{} proposed\ili{} for\ili{} \ili{}\ili\ili{}{English}\ili{} \ili{}\citep\ili{}{Sag02\ili{},\ili{} Baldwin10}\ili{},\ili{} presupposing\ili{} that\ili{} MWEs\ili{} with\ili{} the\ili{} same\ili{} degree\ili{} of\ili{} flexibility\ili{} may\ili{} receive\ili{} the\ili{} same\ili{} or\ili{} similar\ili{} treatment\ili{} in\ili{} NLP\ili{} systems\ili{}.\ili{}
The\ili{} distinction\ili{} between\ili{} fixed\ili{},\ili{} semi\ili{}-fixed\ili{} and\ili{} syntactically\ili{} flexible\ili{} MWEs\ili{} may\ili{} thus\ili{} be\ili{} useful\ili{} also\ili{} for\ili{} other\ili{} languages\ili{} than\ili{} \ili{}\ili\ili{}{English}\ili{},\ili{} although\ili{} the\ili{} criteria\ili{} for\ili{} distinguishing\ili{} between\ili{} the\ili{} classes\ili{} may\ili{} vary\ili{}.\ili{}
\ili{}
Fixed\ili{} MWEs\ili{} are\ili{} found\ili{} in\ili{} most\ili{} languages\ili{} with\ili{} MWEs\ili{} and\ili{} in\ili{} basically\ili{} every\ili{} part\ili{} of\ili{} speech\ili{}.\ili{}
These\ili{} are\ili{} expressions\ili{} that\ili{} are\ili{} completely\ili{} invariable\ili{},\ili{} with\ili{} no\ili{} morphosyntactic\ili{} variation\ili{} or\ili{} internal\ili{} modification\ili{},\ili{} such\ili{} as\ili{} the\ili{} adverb\ili{} \ili{}\emph\ili{}{by\ili{} the\ili{} way}\ili{} and\ili{} the\ili{} determiner\ili{} \ili{}\emph\ili{}{each\ili{} and\ili{} every}\ili{}.\ili{} \ili{}
Semi\ili{}-fixed\ili{} MWEs\ili{},\ili{} as\ili{} defined\ili{} for\ili{} \ili{}\ili\ili{}{English}\ili{},\ili{} allow\ili{} some\ili{} lexical\ili{} and\ili{} morphological\ili{} variation\ili{} such\ili{} as\ili{} limited\ili{} internal\ili{} modification\ili{} and\ili{} inflection\ili{},\ili{} while\ili{} the\ili{} relative\ili{} word\ili{} order\ili{} of\ili{} the\ili{} components\ili{} does\ili{} not\ili{} change\ili{}.\ili{} \ili{} \ili{}
Examples\ili{} are\ili{} compound\ili{} nominals\ili{} \ili{}(\ili{}\emph\ili{}{chicken\ili{} soup}\ili{})\ili{},\ili{} proper\ili{} names\ili{},\ili{} such\ili{} as\ili{} \ili{}\emph\ili{}{Donald\ili{} Duck}\ili{} and\ili{} the\ili{} subset\ili{} of\ili{} verbal\ili{} idioms\ili{} with\ili{} fixed\ili{} word\ili{} order\ili{},\ili{} such\ili{} as\ili{} \ili{}\emph\ili{}{shoot\ili{} the\ili{} breeze}\ili{} \ili{}`chat\ili{}'\ili{} and\ili{} \ili{}\emph\ili{}{kick\ili{} the\ili{} bucket}\ili{} \ili{}`die\ili{}'\ili{}.\ili{}
Syntactically\ili{}-flexible\ili{} expressions\ili{} display\ili{} a\ili{} wider\ili{} range\ili{} of\ili{} flexibility\ili{},\ili{} allowing\ili{} some\ili{} or\ili{} all\ili{} types\ili{} of\ili{} syntactic\ili{} variation\ili{} including\ili{} \ili{}\isi\ili{}{passivization}\ili{},\ili{} \ili{}\isi\ili{}{relativization}\ili{} and\ili{} other\ili{} operations\ili{} that\ili{} are\ili{} not\ili{} possible\ili{} in\ili{} semi\ili{}-fixed\ili{} MWEs\ili{}.\ili{}
All\ili{} flexible\ili{} MWEs\ili{} are\ili{} verbal\ili{}.\ili{} \ili{}
They\ili{} include\ili{} \ili{}\isi\ili{}{verb\ili{}-particle\ili{} constructions}\ili{},\ili{} light\ili{} verbs\ili{},\ili{} and\ili{} the\ili{} subset\ili{} of\ili{} verbal\ili{} idioms\ili{} whose\ili{} word\ili{} order\ili{} is\ili{} less\ili{} restricted\ili{} than\ili{} semi\ili{}-fixed\ili{} expressions\ili{}.\ili{} \ili{}
Table\ili{} \ili{}\ref\ili{}{tab\ili{}:mweiness\ili{}:flexibilityclasses}\ili{} illustrates\ili{} how\ili{} common\ili{} types\ili{} of\ili{} \ili{}\ili\ili{}{English}\ili{} MWEs\ili{} distribute\ili{} over\ili{} these\ili{} classes\ili{}.\ili{}
\ili{}
\ili{}\begin\ili{}{table}\ili{}
\ili{} \ili{} \ili{}\begin\ili{}{tabular}\ili{}{lll}\ili{}
\ili{} \ili{} \ili{} \ili{} \ili{}\lsptoprule\ili{}
\ili{} \ili{} \ili{} \ili{} \ili{}\textbf\ili{}{Flexibility\ili{} class}\ili{} \ili{}&\ili{} \ili{}\textbf\ili{}{Type}\ili{} \ili{}&\ili{} \ili{}\textbf\ili{}{Example\ili{} MWE}\ili{} \ili{}\\ili{}\\ili{}
\ili{} \ili{} \ili{} \ili{} \ili{}\midrule\ili{}
\ili{}	Fixed\ili{} \ili{}&\ili{} \ili{} \ili{}&\ili{} \ili{}\emph\ili{}{by\ili{} the\ili{} way}\ili{} \ili{}\\ili{}\\ili{}\hline\ili{}	\ili{}
\ili{}	Semi\ili{}-fixed\ili{} \ili{}&\ili{} compound\ili{} nominals\ili{} \ili{} \ili{}&\ili{} \ili{}\emph\ili{}{chicken\ili{} soup}\ili{} \ili{}\\ili{}\\ili{}
\ili{}	\ili{}&\ili{} proper\ili{} names\ili{} \ili{}&\ili{} \ili{}\emph\ili{}{Donald\ili{} Duck}\ili{} \ili{}\\ili{}\\ili{}
\ili{}	\ili{}&\ili{} non\ili{}-decomposable\ili{} idioms\ili{} \ili{}&\ili{} \ili{}\emph\ili{}{kick\ili{} the\ili{} bucket}\ili{} \ili{}\\ili{}\\ili{}\hline\ili{}
\ili{}	Flexible\ili{} \ili{}&\ili{} \ili{}\isi\ili{}{verb\ili{}-particle\ili{} constructions}\ili{} \ili{}&\ili{} \ili{} \ili{}\emph\ili{}{give\ili{} up}\ili{} \ili{}\\ili{}\\ili{}
\ili{}	\ili{}&\ili{} light\ili{} verbs\ili{} \ili{}&\ili{} \ili{}\emph\ili{}{give\ili{} a\ili{} speech}\ili{} \ili{}\\ili{}\\ili{}
\ili{}	\ili{}&\ili{} decomposable\ili{} idioms\ili{} \ili{}&\ili{} \ili{}\emph\ili{}{spill\ili{} the\ili{} beans}\ili{} \ili{}\\ili{}\\ili{}
\ili{} \ili{} \ili{} \ili{} \ili{}\lspbottomrule\ili{}
\ili{} \ili{} \ili{}\end\ili{}{tabular}\ili{}
\ili{} \ili{} \ili{}\caption\ili{}{Classes\ili{} of\ili{} flexibility}\ili{}
\ili{} \ili{} \ili{}\label\ili{}{tab\ili{}:mweiness\ili{}:flexibilityclasses}\ili{}
\ili{}\end\ili{}{table}\ili{}
\ili{}
\ili{}%The\ili{} syntactic\ili{} variation\ili{} in\ili{} \ili{}\isi\ili{}{verbal\ili{} MWEs}\ili{} in\ili{} \ili{}\ili\ili{}{English}\ili{} has\ili{} fostered\ili{} a\ili{} theory\ili{} of\ili{} semantic\ili{} decomposability\ili{} \ili{}\cite\ili{}{Nunberg94}\ili{} that\ili{} has\ili{} induced\ili{} widespread\ili{} attention\ili{} to\ili{} the\ili{} relation\ili{} between\ili{} the\ili{} syntax\ili{} and\ili{} semantics\ili{} of\ili{} \ili{}\isi\ili{}{verbal\ili{} MWEs}\ili{}.\ili{}
The\ili{} syntactic\ili{} variation\ili{} in\ili{} \ili{}\isi\ili{}{verbal\ili{} MWEs}\ili{} in\ili{} \ili{}\ili\ili{}{English}\ili{} has\ili{} given\ili{} rise\ili{} to\ili{} a\ili{} theory\ili{} of\ili{} semantic\ili{} decomposability\ili{} \ili{}\citep\ili{}{Nunberg94}\ili{} which\ili{} has\ili{} led\ili{} to\ili{} increased\ili{} interest\ili{} in\ili{} the\ili{} relation\ili{} between\ili{} the\ili{} syntax\ili{} and\ili{} semantics\ili{} of\ili{} \ili{}\isi\ili{}{verbal\ili{} MWEs}\ili{}.\ili{}
Semantic\ili{} decomposabitity\ili{} is\ili{} a\ili{} measure\ili{} of\ili{} whether\ili{} the\ili{} meaning\ili{} of\ili{} the\ili{} expression\ili{} distributes\ili{} over\ili{} the\ili{} MWE\ili{} components\ili{} or\ili{} only\ili{} relates\ili{} to\ili{} the\ili{} expression\ili{} as\ili{} a\ili{} whole\ili{}.\ili{}
It\ili{} may\ili{} explain\ili{} why\ili{} individual\ili{} parts\ili{} of\ili{} an\ili{} expression\ili{} may\ili{} be\ili{} fronted\ili{},\ili{} topicalized\ili{},\ili{} and\ili{} relativized\ili{},\ili{} and\ili{} may\ili{} also\ili{} in\ili{} other\ili{} ways\ili{} contribute\ili{} meaningfully\ili{} to\ili{} the\ili{} information\ili{} structure\ili{} of\ili{} the\ili{} sentence\ili{}.\ili{}
On\ili{} the\ili{} other\ili{} hand\ili{},\ili{} semantic\ili{} non\ili{}-decomposability\ili{} blocks\ili{} compositional\ili{} interpretations\ili{},\ili{} which\ili{} again\ili{} explains\ili{} why\ili{} semi\ili{}-fixed\ili{} MWEs\ili{} are\ili{} not\ili{} subject\ili{} to\ili{} operations\ili{} that\ili{} would\ili{} normally\ili{} indicate\ili{} that\ili{} their\ili{} components\ili{} are\ili{} associated\ili{} with\ili{} some\ili{} independent\ili{} meaning\ili{}.\ili{}
\ili{}
While\ili{} a\ili{} distinction\ili{} between\ili{} semantically\ili{} decomposable\ili{} and\ili{} nondecomposable\ili{} verbal\ili{} idioms\ili{} may\ili{} also\ili{} hold\ili{} for\ili{} \ili{}\ili\ili{}{Norwegian}\ili{},\ili{} the\ili{} correlation\ili{} between\ili{} syntactic\ili{} flexibility\ili{} and\ili{} semantic\ili{} decomposability\ili{} seems\ili{} less\ili{} conspicuous\ili{} than\ili{} for\ili{} \ili{}\ili\ili{}{English}\ili{}.\ili{} \ili{}
In\ili{} particular\ili{},\ili{} \ili{}\ili\ili{}{Norwegian}\ili{} has\ili{} subject\ili{}-verb\ili{} inversion\ili{} in\ili{} interrogative\ili{} main\ili{} clauses\ili{},\ili{} so\ili{} that\ili{} the\ili{} word\ili{} order\ili{} will\ili{} vary\ili{} in\ili{} MWEs\ili{} that\ili{} are\ili{} otherwise\ili{} highly\ili{} restricted\ili{}.\ili{} \ili{}
Most\ili{} verbal\ili{} idioms\ili{} may\ili{} also\ili{} undergo\ili{} at\ili{} least\ili{} some\ili{} modification\ili{} such\ili{} as\ili{} impersonal\ili{} passives\ili{}.\ili{}
\ili{}%The\ili{} correlation\ili{} between\ili{} syntactic\ili{} flexibility\ili{} and\ili{} semantic\ili{} decomposability\ili{} in\ili{} \ili{}\ili\ili{}{Norwegian}\ili{} verbal\ili{} idioms\ili{} is\ili{} thus\ili{} less\ili{} conspicuous\ili{} than\ili{} for\ili{} \ili{}\ili\ili{}{English}\ili{}.\ili{}
Furthermore\ili{},\ili{} the\ili{} mechanisms\ili{} for\ili{} representing\ili{} restrictions\ili{} and\ili{} variation\ili{} in\ili{} NorGram\ili{} are\ili{} technically\ili{} the\ili{} same\ili{} for\ili{} semi\ili{}-fixed\ili{} and\ili{} flexible\ili{} MWEs\ili{}.\ili{} \ili{}
Since\ili{} no\ili{} distinction\ili{} is\ili{} reflected\ili{} in\ili{} the\ili{} way\ili{} \ili{}\isi\ili{}{verbal\ili{} MWEs}\ili{} are\ili{} represented\ili{} in\ili{} the\ili{} lexicon\ili{} and\ili{} grammar\ili{},\ili{} all\ili{} such\ili{} MWEs\ili{} are\ili{} considered\ili{} flexible\ili{},\ili{} \ili{}
and\ili{} MWEs\ili{} with\ili{} similar\ili{} morphosyntactic\ili{} properties\ili{} are\ili{} accounted\ili{} for\ili{} with\ili{} templates\ili{} which\ili{} are\ili{} in\ili{} effect\ili{} mini\ili{}-grammars\ili{} for\ili{} subsets\ili{} of\ili{} MWEs\ili{}.\ili{}
\ili{}%\ili{},\ili{} for\ili{} instance\ili{} verb\ili{}-object\ili{} combinations\ili{} with\ili{} the\ili{} same\ili{} restrictions\ili{} on\ili{} number\ili{},\ili{} definiteness\ili{} and\ili{} modifiability\ili{},\ili{} \ili{}
\ili{}
With\ili{} respect\ili{} to\ili{} subtypes\ili{} of\ili{} MWEs\ili{},\ili{} the\ili{} types\ili{} of\ili{} MWEs\ili{} analyzed\ili{} by\ili{} NorGram\ili{} more\ili{} or\ili{} less\ili{} correspond\ili{} to\ili{} the\ili{} types\ili{} in\ili{} Table\ili{} \ili{}\ref\ili{}{tab\ili{}:mweiness\ili{}:flexibilityclasses}\ili{},\ili{} with\ili{} a\ili{} few\ili{} exceptions\ili{}.\ili{} \ili{}
As\ili{} in\ili{} many\ili{} other\ili{} Germanic\ili{} languages\ili{},\ili{} compound\ili{} nominals\ili{} in\ili{} \ili{}\ili\ili{}{Norwegian}\ili{} form\ili{} single\ili{} graphical\ili{} words\ili{}.\ili{} \ili{}
These\ili{} are\ili{} thus\ili{} not\ili{} considered\ili{} multiword\ili{} expressions\ili{}.\ili{}
In\ili{} addition\ili{} to\ili{} \ili{}\isi\ili{}{prepositional\ili{} verbs}\ili{},\ili{} NorGram\ili{} analyzes\ili{} nouns\ili{} and\ili{} adjectives\ili{} with\ili{} selected\ili{} prepositions\ili{} as\ili{} MWEs\ili{}.\ili{}
Expressions\ili{} that\ili{} are\ili{} completely\ili{} regular\ili{} on\ili{} the\ili{} morphological\ili{} and\ili{} syntactic\ili{} levels\ili{},\ili{} such\ili{} as\ili{} light\ili{} verb\ili{} constructions\ili{},\ili{} are\ili{} analyzed\ili{} compositionally\ili{} by\ili{} the\ili{} grammar\ili{} and\ili{} are\ili{} not\ili{} represented\ili{} in\ili{} the\ili{} lexicon\ili{} as\ili{} MWEs\ili{}.\ili{}
A\ili{} special\ili{} case\ili{} is\ili{} complex\ili{} numerals\ili{} such\ili{} as\ili{} \ili{}\emph\ili{}{hundre\ili{} og\ili{} to}\ili{} \ili{}`one\ili{} hundred\ili{} two\ili{}'\ili{} and\ili{} \ili{} \ili{}\emph\ili{}{to\ili{} og\ili{} nitti}\ili{} \ili{}`ninety\ili{} two\ili{}'\ili{},\ili{} which\ili{} may\ili{} also\ili{} be\ili{} considered\ili{} a\ili{} subtype\ili{} of\ili{} MWE\ili{}.\ili{}
The\ili{} particular\ili{} syntax\ili{} and\ili{} semantics\ili{} of\ili{} such\ili{} expressions\ili{} is\ili{} accounted\ili{} for\ili{} with\ili{} a\ili{} special\ili{} set\ili{} of\ili{} lexical\ili{} entries\ili{} and\ili{} syntactic\ili{} rules\ili{}.\ili{}
\ili{}
NorGramBank\ili{},\ili{} a\ili{} large\ili{} parsebank\ili{} for\ili{} \ili{}\ili\ili{}{Norwegian}\ili{},\ili{} has\ili{} been\ili{} created\ili{} by\ili{} parsing\ili{} a\ili{} corpus\ili{} with\ili{} NorGram\ili{} \ili{}\citep\ili{}{Dyvik16}\ili{}.\ili{}
Because\ili{} of\ili{} lexical\ili{} and\ili{} syntactic\ili{} ambiguity\ili{},\ili{} parsing\ili{} with\ili{} NorGram\ili{} often\ili{} results\ili{} in\ili{} many\ili{} analyses\ili{} for\ili{} each\ili{} sentence\ili{},\ili{} and\ili{} efficient\ili{} disambiguation\ili{} is\ili{} therefore\ili{} necessary\ili{}.\ili{}
The\ili{} INESS\ili{} project\ili{}\footnote\ili{}{\ili{}\url\ili{}{http\ili{}:\ili{}/\ili{}/clarino\ili{}.uib\ili{}.no\ili{}/iness}}\ili{} has\ili{} developed\ili{} a\ili{} treebanking\ili{} infrastructure\ili{} for\ili{} parsing\ili{},\ili{} disambiguating\ili{},\ili{} storing\ili{},\ili{} and\ili{} searching\ili{} the\ili{} texts\ili{} in\ili{} NorGramBank\ili{} \ili{}\citep\ili{}{rosen12lrec}\ili{}.\ili{}
The\ili{} parsebank\ili{} currently\ili{} consists\ili{} of\ili{} about\ili{} 60\ili{} million\ili{} words\ili{} of\ili{} analyzed\ili{} text\ili{},\ili{} of\ili{} which\ili{} sentences\ili{} covering\ili{} 350\ili{},000\ili{} words\ili{} have\ili{} been\ili{} manually\ili{} disambiguated\ili{} by\ili{} computer\ili{}-generated\ili{} discriminants\ili{} \ili{}\citep\ili{}{Rosen07lfg}\ili{}.\ili{}
The\ili{} remainder\ili{} of\ili{} the\ili{} corpus\ili{} has\ili{} been\ili{} stochastically\ili{} disambiguated\ili{}.\ili{}
INESS\ili{} Search\ili{} is\ili{} a\ili{} tool\ili{} for\ili{} searching\ili{} in\ili{} LFG\ili{} and\ili{} other\ili{} \ili{}\isi\ili{}{treebanks}\ili{} in\ili{} the\ili{} treebanking\ili{} infrastructure\ili{} \ili{}\citep\ili{}{Meurer12}\ili{}.\ili{}
MWEs\ili{} are\ili{} analyzed\ili{} by\ili{} NorGram\ili{} in\ili{} such\ili{} a\ili{} way\ili{} that\ili{} the\ili{} different\ili{} types\ili{} may\ili{} be\ili{} searched\ili{} for\ili{}.\ili{}
\ili{}
The\ili{} original\ili{} lexical\ili{} resource\ili{} used\ili{} for\ili{} the\ili{} NorGram\ili{} lexicon\ili{},\ili{} NorKompLeks\ili{},\ili{} contained\ili{} a\ili{} small\ili{} number\ili{} of\ili{} fixed\ili{} expressions\ili{} \ili{}\citep\ili{}{Nordgard00}\ili{}.\ili{}
The\ili{} main\ili{} design\ili{} of\ili{} the\ili{} treatment\ili{} of\ili{} MWEs\ili{} in\ili{} NorGram\ili{} was\ili{} developed\ili{} during\ili{} ParGram\ili{} \ili{}\citep\ili{}{Pargram02}\ili{} and\ili{} especially\ili{} during\ili{} the\ili{} LOGON\ili{} machine\ili{} translation\ili{} project\ili{} \ili{}\citep\ili{}{Lonning04}\ili{}.\ili{}
A\ili{} large\ili{} number\ili{} of\ili{} MWEs\ili{} have\ili{} been\ili{} added\ili{} to\ili{} NorGram\ili{}’s\ili{} lexicon\ili{} during\ili{} the\ili{} construction\ili{} of\ili{} NorGramBank\ili{}.\ili{}
When\ili{} annotators\ili{} discovered\ili{} MWEs\ili{} that\ili{} did\ili{} not\ili{} get\ili{} an\ili{} analysis\ili{} or\ili{} that\ili{} got\ili{} an\ili{} incorrect\ili{} analysis\ili{},\ili{} they\ili{} constructed\ili{} new\ili{} lexical\ili{} entries\ili{} or\ili{} edited\ili{} existing\ili{} lexical\ili{} entries\ili{} as\ili{} needed\ili{} in\ili{} order\ili{} to\ili{} cover\ili{} the\ili{} MWEs\ili{} \ili{}\citep\ili{}{Losnegaard12\ili{},Rosen16lre}\ili{}.\ili{}
\ili{}
This\ili{} chapter\ili{} is\ili{} organized\ili{} as\ili{} follows\ili{}.\ili{}
In\ili{} Section\ili{} \ili{}\ref\ili{}{sec\ili{}:mweiness\ili{}:LFG}\ili{} an\ili{} overview\ili{} of\ili{} the\ili{} basics\ili{} of\ili{} LFG\ili{} is\ili{} given\ili{},\ili{} showing\ili{} how\ili{} constructions\ili{} without\ili{} MWEs\ili{} are\ili{} analyzed\ili{} in\ili{} NorGram\ili{} as\ili{} a\ili{} background\ili{} for\ili{} the\ili{} treatment\ili{} of\ili{} MWEs\ili{} in\ili{} the\ili{} following\ili{} sections\ili{}.\ili{}
Section\ili{} \ili{}\ref\ili{}{sec\ili{}:mweiness\ili{}:fixed}\ili{} illustrates\ili{} the\ili{} analysis\ili{} of\ili{} fixed\ili{} expressions\ili{},\ili{} while\ili{} Section\ili{} \ili{}\ref\ili{}{sec\ili{}:mweiness\ili{}:flexexp}\ili{} is\ili{} about\ili{} the\ili{} analysis\ili{} of\ili{} flexible\ili{} expressions\ili{},\ili{} including\ili{} \ili{}\isi\ili{}{phrasal\ili{} verbs}\ili{},\ili{} verbal\ili{} idioms\ili{},\ili{} and\ili{} nonverbal\ili{} flexible\ili{} expressions\ili{}.\ili{}
Section\ili{} \ili{}\ref\ili{}{sec\ili{}:mweiness\ili{}:variation}\ili{} shows\ili{} how\ili{} various\ili{} syntactic\ili{} modifications\ili{} are\ili{} handled\ili{},\ili{} including\ili{} intervening\ili{} words\ili{},\ili{} \ili{}\isi\ili{}{long\ili{}-distance\ili{} dependencies}\ili{} and\ili{} passive\ili{} alternations\ili{}.\ili{}
Section\ili{} \ili{}\ref\ili{}{sec\ili{}:mweiness\ili{}:complementation}\ili{} discusses\ili{} numerous\ili{} complex\ili{} complementation\ili{} patterns\ili{} that\ili{} are\ili{} covered\ili{} by\ili{} NorGram\ili{} for\ili{} \ili{}\ili\ili{}{Norwegian}\ili{} MWEs\ili{}.\ili{}
Section\ili{} \ili{}\ref\ili{}{sec\ili{}:mweiness\ili{}:conc}\ili{} presents\ili{} our\ili{} conclusions\ili{}.\ili{}
\ili{}
\ili{}\section\ili{}{Syntactic\ili{} analysis\ili{} in\ili{} LFG}\ili{}\label\ili{}{sec\ili{}:mweiness\ili{}:LFG}\ili{}
\ili{}
LFG\ili{} analyses\ili{} have\ili{} two\ili{} distinct\ili{} levels\ili{} of\ili{} syntactic\ili{} representation\ili{}:\ili{} constituent\ili{} structure\ili{} \ili{}(c\ili{}-structure\ili{})\ili{} and\ili{} functional\ili{} structure\ili{} \ili{}(f\ili{}-structure\ili{})\ili{}.\ili{}
The\ili{} c\ili{}-structure\ili{} is\ili{} a\ili{} \ili{}\isi\ili{}{phrase}\ili{} structure\ili{} tree\ili{} that\ili{} represents\ili{} precedence\ili{} and\ili{} dominance\ili{} relations\ili{}.\ili{}
The\ili{} f\ili{}-structure\ili{} is\ili{} an\ili{} attribute\ili{}-value\ili{} matrix\ili{} with\ili{} information\ili{} about\ili{} grammatical\ili{} functions\ili{} such\ili{} as\ili{} subject\ili{} and\ili{} object\ili{} and\ili{} grammatical\ili{} features\ili{} such\ili{} as\ili{} tense\ili{},\ili{} gender\ili{} and\ili{} number\ili{}.\ili{}
An\ili{} example\ili{} of\ili{} a\ili{} NorGram\ili{} analysis\ili{} of\ili{} the\ili{} sentence\ili{} in\ili{} \ili{}(\ili{}\ref\ili{}{ex\ili{}:mweiness\ili{}:thinking\ili{}-while\ili{}-on\ili{}-bus}\ili{})\ili{} is\ili{} given\ili{} in\ili{} Figure\ili{} \ili{}\ref\ili{}{fig\ili{}:mweiness\ili{}:thinking\ili{}-while\ili{}-on\ili{}-bus}\ili{}.\ili{}\footnote\ili{}{In\ili{} this\ili{} example\ili{} the\ili{} morphological\ili{} structure\ili{} of\ili{} the\ili{} word\ili{} form\ili{} \ili{}\textit\ili{}{bussen}\ili{} is\ili{} indicated\ili{} since\ili{} it\ili{} is\ili{} relevant\ili{} for\ili{} the\ili{} analysis\ili{} being\ili{} discussed\ili{}.\ili{}
Otherwise\ili{} we\ili{} simplify\ili{} the\ili{} glossing\ili{} by\ili{} omitting\ili{} morpheme\ili{}-by\ili{}-morpheme\ili{} analysis\ili{} and\ili{} using\ili{} two\ili{} \ili{}\ili\ili{}{English}\ili{} words\ili{} to\ili{} render\ili{} one\ili{} \ili{}\ili\ili{}{Norwegian}\ili{} word\ili{} when\ili{} necessary\ili{}.}\ili{}
\ili{}%In\ili{} this\ili{} and\ili{} subsequent\ili{} examples\ili{} the\ili{} words\ili{} making\ili{} up\ili{} the\ili{} MWE\ili{} are\ili{} highlighted\ili{} with\ili{} boldface\ili{}.\ili{}
\ili{}%\ili{}\ea\ili{}\label\ili{}{ex\ili{}:mweiness\ili{}:thinking\ili{}-while\ili{}-on\ili{}-bus}\ili{}
\ili{}%\ili{}\gll\ili{} Hun\ili{} \ili{}\textbf\ili{}{tenkte}\ili{} \ili{}\textbf\ili{}{på}\ili{} bussen\ili{}.\ili{} \ili{}\\ili{}\\ili{}
\ili{}%\ili{} \ili{} \ili{} \ili{} \ili{} she\ili{} thought\ili{} on\ili{} \ili{}{the\ili{} bus}\ili{}\\ili{}\\ili{}
\ili{}%\ili{}\glt\ili{} \ili{}`She\ili{} was\ili{} thinking\ili{} \ili{}(while\ili{})\ili{} on\ili{} the\ili{} bus\ili{}.\ili{}/She\ili{} thought\ili{} about\ili{} the\ili{} bus\ili{}.\ili{}’\ili{}
\ili{}%\ili{}\z\ili{}
\ili{}
\ili{}\ea\ili{}\label\ili{}{ex\ili{}:mweiness\ili{}:thinking\ili{}-while\ili{}-on\ili{}-bus}\ili{}
\ili{}\gll\ili{} Hun\ili{} tenkte\ili{} på\ili{} buss\ili{}-en\ili{}.\ili{} \ili{}\\ili{}\\ili{}
\ili{} \ili{} \ili{} \ili{} \ili{} she\ili{} thought\ili{} on\ili{} bus\ili{}-\ili{}{\ili{}\definite}\ili{}.\ili{}{\ili{}\sg}\ili{}\\ili{}\\ili{}
\ili{}\glt\ili{} \ili{}`She\ili{} was\ili{} thinking\ili{} \ili{}(while\ili{})\ili{} on\ili{} the\ili{} bus\ili{}.\ili{}/She\ili{} thought\ili{} about\ili{} the\ili{} bus\ili{}.\ili{}’\ili{}
\ili{}\z\ili{}
\ili{}
\ili{}\begin\ili{}{figure}\ili{}
\ili{} \ili{} \ili{}\includegraphics\ili{}[width\ili{}=\ili{}\textwidth\ili{}]\ili{}{figures\ili{}/tenke\ili{}-paa\ili{}-bussen\ili{}-highlight\ili{}.png}\ili{}
\ili{} \ili{} \ili{}\caption\ili{}{C\ili{}-\ili{} and\ili{} f\ili{}-structure\ili{} for\ili{} \ili{}\textit\ili{}{Hun\ili{} tenkte\ili{} på\ili{} bussen}\ili{}.}\ili{}
\ili{} \ili{} \ili{}\label\ili{}{fig\ili{}:mweiness\ili{}:thinking\ili{}-while\ili{}-on\ili{}-bus}\ili{}
\ili{}\end\ili{}{figure}\ili{}
\ili{}
This\ili{} sentence\ili{} is\ili{} ambiguous\ili{},\ili{} as\ili{} shown\ili{} by\ili{} the\ili{} two\ili{} idiomatic\ili{} translations\ili{}.\ili{}
The\ili{} analysis\ili{} in\ili{} Figure\ili{} \ili{}\ref\ili{}{fig\ili{}:mweiness\ili{}:thinking\ili{}-while\ili{}-on\ili{}-bus}\ili{} concerns\ili{} the\ili{} first\ili{} translation\ili{},\ili{} where\ili{} the\ili{} prepositional\ili{} \ili{}\isi\ili{}{phrase}\ili{} \ili{}\textit\ili{}{på\ili{} bussen}\ili{} \ili{}`on\ili{} the\ili{} bus\ili{}'\ili{} is\ili{} an\ili{} adjunct\ili{} \ili{}(adverbial\ili{})\ili{}.\ili{}
The\ili{} second\ili{} reading\ili{},\ili{} where\ili{} \ili{}\textit\ili{}{tenke\ili{} på}\ili{} \ili{}`think\ili{} about\ili{}’\ili{} is\ili{} a\ili{} phrasal\ili{} verb\ili{},\ili{} will\ili{} be\ili{} treated\ili{} in\ili{} Section\ili{} \ili{}\ref\ili{}{sec\ili{}:mweiness\ili{}:flexexp}\ili{}.\ili{}
\ili{}
The\ili{} \ili{}\isi\ili{}{phrase}\ili{} structure\ili{} rules\ili{} and\ili{} lexicon\ili{} of\ili{} an\ili{} LFG\ili{} grammar\ili{} assign\ili{} the\ili{} c\ili{}-struc\ili{}\\ili{}-ture\ili{}.\ili{}
NorGram\ili{} uses\ili{} a\ili{} version\ili{} of\ili{} X\ili{}’\ili{}-syntax\ili{} that\ili{} is\ili{} inspired\ili{} by\ili{} \ili{}\citet\ili{}{BresnanLFS}\ili{},\ili{} with\ili{} some\ili{} adjustments\ili{} which\ili{} depart\ili{} from\ili{} strictly\ili{} binary\ili{} branching\ili{} structures\ili{}.\ili{}
\ili{}
The\ili{} f\ili{}-structure\ili{} is\ili{} projected\ili{} from\ili{} the\ili{} c\ili{}-structure\ili{} by\ili{} the\ili{} functional\ili{} description\ili{} \ili{}(f\ili{}-description\ili{})\ili{},\ili{} which\ili{} describes\ili{} correspondences\ili{} between\ili{} the\ili{} two\ili{} levels\ili{}.\ili{}
\ili{}%Functional\ili{} annotations\ili{} in\ili{} the\ili{} \ili{}\isi\ili{}{phrase}\ili{} structure\ili{} rules\ili{} and\ili{} the\ili{} lexical\ili{} entries\ili{} describe\ili{} parts\ili{} of\ili{} the\ili{} f\ili{}-structure\ili{}.\ili{}
One\ili{} such\ili{} correspondence\ili{} is\ili{} illustrated\ili{} in\ili{} Figure\ili{} \ili{}\ref\ili{}{fig\ili{}:mweiness\ili{}:thinking\ili{}-while\ili{}-on\ili{}-bus}\ili{} by\ili{} the\ili{} highlighting\ili{} of\ili{} the\ili{} PP\ili{} node\ili{} and\ili{} the\ili{} corresponding\ili{} partial\ili{} f\ili{}-structure\ili{}.\ili{}
The\ili{} \ili{}\isi\ili{}{phrase}\ili{} structure\ili{} rules\ili{} that\ili{} assign\ili{} this\ili{} f\ili{}-structure\ili{} are\ili{} given\ili{} in\ili{} \ili{}(\ili{}\ref\ili{}{ex\ili{}:mweiness\ili{}:PP\ili{}-rule}\ili{})\ili{} and\ili{} \ili{}(\ili{}\ref\ili{}{ex\ili{}:mweiness\ili{}:NP\ili{}-rule}\ili{})\ili{}.\ili{}
The\ili{} \ili{}\isi\ili{}{rule}\ili{} daughters\ili{} are\ili{} listed\ili{} vertically\ili{} after\ili{} the\ili{} horizontal\ili{} arrow\ili{},\ili{} with\ili{} each\ili{} node\ili{}'s\ili{} functional\ili{} annotations\ili{} following\ili{} after\ili{} a\ili{} colon\ili{}.\ili{}\footnote\ili{}{The\ili{} examples\ili{} of\ili{} rules\ili{},\ili{} lexical\ili{} entries\ili{},\ili{} and\ili{} templates\ili{} in\ili{} the\ili{} following\ili{} are\ili{} simplified\ili{} for\ili{} the\ili{} purpose\ili{} of\ili{} exposition\ili{}.\ili{} Neither\ili{} the\ili{} format\ili{} nor\ili{} the\ili{} content\ili{} is\ili{} exactly\ili{} the\ili{} same\ili{} as\ili{} in\ili{} NorGram\ili{}.}\ili{}
\ili{}
\ili{}\ea\ili{}\label\ili{}{ex\ili{}:mweiness\ili{}:PP\ili{}-rule}\ili{}
\ili{}{\ili{}\small\ili{} \ili{}
PP\ili{}	\ili{}$\ili{}\rightarrow\ili{}$\ili{}	P\ili{}:\ili{} \ili{}$\ili{}\uparrow\ili{}$\ili{}=\ili{}$\ili{}\downarrow\ili{}$\ili{} \ili{}\\ili{}\\ili{}
\ili{}\hspace\ili{}{2\ili{}.3em}\ili{} NP\ili{}:\ili{} \ili{}(\ili{}$\ili{}\uparrow\ili{}$\ili{}~OBJ\ili{})\ili{}=\ili{}$\ili{}\downarrow\ili{}$\ili{}
}\ili{}
\ili{}\z\ili{}
\ili{}
\ili{}\ea\ili{}\label\ili{}{ex\ili{}:mweiness\ili{}:NP\ili{}-rule}\ili{}
\ili{}{\ili{}\small\ili{} \ili{}
NP\ili{}	\ili{}$\ili{}\rightarrow\ili{}$\ili{}	N\ili{}:\ili{} \ili{}$\ili{}\uparrow\ili{}$\ili{}=\ili{}$\ili{}\downarrow\ili{}$\ili{} \ili{}\\ili{}\\ili{}
}\ili{}
\ili{}\z\ili{}
\ili{}
The\ili{} annotations\ili{} on\ili{} the\ili{} \ili{}\isi\ili{}{rule}\ili{} daughters\ili{} describe\ili{} the\ili{} associated\ili{} f\ili{}-structures\ili{}.\ili{}
In\ili{} the\ili{} equations\ili{},\ili{} \ili{}$\ili{}\uparrow\ili{}$\ili{} refers\ili{} to\ili{} the\ili{} f\ili{}-structure\ili{} of\ili{} the\ili{} mother\ili{} node\ili{} \ili{}(the\ili{} category\ili{} on\ili{} the\ili{} left\ili{}-hand\ili{} side\ili{} of\ili{} the\ili{} \ili{}\isi\ili{}{rule}\ili{})\ili{},\ili{} while\ili{} \ili{}$\ili{}\downarrow\ili{}$\ili{} refers\ili{} to\ili{} the\ili{} f\ili{}-structure\ili{} of\ili{} the\ili{} daughter\ili{} node\ili{} \ili{}(the\ili{} category\ili{} carrying\ili{} the\ili{} annotation\ili{} on\ili{} the\ili{} right\ili{}-hand\ili{} side\ili{} of\ili{} the\ili{} \ili{}\isi\ili{}{rule}\ili{})\ili{}.\ili{}
Thus\ili{} the\ili{} equation\ili{} \ili{}$\ili{}\uparrow\ili{}$\ili{}=\ili{}$\ili{}\downarrow\ili{}$\ili{} annotated\ili{} to\ili{} a\ili{} \ili{}\isi\ili{}{rule}\ili{} daughter\ili{} means\ili{} that\ili{} the\ili{} daughter\ili{} node\ili{} and\ili{} its\ili{} mother\ili{} node\ili{} will\ili{} project\ili{} the\ili{} same\ili{} f\ili{}-structure\ili{}.\ili{}
The\ili{} equation\ili{} \ili{}(\ili{}$\ili{}\uparrow\ili{}$\ili{}~OBJ\ili{})\ili{}=\ili{}$\ili{}\downarrow\ili{}$\ili{} on\ili{} the\ili{} NP\ili{} node\ili{} in\ili{} \ili{}(\ili{}\ref\ili{}{ex\ili{}:mweiness\ili{}:PP\ili{}-rule}\ili{})\ili{} specifies\ili{} that\ili{} the\ili{} f\ili{}-structure\ili{} of\ili{} the\ili{} mother\ili{} node\ili{} \ili{}(PP\ili{})\ili{} has\ili{} an\ili{} object\ili{} \ili{}(OBJ\ili{})\ili{} which\ili{} is\ili{} the\ili{} f\ili{}-structure\ili{} of\ili{} the\ili{} daughter\ili{} node\ili{} \ili{}(NP\ili{})\ili{}.\ili{}
In\ili{} this\ili{} way\ili{} the\ili{} highlighted\ili{} f\ili{}-structure\ili{} with\ili{} the\ili{} index\ili{} \ili{}`\ili{}`2\ili{}'\ili{}'\ili{} at\ili{} its\ili{} lower\ili{} left\ili{} corner\ili{} in\ili{} Figure\ili{} \ili{}\ref\ili{}{fig\ili{}:mweiness\ili{}:thinking\ili{}-while\ili{}-on\ili{}-bus}\ili{} is\ili{} projected\ili{} from\ili{} the\ili{} PP\ili{} node\ili{}.\ili{}
Both\ili{} the\ili{} PP\ili{} node\ili{} and\ili{} the\ili{} P\ili{} node\ili{} are\ili{} highlighted\ili{} in\ili{} the\ili{} c\ili{}-structure\ili{} since\ili{} they\ili{} both\ili{} project\ili{} this\ili{} same\ili{} f\ili{}-structure\ili{}.\ili{}
\ili{}
The\ili{} annotations\ili{} on\ili{} the\ili{} \ili{}\isi\ili{}{phrase}\ili{} structure\ili{} rules\ili{} account\ili{} for\ili{} only\ili{} part\ili{} of\ili{} the\ili{} information\ili{} in\ili{} the\ili{} f\ili{}-structure\ili{}.\ili{}
Other\ili{} information\ili{} comes\ili{} from\ili{} the\ili{} word\ili{} forms\ili{} in\ili{} the\ili{} terminal\ili{} nodes\ili{} of\ili{} the\ili{} tree\ili{}.\ili{}
For\ili{} instance\ili{},\ili{} the\ili{} lexical\ili{} and\ili{} morphological\ili{} information\ili{} for\ili{} the\ili{} word\ili{} \ili{}\textit\ili{}{bussen}\ili{} contributes\ili{} all\ili{} the\ili{} equations\ili{} in\ili{} \ili{}(\ili{}\ref\ili{}{ex\ili{}:mweiness\ili{}:f\ili{}-descr\ili{}-bussen}\ili{})\ili{}.\ili{}
These\ili{} equations\ili{} are\ili{} part\ili{} of\ili{} the\ili{} f\ili{}-description\ili{} for\ili{} the\ili{} f\ili{}-structure\ili{} that\ili{} is\ili{} the\ili{} value\ili{} of\ili{} the\ili{} OBJ\ili{} attribute\ili{} \ili{}(with\ili{} the\ili{} index\ili{} \ili{}`\ili{}`5\ili{}'\ili{}'\ili{})\ili{} in\ili{} Figure\ili{} \ili{}\ref\ili{}{fig\ili{}:mweiness\ili{}:thinking\ili{}-while\ili{}-on\ili{}-bus}\ili{}.\ili{}
\ili{}
\ili{}%The\ili{} annotations\ili{} on\ili{} the\ili{} \ili{}\isi\ili{}{phrase}\ili{} structure\ili{} rules\ili{} account\ili{} for\ili{} only\ili{} part\ili{} of\ili{} the\ili{} information\ili{} in\ili{} the\ili{} f\ili{}-structure\ili{}.\ili{}
\ili{}%Other\ili{} information\ili{} comes\ili{} from\ili{} the\ili{} word\ili{} forms\ili{} in\ili{} the\ili{} terminal\ili{} nodes\ili{} of\ili{} the\ili{} tree\ili{}.\ili{}
\ili{}%The\ili{} equations\ili{} in\ili{} \ili{}(\ili{}\ref\ili{}{ex\ili{}:mweiness\ili{}:f\ili{}-descr\ili{}-bussen}\ili{})\ili{} constitute\ili{} the\ili{} f\ili{}-description\ili{} of\ili{} the\ili{} f\ili{}-structure\ili{} that\ili{} is\ili{} the\ili{} value\ili{} of\ili{} the\ili{} OBJ\ili{} attribute\ili{} \ili{}(with\ili{} the\ili{} index\ili{} \ili{}`5\ili{}'\ili{})\ili{} in\ili{} Figure\ili{} \ili{}\ref\ili{}{fig\ili{}:mweiness\ili{}:thinking\ili{}-while\ili{}-on\ili{}-bus}\ili{}.\ili{}
\ili{}%All\ili{} these\ili{} equations\ili{} except\ili{} for\ili{} \ili{}(\ili{}$\ili{}\uparrow\ili{}$\ili{}~CASE\ili{})\ili{}=obl\ili{} are\ili{} contributed\ili{} by\ili{} the\ili{} word\ili{} \ili{}\textit\ili{}{bussen}\ili{}.\ili{}
\ili{}
\ili{}%A\ili{} simple\ili{} \ili{}\isi\ili{}{lexical\ili{} entry}\ili{} for\ili{} \ili{}\textit\ili{}{bussen}\ili{} could\ili{} be\ili{} as\ili{} in\ili{} \ili{}(\ili{}\ref\ili{}{ex\ili{}:mweiness\ili{}:lexentry\ili{}-bussen}\ili{})\ili{}.\ili{}
\ili{}%Lexical\ili{} entries\ili{} for\ili{} \ili{}\textit\ili{}{på}\ili{} and\ili{} \ili{}\textit\ili{}{buss}\ili{} are\ili{} provided\ili{} in\ili{} \ili{}(\ili{}\ref\ili{}{ex\ili{}:mweiness\ili{}:lexentry\ili{}-paa}\ili{})\ili{} and\ili{} \ili{}(\ili{}\ref\ili{}{ex\ili{}:mweiness\ili{}:lexentry\ili{}-buss}\ili{})\ili{}.\ili{}
\ili{}%The\ili{} \ili{}\isi\ili{}{lexical\ili{} entry}\ili{} for\ili{} \ili{}\textit\ili{}{buss}\ili{} is\ili{} provided\ili{} in\ili{} \ili{}(\ili{}\ref\ili{}{ex\ili{}:mweiness\ili{}:lexentry\ili{}-buss}\ili{})\ili{}.\ili{}
\ili{}
\ili{}%\ili{}\ea\ili{}\label\ili{}{ex\ili{}:mweiness\ili{}:lexentry\ili{}-buss}\ili{}
\ili{}%buss\ili{} \ili{}\hspace\ili{}{0\ili{}.5em}\ili{} N\ili{} \ili{}\hspace\ili{}{0\ili{}.5em}\ili{} XLE\ili{} \ili{}\hspace\ili{}{0\ili{}.5em}\ili{} \ili{}@\ili{}(COUNTNOUN\ili{} buss\ili{})\ili{}.\ili{}
\ili{}%\ili{}\z\ili{}
\ili{}
\ili{}\ea\ili{}\label\ili{}{ex\ili{}:mweiness\ili{}:f\ili{}-descr\ili{}-bussen}\ili{}
\ili{}{\ili{}\small\ili{} \ili{}
\ili{}(\ili{}$\ili{}\uparrow\ili{}$\ili{}~PRED\ili{})\ili{}=\ili{}`buss\ili{}'\ili{} \ili{}\\ili{}\\ili{}
\ili{}(\ili{}$\ili{}\uparrow\ili{}$\ili{}~NTYPE\ili{} NSEM\ili{} COMMON\ili{})\ili{}=count\ili{} \ili{}\\ili{}\\ili{}
\ili{}(\ili{}$\ili{}\uparrow\ili{}$\ili{}~NTYPE\ili{} NSYN\ili{})\ili{}=common\ili{} \ili{}\\ili{}\\ili{}
\ili{}(\ili{}$\ili{}\uparrow\ili{}$\ili{}~GEND\ili{} NEUT\ili{})\ili{}=\ili{}$\ili{}-\ili{}$\ili{} \ili{}\\ili{}\\ili{}
\ili{}(\ili{}$\ili{}\uparrow\ili{}$\ili{}~GEND\ili{} MASC\ili{})\ili{}=\ili{}$\ili{}+\ili{}$\ili{} \ili{}\\ili{}\\ili{}
\ili{}(\ili{}$\ili{}\uparrow\ili{}$\ili{}~GEND\ili{} FEM\ili{})\ili{}=\ili{}$\ili{}-\ili{}$\ili{} \ili{}\\ili{}\\ili{}
\ili{}(\ili{}$\ili{}\uparrow\ili{}$\ili{}~PERS\ili{})\ili{}=3\ili{} \ili{}\\ili{}\\ili{}
\ili{}(\ili{}$\ili{}\uparrow\ili{}$\ili{}~NUM\ili{})\ili{}=sg\ili{} \ili{}\\ili{}\\ili{}
\ili{}(\ili{}$\ili{}\uparrow\ili{}$\ili{}~DEF\ili{}-MORPH\ili{})\ili{}=\ili{}$\ili{}+\ili{}$\ili{} \ili{}\\ili{}\\ili{}
\ili{}(\ili{}$\ili{}\uparrow\ili{}$\ili{}~DEF\ili{})\ili{}=\ili{}$\ili{}+\ili{}$\ili{} \ili{}\\ili{}\\ili{}
\ili{}%\ili{}(\ili{}$\ili{}\uparrow\ili{}$\ili{}~CASE\ili{})\ili{}=obl\ili{} \ili{}\\ili{}\\ili{}
}\ili{}
\ili{}\z\ili{}
\ili{}
\ili{}%All\ili{} except\ili{} for\ili{} the\ili{} first\ili{} one\ili{} are\ili{} common\ili{} to\ili{} many\ili{} other\ili{} nouns\ili{}.\ili{}
The\ili{} first\ili{} equation\ili{},\ili{} which\ili{} assigns\ili{} the\ili{} PRED\ili{}(icate\ili{})\ili{} value\ili{} \ili{}`buss\ili{}'\ili{},\ili{} is\ili{} specific\ili{} to\ili{} this\ili{} noun\ili{},\ili{} but\ili{} the\ili{} others\ili{} are\ili{} common\ili{} to\ili{} many\ili{} other\ili{} words\ili{}.\ili{}
Some\ili{} of\ili{} the\ili{} equations\ili{} come\ili{} from\ili{} features\ili{} assigned\ili{} to\ili{} the\ili{} word\ili{} form\ili{} \ili{}\textit\ili{}{bussen}\ili{} by\ili{} the\ili{} morphological\ili{} analyzer\ili{} run\ili{} prior\ili{} to\ili{} parsing\ili{};\ili{} these\ili{} features\ili{} are\ili{} \ili{}+Noun\ili{},\ili{} \ili{}+Sg\ili{},\ili{} \ili{}+Def\ili{} and\ili{} \ili{}+Masc\ili{},\ili{} and\ili{} they\ili{} will\ili{} appear\ili{} in\ili{} the\ili{} string\ili{} presented\ili{} to\ili{} the\ili{} syntactic\ili{} parser\ili{}.\ili{}
Other\ili{} equations\ili{} come\ili{} from\ili{} the\ili{} \ili{}\isi\ili{}{lexical\ili{} entry}\ili{} for\ili{} the\ili{} noun\ili{} \ili{} \ili{}\textit\ili{}{buss}\ili{}.\ili{}
Both\ili{} the\ili{} features\ili{} and\ili{} the\ili{} noun\ili{} must\ili{} have\ili{} entries\ili{} in\ili{} the\ili{} lexicon\ili{};\ili{} \ili{} these\ili{} are\ili{} shown\ili{} in\ili{} \ili{}(\ili{}\ref\ili{}{ex\ili{}:mweiness\ili{}:entry\ili{}-noun}\ili{})\ili{}-\ili{}-\ili{}(\ili{}\ref\ili{}{ex\ili{}:mweiness\ili{}:entry\ili{}-buss}\ili{})\ili{}.\ili{}
Each\ili{} \ili{}\isi\ili{}{lexical\ili{} entry}\ili{} specifies\ili{} a\ili{} lexical\ili{} category\ili{};\ili{} SUFF\ili{} \ili{}(for\ili{} suffix\ili{})\ili{} is\ili{} the\ili{} category\ili{} for\ili{} morphological\ili{} features\ili{}.\ili{}
\ili{}%Both\ili{} the\ili{} features\ili{} and\ili{} the\ili{} noun\ili{} must\ili{} have\ili{} entries\ili{} in\ili{} the\ili{} NorGram\ili{} lexicon\ili{};\ili{} \ili{} \ili{}(\ili{}\ref\ili{}{ex\ili{}:mweiness\ili{}:entry\ili{}-noun}\ili{})\ili{}-\ili{}-\ili{}(\ili{}\ref\ili{}{ex\ili{}:mweiness\ili{}:entry\ili{}-masc}\ili{})\ili{} shows\ili{} possible\ili{} lexical\ili{} entries\ili{} for\ili{} the\ili{} features\ili{},\ili{} while\ili{} \ili{}(\ili{}\ref\ili{}{ex\ili{}:mweiness\ili{}:entry\ili{}-buss}\ili{})\ili{} provides\ili{} an\ili{} entry\ili{} for\ili{} the\ili{} noun\ili{}.\ili{}
\ili{}
\ili{}\ea\ili{}\label\ili{}{ex\ili{}:mweiness\ili{}:entry\ili{}-noun}\ili{}
\ili{}{\ili{}\small\ili{} \ili{}
\ili{}$\ili{}+\ili{}$Noun\ili{} \ili{}\hspace\ili{}{0\ili{}.4em}\ili{} SUFF\ili{} \ili{}\hspace\ili{}{0\ili{}.4em}\ili{} \ili{}(\ili{}$\ili{}\uparrow\ili{}$\ili{}~PERS\ili{})\ili{}=3\ili{}
}\ili{}
\ili{}\z\ili{}
\ili{}
\ili{}\ea\ili{}\label\ili{}{ex\ili{}:mweiness\ili{}:entry\ili{}-sg}\ili{}
\ili{}{\ili{}\small\ili{} \ili{}
\ili{}$\ili{}+\ili{}$Sg\ili{} \ili{}\hspace\ili{}{1\ili{}.7em}\ili{} SUFF\ili{} \ili{}\hspace\ili{}{0\ili{}.4em}\ili{} \ili{}(\ili{}$\ili{}\uparrow\ili{}$\ili{}~NUM\ili{})\ili{}=sg\ili{}
}\ili{}
\ili{}\z\ili{}
\ili{}
\ili{}\ea\ili{}\label\ili{}{ex\ili{}:mweiness\ili{}:entry\ili{}-def}\ili{}
\ili{}{\ili{}\small\ili{} \ili{}
\ili{}$\ili{}+\ili{}$Def\ili{} \ili{} \ili{}\hspace\ili{}{1\ili{}.2em}\ili{} SUFF\ili{} \ili{}\hspace\ili{}{0\ili{}.4em}\ili{} \ili{}@DEF\ili{}
}\ili{}
\ili{}\z\ili{}
\ili{}%\ili{}\ea\ili{}\label\ili{}{ex\ili{}:mweiness\ili{}:entry\ili{}-def}\ili{}
\ili{}%\ili{}$\ili{}+\ili{}$Def\ili{} \ili{}\hspace\ili{}{1\ili{}.2em}\ili{} \ili{}(\ili{}$\ili{}\uparrow\ili{}$\ili{}~DEF\ili{}-MORPH\ili{})\ili{}=\ili{}$\ili{}+\ili{}$\ili{} \ili{}\\ili{}\\ili{}
\ili{}%\ili{}\hspace\ili{}{3\ili{}.5em}\ili{} \ili{} \ili{}(\ili{}$\ili{}\uparrow\ili{}$\ili{}~DEF\ili{})\ili{}=\ili{}$\ili{}+\ili{}$\ili{} \ili{}\\ili{}\\ili{}
\ili{}%\ili{}\z\ili{}
\ili{}
\ili{}\ea\ili{}\label\ili{}{ex\ili{}:mweiness\ili{}:entry\ili{}-masc}\ili{}
\ili{}{\ili{}\small\ili{} \ili{}
\ili{}$\ili{}+\ili{}$Masc\ili{} \ili{} \ili{}\hspace\ili{}{0\ili{}.6em}\ili{} SUFF\ili{} \ili{}\hspace\ili{}{0\ili{}.4em}\ili{} \ili{}@MASC\ili{}
}\ili{}
\ili{}\z\ili{}
\ili{}
\ili{}%\ili{}\ea\ili{}\label\ili{}{ex\ili{}:mweiness\ili{}:entry\ili{}-masc}\ili{}
\ili{}%\ili{}$\ili{}+\ili{}$Masc\ili{} \ili{}\hspace\ili{}{0\ili{}.6em}\ili{} \ili{}(\ili{}$\ili{}\uparrow\ili{}$\ili{} GEND\ili{} MASC\ili{})\ili{}=\ili{}$\ili{}+\ili{}$\ili{} \ili{}\\ili{}\\ili{}
\ili{}%\ili{}\hspace\ili{}{3\ili{}.5em}\ili{} \ili{} \ili{}(\ili{}$\ili{}\uparrow\ili{}$\ili{} GEND\ili{} FEM\ili{})\ili{}=\ili{}$\ili{}-\ili{}$\ili{} \ili{}\\ili{}\\ili{}
\ili{}%\ili{}\hspace\ili{}{3\ili{}.5em}\ili{} \ili{} \ili{}(\ili{}$\ili{}\uparrow\ili{}$\ili{} GEND\ili{} NEUT\ili{})\ili{}=\ili{}$\ili{}-\ili{}$\ili{} \ili{}\\ili{}\\ili{}
\ili{}%\ili{}\z\ili{}
\ili{}
\ili{}\ea\ili{}\label\ili{}{ex\ili{}:mweiness\ili{}:entry\ili{}-buss}\ili{}
\ili{}{\ili{}\small\ili{} \ili{}
buss\ili{} \ili{} \ili{}\hspace\ili{}{1\ili{}.5em}\ili{} N\ili{} \ili{}\hspace\ili{}{1\ili{}.8em}\ili{} \ili{}@\ili{}(COUNTNOUN\ili{} buss\ili{})\ili{}
}\ili{}
\ili{}\z\ili{}
\ili{}
\ili{}%\ili{}\ea\ili{}\label\ili{}{ex\ili{}:mweiness\ili{}:entry\ili{}-buss}\ili{}
\ili{}%buss\ili{} \ili{}\hspace\ili{}{1\ili{}.4em}\ili{} \ili{}(\ili{}$\ili{}\uparrow\ili{}$\ili{} PRED\ili{})\ili{}=\ili{}`buss\ili{}'\ili{} \ili{}\\ili{}\\ili{}
\ili{}%\ili{}\hspace\ili{}{3\ili{}.5em}\ili{} \ili{} \ili{}(\ili{}$\ili{}\uparrow\ili{}$\ili{} NTYPE\ili{} NSEM\ili{} COMMON\ili{})\ili{}=count\ili{} \ili{}\\ili{}\\ili{}
\ili{}%\ili{}\hspace\ili{}{3\ili{}.5em}\ili{} \ili{} \ili{}(\ili{}$\ili{}\uparrow\ili{}$\ili{} NTYPE\ili{} NSYN\ili{})\ili{}=common\ili{} \ili{}\\ili{}\\ili{}
\ili{}%\ili{}\z\ili{}
\ili{}
The\ili{} equations\ili{} in\ili{} the\ili{} first\ili{} two\ili{} entries\ili{} each\ili{} contribute\ili{} one\ili{} attribute\ili{}-value\ili{} pair\ili{} to\ili{} the\ili{} f\ili{}-structure\ili{}.\ili{}
Entries\ili{} \ili{}(\ili{}\ref\ili{}{ex\ili{}:mweiness\ili{}:entry\ili{}-def}\ili{})\ili{}-\ili{}-\ili{}(\ili{}\ref\ili{}{ex\ili{}:mweiness\ili{}:entry\ili{}-buss}\ili{})\ili{} contain\ili{} \ili{}\isi\ili{}{template}\ili{} invocations\ili{} rather\ili{} than\ili{} equations\ili{}.\ili{}
The\ili{} at\ili{}-sign\ili{} indicates\ili{} a\ili{} call\ili{} to\ili{} a\ili{} \ili{}\isi\ili{}{template}\ili{},\ili{} while\ili{} DEF\ili{},\ili{} MASC\ili{} and\ili{} COUNTNOUN\ili{} are\ili{} names\ili{} of\ili{} templates\ili{}.\ili{}
A\ili{} \ili{}\isi\ili{}{template}\ili{} is\ili{} an\ili{} f\ili{}-description\ili{},\ili{} a\ili{} collection\ili{} of\ili{} equations\ili{} which\ili{} it\ili{} is\ili{} convenient\ili{} to\ili{} refer\ili{} to\ili{} by\ili{} a\ili{} name\ili{} rather\ili{} than\ili{} listing\ili{} all\ili{} the\ili{} equations\ili{}.\ili{}
Templates\ili{} can\ili{} be\ili{} used\ili{} in\ili{} different\ili{} places\ili{} in\ili{} the\ili{} grammar\ili{} and\ili{} lexicon\ili{},\ili{} and\ili{} \ili{}\isi\ili{}{template}\ili{} definitions\ili{} may\ili{} refer\ili{} to\ili{} other\ili{} templates\ili{}.\ili{}
\ili{}
The\ili{} definition\ili{} of\ili{} the\ili{} \ili{}\isi\ili{}{template}\ili{} named\ili{} DEF\ili{} is\ili{} shown\ili{} in\ili{} \ili{}(\ili{}\ref\ili{}{ex\ili{}:mweiness\ili{}:def\ili{}-template}\ili{})\ili{}.\ili{}
All\ili{} nouns\ili{} inflected\ili{} in\ili{} the\ili{} definite\ili{} form\ili{} will\ili{} carry\ili{} these\ili{} two\ili{} equations\ili{},\ili{} so\ili{} it\ili{} can\ili{} be\ili{} convenient\ili{} to\ili{} refer\ili{} to\ili{} them\ili{} together\ili{}.\ili{}
\ili{}%They\ili{} can\ili{} be\ili{} factored\ili{} out\ili{} into\ili{} a\ili{} \ili{}\isi\ili{}{template}\ili{},\ili{} a\ili{} named\ili{} f\ili{}-description\ili{},\ili{} as\ili{} in\ili{} \ili{}(\ili{}\ref\ili{}{ex\ili{}:mweiness\ili{}:def\ili{}-template}\ili{})\ili{}.\ili{}
\ili{}
\ili{}\ea\ili{}\label\ili{}{ex\ili{}:mweiness\ili{}:def\ili{}-template}\ili{}
\ili{}{\ili{}\small\ili{} \ili{}
DEF\ili{} \ili{}=\ili{} \ili{}\\ili{}\\ili{}
\ili{}\hspace\ili{}{2em}\ili{} \ili{} \ili{}(\ili{}$\ili{}\uparrow\ili{}$\ili{}~DEF\ili{}-MORPH\ili{})\ili{}=\ili{}$\ili{}+\ili{}$\ili{} \ili{}\\ili{}\\ili{}
\ili{}\hspace\ili{}{2em}\ili{} \ili{} \ili{}(\ili{}$\ili{}\uparrow\ili{}$\ili{}~DEF\ili{})\ili{}=\ili{}$\ili{}+\ili{}$\ili{} \ili{}\\ili{}\\ili{}
}\ili{}
\ili{}\z\ili{}
\ili{}
\ili{}%The\ili{} \ili{}\isi\ili{}{lexical\ili{} entry}\ili{} for\ili{} the\ili{} morphological\ili{} feature\ili{} \ili{}\textit\ili{}{\ili{}$\ili{}+\ili{}$Def}\ili{} can\ili{} therefore\ili{} be\ili{} specified\ili{} as\ili{} in\ili{} \ili{}(\ili{}\ref\ili{}{ex\ili{}:mweiness\ili{}:def\ili{}-entry}\ili{})\ili{}.\ili{}
\ili{}
\ili{}%\ili{}\ea\ili{}\label\ili{}{ex\ili{}:mweiness\ili{}:def\ili{}-entry}\ili{}
\ili{}%\ili{}$\ili{}+\ili{}$Def\ili{} \ili{} \ili{}\hspace\ili{}{1\ili{}.2em}\ili{} \ili{}@DEF\ili{}
\ili{}%\ili{}\z\ili{}
\ili{}
\ili{}\ili\ili{}{Norwegian}\ili{} has\ili{} a\ili{} complicated\ili{} system\ili{} of\ili{} gender\ili{} agreement\ili{} because\ili{} of\ili{} some\ili{} nouns\ili{} that\ili{} may\ili{} have\ili{} either\ili{} masculine\ili{} or\ili{} feminine\ili{} agreement\ili{},\ili{} and\ili{} because\ili{} adjectives\ili{} and\ili{} determiners\ili{} may\ili{} be\ili{} unspecified\ili{} for\ili{} certain\ili{} gender\ili{} distinctions\ili{}.\ili{}
To\ili{} account\ili{} for\ili{} this\ili{},\ili{} each\ili{} noun\ili{} must\ili{} receive\ili{} a\ili{} plus\ili{} or\ili{} minus\ili{} value\ili{} for\ili{} each\ili{} of\ili{} the\ili{} three\ili{} genders\ili{}.\ili{}
The\ili{} equations\ili{} needed\ili{} for\ili{} specifying\ili{} masculine\ili{} gender\ili{} are\ili{} included\ili{} in\ili{} the\ili{} \ili{}\isi\ili{}{template}\ili{} in\ili{} \ili{}(\ili{}\ref\ili{}{ex\ili{}:mweiness\ili{}:masc\ili{}-template}\ili{})\ili{}.\ili{}
These\ili{} equations\ili{} do\ili{} not\ili{} simply\ili{} describe\ili{} attribute\ili{}-value\ili{} pairs\ili{};\ili{} they\ili{} describe\ili{} paths\ili{} through\ili{} the\ili{} f\ili{}-structure\ili{}.\ili{}
The\ili{} equation\ili{} \ili{}(\ili{}$\ili{}\uparrow\ili{}$\ili{} GEND\ili{} MASC\ili{})\ili{}=\ili{}$\ili{}+\ili{}$\ili{} states\ili{} that\ili{} the\ili{} f\ili{}-structure\ili{} has\ili{} an\ili{} attribute\ili{} GEND\ili{} which\ili{} has\ili{} as\ili{} its\ili{} value\ili{} a\ili{} subsidiary\ili{} f\ili{}-structure\ili{} which\ili{} in\ili{} its\ili{} turn\ili{} has\ili{} an\ili{} attribute\ili{} MASC\ili{} with\ili{} the\ili{} value\ili{} \ili{}$\ili{}+\ili{}$\ili{}.\ili{}
\ili{}
\ili{}\ea\ili{}\label\ili{}{ex\ili{}:mweiness\ili{}:masc\ili{}-template}\ili{}
\ili{}{\ili{}\small\ili{} \ili{}
MASC\ili{} \ili{}=\ili{} \ili{}\\ili{}\\ili{}
\ili{}\hspace\ili{}{2em}\ili{} \ili{} \ili{}(\ili{}$\ili{}\uparrow\ili{}$\ili{} GEND\ili{} MASC\ili{})\ili{}=\ili{}$\ili{}+\ili{}$\ili{} \ili{}\\ili{}\\ili{}
\ili{}\hspace\ili{}{2em}\ili{} \ili{} \ili{}(\ili{}$\ili{}\uparrow\ili{}$\ili{} GEND\ili{} FEM\ili{})\ili{}=\ili{}$\ili{}-\ili{}$\ili{} \ili{}\\ili{}\\ili{}
\ili{}\hspace\ili{}{2em}\ili{} \ili{} \ili{}(\ili{}$\ili{}\uparrow\ili{}$\ili{} GEND\ili{} NEUT\ili{})\ili{}=\ili{}$\ili{}-\ili{}$\ili{} \ili{}\\ili{}\\ili{}
}\ili{}
\ili{}\z\ili{}
\ili{}
\ili{}%\ili{}\ea\ili{}\label\ili{}{ex\ili{}:mweiness\ili{}:masc\ili{}-template}\ili{}
\ili{}%\ili{}\verb\ili{}§MASC\ili{} \ili{}=\ili{}§\ili{}\\ili{}\\ili{}
\ili{}%\ili{}\verb\ili{}§\ili{} \ili{} \ili{} \ili{} \ili{}(\ili{}§\ili{}$\ili{}\uparrow\ili{}$\ili{}\verb\ili{}§\ili{} GEND\ili{} MASC\ili{})\ili{}=\ili{}§\ili{}$\ili{}+\ili{}$\ili{}\\ili{}\\ili{}
\ili{}%\ili{}\verb\ili{}§\ili{} \ili{} \ili{} \ili{} \ili{}(\ili{}§\ili{}$\ili{}\uparrow\ili{}$\ili{}\verb\ili{}§\ili{} GEND\ili{} FEM\ili{})\ili{}=\ili{}§\ili{}–\ili{}\\ili{}\\ili{}
\ili{}%\ili{}\verb\ili{}§\ili{} \ili{} \ili{} \ili{} \ili{}(\ili{}§\ili{}$\ili{}\uparrow\ili{}$\ili{}\verb\ili{}§\ili{} GEND\ili{} NEUT\ili{})\ili{}=\ili{}§\ili{}$\ili{}-\ili{}$\ili{}
\ili{}%\ili{}\z\ili{}
\ili{}%\ili{}
\ili{}%\ili{}\setlength\ili{}{\ili{}\tabcolsep}\ili{}{0\ili{}.1em}\ili{} \ili{}%\ili{} for\ili{} the\ili{} horizontal\ili{} padding\ili{}
\ili{}%\ili{}\ea\ili{}\label\ili{}{ex\ili{}:rules}\ili{}
\ili{}%\ili{}\begin\ili{}{tabular}\ili{}{ll}\ili{}
\ili{}%I\ili{}'\ili{} \ili{}$\ili{}\rightarrow\ili{}$\ili{} \ili{}&\ili{} \ili{} Vfin\ili{}:\ili{} \ili{}$\ili{}\uparrow\ili{}$\ili{}=\ili{}$\ili{}\downarrow\ili{}$\ili{}\\ili{}\\ili{}
\ili{}%\ili{} \ili{}&\ili{} \ili{}(S\ili{}:\ili{} \ili{}$\ili{}\uparrow\ili{}$\ili{}=\ili{}$\ili{}\downarrow\ili{}$\ili{})\ili{}.\ili{}\\ili{}\\ili{}\\ili{}\\ili{}
\ili{}%\ili{}\end\ili{}{tabular}\ili{}
\ili{}%\ili{}\z\ili{}
\ili{}%\ili{}
\ili{}%\ili{}\setlength\ili{}{\ili{}\tabcolsep}\ili{}{0\ili{}.1em}\ili{} \ili{}%\ili{} for\ili{} the\ili{} horizontal\ili{} padding\ili{}
\ili{}%\ili{}\eabox\ili{}{\ili{}\label\ili{}{ex\ili{}:rules}\ili{}
\ili{}%\ili{}\begin\ili{}{tabular}\ili{}{ll}\ili{}
\ili{}%I\ili{}'\ili{} \ili{}$\ili{}\rightarrow\ili{}$\ili{} \ili{}&\ili{} \ili{} Vfin\ili{}:\ili{} \ili{}$\ili{}\uparrow\ili{}$\ili{}=\ili{}$\ili{}\downarrow\ili{}$\ili{}\\ili{}\\ili{}
\ili{}%\ili{} \ili{}&\ili{} \ili{}(S\ili{}:\ili{} \ili{}$\ili{}\uparrow\ili{}$\ili{}=\ili{}$\ili{}\downarrow\ili{}$\ili{})\ili{}.\ili{}\\ili{}\\ili{}\\ili{}\\ili{}
\ili{}%\ili{}\end\ili{}{tabular}}\ili{}
\ili{}%\ili{}
\ili{}%\ili{}\ea\ili{}\label\ili{}{ex\ili{}:rules}\ili{}
\ili{}%\ili{}\begin\ili{}{tabular}\ili{}{lll}\ili{}
\ili{}%I\ili{}'\ili{} \ili{}$\ili{}\rightarrow\ili{}$\ili{} \ili{}&\ili{} \ili{} Vfin\ili{}:\ili{} \ili{}&\ili{} \ili{}$\ili{}\uparrow\ili{}$\ili{}=\ili{}$\ili{}\downarrow\ili{}$\ili{}\\ili{}\\ili{}
\ili{}%\ili{} \ili{}&\ili{} \ili{}(S\ili{}:\ili{} \ili{}&\ili{} \ili{}$\ili{}\uparrow\ili{}$\ili{}=\ili{}$\ili{}\downarrow\ili{}$\ili{})\ili{}.\ili{}\\ili{}\\ili{}\\ili{}\\ili{}
\ili{}%\ili{}\end\ili{}{tabular}\ili{}
\ili{}%\ili{}
\ili{}%\ili{}\begin\ili{}{tabular}\ili{}{lll}\ili{}
\ili{}%S\ili{} \ili{}$\ili{}\rightarrow\ili{}$\ili{} \ili{}&\ili{} \ili{} \ili{}(PRONP\ili{}:\ili{} \ili{}&\ili{} \ili{}(\ili{}$\ili{}\uparrow\ili{}$\ili{} SUBJ\ili{})\ili{}=\ili{}$\ili{}\downarrow\ili{}$\ili{}\\ili{}\\ili{}
\ili{}%\ili{}&\ili{} \ili{}&\ili{}@SUBJCASE\ili{}\\ili{}\\ili{}
\ili{}%\ili{}&\ili{} \ili{}[\ili{}.\ili{}.\ili{}.\ili{}]\ili{}\\ili{}\\ili{}
\ili{}%\ili{}&\ili{} \ili{}(PRONrfl\ili{}:\ili{} \ili{}&\ili{} \ili{}\\ili{}{\ili{} \ili{}\enspace\ili{} \ili{}(\ili{}$\ili{}\uparrow\ili{}$\ili{} OBJ\ili{}-BEN\ili{})\ili{}=\ili{}$\ili{}\downarrow\ili{}$\ili{}\\ili{}\\ili{}
\ili{}%\ili{}&\ili{} \ili{}&\ili{} \ili{}|\ili{} \ili{}\enspace\ili{} \ili{}(\ili{}$\ili{}\uparrow\ili{}$\ili{} OBJ\ili{})\ili{}=\ili{}$\ili{}\downarrow\ili{}$\ili{} \ili{}\enspace\ili{} \ili{}\}\ili{} \ili{}\\ili{}\\ili{}
\ili{}%\ili{}&\ili{} \ili{}[\ili{}.\ili{}.\ili{}.\ili{}]\ili{}\\ili{}\\ili{}
\ili{}%\ili{} \ili{}&\ili{} \ili{}(ADVPs\ili{}+\ili{}:\ili{} \ili{}&\ili{} \ili{}$\ili{}\downarrow\ili{}$\ili{} \ili{}∈\ili{} \ili{}(\ili{}$\ili{}\uparrow\ili{}$\ili{} ADJUNCT\ili{})\ili{})\ili{}\\ili{}\\ili{}
\ili{}%\ili{}&\ili{} \ili{}[\ili{}.\ili{}.\ili{}.\ili{}]\ili{}\\ili{}\\ili{}
\ili{}%\ili{}&\ili{} \ili{}(APsmpl\ili{}:\ili{} \ili{} \ili{}&\ili{} \ili{}$\ili{}\downarrow\ili{}$\ili{} \ili{}∈\ili{} \ili{}(\ili{}$\ili{}\uparrow\ili{}$\ili{} ADJUNCT\ili{})\ili{})\ili{}\\ili{}\\ili{}
\ili{}%\ili{}&\ili{} \ili{}[\ili{}.\ili{}.\ili{}.\ili{}]\ili{}\\ili{}\\ili{}
\ili{}%\ili{} \ili{}&\ili{} \ili{}(VPmain\ili{}:\ili{} \ili{}&\ili{} \ili{}$\ili{}\uparrow\ili{}$\ili{}=\ili{}$\ili{}\downarrow\ili{}$\ili{})\ili{}\\ili{}\\ili{}
\ili{}%\ili{}&\ili{} \ili{}[\ili{}.\ili{}.\ili{}.\ili{}]\ili{}.\ili{}\\ili{}\\ili{}\\ili{}\\ili{}
\ili{}%\ili{}\end\ili{}{tabular}\ili{}
\ili{}%\ili{}
\ili{}%\ili{}\begin\ili{}{tabular}\ili{}{lll}\ili{}
\ili{}%VPmain\ili{} \ili{}$\ili{}\rightarrow\ili{}$\ili{} \ili{}&\ili{} \ili{}[\ili{}.\ili{}.\ili{}.\ili{}]\ili{}\\ili{}\\ili{}
\ili{}%\ili{}&\ili{} \ili{}(PRTP\ili{}:\ili{} \ili{}&\ili{} \ili{}$\ili{}\uparrow\ili{}$\ili{}=\ili{}$\ili{}\downarrow\ili{}$\ili{}\\ili{}\\ili{}
\ili{}%\ili{}&\ili{} \ili{}&\ili{} \ili{}(\ili{}$\ili{}\uparrow\ili{}$\ili{} CHECK\ili{} PRT\ili{}-VERB\ili{})\ili{}=c\ili{} \ili{}+\ili{})\ili{}\\ili{}\\ili{}
\ili{}%\ili{}&\ili{} \ili{}[\ili{}.\ili{}.\ili{}.\ili{}]\ili{}.\ili{}\\ili{}\\ili{}\\ili{}\\ili{}
\ili{}%\ili{}\end\ili{}{tabular}\ili{}
\ili{}%\ili{}\z\ili{}
\ili{}%\ili{}
\ili{}
\ili{}
Like\ili{} the\ili{} \ili{}\isi\ili{}{template}\ili{} MASC\ili{},\ili{} the\ili{} \ili{}\isi\ili{}{template}\ili{} COUNTNOUN\ili{} also\ili{} describes\ili{} paths\ili{} through\ili{} the\ili{} f\ili{}-structure\ili{}.\ili{}
The\ili{} NTYPE\ili{} SYN\ili{} features\ili{} distinguish\ili{} between\ili{} common\ili{} nouns\ili{},\ili{} proper\ili{} nouns\ili{},\ili{} pronouns\ili{},\ili{} etc\ili{}.\ili{} while\ili{} the\ili{} NTYPE\ili{} SEM\ili{} COMMON\ili{} features\ili{} distinguish\ili{} between\ili{} count\ili{} nouns\ili{},\ili{} mass\ili{} nouns\ili{},\ili{} etc\ili{}.\ili{}
All\ili{} nouns\ili{} must\ili{} contribute\ili{} a\ili{} PRED\ili{} feature\ili{} to\ili{} the\ili{} f\ili{}-structure\ili{},\ili{} but\ili{} the\ili{} PRED\ili{} feature\ili{} itself\ili{} will\ili{} differ\ili{} from\ili{} noun\ili{} to\ili{} noun\ili{}.\ili{}
The\ili{} \ili{}\isi\ili{}{template}\ili{} in\ili{} \ili{}(\ili{}\ref\ili{}{ex\ili{}:mweiness\ili{}:countnoun\ili{}-template}\ili{})\ili{} is\ili{} parameterized\ili{};\ili{} the\ili{} parameter\ili{} P\ili{} will\ili{} be\ili{} substituted\ili{} by\ili{} the\ili{} argument\ili{} supplied\ili{} in\ili{} the\ili{} invocation\ili{} of\ili{} the\ili{} \ili{}\isi\ili{}{template}\ili{},\ili{} for\ili{} example\ili{} the\ili{} word\ili{} \ili{}\textit\ili{}{buss}\ili{} in\ili{} \ili{}(\ili{}\ref\ili{}{ex\ili{}:mweiness\ili{}:entry\ili{}-buss}\ili{})\ili{}.\ili{}
\ili{}%All\ili{} count\ili{} nouns\ili{} have\ili{} in\ili{} common\ili{} certain\ili{} NTYPE\ili{} features\ili{}.\ili{}
\ili{}%All\ili{} count\ili{} nouns\ili{} must\ili{} also\ili{} contribute\ili{} a\ili{} PRED\ili{} feature\ili{} to\ili{} the\ili{} f\ili{}-structure\ili{},\ili{} but\ili{} the\ili{} PRED\ili{} feature\ili{} itself\ili{} will\ili{} differ\ili{} from\ili{} noun\ili{} to\ili{} noun\ili{}.\ili{}
\ili{}%The\ili{} \ili{}\isi\ili{}{template}\ili{} in\ili{} \ili{}(\ili{}\ref\ili{}{ex\ili{}:mweiness\ili{}:countnoun\ili{}-template}\ili{})\ili{} is\ili{} parameterized\ili{};\ili{} the\ili{} parameter\ili{} P\ili{} will\ili{} be\ili{} substituted\ili{} by\ili{} the\ili{} argument\ili{} supplied\ili{} in\ili{} the\ili{} invocation\ili{} of\ili{} the\ili{} \ili{}\isi\ili{}{template}\ili{}.\ili{}
\ili{}%For\ili{} the\ili{} \ili{}\isi\ili{}{lexical\ili{} entry}\ili{} in\ili{} \ili{}(\ili{}\ref\ili{}{ex\ili{}:mweiness\ili{}:entry\ili{}-buss}\ili{})\ili{},\ili{} the\ili{} argument\ili{} \ili{}\textit\ili{}{buss}\ili{} will\ili{} be\ili{} substituted\ili{} for\ili{} P\ili{},\ili{} providing\ili{} the\ili{} PRED\ili{} value\ili{} \ili{}`buss\ili{}'\ili{}.\ili{}
\ili{}
\ili{}
\ili{}\ea\ili{}\label\ili{}{ex\ili{}:mweiness\ili{}:countnoun\ili{}-template}\ili{}
\ili{}{\ili{}\small\ili{} \ili{}
COUNTNOUN\ili{} \ili{}(P\ili{})\ili{} \ili{}=\ili{} \ili{}\\ili{}\\ili{}
\ili{}\hspace\ili{}{2em}\ili{} \ili{} \ili{}(\ili{}$\ili{}\uparrow\ili{}$\ili{} PRED\ili{})\ili{}=P\ili{} \ili{}\\ili{}\\ili{}
\ili{}\hspace\ili{}{2em}\ili{} \ili{} \ili{}(\ili{}$\ili{}\uparrow\ili{}$\ili{} NTYPE\ili{} NSEM\ili{} COMMON\ili{})\ili{}=count\ili{} \ili{}\\ili{}\\ili{}
\ili{}\hspace\ili{}{2em}\ili{} \ili{} \ili{}(\ili{}$\ili{}\uparrow\ili{}$\ili{} NTYPE\ili{} NSYN\ili{})\ili{}=common\ili{} \ili{}\\ili{}\\ili{}
}\ili{}
\ili{}\z\ili{}
\ili{}
The\ili{} value\ili{} of\ili{} a\ili{} PRED\ili{} attribute\ili{} is\ili{} a\ili{} semantic\ili{} form\ili{}.\ili{}
A\ili{} semantic\ili{} form\ili{} is\ili{} always\ili{} enclosed\ili{} in\ili{} single\ili{} quotation\ili{} marks\ili{},\ili{} indicating\ili{} that\ili{} the\ili{} value\ili{} is\ili{} unique\ili{},\ili{} which\ili{} means\ili{} that\ili{} it\ili{} cannot\ili{} be\ili{} unified\ili{} even\ili{} with\ili{} an\ili{} identical\ili{}-looking\ili{} value\ili{} of\ili{} some\ili{} other\ili{} attribute\ili{}.\ili{}
For\ili{} some\ili{} words\ili{} the\ili{} semantic\ili{} form\ili{} includes\ili{} not\ili{} only\ili{} the\ili{} word\ili{} itself\ili{},\ili{} but\ili{} also\ili{} a\ili{} syntactic\ili{} argument\ili{} list\ili{}.\ili{}
This\ili{} is\ili{} the\ili{} case\ili{} for\ili{} two\ili{} of\ili{} the\ili{} words\ili{} in\ili{} \ili{}\textit\ili{}{hun\ili{} tenker\ili{} på\ili{} bussen}\ili{} \ili{}(in\ili{} the\ili{} interpretation\ili{} being\ili{} considered\ili{} in\ili{} this\ili{} section\ili{})\ili{}.\ili{}
The\ili{} verb\ili{} has\ili{} the\ili{} semantic\ili{} form\ili{} \ili{}`tenke\ili{}<\ili{}[SUBJ\ili{}]\ili{}>\ili{}'\ili{},\ili{} meaning\ili{} that\ili{} the\ili{} verb\ili{} is\ili{} intransitive\ili{} and\ili{} subcategorizes\ili{} only\ili{} for\ili{} a\ili{} subject\ili{},\ili{} and\ili{} the\ili{} preposition\ili{} has\ili{} the\ili{} semantic\ili{} form\ili{} \ili{}`på\ili{}<\ili{}[OBJ\ili{}]\ili{}>\ili{}'\ili{},\ili{} indicating\ili{} that\ili{} it\ili{} requires\ili{} an\ili{} object\ili{}.\ili{}
\ili{}%This\ili{} is\ili{} the\ili{} case\ili{} for\ili{} two\ili{} of\ili{} the\ili{} words\ili{} in\ili{} \ili{}(\ili{}\ref\ili{}{ex\ili{}:mweiness\ili{}:thinking\ili{}-while\ili{}-on\ili{}-bus}\ili{})\ili{};\ili{} the\ili{} verb\ili{} has\ili{} the\ili{} semantic\ili{} form\ili{} \ili{}`tenke\ili{}<\ili{}[SUBJ\ili{}]\ili{}>\ili{}'\ili{},\ili{} meaning\ili{} that\ili{} the\ili{} verb\ili{} subcategorizes\ili{} for\ili{} a\ili{} subject\ili{},\ili{} and\ili{} the\ili{} preposition\ili{} has\ili{} the\ili{} semantic\ili{} form\ili{} \ili{}`på\ili{}<\ili{}[OBJ\ili{}]\ili{}>\ili{}'\ili{},\ili{} indicating\ili{} that\ili{} it\ili{} requires\ili{} an\ili{} object\ili{}.\ili{}
\ili{}
Verbs\ili{} can\ili{} of\ili{} course\ili{} subcategorize\ili{} for\ili{} several\ili{} arguments\ili{}.\ili{}
For\ili{} example\ili{},\ili{} the\ili{} verb\ili{} \ili{}\textit\ili{}{slå}\ili{} \ili{}`hit\ili{}'\ili{} has\ili{} the\ili{} semantic\ili{} form\ili{} \ili{}`slå\ili{}<\ili{}[SUBJ\ili{},OBJ\ili{}]\ili{}>\ili{}'\ili{} since\ili{} it\ili{} requires\ili{} a\ili{} subject\ili{} and\ili{} an\ili{} object\ili{}.\ili{}
The\ili{} completeness\ili{} requirement\ili{} for\ili{} f\ili{}-structures\ili{} stipulates\ili{} that\ili{} each\ili{} of\ili{} the\ili{} syntactic\ili{} functions\ili{} mentioned\ili{} in\ili{} the\ili{} semantic\ili{} form\ili{} of\ili{} a\ili{} PRED\ili{} feature\ili{} must\ili{} occur\ili{} on\ili{} the\ili{} same\ili{} level\ili{} of\ili{} f\ili{}-structure\ili{} as\ili{} that\ili{} PRED\ili{}.\ili{}
There\ili{} is\ili{} also\ili{} a\ili{} coherence\ili{} requirement\ili{} to\ili{} the\ili{} effect\ili{} that\ili{} subcategorizable\ili{} syntactic\ili{} functions\ili{} may\ili{} only\ili{} occur\ili{} on\ili{} the\ili{} same\ili{} level\ili{} of\ili{} f\ili{}-structure\ili{} as\ili{} a\ili{} PRED\ili{} feature\ili{} if\ili{} they\ili{} are\ili{} mentioned\ili{} in\ili{} its\ili{} semantic\ili{} form\ili{}.\ili{}
The\ili{} argument\ili{} lists\ili{} in\ili{} semantic\ili{} forms\ili{} are\ili{} thus\ili{} crucial\ili{} for\ili{} determining\ili{} grammaticality\ili{}.\ili{}
The\ili{} semantic\ili{} forms\ili{} \ili{}`\ili{}`govern\ili{} the\ili{} process\ili{} of\ili{} semantic\ili{} interpretation\ili{}”\ili{} \ili{}\citep\ili{}[177\ili{}]\ili{}{KaplanBresnan82}\ili{}.\ili{}
\ili{}
The\ili{} crucial\ili{} challenge\ili{} of\ili{} representing\ili{} MWEs\ili{} is\ili{} that\ili{} they\ili{} defy\ili{} normal\ili{} compositional\ili{} analysis\ili{}.\ili{}
The\ili{} LFG\ili{} solution\ili{} that\ili{} is\ili{} implemented\ili{} in\ili{} NorGram\ili{} is\ili{} to\ili{} assign\ili{} to\ili{} each\ili{} MWE\ili{} a\ili{} special\ili{} \ili{}\isi\ili{}{lexical\ili{} entry}\ili{} that\ili{} has\ili{} its\ili{} own\ili{} PRED\ili{} value\ili{} and\ili{} thus\ili{} its\ili{} own\ili{} argument\ili{} structure\ili{}.\ili{}
Each\ili{} MWE\ili{} has\ili{} a\ili{} semantic\ili{} form\ili{} with\ili{} a\ili{} special\ili{} predicate\ili{} name\ili{} and\ili{} a\ili{} list\ili{} of\ili{} any\ili{} syntactic\ili{} arguments\ili{} that\ili{} this\ili{} predicate\ili{} requires\ili{}.\ili{}
\ili{}%This\ili{} approach\ili{} makes\ili{} it\ili{} possible\ili{} to\ili{} have\ili{} a\ili{} clear\ili{} separation\ili{} between\ili{} the\ili{} \ili{}
This\ili{} will\ili{} be\ili{} shown\ili{} in\ili{} detail\ili{} for\ili{} the\ili{} various\ili{} types\ili{} of\ili{} MWEs\ili{} in\ili{} the\ili{} following\ili{} sections\ili{}.\ili{}
\ili{}%This\ili{} \ili{}\isi\ili{}{template}\ili{} can\ili{} be\ili{} invoked\ili{} in\ili{} the\ili{} \ili{}\isi\ili{}{lexical\ili{} entry}\ili{} in\ili{} \ili{}(\ili{}\ref\ili{}{ex\ili{}:mweiness\ili{}:def\ili{}-entry}\ili{})\ili{}.\ili{}
\ili{}
\ili{}%\ili{}\ea\ili{}\label\ili{}{ex\ili{}:mweiness\ili{}:entry\ili{}-def\ili{}-new}\ili{}
\ili{}%\ili{}$\ili{}+\ili{}$Def\ili{} \ili{} \ili{}\hspace\ili{}{1\ili{}.2em}\ili{} \ili{}@DEF\ili{}
\ili{}%\ili{}\z\ili{}
\ili{}%\ili{}
\ili{}%\ili{}\ea\ili{}\label\ili{}{ex\ili{}:mweiness\ili{}:entry\ili{}-masc\ili{}-new}\ili{}
\ili{}%\ili{}$\ili{}+\ili{}$Masc\ili{} \ili{} \ili{}\hspace\ili{}{1\ili{}.2em}\ili{} \ili{}@MASC\ili{}
\ili{}%\ili{}\z\ili{}
\ili{}%\ili{}
\ili{}%\ili{}\ea\ili{}\label\ili{}{ex\ili{}:mweiness\ili{}:entry\ili{}-buss\ili{}-new}\ili{}
\ili{}%buss\ili{} \ili{} \ili{}\hspace\ili{}{1\ili{}.2em}\ili{} \ili{}@\ili{}(COUNTNOUN\ili{} buss\ili{})\ili{}
\ili{}%\ili{}\z\ili{}
\ili{}
\ili{}
\ili{}%\ili{}\ea\ili{}\label\ili{}{ex\ili{}:mweiness\ili{}:f\ili{}-descr\ili{}-bussen}\ili{}
\ili{}%bussen\ili{} \ili{}\hspace\ili{}{0\ili{}.5em}\ili{} \ili{}(\ili{}$\ili{}\uparrow\ili{}$\ili{}~PRED\ili{})\ili{}=\ili{}`buss\ili{}'\ili{} \ili{}\\ili{}\\ili{}
\ili{}%\ili{}\hspace\ili{}{3\ili{}.6em}\ili{} \ili{} \ili{}(\ili{}$\ili{}\uparrow\ili{}$\ili{}~NTYPE\ili{} NSEM\ili{} COMMON\ili{})\ili{}=count\ili{} \ili{}\\ili{}\\ili{}
\ili{}%\ili{}\hspace\ili{}{3\ili{}.6em}\ili{} \ili{} \ili{}(\ili{}$\ili{}\uparrow\ili{}$\ili{}~NTYPE\ili{} NSYN\ili{})\ili{}=common\ili{} \ili{}\\ili{}\\ili{}
\ili{}%\ili{}\hspace\ili{}{3\ili{}.6em}\ili{} \ili{} \ili{}(\ili{}$\ili{}\uparrow\ili{}$\ili{}~GEND\ili{} NEUT\ili{})\ili{}=\ili{}-\ili{} \ili{}\\ili{}\\ili{}
\ili{}%\ili{}\hspace\ili{}{3\ili{}.6em}\ili{} \ili{} \ili{}(\ili{}$\ili{}\uparrow\ili{}$\ili{}~GEND\ili{} MASC\ili{})\ili{}=\ili{}+\ili{} \ili{}\\ili{}\\ili{}
\ili{}%\ili{}\hspace\ili{}{3\ili{}.6em}\ili{} \ili{} \ili{}(\ili{}$\ili{}\uparrow\ili{}$\ili{}~GEND\ili{} FEM\ili{})\ili{}=\ili{}-\ili{} \ili{}\\ili{}\\ili{}
\ili{}%\ili{}\hspace\ili{}{3\ili{}.6em}\ili{} \ili{} \ili{}(\ili{}$\ili{}\uparrow\ili{}$\ili{}~PERS\ili{})\ili{}=3\ili{} \ili{}\\ili{}\\ili{}
\ili{}%\ili{}\hspace\ili{}{3\ili{}.6em}\ili{} \ili{} \ili{}(\ili{}$\ili{}\uparrow\ili{}$\ili{}~NUM\ili{})\ili{}=sg\ili{} \ili{}\\ili{}\\ili{}
\ili{}%\ili{}\hspace\ili{}{3\ili{}.6em}\ili{} \ili{} \ili{}(\ili{}$\ili{}\uparrow\ili{}$\ili{}~DEF\ili{}-MORPH\ili{})\ili{}=\ili{}+\ili{} \ili{}\\ili{}\\ili{}
\ili{}%\ili{}\hspace\ili{}{3\ili{}.6em}\ili{} \ili{} \ili{}(\ili{}$\ili{}\uparrow\ili{}$\ili{}~DEF\ili{})\ili{}=\ili{}+\ili{} \ili{}\\ili{}\\ili{}
\ili{}%\ili{}\z\ili{}
\ili{}
\ili{}%This\ili{} entry\ili{} includes\ili{} all\ili{} the\ili{} information\ili{} passed\ili{} to\ili{} the\ili{} f\ili{}-structure\ili{} by\ili{} the\ili{} word\ili{} form\ili{} \ili{}\textit\ili{}{bussen}\ili{}.\ili{}
\ili{}%But\ili{} since\ili{} NorGram\ili{} is\ili{} used\ili{} with\ili{} a\ili{} morphological\ili{} analyzer\ili{},\ili{} the\ili{} lexicon\ili{} does\ili{} not\ili{} need\ili{} to\ili{} have\ili{} full\ili{}-form\ili{} entries\ili{}.\ili{} The\ili{} equations\ili{} for\ili{} gender\ili{},\ili{} person\ili{},\ili{} number\ili{} and\ili{} definiteness\ili{} are\ili{} all\ili{} provided\ili{} through\ili{} the\ili{} morphological\ili{} analysis\ili{}.\ili{}
\ili{}%This\ili{} means\ili{} that\ili{} the\ili{} entry\ili{} \ili{}
\ili{}%Although\ili{} the\ili{} usual\ili{} display\ili{} in\ili{} XLE\ili{}-Web\ili{} does\ili{} not\ili{} show\ili{} sublexical\ili{} features\ili{},\ili{} clicking\ili{} on\ili{} a\ili{} preterminal\ili{} node\ili{} expands\ili{} the\ili{} view\ili{} to\ili{} display\ili{} those\ili{} features\ili{}.\ili{}
\ili{}%We\ili{} see\ili{} this\ili{} view\ili{} of\ili{} the\ili{} word\ili{} \ili{}\textit\ili{}{bussen}\ili{} in\ili{} \ili{}(\ili{}\ref\ili{}{fig\ili{}:mweiness\ili{}:sublexical}\ili{})\ili{}.\ili{}
\ili{}%The\ili{} morphological\ili{} features\ili{} \ili{}\textit\ili{}{\ili{}+Noun}\ili{},\ili{} \ili{}\textit\ili{}{\ili{}+Masc}\ili{},\ili{} \ili{}\textit\ili{}{\ili{}+Sg}\ili{} and\ili{} \ili{}\textit\ili{}{\ili{}+Def}\ili{} have\ili{} entries\ili{} in\ili{} the\ili{} lexicon\ili{},\ili{} as\ili{} shown\ili{} in\ili{} \ili{}(\ili{})\ili{}.\ili{}
\ili{}%\ili{}
\ili{}%\ili{}\begin\ili{}{figure}\ili{}
\ili{}%\ili{} \ili{} \ili{}\includegraphics\ili{}[width\ili{}=0\ili{}.7\ili{}\textwidth\ili{}]\ili{}{figures\ili{}/sublexical\ili{}.png}\ili{}
\ili{}%\ili{} \ili{} \ili{}\caption\ili{}{Sublexical\ili{} features\ili{} for\ili{} \ili{}\textit\ili{}{bussen}\ili{}.}\ili{}
\ili{}%\ili{} \ili{} \ili{}\label\ili{}{fig\ili{}:mweiness\ili{}:sublexical}\ili{}
\ili{}%\ili{}\end\ili{}{figure}\ili{}
\ili{}%\ili{}
\ili{}%\ili{}\ea\ili{}\label\ili{}{ex\ili{}:mweiness\ili{}:sublexicalentries}\ili{}
\ili{}%\ili{}+Noun\ili{} \ili{}\hspace\ili{}{0\ili{}.5em}\ili{} \ili{}(\ili{}$\ili{}\uparrow\ili{}$\ili{}~PERS\ili{})\ili{}=3\ili{} \ili{}\\ili{}\\ili{}
\ili{}%\ili{}+Masc\ili{} \ili{}\hspace\ili{}{0\ili{}.7em}\ili{} \ili{}@MASC\ili{} \ili{}\\ili{}\\ili{}
\ili{}%\ili{}+Sg\ili{} \ili{}\hspace\ili{}{1\ili{}.8em}\ili{} \ili{}(\ili{}$\ili{}\uparrow\ili{}$\ili{}~NUM\ili{})\ili{}=sg\ili{} \ili{}\\ili{}\\ili{}
\ili{}%\ili{}+Def\ili{} \ili{}\hspace\ili{}{1\ili{}.3em}\ili{} \ili{}@DEF\ili{} \ili{}\\ili{}\\ili{}
\ili{}%\ili{}\z\ili{}
\ili{}%\ili{}
\ili{}%The\ili{} entries\ili{} for\ili{} \ili{}\textit\ili{}{\ili{}+Noun}\ili{} and\ili{} \ili{}\textit\ili{}{\ili{}+Sg}\ili{} each\ili{} contribute\ili{} one\ili{} equation\ili{}.\ili{}
\ili{} \ili{}
\ili{} \ili{}\section\ili{}{Fixed\ili{} expressions}\ili{}\label\ili{}{sec\ili{}:mweiness\ili{}:fixed}\ili{}
\ili{}
Fixed\ili{} expressions\ili{},\ili{} such\ili{} as\ili{} \ili{}\textit\ili{}{ad\ili{} hoc}\ili{},\ili{} \ili{}\textit\ili{}{déjà\ili{} vu}\ili{},\ili{} and\ili{} \ili{}\textit\ili{}{vice\ili{} versa}\ili{},\ili{} are\ili{} those\ili{} that\ili{} do\ili{} not\ili{} vary\ili{} with\ili{} respect\ili{} to\ili{} inflection\ili{} and\ili{} that\ili{} do\ili{} not\ili{} admit\ili{} any\ili{} internal\ili{} modification\ili{}.\ili{}
They\ili{} are\ili{} also\ili{} called\ili{} inflexible\ili{} expressions\ili{} or\ili{} \ili{}`\ili{}`words\ili{} with\ili{} spaces\ili{}'\ili{}'\ili{}.\ili{}
Fixed\ili{} expressions\ili{} are\ili{} the\ili{} simplest\ili{} MWEs\ili{} to\ili{} implement\ili{};\ili{} they\ili{} are\ili{} entered\ili{} into\ili{} the\ili{} NorGram\ili{} lexicon\ili{} as\ili{} single\ili{} graphical\ili{} words\ili{} containing\ili{} white\ili{} space\ili{},\ili{} so\ili{} they\ili{} are\ili{} literally\ili{} treated\ili{} as\ili{} words\ili{} with\ili{} spaces\ili{}.\ili{}
\ili{}
\ili{}\ea\ili{}\label\ili{}{ex\ili{}:mweiness\ili{}:wws}\ili{}
\ili{}\gll\ili{} Hun\ili{} likte\ili{} \ili{}\textbf\ili{}{i}\ili{} \ili{}\textbf\ili{}{bunn}\ili{} \ili{}\textbf\ili{}{og}\ili{} \ili{}\textbf\ili{}{grunn}\ili{} ikke\ili{} \ili{}\textbf\ili{}{New}\ili{} \ili{}\textbf\ili{}{York}\ili{} \ili{}\textbf\ili{}{i}\ili{} \ili{}\textbf\ili{}{det}\ili{} \ili{}\textbf\ili{}{hele}\ili{} \ili{}\textbf\ili{}{tatt}\ili{}.\ili{} \ili{}\\ili{}\\ili{}
\ili{} \ili{} \ili{} \ili{} \ili{} she\ili{} liked\ili{} in\ili{} bottom\ili{} and\ili{} ground\ili{} not\ili{} New\ili{} York\ili{} in\ili{} the\ili{} whole\ili{} taken\ili{}\\ili{}\\ili{}
\ili{}\glt\ili{} \ili{}`She\ili{} basically\ili{} didn\ili{}’t\ili{} like\ili{} New\ili{} York\ili{} at\ili{} all\ili{}.\ili{}’\ili{}
\ili{}\z\ili{}
\ili{}
The\ili{} sentence\ili{} in\ili{} \ili{}(\ili{}\ref\ili{}{ex\ili{}:mweiness\ili{}:wws}\ili{})\ili{} contains\ili{} three\ili{} such\ili{} expressions\ili{}:\ili{} \ili{}\textit\ili{}{i\ili{} bunn\ili{} og\ili{} grunn}\ili{},\ili{} \ili{}\textit\ili{}{New\ili{} York}\ili{},\ili{} and\ili{} \ili{}\textit\ili{}{i\ili{} det\ili{} hele\ili{} tatt}\ili{}.\ili{}\footnote\ili{}{In\ili{} this\ili{} and\ili{} subsequent\ili{} examples\ili{} the\ili{} lexically\ili{} fixed\ili{} words\ili{} making\ili{} up\ili{} the\ili{} MWE\ili{} are\ili{} highlighted\ili{} with\ili{} boldface\ili{}.}\ili{}
The\ili{} c\ili{}-structure\ili{} of\ili{} \ili{}(\ili{}\ref\ili{}{ex\ili{}:mweiness\ili{}:wws}\ili{})\ili{} is\ili{} shown\ili{} in\ili{} Figure\ili{} \ili{}\ref\ili{}{fig\ili{}:mweiness\ili{}:wws\ili{}-cstr}\ili{}.\ili{}
The\ili{} simplified\ili{} f\ili{}-structure\ili{} is\ili{} shown\ili{} in\ili{} Figure\ili{} \ili{}\ref\ili{}{fig\ili{}:mweiness\ili{}:wws\ili{}-fstr}\ili{};\ili{} this\ili{} is\ili{} the\ili{} \ili{}`\ili{}`PREDs\ili{} only\ili{}'\ili{}'\ili{} view\ili{} of\ili{} f\ili{}-structure\ili{} where\ili{} feature\ili{} paths\ili{} that\ili{} do\ili{} not\ili{} end\ili{} in\ili{} PRED\ili{} values\ili{} are\ili{} suppressed\ili{}.\ili{} \ili{}
The\ili{} three\ili{} expressions\ili{} belong\ili{} to\ili{} different\ili{} parts\ili{} of\ili{} speech\ili{}:\ili{} ADVcmt\ili{} \ili{}(commitment\ili{} adverb\ili{})\ili{},\ili{} PROP\ili{} \ili{}(proper\ili{} noun\ili{})\ili{},\ili{} and\ili{} ADVs\ili{} \ili{}(sentence\ili{} adverb\ili{})\ili{}.\ili{}
The\ili{} adverbs\ili{} have\ili{} the\ili{} function\ili{} ADJUNCT\ili{} in\ili{} the\ili{} f\ili{}-structure\ili{} while\ili{} the\ili{} proper\ili{} noun\ili{} functions\ili{} as\ili{} the\ili{} OBJ\ili{}.\ili{}
There\ili{} are\ili{} numerous\ili{} fixed\ili{} expressions\ili{} in\ili{} most\ili{} parts\ili{} of\ili{} speech\ili{} in\ili{} \ili{}\ili\ili{}{Norwegian}\ili{}.\ili{}
\ili{}
\ili{}%\ili{}\ea\ili{}\label\ili{}{ex\ili{}:mweiness\ili{}:wws}\ili{}
\ili{}%\ili{}\gll\ili{} Hun\ili{} likte\ili{} \ili{}\textbf\ili{}{i}\ili{} \ili{}\textbf\ili{}{bunn}\ili{} \ili{}\textbf\ili{}{og}\ili{} \ili{}\textbf\ili{}{grunn}\ili{} ikke\ili{} \ili{}\textbf\ili{}{New}\ili{} \ili{}\textbf\ili{}{York}\ili{} \ili{}\textbf\ili{}{i}\ili{} \ili{}\textbf\ili{}{det}\ili{} \ili{}\textbf\ili{}{hele}\ili{} \ili{}\textbf\ili{}{tatt}\ili{}.\ili{} \ili{}\\ili{}\\ili{}
\ili{}%\ili{} \ili{} \ili{} \ili{} \ili{} she\ili{} liked\ili{} in\ili{} bottom\ili{} and\ili{} ground\ili{} not\ili{} New\ili{} York\ili{} in\ili{} the\ili{} whole\ili{} taken\ili{}\\ili{}\\ili{}
\ili{}%\ili{}\glt\ili{} \ili{}`She\ili{} basically\ili{} didn\ili{}’t\ili{} like\ili{} New\ili{} York\ili{} at\ili{} all\ili{}.\ili{}’\ili{}
\ili{}%\ili{}\z\ili{}
\ili{}%\ili{}
\ili{}\begin\ili{}{figure}\ili{}
\ili{} \ili{} \ili{}\includegraphics\ili{}[width\ili{}=0\ili{}.6\ili{}\textwidth\ili{}]\ili{}{figures\ili{}/wws\ili{}-cstr}\ili{}
\ili{} \ili{} \ili{}\caption\ili{}{C\ili{}-structure\ili{} for\ili{} example\ili{} \ili{}(\ili{}\ref\ili{}{ex\ili{}:mweiness\ili{}:wws}\ili{})}\ili{}
\ili{} \ili{} \ili{}\label\ili{}{fig\ili{}:mweiness\ili{}:wws\ili{}-cstr}\ili{}
\ili{}\end\ili{}{figure}\ili{}
\ili{}
\ili{}\begin\ili{}{figure}\ili{}
\ili{} \ili{} \ili{}\includegraphics\ili{}[width\ili{}=0\ili{}.9\ili{}\textwidth\ili{}]\ili{}{figures\ili{}/wws\ili{}-fstr}\ili{}
\ili{} \ili{} \ili{}\caption\ili{}{F\ili{}-structure\ili{} for\ili{} example\ili{} \ili{}(\ili{}\ref\ili{}{ex\ili{}:mweiness\ili{}:wws}\ili{})}\ili{}
\ili{} \ili{} \ili{}\label\ili{}{fig\ili{}:mweiness\ili{}:wws\ili{}-fstr}\ili{}
\ili{}\end\ili{}{figure}\ili{}
\ili{}
\ili{}%Fixed\ili{} expressions\ili{} are\ili{} easily\ili{} searchable\ili{} in\ili{} NorGramBank\ili{}.\ili{}
\ili{}%In\ili{} the\ili{} c\ili{}-structure\ili{} they\ili{} may\ili{} be\ili{} found\ili{} by\ili{} searching\ili{} for\ili{} word\ili{} forms\ili{} that\ili{} include\ili{} one\ili{} or\ili{} more\ili{} white\ili{} spaces\ili{}.\ili{}
\ili{}%The\ili{} search\ili{} expression\ili{} \ili{}\textit\ili{}{\ili{}`\ili{}`\ili{}.\ili{}*\ili{} \ili{}.\ili{}*\ili{}'\ili{}'}\ili{} means\ili{}:\ili{} search\ili{} for\ili{} a\ili{} word\ili{} form\ili{} \ili{}(anything\ili{} included\ili{} within\ili{} double\ili{} quotations\ili{} marks\ili{})\ili{} consisting\ili{} of\ili{} any\ili{} character\ili{} \ili{}(the\ili{} period\ili{})\ili{} zero\ili{} or\ili{} more\ili{} times\ili{} \ili{}(the\ili{} Kleene\ili{} star\ili{})\ili{} followed\ili{} by\ili{} a\ili{} white\ili{} space\ili{} followed\ili{} by\ili{} any\ili{} character\ili{} zero\ili{} or\ili{} more\ili{} times\ili{}.\ili{}
\ili{}
\ili{}%NOTES\ili{}:\ili{}
\ili{}%\ili{}
\ili{}%This\ili{} usually\ili{} includes\ili{} lots\ili{} about\ili{} nominal\ili{} compounds\ili{},\ili{} which\ili{} are\ili{} very\ili{} common\ili{} in\ili{} \ili{}\ili\ili{}{English}\ili{}.\ili{}
\ili{}%We\ili{} should\ili{} perhaps\ili{} point\ili{} out\ili{} that\ili{} these\ili{} are\ili{} normally\ili{} written\ili{} as\ili{} single\ili{} words\ili{} in\ili{} \ili{}\ili\ili{}{Norwegian}\ili{}.\ili{}
\ili{}%\ili{}
\ili{}%We\ili{} do\ili{} have\ili{} a\ili{} huge\ili{} number\ili{} of\ili{} names\ili{} that\ili{} are\ili{} fixed\ili{} expressions\ili{}.\ili{}
\ili{}%Perhaps\ili{} we\ili{} should\ili{} look\ili{} into\ili{} to\ili{} what\ili{} degree\ili{} they\ili{} are\ili{} really\ili{} fixed\ili{},\ili{} see\ili{} e\ili{}.g\ili{}.\ili{} Sag\ili{} et\ili{} al\ili{}.\ili{}'s\ili{} stuff\ili{} about\ili{} the\ili{} San\ili{} Francisco\ili{} 49\ili{}'ers\ili{},\ili{} the\ili{} 49\ili{}'ers\ili{},\ili{} etc\ili{}.\ili{}
\ili{}%\ili{}
\ili{}%Prepositional\ili{} MWEs\ili{}
\ili{}%\ili{}
\ili{}%determinerless\ili{}-prepositional\ili{} phrases\ili{} \ili{}(do\ili{} we\ili{} have\ili{} any\ili{} of\ili{} these\ili{}?\ili{})\ili{} \ili{}
\ili{}%\ili{}
\ili{}%complex\ili{} prepositions\ili{} \ili{}(sammen\ili{} med\ili{})\ili{}
\ili{}%\ili{}
\ili{}%conjunctionless\ili{} coordinated\ili{} PPs\ili{}
\ili{}%\ili{}
\ili{}%We\ili{} also\ili{} have\ili{} adverbial\ili{} \ili{}(ansikt\ili{} til\ili{} ansikt\ili{})\ili{},\ili{} pronominal\ili{} \ili{}(noe\ili{} som\ili{} helst\ili{})\ili{}.\ili{}
\ili{}%Other\ili{} types\ili{}?\ili{}
\ili{}
\ili{}
\ili{}\section\ili{}{Flexible\ili{} expressions}\ili{}\label\ili{}{sec\ili{}:mweiness\ili{}:flexexp}\ili{}
\ili{}
Flexible\ili{} expressions\ili{} may\ili{} exhibit\ili{} a\ili{} great\ili{} deal\ili{} of\ili{} syntactic\ili{} variation\ili{},\ili{} but\ili{} in\ili{} some\ili{} respects\ili{} they\ili{} are\ili{} inherently\ili{} fixed\ili{} or\ili{} restricted\ili{}.\ili{}
One\ili{} of\ili{} the\ili{} characterizing\ili{} features\ili{} of\ili{} MWEs\ili{} is\ili{} that\ili{} they\ili{} are\ili{} lexically\ili{} fixed\ili{},\ili{} meaning\ili{} that\ili{} they\ili{} consist\ili{} of\ili{} at\ili{} least\ili{} two\ili{} words\ili{} that\ili{} cannot\ili{} be\ili{} substituted\ili{} with\ili{} near\ili{}-synonyms\ili{} or\ili{} semantically\ili{} related\ili{} words\ili{} without\ili{} the\ili{} expression\ili{} losing\ili{} its\ili{} idiomatic\ili{} meaning\ili{}.\ili{} \ili{}
The\ili{} verbal\ili{} idiom\ili{} \ili{}\emph\ili{}{komme\ili{} på\ili{} kant\ili{} med}\ili{} in\ili{} \ili{}(\ili{}\ref\ili{}{ex\ili{}:mweiness\ili{}:kommepåkantmed}\ili{})\ili{} has\ili{} four\ili{} such\ili{} fixed\ili{} lexical\ili{} words\ili{}.\ili{}
\ili{}
\ili{}%Flexible\ili{} expressions\ili{} may\ili{} exhibit\ili{} a\ili{} great\ili{} deal\ili{} of\ili{} syntactic\ili{} variation\ili{},\ili{} but\ili{} in\ili{} some\ili{} respects\ili{} they\ili{} are\ili{} inherently\ili{} fixed\ili{} or\ili{} restricted\ili{}.\ili{}
\ili{}%As\ili{} one\ili{} of\ili{} their\ili{} characterizing\ili{} features\ili{},\ili{} MWEs\ili{} are\ili{} lexically\ili{} fixed\ili{} with\ili{} at\ili{} least\ili{} two\ili{} fixed\ili{} components\ili{} that\ili{} cannot\ili{} be\ili{} substituted\ili{} with\ili{} near\ili{}-synonyms\ili{} or\ili{} semantically\ili{} related\ili{} words\ili{}.\ili{} \ili{}
\ili{}%\ili{}%One\ili{} characterizing\ili{} feature\ili{} of\ili{} MWEs\ili{} is\ili{} that\ili{} they\ili{} are\ili{} lexically\ili{} fixed\ili{},\ili{} with\ili{} at\ili{} least\ili{} two\ili{} fixed\ili{} components\ili{}.\ili{} \ili{}
\ili{}%The\ili{} verbal\ili{} idiom\ili{} \ili{}\emph\ili{}{komme\ili{} på\ili{} kant\ili{} med}\ili{} in\ili{} \ili{}(\ili{}\ref\ili{}{ex\ili{}:mweiness\ili{}:kommepåkantmed}\ili{})\ili{} has\ili{} four\ili{} fixed\ili{} lexemes\ili{}.\ili{}
\ili{}
\ili{}%Treebank\ili{}:\ili{} nob\ili{}-fn\ili{} version\ili{}:\ili{} 2016\ili{}-05\ili{}-17\ili{};\ili{} Document\ili{}:\ili{} Sandnes\ili{},\ili{} Heidi\ili{} Elisabeth\ili{}:\ili{} I\ili{} skyggen\ili{} av\ili{} Grieg\ili{};\ili{} grammar\ili{}:\ili{} \ili{}\ili\ili{}{Norwegian}\ili{} Bokmål\ili{}
\ili{}%Sentence\ili{} \ili{}#30\ili{} Hun\ili{} var\ili{} ikke\ili{} villig\ili{} til\ili{} å\ili{} komme\ili{} på\ili{} kant\ili{} med\ili{} det\ili{} gode\ili{} selskap\ili{}.\ili{}
\ili{}\ea\ili{}\label\ili{}{ex\ili{}:mweiness\ili{}:kommepåkantmed}\ili{}
\ili{}\gll\ili{} Hun\ili{} var\ili{} ikke\ili{} villig\ili{} til\ili{} å\ili{} \ili{}\textbf\ili{}{komme}\ili{} \ili{}\textbf\ili{}{på}\ili{} \ili{}\textbf\ili{}{kant}\ili{} \ili{}\textbf\ili{}{med}\ili{} det\ili{} gode\ili{} selskap\ili{}.\ili{} \ili{}\\ili{}\\ili{}
\ili{} \ili{} \ili{} \ili{} \ili{} she\ili{} was\ili{} not\ili{} willing\ili{} to\ili{} to\ili{} come\ili{} on\ili{} edge\ili{} with\ili{} the\ili{} good\ili{} company\ili{} \ili{}\\ili{}\\ili{}
\ili{}\glt\ili{} \ili{}`She\ili{} was\ili{} not\ili{} willing\ili{} to\ili{} fall\ili{} out\ili{} with\ili{} the\ili{} in\ili{} crowd\ili{}.\ili{}’\ili{}
\ili{}\z\ili{}
\ili{}
Flexible\ili{} MWEs\ili{} are\ili{} often\ili{} also\ili{} morphosyntactically\ili{} restricted\ili{},\ili{} with\ili{} constraints\ili{} on\ili{} grammatical\ili{} features\ili{},\ili{} on\ili{} the\ili{} modification\ili{} of\ili{} component\ili{} words\ili{},\ili{} and\ili{} on\ili{} specifiers\ili{} such\ili{} as\ili{} quantifiers\ili{} and\ili{} determiners\ili{}.\ili{}
For\ili{} instance\ili{},\ili{} the\ili{} noun\ili{} \ili{}\emph\ili{}{kant}\ili{} in\ili{} \ili{}(\ili{}\ref\ili{}{ex\ili{}:mweiness\ili{}:kommepåkantmed}\ili{})\ili{} can\ili{} only\ili{} be\ili{} in\ili{} the\ili{} singular\ili{} indefinite\ili{} form\ili{},\ili{} and\ili{} it\ili{} does\ili{} not\ili{} admit\ili{} any\ili{} specifiers\ili{} or\ili{} modifiers\ili{}.\ili{}
\ili{}%While\ili{} \ili{}\emph\ili{}{kant}\ili{} does\ili{} not\ili{} admit\ili{} specifiers\ili{} or\ili{} modifiers\ili{} in\ili{} this\ili{} expression\ili{},\ili{} the\ili{} PP\ili{} \ili{}\emph\ili{}{på\ili{} kant\ili{} med}\ili{} does\ili{}.\ili{} \ili{}
The\ili{} PP\ili{} \ili{}\emph\ili{}{på\ili{} kant\ili{} med}\ili{},\ili{} however\ili{},\ili{} does\ili{} admit\ili{} modifiers\ili{};\ili{} in\ili{} \ili{}(\ili{}\ref\ili{}{ex\ili{}:mweiness\ili{}:modification}\ili{})\ili{} the\ili{} modifier\ili{} \ili{}\emph\ili{}{helt}\ili{} \ili{}`completely\ili{}'\ili{} has\ili{} scope\ili{} over\ili{} the\ili{} entire\ili{} expression\ili{}.\ili{}
In\ili{} NorGram\ili{},\ili{} no\ili{} distinction\ili{} is\ili{} currently\ili{} made\ili{} in\ili{} the\ili{} representation\ili{} of\ili{} internal\ili{} and\ili{} \ili{}(semantically\ili{})\ili{} external\ili{} modification\ili{} of\ili{} MWEs\ili{}.\ili{} \ili{}
\ili{}
\ili{}\ea\ili{}\label\ili{}{ex\ili{}:mweiness\ili{}:modification}\ili{}
\ili{}\gll\ili{} Hun\ili{} \ili{}\textbf\ili{}{kom}\ili{} helt\ili{} \ili{}\textbf\ili{}{på}\ili{} \ili{}\textbf\ili{}{kant}\ili{} \ili{}\textbf\ili{}{med}\ili{} det\ili{} gode\ili{} selskap\ili{}.\ili{} \ili{}\\ili{}\\ili{}
she\ili{} came\ili{} fully\ili{} on\ili{} edge\ili{} with\ili{} the\ili{} good\ili{} society\ili{} \ili{}\\ili{}\\ili{}
\ili{}\glt\ili{} \ili{}`She\ili{} completely\ili{} fell\ili{} out\ili{} with\ili{} the\ili{} in\ili{} crowd\ili{}.\ili{}'\ili{} \ili{}\\ili{}\\ili{}
\ili{}\z\ili{}
\ili{}%\ili{}\ea\ili{}\label\ili{}{ex\ili{}:mweiness\ili{}:modification}\ili{}
\ili{}%hun\ili{} \ili{}\textbf\ili{}{kom}\ili{} helt\ili{} \ili{}\textbf\ili{}{på}\ili{} \ili{}\textbf\ili{}{kant}\ili{} \ili{}\textbf\ili{}{med}\ili{} naboene\ili{} \ili{}\\ili{}\\ili{}
\ili{}%she\ili{} came\ili{} fully\ili{} on\ili{} edge\ili{} with\ili{} \ili{}{the\ili{} neighbours}\ili{} \ili{}\\ili{}\\ili{}
\ili{}%\ili{}`she\ili{} fell\ili{} out\ili{} with\ili{} her\ili{} neighbours\ili{} completely\ili{}'\ili{} \ili{}\\ili{}\\ili{}
\ili{}%\ili{}\z\ili{}
\ili{}
The\ili{} mechanisms\ili{} for\ili{} representing\ili{} lexical\ili{} and\ili{} morphological\ili{} restrictions\ili{} in\ili{} flexible\ili{} MWEs\ili{} are\ili{} the\ili{} same\ili{} as\ili{} the\ili{} ones\ili{} used\ili{} for\ili{} regular\ili{} constructions\ili{}.\ili{}
As\ili{} described\ili{} in\ili{} Section\ili{} \ili{}\ref\ili{}{sec\ili{}:mweiness\ili{}:LFG}\ili{},\ili{} simplex\ili{} words\ili{} are\ili{} assigned\ili{} predicate\ili{} values\ili{} through\ili{} equations\ili{} in\ili{} the\ili{} \ili{}\isi\ili{}{lexical\ili{} entry}\ili{},\ili{} as\ili{} in\ili{} the\ili{} entry\ili{} for\ili{} the\ili{} simplex\ili{} lexeme\ili{} \ili{}\emph\ili{}{buss}\ili{} in\ili{} example\ili{} \ili{}(\ili{}\ref\ili{}{ex\ili{}:mweiness\ili{}:f\ili{}-descr\ili{}-bussen}\ili{})\ili{},\ili{} which\ili{} has\ili{} the\ili{} predicate\ili{} assignment\ili{} equation\ili{} \ili{}(\ili{}$\ili{}\uparrow\ili{}$\ili{}~PRED\ili{})\ili{}=\ili{}`buss\ili{}'\ili{}.\ili{} \ili{}
For\ili{} words\ili{} that\ili{} subcategorize\ili{} for\ili{} other\ili{} elements\ili{},\ili{} such\ili{} as\ili{} verbs\ili{},\ili{} this\ili{} is\ili{} done\ili{} through\ili{} the\ili{} assignment\ili{} of\ili{} a\ili{} \ili{}\isi\ili{}{predicate\ili{}-argument\ili{} structure}\ili{} \ili{}(or\ili{} \ili{}\isi\ili{}{subcategorization}\ili{} frame\ili{})\ili{}.\ili{} \ili{}
For\ili{} instance\ili{},\ili{} the\ili{} intransitive\ili{} verb\ili{} \ili{}\emph\ili{}{klage}\ili{} \ili{}`complain\ili{}'\ili{} is\ili{} assigned\ili{} a\ili{} frame\ili{} through\ili{} the\ili{} \ili{}\isi\ili{}{template}\ili{} call\ili{} \ili{}@\ili{}(V\ili{}-SUBJ\ili{} klage\ili{})\ili{} in\ili{} the\ili{} \ili{}\isi\ili{}{lexical\ili{} entry}\ili{},\ili{} invoking\ili{} the\ili{} \ili{}\isi\ili{}{template}\ili{} V\ili{}-SUBJ\ili{}.\ili{} \ili{}
Part\ili{} of\ili{} this\ili{} \ili{}\isi\ili{}{template}\ili{} is\ili{} shown\ili{} in\ili{} \ili{}(\ili{}\ref\ili{}{ex\ili{}:mweiness\ili{}:verb\ili{}-template}\ili{})\ili{}.\ili{}
\ili{}
\ili{}\ea\ili{}\label\ili{}{ex\ili{}:mweiness\ili{}:verb\ili{}-template}\ili{}
\ili{}{\ili{}\small\ili{} \ili{}
\ili{} V\ili{}-SUBJ\ili{} \ili{}(P\ili{})\ili{} \ili{}=\ili{} \ili{}\\ili{}\\ili{}
\ili{}\hspace\ili{}{2em}\ili{} \ili{}(\ili{}$\ili{}\uparrow\ili{}$\ili{} PRED\ili{})\ili{}=\ili{}`P\ili{}<\ili{}(\ili{}$\ili{}\uparrow\ili{}$\ili{} SUBJ\ili{})\ili{}>\ili{}'\ili{} \ili{}\\ili{}\\ili{}
}\ili{}
\ili{}\z\ili{}
\ili{}
The\ili{} predicate\ili{} value\ili{} of\ili{} the\ili{} verb\ili{} is\ili{} parameterized\ili{} and\ili{} listed\ili{} together\ili{} with\ili{} its\ili{} arguments\ili{} in\ili{} the\ili{} \ili{}\isi\ili{}{subcategorization}\ili{} frame\ili{},\ili{} which\ili{} is\ili{} everything\ili{} between\ili{} quotation\ili{} marks\ili{} in\ili{} \ili{}(\ili{}\ref\ili{}{ex\ili{}:mweiness\ili{}:verb\ili{}-template}\ili{})\ili{}.\ili{}
When\ili{} the\ili{} \ili{}\isi\ili{}{template}\ili{} is\ili{} invoked\ili{},\ili{} the\ili{} lemma\ili{} form\ili{} \ili{}`klage\ili{}'\ili{} in\ili{} the\ili{} \ili{}\isi\ili{}{template}\ili{} call\ili{} replaces\ili{} the\ili{} parameter\ili{} P\ili{}.\ili{} \ili{}
The\ili{} equation\ili{} on\ili{} the\ili{} second\ili{} line\ili{} assigns\ili{} one\ili{} argument\ili{},\ili{} the\ili{} subject\ili{},\ili{} to\ili{} P\ili{},\ili{} yielding\ili{} the\ili{} predicate\ili{}-argument\ili{} \ili{} structure\ili{} \ili{}`klage\ili{}<\ili{}(\ili{}$\ili{}\uparrow\ili{}$\ili{}~SUBJ\ili{})\ili{}>\ili{}'\ili{} as\ili{} the\ili{} PRED\ili{} value\ili{} for\ili{} the\ili{} intransitive\ili{} reading\ili{} of\ili{} this\ili{} verb\ili{}.\ili{}
\ili{}
\ili{}
Lexical\ili{} fixedness\ili{} in\ili{} flexible\ili{} MWEs\ili{} is\ili{} handled\ili{} through\ili{} lexical\ili{} selection\ili{} in\ili{} the\ili{} entry\ili{} of\ili{} the\ili{} subcategorizing\ili{} word\ili{}.\ili{} \ili{}
In\ili{} addition\ili{} to\ili{} its\ili{} usual\ili{} intransitive\ili{} reading\ili{},\ili{} the\ili{} verb\ili{} \ili{}\emph\ili{}{klage}\ili{} \ili{}`complain\ili{}'\ili{} is\ili{} the\ili{} syntactic\ili{} head\ili{} of\ili{} the\ili{} VP\ili{} idiom\ili{} \ili{}\emph\ili{}{klage\ili{} sin\ili{} nød}\ili{} \ili{}`pour\ili{} out\ili{} one\ili{}'s\ili{} troubles\ili{}'\ili{},\ili{} where\ili{} it\ili{} subcategorizes\ili{} for\ili{} the\ili{} object\ili{} noun\ili{} \ili{}\emph\ili{}{nød}\ili{} \ili{}`need\ili{}'\ili{}.\ili{} \ili{}
\ili{}
\ili{}%Treebank\ili{}:\ili{} nob\ili{}-novel_5\ili{} version\ili{}:\ili{} 2016\ili{}-05\ili{}-17\ili{};\ili{} Document\ili{}:\ili{} Henriksen\ili{},\ili{} Vera\ili{}:\ili{} Helgenkongen\ili{};\ili{} grammar\ili{}:\ili{} \ili{}\ili\ili{}{Norwegian}\ili{} Bokmål\ili{}
\ili{}%Sentence\ili{} \ili{}#3447\ili{}:\ili{} Og\ili{} hun\ili{} klaget\ili{} sin\ili{} nød\ili{} til\ili{} alle\ili{} som\ili{} ville\ili{} høre\ili{}.\ili{}
\ili{} \ili{}\ea\ili{}\label\ili{}{ex\ili{}:mweiness\ili{}:klagesinnød}\ili{}
\ili{}\gll\ili{} Og\ili{} hun\ili{} \ili{}\textbf\ili{}{klaget}\ili{} \ili{}\textbf\ili{}{sin}\ili{} \ili{}\textbf\ili{}{nød}\ili{} til\ili{} alle\ili{} som\ili{} ville\ili{} høre\ili{}.\ili{} \ili{}\\ili{}\\ili{}
\ili{} \ili{} \ili{} \ili{} \ili{} and\ili{} she\ili{} complained\ili{} her\ili{} need\ili{} to\ili{} all\ili{} who\ili{} wanted\ili{} hear\ili{} \ili{}\\ili{}\\ili{}
\ili{}\glt\ili{} \ili{}`And\ili{} she\ili{} poured\ili{} out\ili{} her\ili{} troubles\ili{} to\ili{} everyone\ili{} who\ili{} wanted\ili{} to\ili{} listen\ili{}.\ili{}’\ili{}
\ili{}\z\ili{}
\ili{}%\ili{} \ili{}\ea\ili{}\label\ili{}{ex\ili{}:mweiness\ili{}:klagesinnød}\ili{}
\ili{}%\ili{}\gll\ili{} Hun\ili{} \ili{}\textbf\ili{}{klaget}\ili{} \ili{}\textbf\ili{}{sin}\ili{} \ili{}\textbf\ili{}{nød}\ili{} til\ili{} naboene\ili{}.\ili{} \ili{}\\ili{}\\ili{}
\ili{}%\ili{} \ili{} \ili{} \ili{} \ili{} she\ili{} complained\ili{} \ili{} her\ili{} need\ili{} to\ili{} \ili{}{the\ili{} neighbours}\ili{} \ili{}\\ili{}\\ili{}
\ili{}%\ili{}\glt\ili{} \ili{}`She\ili{} complained\ili{} to\ili{} the\ili{} neighbours\ili{}.\ili{}’\ili{}
\ili{}%\ili{}\z\ili{}
\ili{}
Lexical\ili{} entries\ili{} for\ili{} VP\ili{} idioms\ili{} are\ili{} listed\ili{} as\ili{} alternative\ili{} \ili{}\isi\ili{}{subcategorization}\ili{} frames\ili{} under\ili{} the\ili{} entry\ili{} of\ili{} the\ili{} verb\ili{}.\ili{}
As\ili{} in\ili{} the\ili{} case\ili{} of\ili{} simplex\ili{} verbs\ili{},\ili{} templates\ili{} assign\ili{} predicate\ili{}-argument\ili{} structures\ili{} and\ili{} other\ili{} relevant\ili{} features\ili{}.\ili{} \ili{} \ili{}
The\ili{} word\ili{} that\ili{} subcategorizes\ili{} for\ili{} the\ili{} other\ili{} parts\ili{} of\ili{} the\ili{} MWE\ili{} lists\ili{} the\ili{} predicate\ili{} values\ili{} of\ili{} the\ili{} selected\ili{} arguments\ili{} together\ili{} with\ili{} its\ili{} own\ili{} predicate\ili{} value\ili{} in\ili{} the\ili{} \ili{}\isi\ili{}{template}\ili{} invocation\ili{}.\ili{} \ili{}
The\ili{} \ili{}\isi\ili{}{template}\ili{} call\ili{} in\ili{} \ili{}(\ili{}\ref\ili{}{ex\ili{}:mweiness\ili{}:lexicalselection}\ili{})\ili{} shows\ili{} that\ili{} the\ili{} verb\ili{} \ili{}\emph\ili{}{klage}\ili{} selects\ili{} the\ili{} noun\ili{} \ili{}\emph\ili{}{nød}\ili{}.\ili{} \ili{}
\ili{}
\ili{}\ea\ili{}\label\ili{}{ex\ili{}:mweiness\ili{}:lexicalselection}\ili{}
\ili{}{\ili{}\small\ili{} \ili{}
\ili{}@\ili{}(VPIDIOM\ili{}-DEFOBJ\ili{} klage\ili{} nød\ili{})\ili{}
}\ili{}
\ili{}\z\ili{}
\ili{}
\ili{}\ea\ili{}\label\ili{}{ex\ili{}:mweiness\ili{}:subcatframe}\ili{}
\ili{}{\ili{}\small\ili{}
\ili{}%VPIDIOM\ili{}-DEFOBJ\ili{} \ili{}(P\ili{} S\ili{} OP\ili{})\ili{} \ili{}=\ili{} \ili{} \ili{}\\ili{}\\ili{}
\ili{}%\ili{}	\ili{} \ili{} \ili{}@\ili{}(CONCAT\ili{} P\ili{} \ili{}`\ili{}#\ili{} OP\ili{} \ili{}\\ili{}%FN\ili{})\ili{} \ili{}\\ili{}\\ili{}
\ili{}	\ili{} \ili{} \ili{}(\ili{}$\ili{}\uparrow\ili{}$\ili{} PRED\ili{})\ili{}=\ili{}`\ili{}\\ili{}%FN\ili{}<\ili{}(\ili{}$\ili{}\uparrow\ili{}$\ili{} SUBJ\ili{})\ili{}>\ili{}(\ili{}$\ili{}\uparrow\ili{}$\ili{} OBJ\ili{})\ili{}'\ili{} \ili{} \ili{}\\ili{}\\ili{}
\ili{}%\ili{}	\ili{} \ili{} \ili{}(\ili{}^\ili{} OBJ\ili{} PRED\ili{} FN\ili{})\ili{}=c\ili{} OP\ili{}
\ili{}%\ili{}	\ili{} \ili{} \ili{}(\ili{}^\ili{} OBJ\ili{} DEF\ili{})\ili{}=\ili{}+\ili{}
\ili{}%\ili{} \ili{} \ili{} \ili{} \ili{} \ili{} \ili{} \ili{} \ili{} \ili{} \ili{}(\ili{}^\ili{} OBJ\ili{} NUM\ili{})\ili{}=c\ili{} sg\ili{}
}\ili{}
\ili{}\z\ili{}
\ili{}
\ili{}%NOTE\ili{}-GSL\ili{}:\ili{} the\ili{} next\ili{} example\ili{} and\ili{} paragraph\ili{} is\ili{} slightly\ili{} more\ili{} detailed\ili{} than\ili{} I\ili{} wanted\ili{} it\ili{} to\ili{} be\ili{},\ili{} and\ili{} probably\ili{} repeats\ili{} things\ili{} that\ili{} are\ili{} already\ili{} said\ili{} in\ili{} the\ili{} subsequent\ili{} sections\ili{}.\ili{} However\ili{},\ili{} I\ili{} think\ili{} it\ili{} may\ili{} be\ili{} useful\ili{} to\ili{} explain\ili{} how\ili{} we\ili{} distinguish\ili{} between\ili{} fixed\ili{} and\ili{} variable\ili{} \ili{}(aka\ili{} selected\ili{} and\ili{} semantic\ili{})\ili{} arguments\ili{} \ili{} in\ili{} \ili{}`MWE\ili{} terms\ili{}'\ili{},\ili{} in\ili{} order\ili{} to\ili{} introduce\ili{} and\ili{} explain\ili{} these\ili{} terms\ili{}.\ili{} Maybe\ili{} move\ili{} this\ili{} part\ili{} so\ili{} that\ili{} it\ili{} comes\ili{} after\ili{} the\ili{} \ili{}(simpler\ili{})\ili{} description\ili{} of\ili{} how\ili{} grammatical\ili{} constraints\ili{} are\ili{} handled\ili{}?\ili{}
\ili{}
\ili{}%\ili{}\ea\ili{}\label\ili{}{ex\ili{}:mweiness\ili{}:mweiness\ili{}:subcatframe}\ili{}
\ili{}%VPIDIOM\ili{}-DEFOBJ\ili{} \ili{}(P\ili{} S\ili{} OP\ili{})\ili{} \ili{}=\ili{} \ili{} \ili{}\\ili{}\\ili{}
\ili{}%\ili{}	\ili{} \ili{} \ili{}@\ili{}(CONCAT\ili{} P\ili{} \ili{}`\ili{}#\ili{} OP\ili{} \ili{}\\ili{}%FN\ili{})\ili{} \ili{}\\ili{}\\ili{}
\ili{}%\ili{}	\ili{} \ili{} \ili{}(\ili{}$\ili{}\uparrow\ili{}$\ili{} PRED\ili{})\ili{}=\ili{}'\ili{}\\ili{}%FN\ili{}<\ili{}(\ili{}$\ili{}\uparrow\ili{}$\ili{} SUBJ\ili{})\ili{}>\ili{} \ili{}(\ili{}$\ili{}\uparrow\ili{}$\ili{} OBJ\ili{})\ili{}'\ili{} \ili{} \ili{}\\ili{}\\ili{}
\ili{}%\ili{}	\ili{} \ili{} \ili{}(\ili{}^\ili{} OBJ\ili{} PRED\ili{} FN\ili{})\ili{}=c\ili{} OP\ili{}
\ili{}%\ili{}	\ili{} \ili{} \ili{}(\ili{}^\ili{} OBJ\ili{} DEF\ili{})\ili{}=\ili{}+\ili{}
\ili{}%\ili{} \ili{} \ili{} \ili{} \ili{} \ili{} \ili{} \ili{} \ili{} \ili{} \ili{}(\ili{}^\ili{} OBJ\ili{} NUM\ili{})\ili{}=c\ili{} sg\ili{}
\ili{}%\ili{}\z\ili{}
\ili{}
The\ili{} predicate\ili{} values\ili{} of\ili{} the\ili{} fixed\ili{} MWE\ili{} components\ili{},\ili{} i\ili{}.e\ili{}.\ili{} the\ili{} verb\ili{} and\ili{} its\ili{} selected\ili{} complements\ili{},\ili{} are\ili{} merged\ili{} to\ili{} form\ili{} one\ili{} single\ili{} idiom\ili{} predicate\ili{} which\ili{} is\ili{} substituted\ili{} for\ili{} the\ili{} relevant\ili{} parameter\ili{} in\ili{} the\ili{} \ili{}\isi\ili{}{predicate\ili{}-argument\ili{} structure}\ili{}.\ili{}
In\ili{} one\ili{} of\ili{} the\ili{} equations\ili{} in\ili{} the\ili{} \ili{}\isi\ili{}{template}\ili{},\ili{} the\ili{} predicate\ili{} assignment\ili{} in\ili{} \ili{}(\ili{}\ref\ili{}{ex\ili{}:mweiness\ili{}:subcatframe}\ili{})\ili{},\ili{} \ili{} the\ili{} parameter\ili{} \ili{}\\ili{}%FN\ili{} is\ili{} replaced\ili{} by\ili{} the\ili{} predicate\ili{} name\ili{} \ili{}\emph\ili{}{klage\ili{}\\ili{}#nød}\ili{},\ili{} where\ili{} we\ili{} use\ili{} the\ili{} symbol\ili{} \ili{}`\ili{}`\ili{}\\ili{}#\ili{}'\ili{}'\ili{} to\ili{} signal\ili{} idiomatic\ili{} combinations\ili{} of\ili{} this\ili{} kind\ili{}.\ili{}
Only\ili{} the\ili{} free\ili{} arguments\ili{} of\ili{} \ili{}\isi\ili{}{verbal\ili{} MWEs}\ili{} are\ili{} specified\ili{} as\ili{} semantic\ili{} arguments\ili{} to\ili{} the\ili{} verb\ili{}.\ili{}
The\ili{} \ili{}\isi\ili{}{subcategorization}\ili{} frame\ili{} \ili{}`klage\ili{}\\ili{}#nød\ili{}<\ili{}(\ili{}$\ili{}\uparrow\ili{}$\ili{}~SUBJ\ili{})\ili{}>\ili{}(\ili{}$\ili{}\uparrow\ili{}$\ili{}~OBJ\ili{})\ili{}'\ili{} lists\ili{} the\ili{} semantic\ili{} argument\ili{},\ili{} in\ili{} this\ili{} case\ili{} SUBJ\ili{},\ili{} inside\ili{} the\ili{} angled\ili{} brackets\ili{},\ili{} while\ili{} the\ili{} selected\ili{} argument\ili{} OBJ\ili{} is\ili{} placed\ili{} outside\ili{} the\ili{} brackets\ili{}.\ili{}
The\ili{} parameter\ili{} \ili{}\\ili{}%FN\ili{} and\ili{} the\ili{} construction\ili{} of\ili{} predicate\ili{} names\ili{} such\ili{} as\ili{} \ili{}\emph\ili{}{klage\ili{}\\ili{}#nød}\ili{} are\ili{} accounted\ili{} for\ili{} in\ili{} Section\ili{} \ili{}\ref\ili{}{sec\ili{}:mweiness\ili{}:prepverbs}\ili{}.\ili{} \ili{} \ili{}
\ili{}
\ili{}%Fixed\ili{} or\ili{} lexically\ili{} selected\ili{} arguments\ili{} are\ili{} non\ili{}-thematic\ili{} and\ili{} thus\ili{} listed\ili{} outside\ili{} the\ili{} angled\ili{} brackets\ili{},\ili{} like\ili{} the\ili{} OBJ\ili{} in\ili{} \ili{}(\ili{}\ref\ili{}{ex\ili{}:mweiness\ili{}:subcatframe}\ili{})\ili{} which\ili{} in\ili{} this\ili{} case\ili{} refers\ili{} to\ili{} the\ili{} selected\ili{} object\ili{} \ili{}\emph\ili{}{nød}\ili{}.\ili{}
\ili{}
Constraints\ili{} on\ili{} grammatical\ili{} features\ili{} are\ili{} specified\ili{} with\ili{} constraining\ili{} equations\ili{} and\ili{} existential\ili{} constraints\ili{} in\ili{} the\ili{} entries\ili{} or\ili{} templates\ili{}.\ili{}
A\ili{} constraining\ili{} equation\ili{} is\ili{} an\ili{} equation\ili{} with\ili{} a\ili{} \ili{}`\ili{}`c\ili{}'\ili{}'\ili{} attached\ili{} to\ili{} the\ili{} equal\ili{} sign\ili{}.\ili{}
This\ili{} means\ili{} that\ili{} the\ili{} equation\ili{} does\ili{} not\ili{} actually\ili{} assign\ili{} the\ili{} specified\ili{} value\ili{} to\ili{} the\ili{} attribute\ili{} in\ili{} the\ili{} f\ili{}-structure\ili{};\ili{} instead\ili{} it\ili{} requires\ili{} that\ili{} this\ili{} value\ili{} has\ili{} been\ili{} assigned\ili{} to\ili{} the\ili{} attribute\ili{} somewhere\ili{} else\ili{}.\ili{}
The\ili{} restriction\ili{} that\ili{} the\ili{} object\ili{} \ili{}\emph\ili{}{sin\ili{} nød}\ili{} in\ili{} \ili{}(\ili{}\ref\ili{}{ex\ili{}:mweiness\ili{}:klagesinnød}\ili{})\ili{} must\ili{} be\ili{} definite\ili{} is\ili{} specified\ili{} with\ili{} the\ili{} equation\ili{} in\ili{} \ili{}(\ili{}\ref\ili{}{ex\ili{}:mweiness\ili{}:defrestriction}\ili{})\ili{}.\ili{}
The\ili{} constraint\ili{} in\ili{} \ili{}(\ili{}\ref\ili{}{ex\ili{}:mweiness\ili{}:possrestriction}\ili{})\ili{} is\ili{} an\ili{} existential\ili{} constraint\ili{} which\ili{} simply\ili{} provides\ili{} \ili{} a\ili{} path\ili{} of\ili{} attributes\ili{} without\ili{} assigning\ili{} a\ili{} particular\ili{} value\ili{}.\ili{}
The\ili{} interpretation\ili{} is\ili{} that\ili{} this\ili{} path\ili{} of\ili{} attributes\ili{} must\ili{} have\ili{} some\ili{} value\ili{} in\ili{} the\ili{} f\ili{}-structure\ili{},\ili{} thus\ili{} ensuring\ili{} that\ili{} there\ili{} is\ili{} a\ili{} possessive\ili{}.\ili{}
\ili{}
\ili{}\ea\ili{}\label\ili{}{ex\ili{}:mweiness\ili{}:defrestriction}\ili{}
\ili{}{\ili{}\small\ili{} \ili{}
\ili{}(\ili{}$\ili{}\uparrow\ili{}$\ili{} OBJ\ili{} DEF\ili{})\ili{}=c\ili{} \ili{}+\ili{}.\ili{}
}\ili{}
\ili{}\z\ili{}
\ili{}
\ili{}\ea\ili{}\label\ili{}{ex\ili{}:mweiness\ili{}:possrestriction}\ili{}
\ili{}{\ili{}\small\ili{} \ili{}
\ili{}(\ili{}$\ili{}\uparrow\ili{}$\ili{} OBJ\ili{} SPEC\ili{} POSS\ili{} POSS\ili{}-TYPE\ili{})\ili{} \ili{}\\ili{}\\ili{}
}\ili{}
\ili{}\z\ili{}
\ili{}
\ili{}%\ili{}\ea\ili{}\label\ili{}{ex\ili{}:mweiness\ili{}:mweiness\ili{}:VP\ili{}\isi\ili{}{template}\ili{}:klage}\ili{}
\ili{}%\ili{} \ili{} \ili{}@\ili{}(VPIDIOM\ili{}-DEFOBJ\ili{} klage\ili{} klage\ili{} nød\ili{})\ili{} \ili{}\\ili{}\\ili{}
\ili{}%\ili{}	\ili{}	\ili{} \ili{} \ili{}(\ili{}$\ili{}\uparrow\ili{}$\ili{} OBJ\ili{} SPEC\ili{} POSS\ili{} POSS\ili{}-TYPE\ili{})\ili{} \ili{}\\ili{}\\ili{}
\ili{}%\ili{}	\ili{}	\ili{} \ili{} \ili{}(\ili{}$\ili{}\uparrow\ili{}$\ili{} OBJ\ili{} NUM\ili{})\ili{}=c\ili{} sg\ili{} \ili{}\\ili{}\\ili{}
\ili{}%\ili{}\z\ili{}
\ili{}
\ili{}\ea\ili{}\label\ili{}{ex\ili{}:mweiness\ili{}:nospec}\ili{}
\ili{}{\ili{}\small\ili{} \ili{}
\ili{}{\ili{}\textasciitilde}\ili{}(\ili{}$\ili{}\uparrow\ili{}$\ili{} OBJ\ili{} SPEC\ili{})\ili{} \ili{}\\ili{}\\ili{}
}\ili{}
\ili{}\z\ili{}
\ili{}
The\ili{} selection\ili{} of\ili{} grammatical\ili{} words\ili{} and\ili{} modifiers\ili{} is\ili{} handled\ili{} in\ili{} a\ili{} slightly\ili{} different\ili{} way\ili{} from\ili{} the\ili{} selection\ili{} of\ili{} syntactic\ili{} heads\ili{}.\ili{}
If\ili{} a\ili{} determiner\ili{} is\ili{} selected\ili{} or\ili{} otherwise\ili{} restricted\ili{},\ili{} this\ili{} is\ili{} specified\ili{} with\ili{} a\ili{} constraint\ili{} requiring\ili{} that\ili{} the\ili{} type\ili{} or\ili{} form\ili{} of\ili{} the\ili{} determiner\ili{} must\ili{} match\ili{} the\ili{} specification\ili{}.\ili{}
The\ili{} existential\ili{} constraint\ili{} in\ili{} \ili{}(\ili{}\ref\ili{}{ex\ili{}:mweiness\ili{}:possrestriction}\ili{})\ili{} ensures\ili{} that\ili{} a\ili{} possessive\ili{} will\ili{} specify\ili{} \ili{}\emph\ili{}{nød}\ili{}.\ili{}
If\ili{} no\ili{} determiner\ili{} is\ili{} possible\ili{} in\ili{} an\ili{} idiom\ili{},\ili{} this\ili{} is\ili{} specified\ili{} with\ili{} a\ili{} negative\ili{} constraint\ili{},\ili{} as\ili{} in\ili{} \ili{}(\ili{}\ref\ili{}{ex\ili{}:mweiness\ili{}:nospec}\ili{})\ili{}.\ili{}
\ili{}
Lexical\ili{} constraints\ili{} on\ili{} modifiers\ili{} are\ili{} represented\ili{} in\ili{} the\ili{} same\ili{} way\ili{} as\ili{} grammatical\ili{} constraints\ili{},\ili{} using\ili{} equations\ili{}.\ili{} \ili{}
Some\ili{} nouns\ili{} do\ili{} not\ili{} admit\ili{} modification\ili{} at\ili{} all\ili{},\ili{} such\ili{} as\ili{} \ili{}\emph\ili{}{kant}\ili{} in\ili{} \ili{}(\ili{}\ref\ili{}{ex\ili{}:mweiness\ili{}:kommepåkantmed}\ili{})\ili{}.\ili{}
Others\ili{} may\ili{} require\ili{} that\ili{} the\ili{} choice\ili{} of\ili{} modifier\ili{} is\ili{} restricted\ili{} to\ili{} a\ili{} specific\ili{} predicate\ili{} or\ili{} set\ili{} of\ili{} predicates\ili{},\ili{} such\ili{} as\ili{} \ili{}\textit\ili{}{øye}\ili{} \ili{}`eye\ili{}'\ili{} in\ili{} the\ili{} VP\ili{} idiom\ili{} \ili{}\emph\ili{}{ha\ili{} et\ili{} godt\ili{} øye\ili{} til}\ili{} \ili{}`have\ili{} eyes\ili{} for\ili{}'\ili{} in\ili{} \ili{}(\ili{}\ref\ili{}{ex\ili{}:mweiness\ili{}:haøyetil}\ili{})\ili{},\ili{} where\ili{} the\ili{} only\ili{} possible\ili{} modifier\ili{} is\ili{} the\ili{} adjective\ili{} \ili{}\emph\ili{}{god}\ili{} \ili{}`good\ili{}'\ili{}.\ili{}
\ili{}
\ili{}%Treebank\ili{}:\ili{} nob\ili{}-novel_4\ili{} version\ili{}:\ili{} 2016\ili{}-05\ili{}-17\ili{};\ili{} Document\ili{}:\ili{} Larssen\ili{},\ili{} Trude\ili{} Brænne\ili{}:\ili{} Prisgitt\ili{};\ili{} grammar\ili{}:\ili{} \ili{}\ili\ili{}{Norwegian}\ili{} Bokmål\ili{}
\ili{}%Sentence\ili{} \ili{}#4378\ili{}:\ili{} \ili{}–\ili{} Det\ili{} kan\ili{} være\ili{} han\ili{} har\ili{} et\ili{} godt\ili{} øye\ili{} til\ili{} deg\ili{}.\ili{} \ili{}
\ili{}\ea\ili{}\label\ili{}{ex\ili{}:mweiness\ili{}:haøyetil}\ili{}
\ili{}\gll\ili{} Det\ili{} kan\ili{} være\ili{} han\ili{} \ili{}\textbf\ili{}{har}\ili{} \ili{}\textbf\ili{}{et}\ili{} \ili{}\textbf\ili{}{godt}\ili{} \ili{}\textbf\ili{}{øye}\ili{} \ili{}\textbf\ili{}{til}\ili{} deg\ili{}.\ili{} \ili{}\\ili{}\\ili{}
it\ili{} can\ili{} be\ili{} he\ili{} has\ili{} a\ili{} good\ili{} eye\ili{} to\ili{} you\ili{} \ili{}\\ili{}\\ili{}
\ili{}\glt\ili{} \ili{}`He\ili{} might\ili{} have\ili{} eyes\ili{} for\ili{} you\ili{}.\ili{}'\ili{} \ili{}
\ili{}\z\ili{}
\ili{}
When\ili{} a\ili{} modifier\ili{} is\ili{} lexically\ili{} restricted\ili{},\ili{} a\ili{} constraint\ili{} equation\ili{} is\ili{} used\ili{} to\ili{} specify\ili{} the\ili{} possible\ili{} modifier\ili{} predicate\ili{}(s\ili{})\ili{}.\ili{}
In\ili{} the\ili{} entry\ili{} for\ili{} \ili{}\emph\ili{}{ha\ili{} et\ili{} godt\ili{} øye\ili{} til}\ili{},\ili{} the\ili{} equation\ili{} in\ili{} \ili{}(\ili{}\ref\ili{}{ex\ili{}:mweiness\ili{}:modrestriction}\ili{})\ili{} ensures\ili{} that\ili{} the\ili{} modifier\ili{} \ili{}(ADJUNCT\ili{})\ili{} of\ili{} the\ili{} selected\ili{} object\ili{} \ili{}(the\ili{} noun\ili{} \ili{}\emph\ili{}{øye}\ili{})\ili{} has\ili{} the\ili{} PRED\ili{} value\ili{} \ili{}`god\ili{}'\ili{}.\ili{} \ili{}
\ili{}
\ili{}\ea\ili{}\label\ili{}{ex\ili{}:mweiness\ili{}:modrestriction}\ili{}
\ili{}%\ili{}@\ili{}(VPIDIOM\ili{}-INDEFOBJnonth\ili{}-POBJ\ili{} ha\ili{} ha\ili{} øye\ili{} til\ili{})\ili{} \ili{}\\ili{}\\ili{}
\ili{}%\ili{}(\ili{}$\ili{}\uparrow\ili{}$\ili{} OBJ\ili{} ADJUNCT\ili{} \ili{}\\ili{}$\ili{} PRED\ili{} FN\ili{})\ili{}=c\ili{} god\ili{} \ili{}\\ili{}\\ili{}
\ili{}%\ili{}	\ili{}	\ili{} \ili{} \ili{}(\ili{}^\ili{} OBJ\ili{} SPEC\ili{} DET\ili{} DET\ili{}-TYPE\ili{})\ili{}=c\ili{} article\ili{} \ili{}\\ili{}\\ili{}
\ili{}%\ili{}	\ili{}	\ili{} \ili{} \ili{}(\ili{}^\ili{} VTYPE\ili{})\ili{}=c\ili{} main\ili{} \ili{}\\ili{}\\ili{}
\ili{}{\ili{}\small\ili{} \ili{}
\ili{}(\ili{}$\ili{}\uparrow\ili{}$\ili{} OBJ\ili{} ADJUNCT\ili{} PRED\ili{})\ili{}=c\ili{} god\ili{} \ili{}\\ili{}\\ili{}
}\ili{}
\ili{}\z\ili{}
\ili{}%NOTE\ili{}-GSL\ili{}:\ili{} simplified\ili{} the\ili{} equation\ili{} to\ili{} avoid\ili{} having\ili{} to\ili{} explain\ili{} that\ili{} a\ili{})\ili{} the\ili{} path\ili{} refers\ili{} to\ili{} the\ili{} set\ili{} of\ili{} adjuncts\ili{},\ili{} and\ili{} b\ili{})\ili{} that\ili{} FN\ili{} is\ili{} a\ili{} local\ili{} parameter\ili{} \ili{}
\ili{}
The\ili{} treatment\ili{} of\ili{} lexical\ili{} restrictions\ili{} in\ili{} VP\ili{} idioms\ili{} in\ili{} NorGram\ili{} thus\ili{} depends\ili{} on\ili{} the\ili{} function\ili{} of\ili{} the\ili{} component\ili{} word\ili{} within\ili{} the\ili{} MWE\ili{}.\ili{} \ili{}
While\ili{} syntactic\ili{} heads\ili{} are\ili{} subcategorized\ili{} for\ili{} by\ili{} the\ili{} verb\ili{},\ili{} dependents\ili{} are\ili{} specified\ili{} using\ili{} constraint\ili{} equations\ili{}.\ili{}
\ili{}
\ili{}%The\ili{} VP\ili{} idiom\ili{} \ili{}\emph\ili{}{klage\ili{} sin\ili{} nød}\ili{} is\ili{} relatively\ili{} restricted\ili{} in\ili{} the\ili{} sense\ili{} that\ili{} it\ili{} does\ili{} not\ili{} allow\ili{} many\ili{} syntactic\ili{} \ili{}\isi\ili{}{transformations}\ili{}.\ili{}
\ili{}%However\ili{},\ili{} the\ili{} same\ili{} analysis\ili{} strategies\ili{} are\ili{} employed\ili{} as\ili{} for\ili{} more\ili{} flexible\ili{} verbal\ili{} expressions\ili{}.\ili{}
\ili{}%This\ili{} also\ili{} holds\ili{} for\ili{} the\ili{} other\ili{} types\ili{} of\ili{} flexible\ili{} but\ili{} restricted\ili{} MWEs\ili{} in\ili{} NorGram\ili{},\ili{} nouns\ili{} and\ili{} adjectives\ili{} taking\ili{} selected\ili{} prepositions\ili{}.\ili{}
\ili{}%These\ili{} are\ili{} handled\ili{} similar\ili{} to\ili{} verbs\ili{},\ili{} through\ili{} the\ili{} specification\ili{} of\ili{} \ili{}\isi\ili{}{subcategorization}\ili{} frames\ili{}.\ili{}
\ili{}%However\ili{},\ili{} since\ili{} prepositions\ili{} are\ili{} treated\ili{} as\ili{} grammatical\ili{} words\ili{} that\ili{} do\ili{} not\ili{} contribute\ili{} to\ili{} the\ili{} f\ili{}-structure\ili{} with\ili{} their\ili{} own\ili{} PRED\ili{} values\ili{},\ili{} the\ili{} lexical\ili{} selection\ili{} is\ili{} ensured\ili{} not\ili{} through\ili{} predicate\ili{} assignment\ili{},\ili{} but\ili{} with\ili{} a\ili{} constraint\ili{} equation\ili{} requiring\ili{} that\ili{} the\ili{} preposition\ili{} must\ili{} have\ili{} a\ili{} specific\ili{} form\ili{},\ili{} as\ili{} is\ili{} done\ili{} for\ili{} specifiers\ili{}.\ili{}
\ili{}%Examples\ili{} of\ili{} this\ili{} are\ili{} given\ili{} in\ili{} sections\ili{} \ili{}(\ili{}\ref\ili{}{sec\ili{}:prepnoun}\ili{})\ili{} and\ili{} \ili{}(\ili{}\ref\ili{}{sec\ili{}:prepadj}\ili{})\ili{}.\ili{}
\ili{}%According\ili{} to\ili{} the\ili{} criteria\ili{} presented\ili{} in\ili{} section\ili{} \ili{}(\ili{}\ref\ili{}{sec\ili{}:intro}\ili{})\ili{} both\ili{} the\ili{} VP\ili{} idiom\ili{} and\ili{} the\ili{} expressions\ili{} with\ili{} selected\ili{} prepositions\ili{} should\ili{} be\ili{} treated\ili{} semi\ili{}-fixed\ili{}.\ili{}
\ili{}%The\ili{} strategies\ili{} used\ili{} to\ili{} represent\ili{} them\ili{},\ili{} however\ili{},\ili{} does\ili{} not\ili{} differ\ili{} from\ili{} more\ili{} flexible\ili{} MWEs\ili{}.\ili{} \ili{}
\ili{}%Since\ili{} no\ili{} distinction\ili{} is\ili{} reflected\ili{} in\ili{} the\ili{} way\ili{} they\ili{} are\ili{} being\ili{} analyzed\ili{},\ili{} NorGram\ili{} assumes\ili{} a\ili{} distinction\ili{} only\ili{} between\ili{} fixed\ili{} and\ili{} flexible\ili{} expressions\ili{},\ili{} where\ili{} flexible\ili{} MWEs\ili{} distribute\ili{} over\ili{} a\ili{} fixedness\ili{} scale\ili{} ranging\ili{} from\ili{} syntactically\ili{} restricted\ili{} to\ili{} almost\ili{} full\ili{} syntactic\ili{} flexibility\ili{}.\ili{}\footnote\ili{}{That\ili{} is\ili{},\ili{} \ili{}`\ili{}`full\ili{}'\ili{}'\ili{} within\ili{} the\ili{} boundaries\ili{} of\ili{} the\ili{} lexicon\ili{} and\ili{} grammar\ili{}.}\ili{}
\ili{} \ili{} \ili{}
\ili{}\subsection\ili{}{Phrasal\ili{} verbs}\ili{}\label\ili{}{sec\ili{}:mweiness\ili{}:phrasal}\ili{}
\ili{}
Phrasal\ili{} verbs\ili{} are\ili{} MWEs\ili{} consisting\ili{} of\ili{} a\ili{} verb\ili{} and\ili{} an\ili{} adverb\ili{},\ili{} preposition\ili{} or\ili{} other\ili{} word\ili{} that\ili{} together\ili{} have\ili{} a\ili{} meaning\ili{} that\ili{} is\ili{} in\ili{} some\ili{} way\ili{} idiosyncratic\ili{}.\ili{}
It\ili{} is\ili{} common\ili{} to\ili{} distinguish\ili{} between\ili{} two\ili{} main\ili{} classes\ili{} of\ili{} \ili{}\isi\ili{}{phrasal\ili{} verbs}\ili{},\ili{} \ili{}\isi\ili{}{prepositional\ili{} verbs}\ili{} and\ili{} \ili{}\isi\ili{}{verb\ili{}-particle\ili{} constructions}\ili{}.\ili{}
We\ili{} present\ili{} these\ili{} two\ili{} types\ili{} in\ili{} \ili{}\ref\ili{}{sec\ili{}:mweiness\ili{}:prepverbs}\ili{} and\ili{} \ili{}\ref\ili{}{sec\ili{}:mweiness\ili{}:prtverbs}\ili{},\ili{} respectively\ili{}.\ili{}
There\ili{} are\ili{} also\ili{} constructions\ili{} where\ili{} both\ili{} prepositions\ili{} and\ili{} particles\ili{} occur\ili{};\ili{} these\ili{} are\ili{} presented\ili{} in\ili{} \ili{}\ref\ili{}{sec\ili{}:mweiness\ili{}:prtprepverbs}\ili{}.\ili{}
\ili{}
\ili{}\subsubsection\ili{}{Prepositional\ili{} verbs}\ili{}\label\ili{}{sec\ili{}:mweiness\ili{}:prepverbs}\ili{}
\ili{}
In\ili{} Section\ili{} \ili{}\ref\ili{}{sec\ili{}:mweiness\ili{}:LFG}\ili{} the\ili{} sentence\ili{} in\ili{} \ili{}(\ili{}\ref\ili{}{ex\ili{}:mweiness\ili{}:thinking\ili{}-while\ili{}-on\ili{}-bus}\ili{})\ili{} was\ili{} shown\ili{} to\ili{} have\ili{} two\ili{} readings\ili{}.\ili{}
When\ili{} the\ili{} prepositional\ili{} \ili{}\isi\ili{}{phrase}\ili{} functions\ili{} as\ili{} an\ili{} adjunct\ili{},\ili{} the\ili{} analysis\ili{} shown\ili{} in\ili{} Figure\ili{} \ili{}\ref\ili{}{fig\ili{}:mweiness\ili{}:thinking\ili{}-while\ili{}-on\ili{}-bus}\ili{} obtains\ili{}.\ili{}
When\ili{} the\ili{} preposition\ili{} is\ili{} selected\ili{} by\ili{} the\ili{} verb\ili{},\ili{} the\ili{} verb\ili{} and\ili{} the\ili{} preposition\ili{} constitute\ili{} a\ili{} MWE\ili{},\ili{} as\ili{} indicated\ili{} in\ili{} \ili{}(\ili{}\ref\ili{}{ex\ili{}:mweiness\ili{}:tenkepå}\ili{})\ili{},\ili{} where\ili{} these\ili{} words\ili{} are\ili{} boldfaced\ili{}.\ili{}
The\ili{} analysis\ili{} corresponding\ili{} to\ili{} this\ili{} reading\ili{} is\ili{} shown\ili{} in\ili{} Figure\ili{} \ili{}\ref\ili{}{fig\ili{}:mweiness\ili{}:selprepcons}\ili{}.\ili{}
\ili{}
\ili{}%In\ili{} Section\ili{} \ili{}\ref\ili{}{sec\ili{}:mweiness\ili{}:LFG}\ili{} the\ili{} sentence\ili{} in\ili{} \ili{}(\ili{}\ref\ili{}{ex\ili{}:mweiness\ili{}:thinking\ili{}-while\ili{}-on\ili{}-bus}\ili{})\ili{},\ili{} repeated\ili{} here\ili{} as\ili{} \ili{} \ili{}(\ili{}\ref\ili{}{ex\ili{}:mweiness\ili{}:tenkepå}\ili{})\ili{},\ili{} was\ili{} shown\ili{} to\ili{} have\ili{} two\ili{} readings\ili{}.\ili{}
\ili{}%The\ili{} first\ili{} reading\ili{},\ili{} with\ili{} the\ili{} prepositional\ili{} \ili{}\isi\ili{}{phrase}\ili{} functioning\ili{} as\ili{} an\ili{} adjunct\ili{},\ili{} was\ili{} shown\ili{} in\ili{} Figure\ili{} \ili{}\ref\ili{}{fig\ili{}:mweiness\ili{}:thinking\ili{}-while\ili{}-on\ili{}-bus}\ili{}.\ili{}
\ili{}%When\ili{} the\ili{} preposition\ili{} is\ili{} selected\ili{} by\ili{} the\ili{} verb\ili{},\ili{} the\ili{} second\ili{} reading\ili{} obtains\ili{}.\ili{}
\ili{}%The\ili{} analysis\ili{} with\ili{} the\ili{} prepositional\ili{} verb\ili{} is\ili{} shown\ili{} in\ili{} Figure\ili{} \ili{}\ref\ili{}{fig\ili{}:mweiness\ili{}:selprepcons}\ili{}.\ili{}
\ili{}
\ili{}\ea\ili{}\label\ili{}{ex\ili{}:mweiness\ili{}:tenkepå}\ili{}
\ili{}\gll\ili{} Hun\ili{} \ili{}\textbf\ili{}{tenkte}\ili{} \ili{}\textbf\ili{}{på}\ili{} bussen\ili{}.\ili{} \ili{}\\ili{}\\ili{}
\ili{} \ili{} \ili{} \ili{} \ili{} she\ili{} thought\ili{} on\ili{} \ili{}{the\ili{} bus}\ili{}\\ili{}\\ili{}
\ili{}\glt\ili{} \ili{}`She\ili{} thought\ili{} about\ili{} the\ili{} bus\ili{}.\ili{}'\ili{}
\ili{}\z\ili{}
\ili{}
\ili{}%\ili{}\ea\ili{}\label\ili{}{ex\ili{}:mweiness\ili{}:tenkepå}\ili{}
\ili{}%\ili{}\gll\ili{} Hun\ili{} \ili{}\textbf\ili{}{tenkte}\ili{} \ili{}\textbf\ili{}{på}\ili{} buss\ili{}-en\ili{}.\ili{} \ili{}\\ili{}\\ili{}
\ili{}%\ili{} \ili{} \ili{} \ili{} \ili{} she\ili{} thought\ili{} on\ili{} bus\ili{}-\ili{}\textsc\ili{}{def}\ili{}.\ili{}\textsc\ili{}{sg}\ili{}\\ili{}\\ili{}
\ili{}%\ili{}\glt\ili{} \ili{}`She\ili{} was\ili{} thinking\ili{} \ili{}(while\ili{})\ili{} on\ili{} the\ili{} bus\ili{}.\ili{}/She\ili{} thought\ili{} about\ili{} the\ili{} bus\ili{}.\ili{}
\ili{}%\ili{}\z\ili{}
\ili{}
\ili{}\begin\ili{}{figure}\ili{}
\ili{} \ili{} \ili{}\includegraphics\ili{}[width\ili{}=\ili{}\textwidth\ili{}]\ili{}{figures\ili{}/selprepcons\ili{}.png}\ili{}
\ili{} \ili{} \ili{}\caption\ili{}{C\ili{}-\ili{} and\ili{} f\ili{}-structure\ili{} for\ili{} example\ili{} \ili{}(\ili{}\ref\ili{}{ex\ili{}:mweiness\ili{}:tenkepå}\ili{})}\ili{}
\ili{} \ili{} \ili{}\label\ili{}{fig\ili{}:mweiness\ili{}:selprepcons}\ili{}
\ili{}\end\ili{}{figure}\ili{}
\ili{}
In\ili{} the\ili{} c\ili{}-structure\ili{} \ili{}\textit\ili{}{på\ili{} bussen}\ili{} forms\ili{} a\ili{} prepositional\ili{} \ili{}\isi\ili{}{phrase}\ili{} PPsel\ili{}-n\ili{},\ili{} marked\ili{} as\ili{} selected\ili{} by\ili{} \ili{}`\ili{}`sel\ili{}-n\ili{}'\ili{}'\ili{} in\ili{} the\ili{} node\ili{} label\ili{}.\ili{}
This\ili{} analysis\ili{} captures\ili{} the\ili{} fact\ili{} that\ili{} the\ili{} selected\ili{} preposition\ili{} \ili{}\textit\ili{}{på}\ili{} can\ili{} only\ili{} occur\ili{} before\ili{} the\ili{} object\ili{},\ili{} and\ili{} that\ili{} the\ili{} preposition\ili{} and\ili{} its\ili{} complement\ili{} behave\ili{} as\ili{} one\ili{} constituent\ili{} with\ili{} respect\ili{} to\ili{} movement\ili{},\ili{} as\ili{} in\ili{} the\ili{} topicalized\ili{} version\ili{} \ili{}\textit\ili{}{På\ili{} bussen\ili{} tenkte\ili{} hun\ili{} ofte}\ili{} \ili{}`The\ili{} bus\ili{} she\ili{} was\ili{} often\ili{} thinking\ili{} of\ili{}'\ili{}.\ili{}
The\ili{} preposition\ili{} does\ili{} not\ili{} provide\ili{} its\ili{} own\ili{} predicate\ili{} in\ili{} the\ili{} f\ili{}-structure\ili{},\ili{} but\ili{} is\ili{} analyzed\ili{} as\ili{} incorporated\ili{} in\ili{} the\ili{} predicate\ili{} expressed\ili{} by\ili{} the\ili{} verb\ili{} to\ili{} form\ili{} the\ili{} predicate\ili{} name\ili{} \ili{}`tenke\ili{}*på\ili{}'\ili{}.\ili{}
In\ili{} predicate\ili{} names\ili{} the\ili{} symbol\ili{} \ili{}`\ili{}`\ili{}*\ili{}'\ili{}'\ili{} is\ili{} used\ili{} to\ili{} signal\ili{} such\ili{} combinations\ili{} of\ili{} a\ili{} lexical\ili{} predicate\ili{} with\ili{} a\ili{} selected\ili{} particle\ili{} or\ili{} preposition\ili{}.\ili{}
The\ili{} complement\ili{} of\ili{} the\ili{} preposition\ili{},\ili{} \ili{}\textit\ili{}{bussen}\ili{},\ili{} fills\ili{} the\ili{} function\ili{} OBL\ili{}-TH\ili{} \ili{}-\ili{}-\ili{} oblique\ili{}-theta\ili{} \ili{}-\ili{}-\ili{} as\ili{} an\ili{} argument\ili{} of\ili{} this\ili{} predicate\ili{},\ili{} i\ili{}.e\ili{}.\ili{},\ili{} an\ili{} oblique\ili{} argument\ili{} expressing\ili{} a\ili{} theta\ili{} role\ili{}.\ili{}
\ili{}
The\ili{} \ili{}\isi\ili{}{lexical\ili{} entry}\ili{} for\ili{} \ili{}\textit\ili{}{tenke}\ili{} is\ili{} associated\ili{} with\ili{} the\ili{} relevant\ili{} frame\ili{} through\ili{} an\ili{} invocation\ili{} of\ili{} the\ili{} \ili{}\isi\ili{}{template}\ili{} describing\ili{} this\ili{} class\ili{} of\ili{} constructions\ili{}.\ili{}
The\ili{} relevant\ili{} part\ili{} of\ili{} the\ili{} \ili{}\isi\ili{}{lexical\ili{} entry}\ili{} for\ili{} \ili{}\textit\ili{}{tenke}\ili{} is\ili{} shown\ili{} in\ili{} \ili{} \ili{}(\ili{}\ref\ili{}{ex\ili{}:mweiness\ili{}:tenke\ili{}-lex}\ili{})\ili{}.\ili{}
\ili{}
\ili{}\eabox\ili{}{\ili{}\label\ili{}{ex\ili{}:mweiness\ili{}:tenke\ili{}-lex}\ili{}
\ili{}{\ili{}\small\ili{} \ili{}
\ili{}\begin\ili{}{tabular}\ili{}{ll}\ili{}
tenke\ili{} V\ili{} \ili{}&\ili{} \ili{}\\ili{}{\ili{} \ili{}\enspace\ili{} \ili{}[\ili{} \ili{}.\ili{}.\ili{}.\ili{} \ili{}]\ili{}\\ili{}\\ili{}
\ili{}&\ili{} \ili{}|\ili{} \ili{}\enspace\ili{} \ili{}@\ili{}(V\ili{}-SUBJ\ili{}-POBJ\ili{} tenke\ili{} på\ili{})\ili{}\\ili{}\\ili{}
\ili{}&\ili{} \ili{}|\ili{} \ili{}\enspace\ili{} \ili{}[\ili{} \ili{}.\ili{}.\ili{}.\ili{} \ili{}]\ili{} \ili{}\enspace\ili{} \ili{}\}\ili{}
\ili{}\end\ili{}{tabular}}}\ili{}
\ili{}
The\ili{} invocation\ili{} of\ili{} the\ili{} \ili{}\isi\ili{}{template}\ili{} V\ili{}-SUBJ\ili{}-POBJ\ili{} has\ili{} two\ili{} parameters\ili{},\ili{} the\ili{} predicate\ili{} name\ili{} for\ili{} the\ili{} verb\ili{} \ili{}\textit\ili{}{tenke}\ili{} and\ili{} the\ili{} form\ili{} of\ili{} the\ili{} selected\ili{} preposition\ili{} \ili{}\textit\ili{}{på}\ili{}.\ili{}
In\ili{} \ili{}(\ili{}\ref\ili{}{ex\ili{}:mweiness\ili{}:tenke\ili{}-template}\ili{})\ili{} part\ili{} of\ili{} the\ili{} \ili{}\isi\ili{}{template}\ili{} is\ili{} shown\ili{} \ili{}(other\ili{} parts\ili{} of\ili{} this\ili{} \ili{}\isi\ili{}{template}\ili{} for\ili{} handling\ili{} passive\ili{} and\ili{} other\ili{} modifications\ili{} are\ili{} discussed\ili{} in\ili{} Section\ili{} \ili{}\ref\ili{}{sec\ili{}:mweiness\ili{}:variation}\ili{})\ili{}.\ili{}
\ili{}
\ili{}\ea\ili{}\label\ili{}{ex\ili{}:mweiness\ili{}:tenke\ili{}-template}\ili{}
\ili{}{\ili{}\small\ili{} \ili{}
V\ili{}-SUBJ\ili{}-POBJ\ili{} \ili{}(P\ili{} prp\ili{})\ili{} \ili{}=\ili{}\\ili{}\\ili{}
\ili{}\hspace\ili{}{2em}\ili{} \ili{}@\ili{}(CONCAT\ili{} P\ili{} \ili{}`\ili{}*\ili{} prp\ili{} \ili{}\\ili{}%FN\ili{})\ili{}\\ili{}\\ili{}
\ili{}\hspace\ili{}{2em}\ili{} \ili{} \ili{}(\ili{}$\ili{}\uparrow\ili{}$\ili{} PRED\ili{})\ili{}=\ili{}`\ili{}\\ili{}%FN\ili{}<\ili{}(\ili{}$\ili{}\uparrow\ili{}$\ili{} SUBJ\ili{})\ili{}(\ili{}$\ili{}\uparrow\ili{}$\ili{} OBL\ili{}-TH\ili{})\ili{}>\ili{}'\ili{}\\ili{}\\ili{}
\ili{}\hspace\ili{}{2em}\ili{} \ili{} \ili{}(\ili{}$\ili{}\uparrow\ili{}$\ili{} OBL\ili{}-TH\ili{} CHECK\ili{} P\ili{}-SELFORM\ili{})\ili{}=prp\ili{}
}\ili{}
\ili{}\z\ili{}
\ili{}
The\ili{} \ili{}\isi\ili{}{template}\ili{} invokes\ili{} another\ili{} \ili{}\isi\ili{}{template}\ili{} CONCAT\ili{},\ili{} which\ili{} concatenates\ili{} the\ili{} pre\ili{}\\ili{}-dicate\ili{} name\ili{} P\ili{} and\ili{} the\ili{} preposition\ili{} form\ili{} prp\ili{} as\ili{} the\ili{} value\ili{} of\ili{} the\ili{} variable\ili{} \ili{}\\ili{}%FN\ili{}.\ili{}
In\ili{} this\ili{} example\ili{} the\ili{} result\ili{} is\ili{} the\ili{} predicate\ili{} name\ili{} \ili{}`tenke\ili{}*på\ili{}'\ili{},\ili{} which\ili{} is\ili{} then\ili{} included\ili{} in\ili{} the\ili{} value\ili{} of\ili{} PRED\ili{}.\ili{}
The\ili{} last\ili{} line\ili{} assigns\ili{} the\ili{} value\ili{} of\ili{} prp\ili{} \ili{}(\ili{}\textit\ili{}{på}\ili{} in\ili{} the\ili{} example\ili{})\ili{} as\ili{} the\ili{} value\ili{} of\ili{} the\ili{} attribute\ili{} P\ili{}-SELFORM\ili{} under\ili{} the\ili{} OBL\ili{}-TH\ili{} argument\ili{}.\ili{}
This\ili{} feature\ili{} is\ili{} checked\ili{} by\ili{} the\ili{} syntactic\ili{} \ili{}\isi\ili{}{rule}\ili{} which\ili{} introduces\ili{} the\ili{} selected\ili{} PP\ili{},\ili{} ensuring\ili{} that\ili{} only\ili{} the\ili{} preposition\ili{} selected\ili{} by\ili{} the\ili{} verb\ili{} is\ili{} accepted\ili{}.\ili{}
\ili{}
\ili{}%The\ili{} criterion\ili{} for\ili{} assigning\ili{} a\ili{} construction\ili{} to\ili{} the\ili{} selected\ili{} preposition\ili{} type\ili{} rather\ili{} that\ili{} the\ili{} particle\ili{} construction\ili{} type\ili{} is\ili{} the\ili{} restriction\ili{} of\ili{} the\ili{} preposition\ili{} to\ili{} occur\ili{} only\ili{} before\ili{} an\ili{} object\ili{} occurring\ili{} in\ili{} the\ili{} verb\ili{} \ili{}\isi\ili{}{phrase}\ili{}.\ili{}
\ili{}%There\ili{} is\ili{} a\ili{} set\ili{} of\ili{} cases\ili{} where\ili{} this\ili{} criterion\ili{} is\ili{} not\ili{} totally\ili{} satisfactory\ili{}:\ili{} cases\ili{} where\ili{} the\ili{} particle\ili{}/preposition\ili{} receives\ili{} stress\ili{} and\ili{} does\ili{} not\ili{} clearly\ili{} form\ili{} a\ili{} constituent\ili{} with\ili{} the\ili{} object\ili{},\ili{} but\ili{} is\ili{} still\ili{} restricted\ili{} to\ili{} occurring\ili{} before\ili{} it\ili{}.\ili{}
\ili{}%Such\ili{} cases\ili{} are\ili{} analyzed\ili{} as\ili{} involving\ili{} selected\ili{} prepositions\ili{},\ili{} in\ili{} accordance\ili{} with\ili{} the\ili{} criterion\ili{}.\ili{}
\ili{}%An\ili{} example\ili{} is\ili{} \ili{}\textit\ili{}{daske\ili{} til}\ili{} \ili{}=\ili{} \ili{}`slap\ili{}'\ili{} in\ili{} a\ili{} treebank\ili{} sentence\ili{} like\ili{} \ili{}(\ili{}\ref\ili{}{ex\ili{}:mweiness\ili{}:dasketil}\ili{})\ili{}.\ili{}
\ili{}
\ili{}%\ili{}\ea\ili{}\label\ili{}{ex\ili{}:mweiness\ili{}:dasketil}\ili{}
\ili{}%\ili{}\gll\ili{} Elise\ili{} dasket\ili{} til\ili{} ham\ili{}.\ili{} \ili{}\\ili{}\\ili{}
\ili{}%\ili{} \ili{} \ili{} \ili{} \ili{} Elise\ili{} slap\ili{}.\ili{}\textsc\ili{}{pst}\ili{} to\ili{} him\ili{} \ili{}\\ili{}\\ili{}
\ili{}%\ili{}\glt\ili{} \ili{}`Elise\ili{} slapped\ili{} him\ili{}'\ili{}
\ili{}%\ili{}\z\ili{}
\ili{}
\ili{}%This\ili{} leads\ili{} to\ili{} a\ili{} slight\ili{} overgeneration\ili{} since\ili{} the\ili{} grammar\ili{} will\ili{} accept\ili{} the\ili{} ungrammatical\ili{} topicalization\ili{} of\ili{} \ili{}\textit\ili{}{til\ili{} ham}\ili{}:\ili{} \ili{}\textit\ili{}{\ili{}*Til\ili{} ham\ili{} dasket\ili{} Elise}\ili{}.\ili{}
\ili{}%However\ili{},\ili{} we\ili{} consider\ili{} that\ili{} an\ili{} acceptable\ili{} price\ili{} to\ili{} pay\ili{} to\ili{} avoid\ili{} further\ili{} complication\ili{} of\ili{} the\ili{} grammar\ili{} and\ili{} lexicon\ili{}.\ili{}
\ili{}
\ili{}\subsubsection\ili{}{Verb\ili{}-particle\ili{} constructions}\ili{}\label\ili{}{sec\ili{}:mweiness\ili{}:prtverbs}\ili{}
\ili{}
Verb\ili{}-particle\ili{} constructions\ili{} consist\ili{} of\ili{} a\ili{} verb\ili{} and\ili{} a\ili{} selected\ili{} particle\ili{} in\ili{} the\ili{} form\ili{} of\ili{} an\ili{} adverb\ili{} or\ili{} an\ili{} intransitively\ili{} used\ili{} preposition\ili{};\ili{} in\ili{} NorGram\ili{} such\ili{} elements\ili{} are\ili{} classified\ili{} as\ili{} PRT\ili{} in\ili{} the\ili{} c\ili{}-structure\ili{}.\ili{}
The\ili{} verb\ili{} and\ili{} the\ili{} particle\ili{} express\ili{} an\ili{} idiosyncratic\ili{} meaning\ili{}.\ili{}
As\ili{} in\ili{} \ili{}\ili\ili{}{English}\ili{},\ili{} \ili{}\isi\ili{}{verb\ili{}-particle\ili{} constructions}\ili{} in\ili{} \ili{}\ili\ili{}{Norwegian}\ili{} can\ili{} have\ili{} the\ili{} particle\ili{} either\ili{} before\ili{} or\ili{} after\ili{} an\ili{} object\ili{},\ili{} and\ili{} obligatorily\ili{} after\ili{} if\ili{} the\ili{} object\ili{} is\ili{} pronominal\ili{};\ili{} cf\ili{}.\ili{} \ili{}\cite\ili{}[276\ili{}]\ili{}{Baldwin10}\ili{}.\ili{}
The\ili{} analysis\ili{} is\ili{} illustrated\ili{} in\ili{} Figures\ili{} \ili{}\ref\ili{}{fig\ili{}:mweiness\ili{}:particlecons\ili{}-c}\ili{} and\ili{} \ili{}\ref\ili{}{fig\ili{}:mweiness\ili{}:particlecons\ili{}-f}\ili{} for\ili{} the\ili{} sentence\ili{} in\ili{} \ili{}(\ili{}\ref\ili{}{ex\ili{}:mweiness\ili{}:skriveopp}\ili{})\ili{}.\ili{}
\ili{}
\ili{}\ea\ili{}\label\ili{}{ex\ili{}:mweiness\ili{}:skriveopp}\ili{}
\ili{}\gll\ili{} Han\ili{} \ili{}\textbf\ili{}{skrev}\ili{} \ili{}\textbf\ili{}{opp}\ili{} nummeret\ili{}.\ili{} \ili{}\\ili{}\\ili{}
\ili{} \ili{} \ili{} \ili{} \ili{} he\ili{} wrote\ili{} up\ili{} \ili{}{the\ili{} number}\ili{}\\ili{}\\ili{}
\ili{}\glt\ili{} \ili{}`He\ili{} wrote\ili{} down\ili{} the\ili{} number\ili{}.\ili{}'\ili{}
\ili{}\z\ili{}
\ili{}
\ili{}%\ili{}\ea\ili{}\label\ili{}{ex\ili{}:mweiness\ili{}:skriveopp}\ili{}
\ili{}%\ili{}\gll\ili{} Han\ili{} \ili{}\textbf\ili{}{skrev}\ili{} \ili{}\textbf\ili{}{opp}\ili{} nummeret\ili{}.\ili{} \ili{}\\ili{}\\ili{}
\ili{}%\ili{} \ili{} \ili{} \ili{} \ili{} he\ili{} wrote\ili{} up\ili{} number\ili{}.\ili{}\textsc\ili{}{def}\ili{}.\ili{}\textsc\ili{}{sg}\ili{}\\ili{}\\ili{}
\ili{}%\ili{}\glt\ili{} \ili{}`He\ili{} wrote\ili{} up\ili{} the\ili{} number\ili{}.\ili{}'\ili{}
\ili{}%\ili{}\z\ili{}
\ili{}
\ili{}\begin\ili{}{figure}\ili{}
\ili{} \ili{} \ili{}\includegraphics\ili{}[width\ili{}=0\ili{}.27\ili{}\textwidth\ili{}]\ili{}{figures\ili{}/particlecons\ili{}-c\ili{}.png}\ili{}
\ili{} \ili{} \ili{}\caption\ili{}{C\ili{}-structure\ili{} for\ili{} example\ili{} \ili{}(\ili{}\ref\ili{}{ex\ili{}:mweiness\ili{}:skriveopp}\ili{})}\ili{}
\ili{} \ili{} \ili{}\label\ili{}{fig\ili{}:mweiness\ili{}:particlecons\ili{}-c}\ili{}
\ili{}\end\ili{}{figure}\ili{}
\ili{}
\ili{}\begin\ili{}{figure}\ili{}
\ili{} \ili{} \ili{}\includegraphics\ili{}[width\ili{}=0\ili{}.75\ili{}\textwidth\ili{}]\ili{}{figures\ili{}/particlecons\ili{}-f\ili{}.png}\ili{}
\ili{} \ili{} \ili{}\caption\ili{}{F\ili{}-structure\ili{} for\ili{} example\ili{} \ili{}(\ili{}\ref\ili{}{ex\ili{}:mweiness\ili{}:skriveopp}\ili{})}\ili{}
\ili{} \ili{} \ili{}\label\ili{}{fig\ili{}:mweiness\ili{}:particlecons\ili{}-f}\ili{}
\ili{}\end\ili{}{figure}\ili{}
\ili{}
\ili{}%\ili{}\begin\ili{}{figure}\ili{}
\ili{}%\ili{} \ili{} \ili{}\includegraphics\ili{}[width\ili{}=\ili{}\textwidth\ili{}]\ili{}{figures\ili{}/particlecons\ili{}.png}\ili{}
\ili{}%\ili{} \ili{} \ili{}\caption\ili{}{C\ili{}-\ili{} and\ili{} f\ili{}-structure\ili{} for\ili{} example\ili{} \ili{}(\ili{}\ref\ili{}{ex\ili{}:mweiness\ili{}:skriveopp}\ili{})}\ili{}
\ili{}%\ili{} \ili{} \ili{}\label\ili{}{fig\ili{}:mweiness\ili{}:particlecons}\ili{}
\ili{}%\ili{}\end\ili{}{figure}\ili{}
\ili{}
In\ili{} the\ili{} c\ili{}-structure\ili{} the\ili{} particle\ili{} PRT\ili{} is\ili{} a\ili{} separate\ili{} constituent\ili{} which\ili{} may\ili{} also\ili{} occur\ili{} after\ili{} the\ili{} NP\ili{} under\ili{} VPmain\ili{}.\ili{}
In\ili{} the\ili{} f\ili{}-structure\ili{} the\ili{} verb\ili{} and\ili{} the\ili{} particle\ili{} are\ili{} analyzed\ili{} as\ili{} forming\ili{} one\ili{} predicate\ili{} \ili{}`skrive\ili{}*opp\ili{}'\ili{},\ili{} and\ili{} the\ili{} particle\ili{} also\ili{} provides\ili{} a\ili{} value\ili{} to\ili{} the\ili{} feature\ili{} PRT\ili{}-FORM\ili{}.\ili{}
\ili{}
As\ili{} in\ili{} the\ili{} case\ili{} of\ili{} selected\ili{} prepositions\ili{},\ili{} the\ili{} \ili{}\isi\ili{}{lexical\ili{} entry}\ili{} for\ili{} \ili{}\textit\ili{}{skrive}\ili{} is\ili{} associated\ili{} with\ili{} the\ili{} relevant\ili{} frame\ili{} through\ili{} an\ili{} invocation\ili{} of\ili{} the\ili{} \ili{}\isi\ili{}{template}\ili{} describing\ili{} this\ili{} class\ili{} of\ili{} constructions\ili{}.\ili{}
Part\ili{} of\ili{} the\ili{} \ili{}\isi\ili{}{lexical\ili{} entry}\ili{} for\ili{} \ili{}\textit\ili{}{skrive}\ili{} is\ili{} shown\ili{} in\ili{} \ili{} \ili{}(\ili{}\ref\ili{}{ex\ili{}:mweiness\ili{}:skrive\ili{}-lex}\ili{})\ili{}.\ili{}
\ili{}
\ili{}\eabox\ili{}{\ili{}\label\ili{}{ex\ili{}:mweiness\ili{}:skrive\ili{}-lex}\ili{}
\ili{}{\ili{}\small\ili{} \ili{}
\ili{}\begin\ili{}{tabular}\ili{}{ll}\ili{}
skrive\ili{} V\ili{} \ili{}&\ili{} \ili{}\\ili{}{\ili{} \ili{}\enspace\ili{} \ili{}[\ili{} \ili{}.\ili{}.\ili{}.\ili{} \ili{}]\ili{}\\ili{}\\ili{}
\ili{}&\ili{} \ili{}|\ili{} \ili{}\enspace\ili{} \ili{}@\ili{}(V\ili{}-SUBJ\ili{}-PRT\ili{}-OBJ\ili{} skrive\ili{} opp\ili{})\ili{}\\ili{}\\ili{}
\ili{}&\ili{} \ili{}|\ili{} \ili{}\enspace\ili{} \ili{}[\ili{} \ili{}.\ili{}.\ili{}.\ili{} \ili{}]\ili{} \ili{}\enspace\ili{} \ili{}\}\ili{}
\ili{}\end\ili{}{tabular}}}\ili{}
\ili{}
Part\ili{} of\ili{} the\ili{} invoked\ili{} \ili{}\isi\ili{}{template}\ili{} V\ili{}-SUBJ\ili{}-PRT\ili{}-OBJ\ili{} is\ili{} shown\ili{} in\ili{} \ili{} \ili{}(\ili{}\ref\ili{}{ex\ili{}:mweiness\ili{}:skrive\ili{}-template}\ili{})\ili{}.\ili{}
\ili{}
\ili{}\ea\ili{}\label\ili{}{ex\ili{}:mweiness\ili{}:skrive\ili{}-template}\ili{}
\ili{}{\ili{}\small\ili{} \ili{}
V\ili{}-SUBJ\ili{}-PRT\ili{}-OBJ\ili{} \ili{}(P\ili{} prt\ili{})\ili{} \ili{}=\ili{}\\ili{}\\ili{}
\ili{}\hspace\ili{}{2em}\ili{} \ili{}@\ili{}(CONCAT\ili{} P\ili{} \ili{}`\ili{}*\ili{} prt\ili{} \ili{}\\ili{}%FN\ili{})\ili{}\\ili{}\\ili{}
\ili{}\hspace\ili{}{2em}\ili{} \ili{} \ili{}(\ili{}$\ili{}\uparrow\ili{}$\ili{} PRED\ili{})\ili{}=\ili{}`\ili{}\\ili{}%FN\ili{}<\ili{}(\ili{}$\ili{}\uparrow\ili{}$\ili{} SUBJ\ili{})\ili{}(\ili{}$\ili{}\uparrow\ili{}$\ili{} OBJ\ili{})\ili{}>\ili{}'\ili{}\\ili{}\\ili{}
\ili{}\hspace\ili{}{2em}\ili{} \ili{} \ili{}(\ili{}$\ili{}\uparrow\ili{}$\ili{} CHECK\ili{} PRT\ili{}-VERB\ili{})\ili{}=\ili{}+\ili{}\\ili{}\\ili{}
\ili{}\hspace\ili{}{2em}\ili{} \ili{} \ili{}(\ili{}$\ili{}\uparrow\ili{}$\ili{} PRT\ili{}-FORM\ili{})\ili{}=c\ili{} prt\ili{}
}\ili{}
\ili{}\z\ili{}
\ili{}
The\ili{} CONCAT\ili{} \ili{}\isi\ili{}{template}\ili{} functions\ili{} as\ili{} in\ili{} the\ili{} \ili{}\isi\ili{}{template}\ili{} \ili{}(\ili{}\ref\ili{}{ex\ili{}:mweiness\ili{}:tenke\ili{}-template}\ili{})\ili{},\ili{} yielding\ili{} the\ili{} \ili{} predicate\ili{} name\ili{} \ili{}`skrive\ili{}*opp\ili{}'\ili{} as\ili{} the\ili{} value\ili{} of\ili{} PRED\ili{}.\ili{}
The\ili{} second\ili{} last\ili{} line\ili{} assigns\ili{} the\ili{} value\ili{} \ili{}`\ili{}`\ili{}+\ili{}'\ili{}'\ili{} to\ili{} the\ili{} path\ili{} CHECK\ili{} PRT\ili{}-VERB\ili{},\ili{} a\ili{} feature\ili{} which\ili{} is\ili{} checked\ili{} by\ili{} the\ili{} syntactic\ili{} \ili{}\isi\ili{}{rule}\ili{} introducing\ili{} the\ili{} particle\ili{} PRT\ili{};\ili{} see\ili{} the\ili{} VPmain\ili{} \ili{}\isi\ili{}{rule}\ili{} in\ili{} \ili{}(\ili{}\ref\ili{}{ex\ili{}:mweiness\ili{}:rule3}\ili{})\ili{} below\ili{}.\ili{}
The\ili{} last\ili{} line\ili{} is\ili{} a\ili{} constraining\ili{} equation\ili{}\footnote\ili{}{This\ili{} concept\ili{} is\ili{} explained\ili{} in\ili{} connection\ili{} with\ili{} example\ili{} \ili{}(\ili{}\ref\ili{}{ex\ili{}:mweiness\ili{}:defrestriction}\ili{})\ili{} above\ili{}.}\ili{} which\ili{} checks\ili{} that\ili{} the\ili{} value\ili{} of\ili{} the\ili{} feature\ili{} PRT\ili{}-FORM\ili{},\ili{} which\ili{} is\ili{} introduced\ili{} in\ili{} the\ili{} sentence\ili{} by\ili{} the\ili{} particle\ili{},\ili{} is\ili{} the\ili{} value\ili{} of\ili{} prt\ili{},\ili{} i\ili{}.e\ili{}.\ili{} \ili{}\emph\ili{}{opp}\ili{} in\ili{} the\ili{} \ili{}\isi\ili{}{template}\ili{} invocation\ili{} in\ili{} \ili{}(\ili{}\ref\ili{}{ex\ili{}:mweiness\ili{}:skrive\ili{}-lex}\ili{})\ili{}.\ili{}
\ili{}
\ili{}\subsubsection\ili{}{Verb\ili{}-particle\ili{} constructions\ili{} with\ili{} selected\ili{} prepositions}\ili{}\label\ili{}{sec\ili{}:mweiness\ili{}:prtprepverbs}\ili{}
\ili{}
The\ili{} preceding\ili{} sections\ili{} have\ili{} shown\ili{} how\ili{} \ili{}\isi\ili{}{prepositional\ili{} verbs}\ili{} and\ili{} \ili{}\isi\ili{}{verb\ili{}-particle\ili{} constructions}\ili{} are\ili{} analyzed\ili{}.\ili{}
Phrasal\ili{} verbs\ili{} also\ili{} allow\ili{} both\ili{} selected\ili{} prepositions\ili{} and\ili{} particles\ili{} in\ili{} the\ili{} same\ili{} MWE\ili{}.\ili{}
An\ili{} example\ili{} involving\ili{} both\ili{},\ili{} in\ili{} addition\ili{} to\ili{} a\ili{} reflexive\ili{} object\ili{},\ili{} is\ili{} provided\ili{} in\ili{} the\ili{} treebank\ili{} example\ili{} in\ili{} \ili{}(\ili{}\ref\ili{}{ex\ili{}:mweiness\ili{}:sette\ili{}-seg\ili{}-inn\ili{}-i}\ili{})\ili{}.\ili{}
The\ili{} analysis\ili{} of\ili{} \ili{}(\ili{}\ref\ili{}{ex\ili{}:mweiness\ili{}:sette\ili{}-seg\ili{}-inn\ili{}-i}\ili{})\ili{} is\ili{} shown\ili{} in\ili{} Figures\ili{} \ili{}\ref\ili{}{fig\ili{}:mweiness\ili{}:sette\ili{}-seg\ili{}-inn\ili{}-i\ili{}-c}\ili{} and\ili{} \ili{}\ref\ili{}{fig\ili{}:mweiness\ili{}:sette\ili{}-seg\ili{}-inn\ili{}-i\ili{}-f}\ili{}.\ili{}
\ili{}
\ili{}\ea\ili{}\label\ili{}{ex\ili{}:mweiness\ili{}:sette\ili{}-seg\ili{}-inn\ili{}-i}\ili{}
\ili{}\gll\ili{} Vi\ili{} har\ili{} et\ili{} så\ili{} enormt\ili{} stort\ili{} område\ili{} å\ili{} \ili{}\textbf\ili{}{sette}\ili{} \ili{}\textbf\ili{}{oss}\ili{} \ili{}\textbf\ili{}{inn}\ili{} \ili{}\textbf\ili{}{i}\ili{}.\ili{}\\ili{}\\ili{}
\ili{} \ili{} \ili{} \ili{} \ili{} we\ili{} have\ili{} a\ili{} such\ili{} enormously\ili{} large\ili{} area\ili{} to\ili{} set\ili{} us\ili{} in\ili{} into\ili{}\\ili{}\\ili{}
\ili{}\glt\ili{} \ili{}`We\ili{} have\ili{} such\ili{} an\ili{} enormously\ili{} large\ili{} area\ili{} to\ili{} immerse\ili{} ourselves\ili{} in\ili{}.\ili{}'\ili{}
\ili{}\z\ili{}
\ili{}
\ili{}\begin\ili{}{figure}\ili{}
\ili{} \ili{} \ili{}\includegraphics\ili{}[height\ili{}=\ili{}.47\ili{}\textheight\ili{}]\ili{}{figures\ili{}/sette\ili{}-seg\ili{}-inn\ili{}-i\ili{}-c\ili{}.png}\ili{}
\ili{} \ili{} \ili{}\caption\ili{}{The\ili{} c\ili{}-structure\ili{} of\ili{} sentence\ili{} \ili{}(\ili{}\ref\ili{}{ex\ili{}:mweiness\ili{}:sette\ili{}-seg\ili{}-inn\ili{}-i}\ili{})}\ili{}
\ili{} \ili{} \ili{}\label\ili{}{fig\ili{}:mweiness\ili{}:sette\ili{}-seg\ili{}-inn\ili{}-i\ili{}-c}\ili{}
\ili{}\end\ili{}{figure}\ili{}
\ili{}
\ili{}\begin\ili{}{figure}\ili{}
\ili{} \ili{} \ili{}\includegraphics\ili{}[width\ili{}=0\ili{}.9\ili{}\textwidth\ili{}]\ili{}{figures\ili{}/sette\ili{}-seg\ili{}-inn\ili{}-i\ili{}-f\ili{}.png}\ili{}
\ili{} \ili{} \ili{}\caption\ili{}{Part\ili{} of\ili{} the\ili{} simplified\ili{} f\ili{}-structure\ili{} of\ili{} sentence\ili{} \ili{}(\ili{}\ref\ili{}{ex\ili{}:mweiness\ili{}:sette\ili{}-seg\ili{}-inn\ili{}-i}\ili{})}\ili{}
\ili{} \ili{} \ili{}\label\ili{}{fig\ili{}:mweiness\ili{}:sette\ili{}-seg\ili{}-inn\ili{}-i\ili{}-f}\ili{}
\ili{}\end\ili{}{figure}\ili{}
\ili{}
In\ili{} this\ili{} example\ili{} the\ili{} MWE\ili{} \ili{} \ili{}\textit\ili{}{sette\ili{} oss\ili{} inn\ili{} i}\ili{} occurs\ili{} in\ili{} an\ili{} infinitival\ili{} relative\ili{} \ili{}(CPinf\ili{})\ili{} in\ili{} an\ili{} NP\ili{} with\ili{} the\ili{} head\ili{} \ili{}\textit\ili{}{område}\ili{} \ili{}`area\ili{}'\ili{}.\ili{}
In\ili{} the\ili{} f\ili{}-structure\ili{} the\ili{} infinitival\ili{} relative\ili{} occurs\ili{} as\ili{} a\ili{} member\ili{} of\ili{} the\ili{} set\ili{} of\ili{} adjuncts\ili{} to\ili{} the\ili{} predicate\ili{} \ili{}`område\ili{}'\ili{},\ili{} also\ili{} occurring\ili{} as\ili{} the\ili{} second\ili{} argument\ili{} of\ili{} \ili{}`sette\ili{}*seg\ili{}*inn\ili{}*i\ili{}'\ili{} as\ili{} its\ili{} relativized\ili{} argument\ili{} \ili{}(see\ili{} Section\ili{} \ili{}\ref\ili{}{sec\ili{}:mweiness\ili{}:longdist}\ili{} for\ili{} the\ili{} analysis\ili{} of\ili{} \ili{}\isi\ili{}{long\ili{}-distance\ili{} dependencies}\ili{} like\ili{} \ili{}\isi\ili{}{relativization}\ili{} and\ili{} topicalization\ili{})\ili{}.\ili{}
The\ili{} \ili{}\isi\ili{}{template}\ili{} invoked\ili{} by\ili{} the\ili{} verb\ili{} \ili{}\textit\ili{}{sette}\ili{},\ili{} V\ili{}-SUBJ\ili{}-OBJrefl\ili{}-PRT\ili{}-POBJ\ili{},\ili{} provides\ili{} an\ili{} analysis\ili{} along\ili{} the\ili{} lines\ili{} of\ili{} the\ili{} templates\ili{} in\ili{} \ili{}(\ili{}\ref\ili{}{ex\ili{}:mweiness\ili{}:tenke\ili{}-template}\ili{})\ili{} and\ili{} \ili{}(\ili{}\ref\ili{}{ex\ili{}:mweiness\ili{}:skrive\ili{}-template}\ili{})\ili{}.\ili{}
\ili{}
\ili{}\subsection\ili{}{Verbal\ili{} idioms}\ili{}\label\ili{}{sec\ili{}:mweiness\ili{}:verbalidioms}\ili{}
\ili{}
A\ili{} VP\ili{} idiom\ili{} is\ili{} a\ili{} flexible\ili{} MWE\ili{} in\ili{} which\ili{} at\ili{} least\ili{} one\ili{} predicate\ili{}-bearing\ili{} lexeme\ili{} \ili{}(such\ili{} as\ili{} a\ili{} noun\ili{} or\ili{} an\ili{} adjective\ili{})\ili{} is\ili{} selected\ili{},\ili{} with\ili{} possible\ili{} restrictions\ili{} as\ili{} to\ili{} number\ili{},\ili{} definiteness\ili{} or\ili{} other\ili{} morphological\ili{} properties\ili{} applying\ili{}.\ili{}
VP\ili{} idioms\ili{} are\ili{} handled\ili{} by\ili{} a\ili{} specific\ili{} set\ili{} of\ili{} templates\ili{}.\ili{}
For\ili{} example\ili{},\ili{} an\ili{} idiom\ili{} like\ili{} \ili{}\textit\ili{}{holde\ili{} øye\ili{} med}\ili{} \ili{}`keep\ili{} an\ili{} eye\ili{} on\ili{}'\ili{} is\ili{} analyzed\ili{} by\ili{} means\ili{} of\ili{} a\ili{} lexical\ili{} \ili{}\isi\ili{}{template}\ili{} covering\ili{} idioms\ili{} consisting\ili{} of\ili{} a\ili{} selected\ili{} indefinite\ili{} object\ili{} plus\ili{} a\ili{} selected\ili{} prepositional\ili{} \ili{}\isi\ili{}{phrase}\ili{}.\ili{}
The\ili{} treebank\ili{} sentence\ili{} in\ili{} \ili{}(\ili{}\ref\ili{}{ex\ili{}:mweiness\ili{}:øyemed}\ili{})\ili{} is\ili{} analyzed\ili{} as\ili{} shown\ili{} in\ili{} Figures\ili{} \ili{}\ref\ili{}{fig\ili{}:mweiness\ili{}:øyemed\ili{}-c}\ili{} and\ili{} \ili{}\ref\ili{}{fig\ili{}:mweiness\ili{}:øyemed\ili{}-f}\ili{}.\ili{}
\ili{}
\ili{}\ea\ili{}\label\ili{}{ex\ili{}:mweiness\ili{}:øyemed}\ili{}
\ili{}\gll\ili{} Samtidig\ili{} \ili{}\textbf\ili{}{holdt}\ili{} han\ili{} umerkelig\ili{} et\ili{} skarpt\ili{} \ili{}\textbf\ili{}{øye}\ili{} \ili{}\textbf\ili{}{med}\ili{} jentungen\ili{}.\ili{} \ili{}\\ili{}\\ili{}
\ili{} \ili{} \ili{} \ili{} \ili{} simultaneously\ili{} kept\ili{} he\ili{} unnoticeably\ili{} a\ili{} sharp\ili{} eye\ili{} with\ili{} \ili{}{the\ili{} girl\ili{} child}\ili{}\\ili{}\\ili{}
\ili{}\glt\ili{} \ili{}`At\ili{} the\ili{} same\ili{} time\ili{} he\ili{} furtively\ili{} kept\ili{} a\ili{} close\ili{} eye\ili{} on\ili{} the\ili{} girl\ili{}.\ili{}'\ili{}
\ili{}\z\ili{}
\ili{}
\ili{}%\ili{}\ea\ili{}\label\ili{}{ex\ili{}:mweiness\ili{}:øyemed}\ili{}
\ili{}%\ili{}\gll\ili{} Samtidig\ili{} \ili{}\textbf\ili{}{holdt}\ili{} han\ili{} umerkelig\ili{} et\ili{} skarpt\ili{} \ili{}\textbf\ili{}{øye}\ili{} \ili{}\textbf\ili{}{med}\ili{} jentungen\ili{}.\ili{} \ili{}\\ili{}\\ili{}
\ili{}%\ili{} \ili{} \ili{} \ili{} \ili{} simultaneously\ili{} kept\ili{} he\ili{} unnoticably\ili{} a\ili{} sharp\ili{} eye\ili{} with\ili{} girl\ili{}-child\ili{}.\ili{}\textsc\ili{}{def}\ili{}.\ili{}\textsc\ili{}{sg}\ili{}\\ili{}\\ili{}
\ili{}%\ili{}\glt\ili{} \ili{}`At\ili{} the\ili{} same\ili{} time\ili{} he\ili{} furtively\ili{} kept\ili{} a\ili{} sharp\ili{} eye\ili{} on\ili{} the\ili{} girl\ili{}'\ili{}
\ili{}%\ili{}\z\ili{}
\ili{}
\ili{}\begin\ili{}{figure}\ili{}
\ili{} \ili{} \ili{}\includegraphics\ili{}[height\ili{}=\ili{}.4\ili{}\textheight\ili{}]\ili{}{figures\ili{}/øyemed\ili{}-c\ili{}.png}\ili{}
\ili{} \ili{} \ili{}\caption\ili{}{The\ili{} c\ili{}-structure\ili{} of\ili{} sentence\ili{} \ili{}(\ili{}\ref\ili{}{ex\ili{}:mweiness\ili{}:øyemed}\ili{})\ili{} }\ili{}
\ili{} \ili{} \ili{}\label\ili{}{fig\ili{}:mweiness\ili{}:øyemed\ili{}-c}\ili{}
\ili{}\end\ili{}{figure}\ili{}
\ili{}
\ili{}\begin\ili{}{figure}\ili{}
\ili{} \ili{} \ili{}\includegraphics\ili{}[height\ili{}=\ili{}.25\ili{}\textheight\ili{}]\ili{}{figures\ili{}/øyemed\ili{}-f\ili{}.png}\ili{}
\ili{} \ili{} \ili{}\caption\ili{}{The\ili{} simplified\ili{} f\ili{}-structure\ili{} of\ili{} sentence\ili{} \ili{}(\ili{}\ref\ili{}{ex\ili{}:mweiness\ili{}:øyemed}\ili{})\ili{} }\ili{}
\ili{} \ili{} \ili{}\label\ili{}{fig\ili{}:mweiness\ili{}:øyemed\ili{}-f}\ili{}
\ili{}\end\ili{}{figure}\ili{}
\ili{}
The\ili{} analysis\ili{} of\ili{} the\ili{} selected\ili{} prepositional\ili{} \ili{}\isi\ili{}{phrase}\ili{} \ili{}\textit\ili{}{med\ili{} jentungen}\ili{} is\ili{} as\ili{} described\ili{} in\ili{} subsection\ili{} \ili{}\ref\ili{}{sec\ili{}:mweiness\ili{}:prepverbs}\ili{} for\ili{} example\ili{} \ili{}(\ili{}\ref\ili{}{ex\ili{}:mweiness\ili{}:tenkepå}\ili{})\ili{}.\ili{}
The\ili{} selected\ili{} lexeme\ili{} in\ili{} \ili{}(\ili{}\ref\ili{}{ex\ili{}:mweiness\ili{}:øyemed}\ili{})\ili{} is\ili{} \ili{}\textit\ili{}{øye}\ili{}.\ili{}
In\ili{} the\ili{} f\ili{}-structure\ili{} in\ili{} Figure\ili{} \ili{}\ref\ili{}{fig\ili{}:mweiness\ili{}:øyemed\ili{}-f}\ili{} the\ili{} idiomatic\ili{} meaning\ili{} is\ili{} represented\ili{} by\ili{} incorporating\ili{} \ili{}`øye\ili{}'\ili{} in\ili{} the\ili{} predicate\ili{} name\ili{},\ili{} deriving\ili{} the\ili{} predicate\ili{} name\ili{} \ili{}`holde\ili{}\\ili{}#øye\ili{}*med\ili{}'\ili{}.\ili{} \ili{}%\ili{},\ili{} where\ili{} we\ili{} use\ili{} the\ili{} symbol\ili{} \ili{}`\ili{}`\ili{}\\ili{}#\ili{}'\ili{}'\ili{} to\ili{} signal\ili{} an\ili{} idiomatic\ili{} combination\ili{} of\ili{} this\ili{} kind\ili{}.\ili{}
The\ili{} \ili{}\isi\ili{}{phrase}\ili{} \ili{}\textit\ili{}{et\ili{} skarpt\ili{} øye}\ili{} fills\ili{} the\ili{} function\ili{} of\ili{} OBJ\ili{},\ili{} but\ili{} is\ili{} not\ili{} analyzed\ili{} as\ili{} a\ili{} semantic\ili{} argument\ili{} of\ili{} the\ili{} sentence\ili{} predicate\ili{},\ili{} which\ili{} appears\ili{} from\ili{} its\ili{} position\ili{} outside\ili{} the\ili{} angled\ili{} brackets\ili{} \ili{}<\ili{}.\ili{}.\ili{}.\ili{}>\ili{} surrounding\ili{} the\ili{} argument\ili{} list\ili{}.\ili{}
This\ili{} position\ili{} signals\ili{} that\ili{} the\ili{} constituent\ili{} is\ili{} syntactically\ili{} subcategorized\ili{} for\ili{} without\ili{} being\ili{} a\ili{} semantic\ili{} argument\ili{}.\ili{}
\ili{}
The\ili{} \ili{}\isi\ili{}{lexical\ili{} entry}\ili{} for\ili{} \ili{}\textit\ili{}{holde}\ili{} is\ili{} associated\ili{} with\ili{} the\ili{} VP\ili{} idiom\ili{} through\ili{} an\ili{} invocation\ili{} of\ili{} the\ili{} idiom\ili{} \ili{}\isi\ili{}{template}\ili{} describing\ili{} the\ili{} relevant\ili{} class\ili{} of\ili{} idioms\ili{}.\ili{}
Part\ili{} of\ili{} the\ili{} \ili{}\isi\ili{}{lexical\ili{} entry}\ili{} for\ili{} \ili{}\textit\ili{}{holde}\ili{} is\ili{} shown\ili{} in\ili{} \ili{} \ili{}(\ili{}\ref\ili{}{ex\ili{}:mweiness\ili{}:holde\ili{}-lex}\ili{})\ili{}.\ili{}
The\ili{} \ili{}\isi\ili{}{template}\ili{} invocation\ili{} has\ili{} three\ili{} parameters\ili{},\ili{} the\ili{} predicate\ili{} name\ili{} for\ili{} the\ili{} verb\ili{} \ili{}\textit\ili{}{holde}\ili{},\ili{} the\ili{} predicate\ili{} of\ili{} the\ili{} selected\ili{} noun\ili{} \ili{}\textit\ili{}{øye}\ili{},\ili{} \ili{} and\ili{} the\ili{} form\ili{} of\ili{} the\ili{} selected\ili{} preposition\ili{} \ili{}\textit\ili{}{med}\ili{}.\ili{}
Part\ili{} of\ili{} the\ili{} \ili{}\isi\ili{}{template}\ili{} is\ili{} shown\ili{} in\ili{} \ili{} \ili{}(\ili{}\ref\ili{}{ex\ili{}:mweiness\ili{}:holde\ili{}-template}\ili{})\ili{};\ili{} the\ili{} full\ili{} \ili{}\isi\ili{}{template}\ili{} is\ili{} discussed\ili{} in\ili{} Section\ili{} \ili{}\ref\ili{}{sec\ili{}:mweiness\ili{}:vpidiomsyntax}\ili{}.\ili{}
\ili{}
\ili{}\eabox\ili{}{\ili{}\label\ili{}{ex\ili{}:mweiness\ili{}:holde\ili{}-lex}\ili{}
\ili{}{\ili{}\small\ili{} \ili{}
\ili{}\begin\ili{}{tabular}\ili{}{ll}\ili{}
holde\ili{} V\ili{} \ili{}&\ili{} \ili{}\\ili{}{\ili{} \ili{}\enspace\ili{} \ili{}[\ili{} \ili{}.\ili{}.\ili{}.\ili{} \ili{}]\ili{}\\ili{}\\ili{}
\ili{}&\ili{} \ili{}|\ili{} \ili{}\enspace\ili{} \ili{}@\ili{}(VPIDIOM\ili{}-INDEFOBJ\ili{}-POBJ\ili{} holde\ili{} øye\ili{} med\ili{})\ili{}\\ili{}\\ili{}
\ili{}&\ili{} \ili{}\quad\ili{} \ili{}(\ili{}$\ili{}\uparrow\ili{}$\ili{} OBJ\ili{} NUM\ili{})\ili{}=c\ili{} sg\ili{}\\ili{}\\ili{}
\ili{}&\ili{} \ili{}|\ili{} \ili{}\enspace\ili{} \ili{}[\ili{} \ili{}.\ili{}.\ili{}.\ili{} \ili{}]\ili{} \ili{}\enspace\ili{} \ili{}\}\ili{}\\ili{}\\ili{}
\ili{}\end\ili{}{tabular}}}\ili{}
\ili{}
\ili{}\ea\ili{}\label\ili{}{ex\ili{}:mweiness\ili{}:holde\ili{}-template}\ili{}
\ili{}{\ili{}\small\ili{} \ili{}
\ili{} VPIDIOM\ili{}-INDEFOBJ\ili{}-POBJ\ili{} \ili{}(P\ili{} OP\ili{} prp\ili{})\ili{} \ili{}=\ili{}\\ili{}\\ili{}
\ili{}\hspace\ili{}{1\ili{}.5em}\ili{} \ili{}@\ili{}(CONCAT\ili{} P\ili{} \ili{}`\ili{}\\ili{}#\ili{} OP\ili{} \ili{}`\ili{}*\ili{} prp\ili{} \ili{}\\ili{}%FN\ili{})\ili{}\\ili{}\\ili{}
\ili{}\hspace\ili{}{1\ili{}.5em}\ili{} \ili{} \ili{}(\ili{}$\ili{}\uparrow\ili{}$\ili{} \ili{} PRED\ili{})\ili{}=\ili{}`\ili{}\\ili{}%FN\ili{}<\ili{}(\ili{}$\ili{}\uparrow\ili{}$\ili{} SUBJ\ili{})\ili{}(\ili{}$\ili{}\uparrow\ili{}$\ili{} OBL\ili{}-TH\ili{})\ili{}>\ili{}(\ili{}$\ili{}\uparrow\ili{}$\ili{} OBJ\ili{})\ili{}\\ili{}\\ili{}
\ili{}\hspace\ili{}{1\ili{}.5em}\ili{} \ili{}(\ili{}$\ili{}\uparrow\ili{}$\ili{} OBL\ili{}-TH\ili{} CHECK\ili{} P\ili{}-SELFORM\ili{})\ili{}=prp\ili{}\\ili{}\\ili{}
\ili{}\hspace\ili{}{1\ili{}.5em}\ili{} \ili{}(\ili{}$\ili{}\uparrow\ili{}$\ili{} OBJ\ili{} PRED\ili{} FN\ili{})\ili{}=c\ili{} OP\ili{}\\ili{}\\ili{}
\ili{}\hspace\ili{}{1\ili{}.5em}\ili{} \ili{}{\ili{}\textasciitilde}\ili{}(\ili{}$\ili{}\uparrow\ili{}$\ili{} OBJ\ili{} DEF\ili{})\ili{}=\ili{}+\ili{}
}\ili{}
\ili{}\z\ili{}
\ili{}
In\ili{} addition\ili{} to\ili{} the\ili{} \ili{}\isi\ili{}{template}\ili{} call\ili{},\ili{} the\ili{} \ili{}\isi\ili{}{lexical\ili{} entry}\ili{} in\ili{} \ili{} \ili{} \ili{}(\ili{}\ref\ili{}{ex\ili{}:mweiness\ili{}:holde\ili{}-lex}\ili{})\ili{} also\ili{} specifies\ili{} that\ili{} the\ili{} selected\ili{} object\ili{} should\ili{} be\ili{} singular\ili{}.\ili{}
It\ili{} is\ili{} a\ili{} matter\ili{} of\ili{} choice\ili{} whether\ili{} such\ili{} information\ili{} should\ili{} be\ili{} included\ili{} in\ili{} the\ili{} individual\ili{} \ili{}\isi\ili{}{lexical\ili{} entry}\ili{} or\ili{} give\ili{} rise\ili{} to\ili{} a\ili{} distinction\ili{} between\ili{} more\ili{} fine\ili{}-grained\ili{} templates\ili{}.\ili{}
In\ili{} the\ili{} \ili{}\isi\ili{}{template}\ili{} definition\ili{} in\ili{} \ili{}(\ili{}\ref\ili{}{ex\ili{}:mweiness\ili{}:holde\ili{}-template}\ili{})\ili{},\ili{} OP\ili{} \ili{}(object\ili{} predicate\ili{})\ili{} is\ili{} the\ili{} variable\ili{} for\ili{} the\ili{} selected\ili{} noun\ili{} predicate\ili{} and\ili{} prp\ili{} for\ili{} the\ili{} selected\ili{} preposition\ili{}.\ili{}
As\ili{} in\ili{} the\ili{} case\ili{} of\ili{} the\ili{} \ili{}\isi\ili{}{template}\ili{} in\ili{} \ili{}(\ili{}\ref\ili{}{ex\ili{}:mweiness\ili{}:skrive\ili{}-template}\ili{})\ili{},\ili{} the\ili{} \ili{}\isi\ili{}{template}\ili{} invokes\ili{} the\ili{} CONCAT\ili{} \ili{}\isi\ili{}{template}\ili{} which\ili{} builds\ili{} the\ili{} predicate\ili{} name\ili{}.\ili{}
The\ili{} second\ili{} last\ili{} equation\ili{} requires\ili{} the\ili{} object\ili{} to\ili{} have\ili{} the\ili{} value\ili{} of\ili{} OP\ili{} as\ili{} its\ili{} predicate\ili{} \ili{}(in\ili{} this\ili{} case\ili{} \ili{}`øye\ili{}'\ili{})\ili{},\ili{} and\ili{} the\ili{} final\ili{} equation\ili{} requires\ili{} the\ili{} object\ili{} not\ili{} to\ili{} be\ili{} definite\ili{}.\ili{}
As\ili{} for\ili{} the\ili{} equation\ili{} mentioning\ili{} P\ili{}-SELFORM\ili{},\ili{} see\ili{} the\ili{} explanation\ili{} of\ili{} the\ili{} \ili{}\isi\ili{}{template}\ili{} in\ili{} \ili{}(\ili{}\ref\ili{}{ex\ili{}:mweiness\ili{}:tenke\ili{}-template}\ili{})\ili{}.\ili{}
\ili{}
\ili{}\subsection\ili{}{Nonverbal\ili{} flexible\ili{} expressions}\ili{}\label\ili{}{sec\ili{}:mweiness\ili{}:nonverbal}\ili{}
\ili{}
\ili{}\subsubsection\ili{}{Nouns\ili{} with\ili{} selected\ili{} prepositions}\ili{}\label\ili{}{sec\ili{}:mweiness\ili{}:prepnoun}\ili{}
\ili{}
Nouns\ili{} may\ili{} also\ili{} form\ili{} MWEs\ili{} by\ili{} selecting\ili{} prepositional\ili{} phrases\ili{} as\ili{} their\ili{} arguments\ili{}.\ili{}
For\ili{} example\ili{},\ili{} the\ili{} noun\ili{} \ili{}\textit\ili{}{ansvar}\ili{} \ili{}`responsibility\ili{}'\ili{} may\ili{} select\ili{} the\ili{} preposition\ili{} \ili{}\textit\ili{}{for}\ili{} \ili{}`for\ili{}'\ili{},\ili{} which\ili{} can\ili{} take\ili{} a\ili{} nominal\ili{} \ili{}\isi\ili{}{phrase}\ili{},\ili{} an\ili{} infinitival\ili{},\ili{} or\ili{} a\ili{} nominal\ili{} subclause\ili{} as\ili{} complement\ili{},\ili{} as\ili{} in\ili{} the\ili{} treebank\ili{} examples\ili{} in\ili{} \ili{}(\ili{}\ref\ili{}{ex\ili{}:mweiness\ili{}:ansvarfor\ili{}-NP}\ili{})\ili{}–\ili{}(\ili{}\ref\ili{}{ex\ili{}:mweiness\ili{}:ansvarfor\ili{}-CP}\ili{})\ili{}.\ili{}
\ili{}
\ili{}\ea\ili{}\label\ili{}{ex\ili{}:mweiness\ili{}:ansvarfor\ili{}-NP}\ili{}
\ili{}\gll\ili{} Hadde\ili{} jeg\ili{} \ili{}\textbf\ili{}{ansvar}\ili{} \ili{}\textbf\ili{}{for}\ili{} gutten\ili{}?\ili{} \ili{}\\ili{}\\ili{}
\ili{} \ili{} \ili{} \ili{} \ili{} had\ili{} I\ili{} responsibility\ili{} for\ili{} \ili{}{the\ili{} boy}\ili{}\\ili{}\\ili{}
\ili{}\glt\ili{} \ili{}`Did\ili{} I\ili{} have\ili{} responsibility\ili{} for\ili{} the\ili{} boy\ili{}?\ili{}'\ili{}
\ili{}\z\ili{}
\ili{}
\ili{}%\ili{}\ea\ili{}\label\ili{}{ex\ili{}:mweiness\ili{}:ansvarfor\ili{}-NP}\ili{}
\ili{}%\ili{}\gll\ili{} Hadde\ili{} jeg\ili{} \ili{}\textbf\ili{}{ansvar}\ili{} \ili{}\textbf\ili{}{for}\ili{} gutten\ili{}?\ili{} \ili{}\\ili{}\\ili{}
\ili{}%\ili{} \ili{} \ili{} \ili{} \ili{} had\ili{} I\ili{} responsibility\ili{} for\ili{} boy\ili{}.\ili{}\textsc\ili{}{def}\ili{}.\ili{}\textsc\ili{}{sg}\ili{}\\ili{}\\ili{}
\ili{}%\ili{}\glt\ili{} \ili{}`Did\ili{} I\ili{} have\ili{} responsibility\ili{} for\ili{} the\ili{} boy\ili{}?\ili{}'\ili{}
\ili{}%\ili{}\z\ili{}
\ili{}
\ili{}\ea\ili{}\label\ili{}{ex\ili{}:mweiness\ili{}:ansvarfor\ili{}-inf}\ili{}
\ili{}\gll\ili{} Han\ili{} fikk\ili{} \ili{}\textbf\ili{}{ansvar}\ili{} \ili{}\textbf\ili{}{for}\ili{} å\ili{} overta\ili{} søket\ili{}.\ili{} \ili{}\\ili{}\\ili{}
\ili{} \ili{} \ili{} \ili{} \ili{} he\ili{} got\ili{} responsibility\ili{} for\ili{} to\ili{} take\ili{} over\ili{} \ili{}{the\ili{} search}\ili{}\\ili{}\\ili{}
\ili{}\glt\ili{} \ili{}`He\ili{} got\ili{} the\ili{} responsibility\ili{} for\ili{} taking\ili{} over\ili{} the\ili{} search\ili{}.\ili{}'\ili{}
\ili{}\z\ili{}
\ili{}
\ili{}%\ili{}\ea\ili{}\label\ili{}{ex\ili{}:mweiness\ili{}:ansvarfor\ili{}-inf}\ili{}
\ili{}%\ili{}\gll\ili{} Han\ili{} fikk\ili{} \ili{}\textbf\ili{}{ansvar}\ili{} \ili{}\textbf\ili{}{for}\ili{} å\ili{} overta\ili{} søket\ili{}.\ili{} \ili{}\\ili{}\\ili{}
\ili{}%\ili{} \ili{} \ili{} \ili{} \ili{} he\ili{} got\ili{} responsibility\ili{} for\ili{} to\ili{} take\ili{} over\ili{} search\ili{}.\ili{}\textsc\ili{}{def}\ili{}.\ili{}\textsc\ili{}{sg}\ili{}\\ili{}\\ili{}
\ili{}%\ili{}\glt\ili{} \ili{}`He\ili{} got\ili{} the\ili{} responsibility\ili{} for\ili{} taking\ili{} over\ili{} the\ili{} search\ili{}.\ili{}'\ili{}
\ili{}%\ili{}\z\ili{}
\ili{}
\ili{}\ea\ili{}\label\ili{}{ex\ili{}:mweiness\ili{}:ansvarfor\ili{}-CP}\ili{}
\ili{}\gll\ili{} Jeg\ili{} kan\ili{} ikke\ili{} ta\ili{} \ili{}\textbf\ili{}{ansvar}\ili{} \ili{}\textbf\ili{}{for}\ili{} at\ili{} det\ili{} ble\ili{} lekket\ili{}.\ili{} \ili{}\\ili{}\\ili{}
\ili{} \ili{} \ili{} \ili{} I\ili{} can\ili{} not\ili{} take\ili{} responsibility\ili{} for\ili{} that\ili{} it\ili{} became\ili{} leaked\ili{}\\ili{}\\ili{}
\ili{}\glt\ili{} \ili{}`I\ili{} cannot\ili{} take\ili{} responsibility\ili{} for\ili{} its\ili{} having\ili{} been\ili{} leaked\ili{}.\ili{}'\ili{}
\ili{}\z\ili{}
\ili{}
\ili{}\begin\ili{}{figure}\ili{}
\ili{} \ili{} \ili{}\includegraphics\ili{}[width\ili{}=\ili{}\textwidth\ili{}]\ili{}{figures\ili{}/ansvarfor\ili{}-NP\ili{}-c\ili{}-f\ili{}.png}\ili{}
\ili{} \ili{} \ili{}\caption\ili{}{C\ili{}-\ili{} and\ili{} f\ili{}-structure\ili{} for\ili{} the\ili{} sentence\ili{} \ili{}(\ili{}\ref\ili{}{ex\ili{}:mweiness\ili{}:ansvarfor\ili{}-NP}\ili{})\ili{} }\ili{}
\ili{} \ili{} \ili{}\label\ili{}{fig\ili{}:mweiness\ili{}:ansvarfor\ili{}-NP\ili{}-c\ili{}-f}\ili{}
\ili{}\end\ili{}{figure}\ili{}
\ili{}
Example\ili{} \ili{}(\ili{}\ref\ili{}{ex\ili{}:mweiness\ili{}:ansvarfor\ili{}-NP}\ili{})\ili{} is\ili{} analyzed\ili{} as\ili{} in\ili{} Figure\ili{} \ili{}\ref\ili{}{fig\ili{}:mweiness\ili{}:ansvarfor\ili{}-NP\ili{}-c\ili{}-f}\ili{}.\ili{}
As\ili{} in\ili{} the\ili{} case\ili{} of\ili{} \ili{}\isi\ili{}{prepositional\ili{} verbs}\ili{},\ili{} the\ili{} selected\ili{} preposition\ili{} does\ili{} not\ili{} contribute\ili{} a\ili{} PRED\ili{} of\ili{} its\ili{} own\ili{},\ili{} but\ili{} is\ili{} analyzed\ili{} as\ili{} forming\ili{} a\ili{} single\ili{} predicate\ili{} \ili{}`ansvar\ili{}*for\ili{}'\ili{} with\ili{} the\ili{} noun\ili{},\ili{} taking\ili{} the\ili{} complement\ili{} of\ili{} the\ili{} preposition\ili{} as\ili{} an\ili{} argument\ili{} with\ili{} the\ili{} function\ili{} OBL\ili{}-TH\ili{} \ili{}(an\ili{} oblique\ili{} argument\ili{} expressing\ili{} a\ili{} theta\ili{} role\ili{})\ili{}.\ili{}
With\ili{} an\ili{} infinitival\ili{} or\ili{} a\ili{} clausal\ili{} complement\ili{} the\ili{} syntactic\ili{} function\ili{} is\ili{} COMP\ili{}.\ili{}
The\ili{} \ili{}\isi\ili{}{lexical\ili{} entry}\ili{} for\ili{} \ili{}\textit\ili{}{ansvar}\ili{} in\ili{} \ili{}(\ili{}\ref\ili{}{ex\ili{}:mweiness\ili{}:ansvar\ili{}-lex}\ili{})\ili{} invokes\ili{} three\ili{} alternative\ili{} templates\ili{} for\ili{} the\ili{} three\ili{} possible\ili{} kinds\ili{} of\ili{} complements\ili{},\ili{} in\ili{} addition\ili{} to\ili{} its\ili{} basic\ili{} \ili{}\isi\ili{}{template}\ili{} as\ili{} a\ili{} mass\ili{} noun\ili{}.\ili{}
\ili{}
\ili{}\eabox\ili{}{\ili{}\label\ili{}{ex\ili{}:mweiness\ili{}:ansvar\ili{}-lex}\ili{}
\ili{}{\ili{}\small\ili{} \ili{}
\ili{}\begin\ili{}{tabular}\ili{}{ll}\ili{}
ansvar\ili{} N\ili{} \ili{}&\ili{} \ili{}\\ili{}{\ili{} \ili{}\enspace\ili{}@\ili{}(MASSNOUN\ili{} ansvar\ili{})\ili{}\\ili{}\\ili{}
\ili{}&\ili{} \ili{}|\ili{} \ili{}\enspace\ili{} \ili{}@\ili{}(N\ili{}-POBJ\ili{} ansvar\ili{} for\ili{})\ili{}\\ili{}\\ili{}
\ili{}&\ili{} \ili{}|\ili{} \ili{}\enspace\ili{} \ili{}@\ili{}(N\ili{}-PINFCOMP\ili{} ansvar\ili{} for\ili{})\ili{}\\ili{}\\ili{}
\ili{}&\ili{} \ili{}|\ili{} \ili{}\enspace\ili{} \ili{}@\ili{}(N\ili{}-PCOMP\ili{} ansvar\ili{} for\ili{})\ili{} \ili{}\enspace\ili{} \ili{}\}\ili{}
\ili{}\end\ili{}{tabular}}}\ili{}
\ili{}
\ili{}\subsubsection\ili{}{Adjectives\ili{} with\ili{} selected\ili{} prepositions}\ili{}\label\ili{}{sec\ili{}:mweiness\ili{}:prepadj}\ili{}
\ili{}
Similarly\ili{},\ili{} adjectives\ili{} may\ili{} select\ili{} prepositional\ili{} phrases\ili{} as\ili{} complements\ili{},\ili{} for\ili{} instance\ili{} \ili{}\textit\ili{}{flink\ili{} til}\ili{} \ili{}`clever\ili{} at\ili{}'\ili{},\ili{} as\ili{} in\ili{} the\ili{} treebank\ili{} example\ili{} in\ili{} \ili{}(\ili{}\ref\ili{}{ex\ili{}:mweiness\ili{}:flinktil}\ili{})\ili{}.\ili{}
\ili{}
\ili{}\ea\ili{}\label\ili{}{ex\ili{}:mweiness\ili{}:flinktil}\ili{}
\ili{}\gll\ili{} Hva\ili{} er\ili{} egentlig\ili{} du\ili{} \ili{}\textbf\ili{}{flink}\ili{} \ili{}\textbf\ili{}{til}\ili{}?\ili{} \ili{}\\ili{}\\ili{}
\ili{} \ili{} \ili{} \ili{} \ili{} what\ili{} are\ili{} after\ili{} all\ili{} you\ili{} clever\ili{} at\ili{}\\ili{}\\ili{}
\ili{}\glt\ili{} \ili{}`What\ili{} are\ili{} you\ili{} clever\ili{} at\ili{},\ili{} after\ili{} all\ili{}?\ili{}'\ili{}
\ili{}\z\ili{}
\ili{}
\ili{}\begin\ili{}{figure}\ili{}
\ili{} \ili{} \ili{}\includegraphics\ili{}[width\ili{}=\ili{}\textwidth\ili{}]\ili{}{figures\ili{}/flinktil\ili{}-c\ili{}-f\ili{}.png}\ili{}
\ili{} \ili{} \ili{}\caption\ili{}{C\ili{}-\ili{} and\ili{} f\ili{}-structure\ili{} for\ili{} the\ili{} sentence\ili{} \ili{} \ili{}(\ili{}\ref\ili{}{ex\ili{}:mweiness\ili{}:flinktil}\ili{})\ili{} }\ili{}
\ili{} \ili{} \ili{}\label\ili{}{fig\ili{}:mweiness\ili{}:flinktil\ili{}-c\ili{}-f}\ili{}
\ili{}\end\ili{}{figure}\ili{}
\ili{}
Example\ili{} \ili{} \ili{}(\ili{}\ref\ili{}{ex\ili{}:mweiness\ili{}:flinktil}\ili{})\ili{} is\ili{} analyzed\ili{} as\ili{} in\ili{} Figure\ili{} \ili{}\ref\ili{}{fig\ili{}:mweiness\ili{}:flinktil\ili{}-c\ili{}-f}\ili{}.\ili{}
In\ili{} this\ili{} example\ili{} the\ili{} complement\ili{} of\ili{} the\ili{} selected\ili{} preposition\ili{} has\ili{} been\ili{} questioned\ili{} and\ili{} occurs\ili{} in\ili{} the\ili{} f\ili{}-structure\ili{} as\ili{} the\ili{} value\ili{} of\ili{} FOCUS\ili{}-INT\ili{},\ili{} i\ili{}.e\ili{}.\ili{},\ili{} interrogative\ili{} focus\ili{}.\ili{}
The\ili{} predicative\ili{} complement\ili{} \ili{}(PREDLINK\ili{})\ili{} has\ili{} the\ili{} predicate\ili{} \ili{}`flink\ili{}*til\ili{}'\ili{},\ili{} taking\ili{} the\ili{} prepositional\ili{} complement\ili{} as\ili{} its\ili{} OBL\ili{}-TH\ili{}.\ili{}
The\ili{} value\ili{} of\ili{} OBL\ili{}-TH\ili{} is\ili{} identical\ili{} with\ili{} the\ili{} value\ili{} of\ili{} \ili{} FOCUS\ili{}-INT\ili{},\ili{} which\ili{} is\ili{} indicated\ili{} by\ili{} the\ili{} shared\ili{} index\ili{} 8\ili{},\ili{} resulting\ili{} from\ili{} the\ili{} general\ili{} analysis\ili{} of\ili{} \ili{}\textit\ili{}{wh}\ili{}-questions\ili{} in\ili{} the\ili{} grammar\ili{}.\ili{}
\ili{}
\ili{}\section\ili{}{Representing\ili{} flexibility}\ili{}\label\ili{}{sec\ili{}:mweiness\ili{}:variation}\ili{}
\ili{}
\ili{}%\ili{}\subsection\ili{}{Types\ili{} of\ili{} MWEs}\ili{}
\ili{}%\ili{}
\ili{}%NOTE\ili{}-VR\ili{}:\ili{} Light\ili{} verb\ili{} constructions\ili{} should\ili{} be\ili{} mentioned\ili{} here\ili{}.\ili{} \ili{} Are\ili{} some\ili{} actually\ili{} analyzed\ili{} as\ili{} verbal\ili{} idioms\ili{}?\ili{} \ili{} Check\ili{} annotation\ili{} guidelines\ili{} for\ili{} PARSEME\ili{} shared\ili{} task\ili{}.\ili{} \ili{}
\ili{}%NOTE\ili{}-GSL\ili{}:\ili{} Reflexive\ili{} verbs\ili{} should\ili{} perhaps\ili{} also\ili{} be\ili{} mentioned\ili{} here\ili{}.\ili{} \ili{}
\ili{}%Both\ili{} reflexive\ili{} verbs\ili{} and\ili{} LVCs\ili{} are\ili{} borderline\ili{} cases\ili{} if\ili{} we\ili{} assume\ili{} a\ili{} definition\ili{} where\ili{} an\ili{} MWE\ili{} must\ili{} have\ili{} at\ili{} least\ili{} two\ili{} lexically\ili{} fixed\ili{} components\ili{}.\ili{} \ili{}
\ili{}%In\ili{} reflexive\ili{} verbs\ili{},\ili{} the\ili{} variation\ili{} in\ili{} the\ili{} reflexive\ili{} is\ili{} grammatical\ili{} and\ili{} limited\ili{};\ili{} in\ili{} LVCs\ili{},\ili{} the\ili{} variation\ili{} \ili{}(in\ili{} the\ili{} verb\ili{})\ili{} is\ili{} lexical\ili{} and\ili{} limited\ili{}.\ili{} \ili{}
\ili{}%\ili{}
\ili{}%\ili{}%definitions\ili{} by\ili{} Sag\ili{} and\ili{} Baldwin\ili{}:\ili{}
\ili{}%\ili{}%\ili{}\citet\ili{}{Sag\ili{}:02}\ili{}:\ili{} \ili{}`\ili{}`idiosyncratic\ili{} interpretations\ili{} that\ili{} cross\ili{} word\ili{} boundaries\ili{} \ili{}(or\ili{} spaces\ili{})\ili{}'\ili{}'\ili{}
\ili{}%\ili{}%\ili{}\citet\ili{}{Baldwin\ili{}:10}\ili{}:\ili{} \ili{}´\ili{}´lexical\ili{} items\ili{} that\ili{}:\ili{} \ili{}(a\ili{})\ili{} can\ili{} be\ili{} decomposed\ili{} into\ili{} multiple\ili{} lexemes\ili{};\ili{} and\ili{} \ili{}(b\ili{})\ili{} display\ili{} lexical\ili{},\ili{} syntactic\ili{},\ili{} semantic\ili{},\ili{} pragmatic\ili{} and\ili{}/or\ili{} statistical\ili{} \ili{}\isi\ili{}{idiomaticity}\ili{}'\ili{}'\ili{}
\ili{}
\ili{}%\ili{}\subsection\ili{}{Syntactic\ili{} \ili{}`\ili{}\isi\ili{}{transformations}\ili{}'}\ili{}
\ili{}
Flexible\ili{} MWEs\ili{} must\ili{} be\ili{} recognizable\ili{} across\ili{} different\ili{} types\ili{} of\ili{} syntactic\ili{} modifications\ili{} which\ili{} separate\ili{} their\ili{} parts\ili{} from\ili{} each\ili{} other\ili{} in\ili{} the\ili{} sentence\ili{}.\ili{}
Such\ili{} modifications\ili{} include\ili{} the\ili{} simple\ili{} occurrence\ili{} of\ili{} other\ili{} words\ili{} between\ili{} the\ili{} MWE\ili{} parts\ili{},\ili{} \ili{}\isi\ili{}{long\ili{}-distance\ili{} dependencies}\ili{} like\ili{} topicalization\ili{},\ili{} \ili{}\isi\ili{}{relativization}\ili{} and\ili{} \ili{}\textit\ili{}{wh}\ili{}-question\ili{} formation\ili{},\ili{} presentative\ili{} constructions\ili{},\ili{} and\ili{} various\ili{} types\ili{} of\ili{} passive\ili{} constructions\ili{}.\ili{}
When\ili{} flexible\ili{} MWEs\ili{} are\ili{} treated\ili{} by\ili{} means\ili{} of\ili{} LFG\ili{} templates\ili{},\ili{} such\ili{} modifications\ili{} are\ili{} automatically\ili{} taken\ili{} care\ili{} of\ili{} within\ili{} the\ili{} regular\ili{} grammar\ili{}.\ili{}
Having\ili{} both\ili{} a\ili{} c\ili{}-structure\ili{} and\ili{} an\ili{} f\ili{}-structure\ili{} representation\ili{} allows\ili{} us\ili{} to\ili{} capture\ili{} both\ili{} the\ili{} close\ili{} semantic\ili{} and\ili{} functional\ili{} association\ili{} between\ili{} the\ili{} selecting\ili{} and\ili{} the\ili{} selected\ili{} words\ili{} \ili{}(in\ili{} the\ili{} f\ili{}-structure\ili{})\ili{} and\ili{} their\ili{} syntactic\ili{} independence\ili{} as\ili{} different\ili{} constituents\ili{} \ili{}(in\ili{} the\ili{} c\ili{}-structure\ili{})\ili{}.\ili{}
We\ili{} will\ili{} present\ili{} the\ili{} analyses\ili{} of\ili{} some\ili{} cases\ili{}.\ili{}
\ili{}
\ili{}\subsection\ili{}{Intervening\ili{} words}\ili{}
\ili{}
The\ili{} simplest\ili{} case\ili{} of\ili{} syntactic\ili{} modification\ili{} of\ili{} a\ili{} MWE\ili{} is\ili{} when\ili{} other\ili{} words\ili{},\ili{} typically\ili{} adverbs\ili{},\ili{} occur\ili{} between\ili{} the\ili{} MWE\ili{} components\ili{}.\ili{}
In\ili{} a\ili{} verb\ili{}-second\ili{} language\ili{} like\ili{} \ili{}\ili\ili{}{Norwegian}\ili{} the\ili{} sentence\ili{} subject\ili{} also\ili{} frequently\ili{} breaks\ili{} up\ili{} a\ili{} verb\ili{} \ili{}\isi\ili{}{phrase}\ili{} MWE\ili{}.\ili{}
The\ili{} treebank\ili{} example\ili{} in\ili{} \ili{}(\ili{}\ref\ili{}{ex\ili{}:mweiness\ili{}:trekketilbake}\ili{})\ili{} \ili{} illustrates\ili{} the\ili{} recognition\ili{} of\ili{} the\ili{} predicate\ili{} \ili{}`trekke\ili{}*seg\ili{}*tilbake\ili{}'\ili{} \ili{}(\ili{}`withdraw\ili{}'\ili{},\ili{} literally\ili{}:\ili{} \ili{}`draw\ili{} oneself\ili{} back\ili{}'\ili{})\ili{} across\ili{} several\ili{} intervening\ili{} words\ili{}.\ili{}
The\ili{} analysis\ili{} is\ili{} shown\ili{} in\ili{} Figures\ili{} \ili{}\ref\ili{}{fig\ili{}:mweiness\ili{}:trekketilbake\ili{}-c}\ili{} and\ili{} \ili{}\ref\ili{}{fig\ili{}:mweiness\ili{}:trekketilbake\ili{}-f}\ili{}.\ili{} \ili{}
\ili{}
\ili{}\ea\ili{}\label\ili{}{ex\ili{}:mweiness\ili{}:trekketilbake}\ili{}
\ili{}\gll\ili{} Da\ili{} \ili{}\textbf\ili{}{trekker}\ili{} jeg\ili{} \ili{}\textbf\ili{}{meg}\ili{} bare\ili{} stille\ili{} og\ili{} rolig\ili{} \ili{}\textbf\ili{}{tilbake}\ili{}.\ili{}\\ili{}\\ili{}
\ili{} \ili{} \ili{} \ili{} \ili{} then\ili{} draw\ili{} I\ili{} me\ili{} only\ili{} silently\ili{} and\ili{} calmly\ili{} back\ili{}\\ili{}\\ili{}
\ili{}\glt\ili{} \ili{}`Then\ili{} I\ili{} simply\ili{} withdraw\ili{} silently\ili{} and\ili{} calmly\ili{}.\ili{}'\ili{}
\ili{}\z\ili{}
\ili{}
\ili{}%\ili{}\ea\ili{}\label\ili{}{ex\ili{}:mweiness\ili{}:trekketilbake}\ili{}
\ili{}%\ili{}\gll\ili{} Da\ili{} \ili{}\textbf\ili{}{trekker}\ili{} jeg\ili{} \ili{}\textbf\ili{}{meg}\ili{} bare\ili{} stille\ili{} og\ili{} rolig\ili{} \ili{}\textbf\ili{}{tilbake}\ili{}.\ili{}\\ili{}\\ili{}
\ili{}%\ili{} \ili{} \ili{} \ili{} \ili{} then\ili{} draw\ili{}.\ili{}\textsc\ili{}{prs}\ili{} I\ili{} me\ili{} only\ili{} silently\ili{} and\ili{} calmly\ili{} back\ili{}\\ili{}\\ili{}
\ili{}%\ili{}\glt\ili{} \ili{}`Then\ili{} I\ili{} simply\ili{} withdraw\ili{} silently\ili{} and\ili{} calmly\ili{}.\ili{}'\ili{}
\ili{}%\ili{}\z\ili{}
\ili{}
\ili{}\begin\ili{}{figure}\ili{}
\ili{} \ili{} \ili{}\includegraphics\ili{}[height\ili{}=\ili{}.31\ili{}\textheight\ili{}]\ili{}{figures\ili{}/trekketilbake\ili{}-c\ili{}.png}\ili{}
\ili{} \ili{} \ili{}\caption\ili{}{C\ili{}-structure\ili{} of\ili{} sentence\ili{} \ili{}(\ili{}\ref\ili{}{ex\ili{}:mweiness\ili{}:trekketilbake}\ili{})}\ili{}
\ili{} \ili{} \ili{}\label\ili{}{fig\ili{}:mweiness\ili{}:trekketilbake\ili{}-c}\ili{}
\ili{}\end\ili{}{figure}\ili{}
\ili{}
\ili{}\begin\ili{}{figure}\ili{}
\ili{} \ili{} \ili{}\includegraphics\ili{}[height\ili{}=\ili{}.175\ili{}\textheight\ili{}]\ili{}{figures\ili{}/trekketilbake\ili{}-f\ili{}.png}\ili{}
\ili{} \ili{} \ili{}\caption\ili{}{Simplified\ili{} f\ili{}-structure\ili{} of\ili{} sentence\ili{} \ili{}(\ili{}\ref\ili{}{ex\ili{}:mweiness\ili{}:trekketilbake}\ili{})}\ili{}
\ili{} \ili{} \ili{}\label\ili{}{fig\ili{}:mweiness\ili{}:trekketilbake\ili{}-f}\ili{}
\ili{}\end\ili{}{figure}\ili{}
\ili{}
The\ili{} mechanism\ili{} for\ili{} achieving\ili{} this\ili{} lies\ili{} in\ili{} the\ili{} projection\ili{} architecture\ili{} of\ili{} LFG\ili{},\ili{} in\ili{} which\ili{} different\ili{} constituents\ili{} in\ili{} c\ili{}-structure\ili{} may\ili{} project\ili{} the\ili{} same\ili{} f\ili{}-structure\ili{},\ili{} within\ili{} which\ili{} dependencies\ili{} may\ili{} be\ili{} formulated\ili{}.\ili{}
To\ili{} illustrate\ili{} we\ili{} may\ili{} consider\ili{} the\ili{} relevant\ili{} fragments\ili{} of\ili{} the\ili{} c\ili{}-structure\ili{} rules\ili{} for\ili{} I\ili{}′\ili{},\ili{} S\ili{} and\ili{} VPmain\ili{} in\ili{} \ili{}(\ili{}\ref\ili{}{ex\ili{}:mweiness\ili{}:rule1}\ili{})\ili{}–\ili{}(\ili{}\ref\ili{}{ex\ili{}:mweiness\ili{}:rule3}\ili{})\ili{}.\ili{}
\ili{}
\ili{}\eabox\ili{}{\ili{}\label\ili{}{ex\ili{}:mweiness\ili{}:rule1}\ili{}
\ili{}{\ili{}\small\ili{} \ili{}
\ili{}\begin\ili{}{tabular}\ili{}{lll}\ili{}
I\ili{}′\ili{} \ili{}$\ili{}\rightarrow\ili{}$\ili{} \ili{}&\ili{} \ili{} Vfin\ili{}:\ili{} \ili{}&\ili{} \ili{}$\ili{}\uparrow\ili{}$\ili{}=\ili{}$\ili{}\downarrow\ili{}$\ili{}\\ili{}\\ili{}
\ili{} \ili{}&\ili{} \ili{}(S\ili{}:\ili{} \ili{}&\ili{} \ili{}$\ili{}\uparrow\ili{}$\ili{}=\ili{}$\ili{}\downarrow\ili{}$\ili{})\ili{}\\ili{}\\ili{}\\ili{}\\ili{}
\ili{}\end\ili{}{tabular}}}\ili{}
\ili{}
\ili{}\eabox\ili{}{\ili{}\label\ili{}{ex\ili{}:mweiness\ili{}:rule2}\ili{}
\ili{}{\ili{}\small\ili{} \ili{}
\ili{}\begin\ili{}{tabular}\ili{}{lll}\ili{}
S\ili{} \ili{}$\ili{}\rightarrow\ili{}$\ili{} \ili{}&\ili{} \ili{} \ili{}(PRONP\ili{}:\ili{} \ili{}&\ili{} \ili{}(\ili{}$\ili{}\uparrow\ili{}$\ili{} SUBJ\ili{})\ili{}=\ili{}$\ili{}\downarrow\ili{}$\ili{}\\ili{}\\ili{}
\ili{}&\ili{} \ili{}&\ili{}@SUBJCASE\ili{})\ili{}\\ili{}\\ili{}
\ili{}&\ili{} \ili{}[\ili{}.\ili{}.\ili{}.\ili{}]\ili{}\\ili{}\\ili{}
\ili{}&\ili{} \ili{}(PRONrfl\ili{}:\ili{} \ili{}&\ili{} \ili{}\\ili{}{\ili{} \ili{}\enspace\ili{} \ili{}(\ili{}$\ili{}\uparrow\ili{}$\ili{} OBJ\ili{}-BEN\ili{})\ili{}=\ili{}$\ili{}\downarrow\ili{}$\ili{}\\ili{}\\ili{}
\ili{}&\ili{} \ili{}&\ili{} \ili{}|\ili{} \ili{}\enspace\ili{} \ili{}(\ili{}$\ili{}\uparrow\ili{}$\ili{} OBJ\ili{})\ili{}=\ili{}$\ili{}\downarrow\ili{}$\ili{} \ili{}\enspace\ili{} \ili{}\}\ili{})\ili{} \ili{}\\ili{}\\ili{}
\ili{}&\ili{} \ili{}[\ili{}.\ili{}.\ili{}.\ili{}]\ili{}\\ili{}\\ili{}
\ili{} \ili{}&\ili{} \ili{}(ADVPs\ili{}+\ili{}:\ili{} \ili{}&\ili{} \ili{}$\ili{}\downarrow\ili{}$\ili{} \ili{}∈\ili{} \ili{}(\ili{}$\ili{}\uparrow\ili{}$\ili{} ADJUNCT\ili{})\ili{})\ili{}\\ili{}\\ili{}
\ili{}&\ili{} \ili{}[\ili{}.\ili{}.\ili{}.\ili{}]\ili{}\\ili{}\\ili{}
\ili{}&\ili{} \ili{}(APsmpl\ili{}:\ili{} \ili{} \ili{}&\ili{} \ili{}$\ili{}\downarrow\ili{}$\ili{} \ili{}∈\ili{} \ili{}(\ili{}$\ili{}\uparrow\ili{}$\ili{} ADJUNCT\ili{})\ili{})\ili{}\\ili{}\\ili{}
\ili{}&\ili{} \ili{}[\ili{}.\ili{}.\ili{}.\ili{}]\ili{}\\ili{}\\ili{}
\ili{} \ili{}&\ili{} \ili{}(VPmain\ili{}:\ili{} \ili{}&\ili{} \ili{}$\ili{}\uparrow\ili{}$\ili{}=\ili{}$\ili{}\downarrow\ili{}$\ili{})\ili{}\\ili{}\\ili{}
\ili{}&\ili{} \ili{}[\ili{}.\ili{}.\ili{}.\ili{}]\ili{}\\ili{}\\ili{}\\ili{}\\ili{}
\ili{}\end\ili{}{tabular}}}\ili{}
\ili{}
\ili{}\eabox\ili{}{\ili{}\label\ili{}{ex\ili{}:mweiness\ili{}:rule3}\ili{}
\ili{}{\ili{}\small\ili{} \ili{}
\ili{}\begin\ili{}{tabular}\ili{}{lll}\ili{}
VPmain\ili{} \ili{}$\ili{}\rightarrow\ili{}$\ili{} \ili{}&\ili{} \ili{}[\ili{}.\ili{}.\ili{}.\ili{}]\ili{}\\ili{}\\ili{}
\ili{}&\ili{} \ili{}(PRTP\ili{}:\ili{} \ili{}&\ili{} \ili{}$\ili{}\uparrow\ili{}$\ili{}=\ili{}$\ili{}\downarrow\ili{}$\ili{}\\ili{}\\ili{}
\ili{}&\ili{} \ili{}&\ili{} \ili{}(\ili{}$\ili{}\uparrow\ili{}$\ili{} CHECK\ili{} PRT\ili{}-VERB\ili{})\ili{}=c\ili{} \ili{}+\ili{})\ili{}\\ili{}\\ili{}
\ili{}&\ili{} \ili{}[\ili{}.\ili{}.\ili{}.\ili{}]\ili{}\\ili{}\\ili{}\\ili{}\\ili{}
\ili{}\end\ili{}{tabular}}}\ili{}
\ili{}
\ili{}
\ili{}%\ili{}\eabox\ili{}{\ili{}\label\ili{}{ex\ili{}:mweiness\ili{}:rules}\ili{}
\ili{}%\ili{}\begin\ili{}{tabular}\ili{}{lll}\ili{}
\ili{}%I\ili{}'\ili{} \ili{}$\ili{}\rightarrow\ili{}$\ili{} \ili{}&\ili{} \ili{} Vfin\ili{}:\ili{} \ili{}&\ili{} \ili{}$\ili{}\uparrow\ili{}$\ili{}=\ili{}$\ili{}\downarrow\ili{}$\ili{}\\ili{}\\ili{}
\ili{}%\ili{} \ili{}&\ili{} \ili{}(S\ili{}:\ili{} \ili{}&\ili{} \ili{}$\ili{}\uparrow\ili{}$\ili{}=\ili{}$\ili{}\downarrow\ili{}$\ili{})\ili{}\\ili{}\\ili{}\\ili{}\\ili{}
\ili{}%\ili{}\end\ili{}{tabular}\ili{}
\ili{}%\ili{}
\ili{}%\ili{}\begin\ili{}{tabular}\ili{}{lll}\ili{}
\ili{}%S\ili{} \ili{}$\ili{}\rightarrow\ili{}$\ili{} \ili{}&\ili{} \ili{} \ili{}(PRONP\ili{}:\ili{} \ili{}&\ili{} \ili{}(\ili{}$\ili{}\uparrow\ili{}$\ili{} SUBJ\ili{})\ili{}=\ili{}$\ili{}\downarrow\ili{}$\ili{}\\ili{}\\ili{}
\ili{}%\ili{}&\ili{} \ili{}&\ili{}@SUBJCASE\ili{}\\ili{}\\ili{}
\ili{}%\ili{}&\ili{} \ili{}[\ili{}.\ili{}.\ili{}.\ili{}]\ili{}\\ili{}\\ili{}
\ili{}%\ili{}&\ili{} \ili{}(PRONrfl\ili{}:\ili{} \ili{}&\ili{} \ili{}\\ili{}{\ili{} \ili{}\enspace\ili{} \ili{}(\ili{}$\ili{}\uparrow\ili{}$\ili{} OBJ\ili{}-BEN\ili{})\ili{}=\ili{}$\ili{}\downarrow\ili{}$\ili{}\\ili{}\\ili{}
\ili{}%\ili{}&\ili{} \ili{}&\ili{} \ili{}|\ili{} \ili{}\enspace\ili{} \ili{}(\ili{}$\ili{}\uparrow\ili{}$\ili{} OBJ\ili{})\ili{}=\ili{}$\ili{}\downarrow\ili{}$\ili{} \ili{}\enspace\ili{} \ili{}\}\ili{} \ili{}\\ili{}\\ili{}
\ili{}%\ili{}&\ili{} \ili{}[\ili{}.\ili{}.\ili{}.\ili{}]\ili{}\\ili{}\\ili{}
\ili{}%\ili{} \ili{}&\ili{} \ili{}(ADVPs\ili{}+\ili{}:\ili{} \ili{}&\ili{} \ili{}$\ili{}\downarrow\ili{}$\ili{} \ili{}∈\ili{} \ili{}(\ili{}$\ili{}\uparrow\ili{}$\ili{} ADJUNCT\ili{})\ili{})\ili{}\\ili{}\\ili{}
\ili{}%\ili{}&\ili{} \ili{}[\ili{}.\ili{}.\ili{}.\ili{}]\ili{}\\ili{}\\ili{}
\ili{}%\ili{}&\ili{} \ili{}(APsmpl\ili{}:\ili{} \ili{} \ili{}&\ili{} \ili{}$\ili{}\downarrow\ili{}$\ili{} \ili{}∈\ili{} \ili{}(\ili{}$\ili{}\uparrow\ili{}$\ili{} ADJUNCT\ili{})\ili{})\ili{}\\ili{}\\ili{}
\ili{}%\ili{}&\ili{} \ili{}[\ili{}.\ili{}.\ili{}.\ili{}]\ili{}\\ili{}\\ili{}
\ili{}%\ili{} \ili{}&\ili{} \ili{}(VPmain\ili{}:\ili{} \ili{}&\ili{} \ili{}$\ili{}\uparrow\ili{}$\ili{}=\ili{}$\ili{}\downarrow\ili{}$\ili{})\ili{}\\ili{}\\ili{}
\ili{}%\ili{}&\ili{} \ili{}[\ili{}.\ili{}.\ili{}.\ili{}]\ili{}\\ili{}\\ili{}\\ili{}\\ili{}
\ili{}%\ili{}\end\ili{}{tabular}\ili{}
\ili{}%\ili{}
\ili{}%\ili{}\begin\ili{}{tabular}\ili{}{lll}\ili{}
\ili{}%VPmain\ili{} \ili{}$\ili{}\rightarrow\ili{}$\ili{} \ili{}&\ili{} \ili{}[\ili{}.\ili{}.\ili{}.\ili{}]\ili{}\\ili{}\\ili{}
\ili{}%\ili{}&\ili{} \ili{}(PRTP\ili{}:\ili{} \ili{}&\ili{} \ili{}$\ili{}\uparrow\ili{}$\ili{}=\ili{}$\ili{}\downarrow\ili{}$\ili{}\\ili{}\\ili{}
\ili{}%\ili{}&\ili{} \ili{}&\ili{} \ili{}(\ili{}$\ili{}\uparrow\ili{}$\ili{} CHECK\ili{} PRT\ili{}-VERB\ili{})\ili{}=c\ili{} \ili{}+\ili{})\ili{}\\ili{}\\ili{}
\ili{}%\ili{}&\ili{} \ili{}[\ili{}.\ili{}.\ili{}.\ili{}]\ili{}\\ili{}\\ili{}\\ili{}\\ili{}
\ili{}%\ili{}\end\ili{}{tabular}}\ili{}
\ili{}
As\ili{} explained\ili{} in\ili{} section\ili{} \ili{}\ref\ili{}{sec\ili{}:mweiness\ili{}:LFG}\ili{},\ili{} the\ili{} equation\ili{} \ili{}$\ili{}\uparrow\ili{}$\ili{}=\ili{}$\ili{}\downarrow\ili{}$\ili{} annotated\ili{} to\ili{} a\ili{} \ili{}\isi\ili{}{rule}\ili{} daughter\ili{} means\ili{} that\ili{} the\ili{} \ili{}\isi\ili{}{rule}\ili{} daughter\ili{} and\ili{} its\ili{} mother\ili{} will\ili{} project\ili{} the\ili{} same\ili{} f\ili{}-structure\ili{}.\ili{}
Thus\ili{} it\ili{} can\ili{} be\ili{} seen\ili{} that\ili{} the\ili{} Vfin\ili{} daughter\ili{} of\ili{} I\ili{}′\ili{} \ili{}(the\ili{} verb\ili{})\ili{} and\ili{} the\ili{} PRTP\ili{} daughter\ili{} of\ili{} VPmain\ili{} \ili{}(the\ili{} particle\ili{})\ili{} will\ili{} project\ili{} the\ili{} same\ili{} f\ili{}-structure\ili{}.\ili{}
\ili{}
A\ili{} particle\ili{} verb\ili{} presupposes\ili{} the\ili{} presence\ili{} of\ili{} the\ili{} required\ili{} particle\ili{} in\ili{} the\ili{} sentence\ili{},\ili{} and\ili{} a\ili{} particle\ili{} presupposes\ili{} the\ili{} presence\ili{} of\ili{} a\ili{} particle\ili{} verb\ili{}.\ili{}
This\ili{} mutual\ili{} dependency\ili{} is\ili{} captured\ili{} through\ili{} two\ili{} features\ili{},\ili{} one\ili{} feature\ili{} PRT\ili{}-VERB\ili{}=\ili{}+\ili{},\ili{} carried\ili{} by\ili{} the\ili{} verb\ili{} and\ili{} required\ili{} by\ili{} the\ili{} \ili{}\isi\ili{}{rule}\ili{} introducing\ili{} the\ili{} particle\ili{},\ili{} and\ili{},\ili{} conversely\ili{},\ili{} one\ili{} feature\ili{} PRT\ili{}-FORM\ili{},\ili{} carried\ili{} by\ili{} the\ili{} particle\ili{} and\ili{} required\ili{} by\ili{} the\ili{} verb\ili{} to\ili{} have\ili{} the\ili{} appropriate\ili{} value\ili{}.\ili{}
Thus\ili{},\ili{} the\ili{} constraint\ili{} equation\ili{} annotated\ili{} to\ili{} PRTP\ili{},\ili{} \ili{}(\ili{}$\ili{}\uparrow\ili{}$\ili{} CHECK\ili{} PRT\ili{}-VERB\ili{})\ili{}=c\ili{} \ili{}+\ili{},\ili{} demanding\ili{} that\ili{} its\ili{} f\ili{}-structure\ili{} should\ili{} have\ili{} a\ili{} feature\ili{} PRT\ili{}-VERB\ili{}=\ili{}+\ili{} \ili{}(i\ili{}.e\ili{}.\ili{},\ili{} that\ili{} the\ili{} verb\ili{} should\ili{} be\ili{} a\ili{} particle\ili{} verb\ili{})\ili{},\ili{} will\ili{} be\ili{} satisfied\ili{} if\ili{} the\ili{} finite\ili{} verb\ili{} has\ili{} contributed\ili{} such\ili{} a\ili{} feature\ili{} to\ili{} this\ili{} common\ili{} f\ili{}-structure\ili{}.\ili{}
A\ili{} similar\ili{} constraint\ili{} equation\ili{} associated\ili{} with\ili{} the\ili{} verb\ili{},\ili{} \ili{}(\ili{}(\ili{}$\ili{}\uparrow\ili{}$\ili{} PRT\ili{}-FORM\ili{})\ili{}=c\ili{} prt\ili{} in\ili{} \ili{}(\ili{}\ref\ili{}{ex\ili{}:mweiness\ili{}:trekke\ili{}-template}\ili{})\ili{} below\ili{})\ili{},\ili{} ensures\ili{} that\ili{} the\ili{} particle\ili{} has\ili{} the\ili{} form\ili{} required\ili{} by\ili{} the\ili{} verb\ili{}.\ili{}
\ili{}
The\ili{} \ili{}\isi\ili{}{lexical\ili{} entry}\ili{} for\ili{} \ili{}\textit\ili{}{trekke}\ili{} is\ili{} associated\ili{} with\ili{} the\ili{} relevant\ili{} frame\ili{} through\ili{} an\ili{} invocation\ili{} of\ili{} the\ili{} \ili{}\isi\ili{}{template}\ili{} for\ili{} reflexive\ili{} \ili{}\isi\ili{}{verb\ili{}-particle\ili{} constructions}\ili{}.\ili{}
Part\ili{} of\ili{} the\ili{} \ili{}\isi\ili{}{lexical\ili{} entry}\ili{} for\ili{} \ili{}\textit\ili{}{trekke}\ili{} is\ili{} shown\ili{} in\ili{} \ili{} \ili{}(\ili{}\ref\ili{}{ex\ili{}:mweiness\ili{}:trekke\ili{}-lex}\ili{})\ili{}.\ili{}
The\ili{} \ili{}\isi\ili{}{template}\ili{} has\ili{} the\ili{} form\ili{} shown\ili{} in\ili{} \ili{} \ili{}(\ili{}\ref\ili{}{ex\ili{}:mweiness\ili{}:trekke\ili{}-template}\ili{})\ili{}.\ili{}
\ili{}
\ili{}\eabox\ili{}{\ili{}\label\ili{}{ex\ili{}:mweiness\ili{}:trekke\ili{}-lex}\ili{}
\ili{}{\ili{}\small\ili{} \ili{}
\ili{}\begin\ili{}{tabular}\ili{}{ll}\ili{}
trekke\ili{} V\ili{} \ili{}&\ili{} \ili{}\\ili{}{\ili{} \ili{}\enspace\ili{} \ili{}[\ili{} \ili{}.\ili{}.\ili{}.\ili{} \ili{}]\ili{}\\ili{}\\ili{}
\ili{}&\ili{} \ili{}|\ili{} \ili{}\enspace\ili{} \ili{}@\ili{}(V\ili{}-SUBJ\ili{}-OBJrefl\ili{}-PRT\ili{} trekke\ili{} tilbake\ili{})\ili{}\\ili{}\\ili{}
\ili{}&\ili{} \ili{}|\ili{} \ili{}\enspace\ili{} \ili{}[\ili{} \ili{}.\ili{}.\ili{}.\ili{} \ili{}]\ili{} \ili{}\enspace\ili{} \ili{}\}\ili{}\\ili{}\\ili{}
\ili{}\end\ili{}{tabular}}}\ili{}
\ili{}
\ili{}\ea\ili{}\label\ili{}{ex\ili{}:mweiness\ili{}:trekke\ili{}-template}\ili{}
\ili{}{\ili{}\small\ili{} \ili{}
V\ili{}-SUBJ\ili{}-OBJrefl\ili{}-PRT\ili{} \ili{}(P\ili{} prt\ili{})\ili{} \ili{}=\ili{}\\ili{}\\ili{}%\ili{}[\ili{}.5em\ili{}]\ili{}
\ili{}\hspace\ili{}{1\ili{}.5em}\ili{} \ili{}@\ili{}(CONCAT\ili{} P\ili{} \ili{}`\ili{}*\ili{} seg\ili{} \ili{}`\ili{}*\ili{} prt\ili{} \ili{} \ili{}\\ili{}%FN\ili{})\ili{}\\ili{}\\ili{}%\ili{}[\ili{}.5em\ili{}]\ili{}
\ili{}\hspace\ili{}{1\ili{}.5em}\ili{} \ili{} \ili{}\\ili{}{\ili{} \ili{}\enspace\ili{} \ili{}(\ili{}$\ili{}\uparrow\ili{}$\ili{} \ili{} PRED\ili{})\ili{}=\ili{}`\ili{}\\ili{}%FN\ili{}<\ili{}(\ili{}$\ili{}\uparrow\ili{}$\ili{} SUBJ\ili{})\ili{}>\ili{}(\ili{}$\ili{}\uparrow\ili{}$\ili{} OBJ\ili{}-BEN\ili{})\ili{}'\ili{}\\ili{}\\ili{}%\ili{}[\ili{}.5em\ili{}]\ili{}
\ili{}\hspace\ili{}{1\ili{}.5em}\ili{} \ili{}|\ili{} \ili{}\enspace\ili{} \ili{}(\ili{}$\ili{}\uparrow\ili{}$\ili{} \ili{} PRED\ili{})\ili{}=\ili{}`\ili{}\\ili{}%FN\ili{}<\ili{}$\ili{}\uparrow\ili{}$\ili{} OBJ\ili{})\ili{}>\ili{}(\ili{}$\ili{}\uparrow\ili{}$\ili{} \ili{} OBJ\ili{}-BEN\ili{})\ili{}(\ili{}$\ili{}\uparrow\ili{}$\ili{} SUBJ\ili{})\ili{}'\ili{}\\ili{}\\ili{}%\ili{}[\ili{}.5em\ili{}]\ili{}
\ili{}\hspace\ili{}{1\ili{}.5em}\ili{} \ili{}\quad\ili{} \ili{}(\ili{}$\ili{}\uparrow\ili{}$\ili{} PRESENTATIVE\ili{})\ili{}=\ili{}+\ili{}\\ili{}\\ili{}%\ili{}[\ili{}.5em\ili{}]\ili{}
\ili{}\hspace\ili{}{1\ili{}.5em}\ili{} \ili{}\quad\ili{} \ili{}(\ili{}$\ili{}\uparrow\ili{}$\ili{} SUBJ\ili{} PRON\ili{}-TYPE\ili{})\ili{}=c\ili{} expl\ili{}\\ili{}\\ili{}%\ili{}[\ili{}.5em\ili{}]\ili{}
\ili{}\hspace\ili{}{1\ili{}.5em}\ili{} \ili{}\quad\ili{} \ili{}¬\ili{} \ili{}(\ili{}$\ili{}\uparrow\ili{}$\ili{} OBJ\ili{} DEF\ili{})\ili{}=\ili{}+\ili{} \ili{}\enspace\ili{} \ili{}\}\ili{}\\ili{}\\ili{}%\ili{}[\ili{}.5em\ili{}]\ili{}
\ili{}\hspace\ili{}{1\ili{}.5em}\ili{} \ili{}@\ili{}(REFLEXIVE\ili{} OBJ\ili{}-BEN\ili{})\ili{}\\ili{}\\ili{}%\ili{}[\ili{}.5em\ili{}]\ili{}
\ili{}\hspace\ili{}{1\ili{}.5em}\ili{} \ili{}(\ili{}$\ili{}\uparrow\ili{}$\ili{} CHECK\ili{} PRT\ili{}-VRB\ili{})\ili{}=\ili{}+\ili{}\\ili{}\\ili{}%\ili{}[\ili{}.5em\ili{}]\ili{}
\ili{}\hspace\ili{}{1\ili{}.5em}\ili{} \ili{}(\ili{}$\ili{}\uparrow\ili{}$\ili{} PRT\ili{}-FORM\ili{})\ili{}=c\ili{} prt\ili{}\\ili{}\\ili{}%\ili{}[\ili{}.5em\ili{}]\ili{}
\ili{}\hspace\ili{}{1\ili{}.5em}\ili{} \ili{}¬\ili{} \ili{}(\ili{}$\ili{}\uparrow\ili{}$\ili{} PASSIVE\ili{})\ili{}=\ili{}+\ili{}
}\ili{}
\ili{}\z\ili{}
\ili{}
The\ili{} \ili{}\isi\ili{}{template}\ili{} CONCAT\ili{} constructs\ili{} the\ili{} predicate\ili{} name\ili{} \ili{}`trekke\ili{}*seg\ili{}*tilbake\ili{}'\ili{} as\ili{} the\ili{} value\ili{} of\ili{} PRED\ili{}.\ili{}
The\ili{} reflexive\ili{} occurring\ili{} with\ili{} reflexive\ili{} verbs\ili{} is\ili{} analyzed\ili{} as\ili{} OBJ\ili{}-BEN\ili{} \ili{}(indirect\ili{} object\ili{})\ili{}.\ili{}
The\ili{} reason\ili{} for\ili{} this\ili{} is\ili{} that\ili{} there\ili{} will\ili{} be\ili{} a\ili{} direct\ili{} object\ili{} in\ili{} the\ili{} alternative\ili{} presentative\ili{} construction\ili{} with\ili{} an\ili{} expletive\ili{} \ili{}\textit\ili{}{det}\ili{} subject\ili{},\ili{} such\ili{} as\ili{} \ili{}\textit\ili{}{Det\ili{} trekker\ili{} seg\ili{} tilbake\ili{} store\ili{} styrker}\ili{} \ili{}`There\ili{} are\ili{} big\ili{} forces\ili{} withdrawing\ili{}'\ili{},\ili{} in\ili{} which\ili{} \ili{}\textit\ili{}{store\ili{} styrker}\ili{} occurs\ili{} as\ili{} a\ili{} syntactic\ili{} object\ili{} \ili{}(OBJ\ili{})\ili{};\ili{} there\ili{} can\ili{} only\ili{} be\ili{} one\ili{} OBJ\ili{}.\ili{}
The\ili{} presentative\ili{} construction\ili{} is\ili{} described\ili{} as\ili{} the\ili{} second\ili{} alternative\ili{} in\ili{} the\ili{} disjunction\ili{} \ili{}\\ili{}{\ili{}.\ili{}.\ili{}.\ili{}|\ili{}.\ili{}.\ili{}.\ili{}\}\ili{} in\ili{} the\ili{} \ili{}\isi\ili{}{template}\ili{}.\ili{}
The\ili{} reflexive\ili{},\ili{} like\ili{} the\ili{} expletive\ili{} subject\ili{},\ili{} is\ili{} analyzed\ili{} as\ili{} a\ili{} non\ili{}-argument\ili{},\ili{} which\ili{} appears\ili{} from\ili{} the\ili{} fact\ili{} that\ili{} it\ili{} is\ili{} placed\ili{} outside\ili{} the\ili{} argument\ili{} list\ili{} enclosed\ili{} by\ili{} \ili{}<\ili{}.\ili{}.\ili{}.\ili{}>\ili{} in\ili{} the\ili{} value\ili{} of\ili{} PRED\ili{}.\ili{}
The\ili{} features\ili{} PRT\ili{}-VRB\ili{} and\ili{} PRT\ili{}-FORM\ili{} are\ili{} explained\ili{} in\ili{} the\ili{} discussion\ili{} of\ili{} the\ili{} \ili{}\isi\ili{}{template}\ili{} in\ili{} \ili{}(\ili{}\ref\ili{}{ex\ili{}:mweiness\ili{}:skrive\ili{}-template}\ili{})\ili{}.\ili{}
\ili{}
\ili{}\subsection\ili{}{Long\ili{}-distance\ili{} dependencies}\ili{}\label\ili{}{sec\ili{}:mweiness\ili{}:longdist}\ili{}
\ili{}
Long\ili{}-distance\ili{} dependencies\ili{} involve\ili{} syntactic\ili{} dependencies\ili{} across\ili{} an\ili{} arbitrary\ili{} number\ili{} of\ili{} clause\ili{} boundaries\ili{} and\ili{} comprise\ili{} topicalization\ili{} by\ili{} fronting\ili{},\ili{} relative\ili{} clauses\ili{} and\ili{} \ili{}\textit\ili{}{wh}\ili{}-questions\ili{}.\ili{}
Such\ili{} dependencies\ili{} are\ili{} handled\ili{} in\ili{} the\ili{} f\ili{}-structure\ili{} by\ili{} means\ili{} of\ili{} a\ili{} special\ili{} type\ili{} of\ili{} equations\ili{} using\ili{} regular\ili{} expressions\ili{} to\ili{} specify\ili{} a\ili{} set\ili{} of\ili{} alternative\ili{} attribute\ili{} paths\ili{} into\ili{} the\ili{} f\ili{}-structure\ili{}.\ili{}
The\ili{} term\ili{} for\ili{} this\ili{} mechanism\ili{} is\ili{} functional\ili{} uncertainty\ili{}.\ili{}
The\ili{} \ili{}\isi\ili{}{rule}\ili{} in\ili{} \ili{}(\ili{}\ref\ili{}{ex\ili{}:mweiness\ili{}:furule}\ili{})\ili{} shows\ili{} a\ili{} simplified\ili{} version\ili{} of\ili{} the\ili{} functional\ili{} uncertainty\ili{} equation\ili{} handling\ili{} the\ili{} dependency\ili{} between\ili{} the\ili{} topic\ili{} and\ili{} some\ili{} embedded\ili{} gap\ili{} further\ili{} down\ili{} in\ili{} the\ili{} structure\ili{}.\ili{}
\ili{}
\ili{}\eabox\ili{}{\ili{}\label\ili{}{ex\ili{}:mweiness\ili{}:furule}\ili{}
\ili{} \ili{} \ili{}{\ili{}\small\ili{}
\ili{}\begin\ili{}{tabular}\ili{}{lll}\ili{}
IP\ili{} \ili{}$\ili{}\rightarrow\ili{}$\ili{} \ili{}&\ili{} \ili{} XP\ili{}:\ili{} \ili{}&\ili{} \ili{}(\ili{}$\ili{}\uparrow\ili{}$\ili{} TOPIC\ili{})\ili{}=\ili{}$\ili{}\downarrow\ili{}$\ili{}\\ili{}\\ili{}
\ili{}&\ili{} \ili{}&\ili{} \ili{}(\ili{}$\ili{}\uparrow\ili{}$\ili{} \ili{}\\ili{}{COMP\ili{} \ili{}|\ili{} XCOMP\ili{}\}\ili{}*\ili{} \ili{}\\ili{}{SUBJ\ili{} \ili{}|\ili{} OBJ\ili{} \ili{}|\ili{} OBL\ili{}-TH\ili{}\}\ili{})\ili{}=\ili{}$\ili{}\downarrow\ili{}$\ili{}\\ili{}\\ili{}
\ili{}&\ili{} \ili{}[\ili{} \ili{}.\ili{}.\ili{}.\ili{} \ili{}]\ili{}
\ili{}\end\ili{}{tabular}\ili{}
}\ili{}
}\ili{}
\ili{}
The\ili{} first\ili{} equation\ili{} annotated\ili{} to\ili{} XP\ili{} in\ili{} \ili{}(\ili{}\ref\ili{}{ex\ili{}:mweiness\ili{}:furule}\ili{})\ili{} specifies\ili{} that\ili{} the\ili{} f\ili{}-structure\ili{} of\ili{} the\ili{} XP\ili{} daughter\ili{} \ili{}(\ili{}$\ili{}\downarrow\ili{}$\ili{})\ili{} is\ili{} the\ili{} value\ili{} of\ili{} the\ili{} attribute\ili{} TOPIC\ili{} of\ili{} the\ili{} f\ili{}-structure\ili{} of\ili{} the\ili{} IP\ili{} mother\ili{} \ili{}(\ili{}$\ili{}\uparrow\ili{}$\ili{})\ili{}.\ili{}
The\ili{} second\ili{} equation\ili{} specifies\ili{} that\ili{} the\ili{} daughter\ili{} f\ili{}-structure\ili{} is\ili{} also\ili{} the\ili{} value\ili{} of\ili{} one\ili{} of\ili{} a\ili{} set\ili{} of\ili{} alternative\ili{} attribute\ili{} paths\ili{}.\ili{}
COMP\ili{} and\ili{} XCOMP\ili{} are\ili{} the\ili{} attributes\ili{} of\ili{} embedded\ili{} finite\ili{} and\ili{} non\ili{}-finite\ili{} clauses\ili{}.\ili{}
The\ili{} regular\ili{} expression\ili{} \ili{}\\ili{}{COMP\ili{} \ili{}|\ili{} XCOMP\ili{}\}\ili{}*\ili{} describes\ili{} all\ili{} possible\ili{} strings\ili{} over\ili{} the\ili{} elements\ili{} COMP\ili{} and\ili{} XCOMP\ili{} \ili{}(with\ili{} repetitions\ili{})\ili{},\ili{} and\ili{} the\ili{} final\ili{} disjunction\ili{} specifies\ili{} the\ili{} last\ili{} attribute\ili{} of\ili{} the\ili{} string\ili{},\ili{} enabling\ili{} the\ili{} TOPIC\ili{} to\ili{} be\ili{} identical\ili{} with\ili{} an\ili{} embedded\ili{} SUBJ\ili{},\ili{} OBJ\ili{} or\ili{} OBL\ili{}-TH\ili{}.\ili{}
We\ili{} may\ili{} illustrate\ili{} with\ili{} the\ili{} treebank\ili{} example\ili{} in\ili{} \ili{}(\ili{}\ref\ili{}{ex\ili{}:mweiness\ili{}:fortelleom}\ili{})\ili{} of\ili{} a\ili{} prepositional\ili{} verb\ili{} \ili{}\textit\ili{}{fortelle\ili{} om}\ili{} \ili{}`tell\ili{} about\ili{}'\ili{} in\ili{} a\ili{} sentence\ili{} where\ili{} the\ili{} OBL\ili{}-TH\ili{},\ili{} i\ili{}.e\ili{}.\ili{},\ili{} the\ili{} selected\ili{} prepositional\ili{} \ili{}\isi\ili{}{phrase}\ili{},\ili{} has\ili{} been\ili{} topicalized\ili{}.\ili{}
\ili{}
\ili{}\ea\ili{}\label\ili{}{ex\ili{}:mweiness\ili{}:fortelleom}\ili{}
\ili{}\gll\ili{} \ili{}\textbf\ili{}{Om}\ili{} dette\ili{} skal\ili{} jeg\ili{} \ili{}\textbf\ili{}{fortelle}\ili{} nå\ili{}.\ili{}\\ili{}\\ili{}
\ili{} \ili{} \ili{} \ili{} \ili{} about\ili{} this\ili{} shall\ili{} I\ili{} tell\ili{} now\ili{}\\ili{}\\ili{}
\ili{}\glt\ili{} \ili{}`This\ili{} I\ili{} will\ili{} now\ili{} tell\ili{} about\ili{}.\ili{}'\ili{}
\ili{}\z\ili{}
\ili{}
\ili{}%\ili{}\begin\ili{}{figure}\ili{}
\ili{}%\ili{} \ili{} \ili{}\includegraphics\ili{}[width\ili{}=\ili{}\textwidth\ili{}]\ili{}{figures\ili{}/fortelle\ili{}-om\ili{}-c\ili{}.png}\ili{}
\ili{}%\ili{} \ili{} \ili{}\caption\ili{}{C\ili{}-structure\ili{} for\ili{} example\ili{} \ili{}(\ili{}\ref\ili{}{ex\ili{}:mweiness\ili{}:fortelleom}\ili{})}\ili{}
\ili{}%\ili{} \ili{} \ili{}\label\ili{}{fig\ili{}:mweiness\ili{}:fortelle\ili{}-om\ili{}-c\ili{}-f}\ili{}
\ili{}%\ili{}\end\ili{}{figure}\ili{}
\ili{}%\ili{}
\ili{}%\ili{}\begin\ili{}{figure}\ili{}
\ili{}%\ili{} \ili{} \ili{}\includegraphics\ili{}[width\ili{}=\ili{}\textwidth\ili{}]\ili{}{figures\ili{}/fortelle\ili{}-om\ili{}-f\ili{}.png}\ili{}
\ili{}%\ili{} \ili{} \ili{}\caption\ili{}{F\ili{}-structure\ili{} for\ili{} example\ili{} \ili{}(\ili{}\ref\ili{}{ex\ili{}:mweiness\ili{}:fortelleom}\ili{})}\ili{}
\ili{}%\ili{} \ili{} \ili{}\label\ili{}{fig\ili{}:mweiness\ili{}:fortelle\ili{}-om\ili{}-c\ili{}-f}\ili{}
\ili{}%\ili{}\end\ili{}{figure}\ili{}
\ili{}
\ili{}\begin\ili{}{figure}\ili{}
\ili{} \ili{} \ili{}\includegraphics\ili{}[width\ili{}=\ili{}\textwidth\ili{}]\ili{}{figures\ili{}/fortelle\ili{}-om\ili{}-c\ili{}-f\ili{}.png}\ili{}
\ili{} \ili{} \ili{}\caption\ili{}{C\ili{}-\ili{} and\ili{} f\ili{}-structure\ili{} for\ili{} example\ili{} \ili{}(\ili{}\ref\ili{}{ex\ili{}:mweiness\ili{}:fortelleom}\ili{})}\ili{}
\ili{} \ili{} \ili{}\label\ili{}{fig\ili{}:mweiness\ili{}:fortelle\ili{}-om\ili{}-c\ili{}-f}\ili{}
\ili{}\end\ili{}{figure}\ili{}
\ili{}
The\ili{} analysis\ili{} of\ili{} \ili{} \ili{}(\ili{}\ref\ili{}{ex\ili{}:mweiness\ili{}:fortelleom}\ili{})\ili{} is\ili{} shown\ili{} in\ili{} Figure\ili{} \ili{}\ref\ili{}{fig\ili{}:mweiness\ili{}:fortelle\ili{}-om\ili{}-c\ili{}-f}\ili{}.\ili{}
In\ili{} the\ili{} f\ili{}-structure\ili{} the\ili{} value\ili{} of\ili{} TOPIC\ili{},\ili{} indexed\ili{} 6\ili{},\ili{} is\ili{} also\ili{} found\ili{} as\ili{} the\ili{} value\ili{} of\ili{} OBL\ili{}-TH\ili{} in\ili{} the\ili{} embedded\ili{} XCOMP\ili{} with\ili{} \ili{}`fortelle\ili{}*om\ili{}'\ili{} as\ili{} predicate\ili{}.\ili{}
Thus\ili{} the\ili{} attribute\ili{} string\ili{} from\ili{} the\ili{} set\ili{} specified\ili{} by\ili{} the\ili{} functional\ili{} uncertainty\ili{} equation\ili{} in\ili{} \ili{}(\ili{}\ref\ili{}{ex\ili{}:mweiness\ili{}:furule}\ili{})\ili{} for\ili{} this\ili{} example\ili{} is\ili{} \ili{}(\ili{}$\ili{}\uparrow\ili{}$\ili{} XCOMP\ili{} OBL\ili{}-TH\ili{})\ili{}.\ili{}
\ili{}
\ili{}\subsection\ili{}{Passive\ili{} alternations}\ili{}\label\ili{}{sec\ili{}:mweiness\ili{}:vpidiomsyntax}\ili{}
\ili{}
Passive\ili{} is\ili{} another\ili{} source\ili{} of\ili{} verbal\ili{} MWE\ili{} modifications\ili{},\ili{} changing\ili{} the\ili{} syntactic\ili{} functions\ili{} of\ili{} selected\ili{} constituents\ili{}.\ili{}
In\ili{} LFG\ili{} passive\ili{} is\ili{} analyzed\ili{} as\ili{} a\ili{} lexical\ili{} phenomenon\ili{} modifying\ili{} the\ili{} value\ili{} of\ili{} PRED\ili{} in\ili{} a\ili{} \ili{}\isi\ili{}{lexical\ili{} entry}\ili{} for\ili{} a\ili{} verb\ili{},\ili{} changing\ili{} the\ili{} mapping\ili{} between\ili{} argument\ili{} positions\ili{} and\ili{} syntactic\ili{} functions\ili{}.\ili{}
In\ili{} NorGram\ili{} this\ili{} is\ili{} handled\ili{} by\ili{} passive\ili{} templates\ili{} invoked\ili{} by\ili{} the\ili{} verb\ili{} templates\ili{}.\ili{}
The\ili{} full\ili{} version\ili{} of\ili{} the\ili{} VP\ili{} idiom\ili{} \ili{}\isi\ili{}{template}\ili{} in\ili{} \ili{}(\ili{}\ref\ili{}{ex\ili{}:mweiness\ili{}:holde\ili{}-template}\ili{})\ili{} for\ili{} idioms\ili{} like\ili{} \ili{}\textit\ili{}{holde\ili{} øye\ili{} med}\ili{} \ili{}`keep\ili{} an\ili{} eye\ili{} on\ili{}'\ili{} is\ili{} shown\ili{} in\ili{} \ili{}(\ili{}\ref\ili{}{ex\ili{}:mweiness\ili{}:holde\ili{}-template\ili{}-full}\ili{})\ili{},\ili{} where\ili{} different\ili{} types\ili{} of\ili{} passive\ili{} alternations\ili{} are\ili{} handled\ili{}.\ili{}
\ili{}
\ili{}\clearpage\ili{}
\ili{}
\ili{}\ea\ili{}\label\ili{}{ex\ili{}:mweiness\ili{}:holde\ili{}-template\ili{}-full}\ili{}
\ili{}{\ili{}\small\ili{} \ili{}
\ili{} VPIDIOM\ili{}-INDEFOBJ\ili{}-POBJ\ili{} \ili{}(P\ili{} OP\ili{} prp\ili{})\ili{} \ili{}=\ili{}\\ili{}\\ili{}%\ili{}[\ili{}.5em\ili{}]\ili{}
\ili{}\hspace\ili{}{1\ili{}.5em}\ili{} \ili{}@\ili{}(CONCAT\ili{} P\ili{} \ili{}`\ili{}\\ili{}#\ili{} OP\ili{} \ili{}`\ili{}*\ili{} prp\ili{} \ili{}\\ili{}%FN\ili{})\ili{}\\ili{}\\ili{}%\ili{}[\ili{}.5em\ili{}]\ili{}
\ili{}\hspace\ili{}{1\ili{}.5em}\ili{} \ili{} \ili{}\\ili{}{\ili{} \ili{}\enspace\ili{} \ili{}@\ili{}(PASS\ili{}-OBL\ili{}-TH\ili{} \ili{}[\ili{}(\ili{}$\ili{}\uparrow\ili{}$\ili{} \ili{} PRED\ili{})\ili{}=\ili{}`\ili{}\\ili{}%FN\ili{}<\ili{}(\ili{}$\ili{}\uparrow\ili{}$\ili{} SUBJ\ili{})\ili{}(\ili{}$\ili{}\uparrow\ili{}$\ili{} OBL\ili{}-TH\ili{})\ili{}>\ili{}(\ili{}$\ili{}\uparrow\ili{}$\ili{} OBJ\ili{})\ili{} \ili{}]\ili{})\ili{}\\ili{}\\ili{}%\ili{}[\ili{}.5em\ili{}]\ili{}
\ili{}\hspace\ili{}{1\ili{}.5em}\ili{} \ili{}|\ili{} \ili{}\enspace\ili{} \ili{}\\ili{}{\ili{} \ili{}\enspace\ili{} \ili{}(\ili{}$\ili{}\uparrow\ili{}$\ili{} \ili{} PRED\ili{})\ili{}=\ili{}`\ili{}\\ili{}%FN\ili{}<NULL\ili{}(\ili{}$\ili{}\uparrow\ili{}$\ili{} OBL\ili{}-TH\ili{})\ili{}>\ili{}(\ili{}$\ili{}\uparrow\ili{}$\ili{} SUBJ\ili{})\ili{}(\ili{}$\ili{}\uparrow\ili{}$\ili{} OBJ\ili{})\ili{}'\ili{}\\ili{}\\ili{}%\ili{}[\ili{}.5em\ili{}]\ili{}
\ili{}\hspace\ili{}{1\ili{}.5em}\ili{} \ili{}\quad\ili{} \ili{}|\ili{} \ili{}\enspace\ili{} \ili{}(\ili{}$\ili{}\uparrow\ili{}$\ili{} \ili{} PRED\ili{})\ili{}=\ili{}`\ili{}\\ili{}%FN\ili{}<\ili{}$\ili{}\uparrow\ili{}$\ili{} OBL\ili{}-AG\ili{})\ili{}(\ili{}$\ili{}\uparrow\ili{}$\ili{} \ili{} OBL\ili{}-TH\ili{})\ili{}>\ili{}(\ili{}$\ili{}\uparrow\ili{}$\ili{} SUBJ\ili{})\ili{}(\ili{}$\ili{}\uparrow\ili{}$\ili{} OBJ\ili{})\ili{}'\ili{} \ili{}\enspace\ili{} \ili{}\}\ili{}\\ili{}\\ili{}%\ili{}[\ili{}.5em\ili{}]\ili{}
\ili{}\hspace\ili{}{1\ili{}.5em}\ili{} \ili{}\quad\ili{} \ili{}(\ili{}$\ili{}\uparrow\ili{}$\ili{} PASSIVE\ili{})\ili{}=c\ili{} \ili{}+\ili{}\\ili{}\\ili{}%\ili{}[\ili{}.5em\ili{}]\ili{}
\ili{}\hspace\ili{}{1\ili{}.5em}\ili{} \ili{}\quad\ili{} \ili{}(\ili{}$\ili{}\uparrow\ili{}$\ili{} PRESENTATIVE\ili{}-TYPE\ili{})\ili{}=passive\ili{}\\ili{}\\ili{}%\ili{}[\ili{}.5em\ili{}]\ili{}
\ili{}\hspace\ili{}{1\ili{}.5em}\ili{} \ili{}\quad\ili{} \ili{}(\ili{}$\ili{}\uparrow\ili{}$\ili{} SUBJ\ili{} PRON\ili{}-TYPE\ili{})\ili{}=c\ili{} expl\ili{} \ili{}\enspace\ili{} \ili{}\}\ili{}\\ili{}\\ili{}%\ili{}[\ili{}.5em\ili{}]\ili{}
\ili{}\hspace\ili{}{1\ili{}.5em}\ili{} \ili{}(\ili{}$\ili{}\uparrow\ili{}$\ili{} OBL\ili{}-TH\ili{} CHECK\ili{} P\ili{}-SELFORM\ili{})\ili{}=prp\ili{}\\ili{}\\ili{}%\ili{}[\ili{}.5em\ili{}]\ili{}
\ili{}\hspace\ili{}{1\ili{}.5em}\ili{} \ili{}(\ili{}$\ili{}\uparrow\ili{}$\ili{} OBJ\ili{} PRED\ili{} FN\ili{})\ili{}=c\ili{} OP\ili{}\\ili{}\\ili{}%\ili{}[\ili{}.5em\ili{}]\ili{}
\ili{}\hspace\ili{}{1\ili{}.5em}\ili{} \ili{}{\ili{}\textasciitilde}\ili{}(\ili{}$\ili{}\uparrow\ili{}$\ili{} OBJ\ili{} DEF\ili{})\ili{}=\ili{}+\ili{}
}\ili{}
\ili{}\z\ili{}
\ili{}
After\ili{} the\ili{} second\ili{} line\ili{} there\ili{} follows\ili{} a\ili{} disjunction\ili{} of\ili{} two\ili{} alternatives\ili{}.\ili{}
The\ili{} first\ili{} alternative\ili{} invokes\ili{} the\ili{} \ili{}\isi\ili{}{template}\ili{} PASS\ili{}-OBL\ili{}-TH\ili{},\ili{} taking\ili{} the\ili{} \ili{}\isi\ili{}{predicate\ili{}-argument\ili{} structure}\ili{} as\ili{} a\ili{} parameter\ili{}.\ili{}
This\ili{} \ili{}\isi\ili{}{template}\ili{} allows\ili{} the\ili{} active\ili{}/passive\ili{} alternation\ili{} whereby\ili{} the\ili{} OBL\ili{}-TH\ili{},\ili{} i\ili{}.e\ili{}.\ili{},\ili{} the\ili{} complement\ili{} of\ili{} the\ili{} selected\ili{} preposition\ili{} \ili{}(see\ili{} the\ili{} discussion\ili{} of\ili{} example\ili{} \ili{}(\ili{}\ref\ili{}{ex\ili{}:mweiness\ili{}:tenkepå}\ili{})\ili{})\ili{},\ili{} may\ili{} be\ili{} the\ili{} subject\ili{} in\ili{} a\ili{} passive\ili{} construction\ili{},\ili{} as\ili{} in\ili{} the\ili{} treebank\ili{} example\ili{} in\ili{} \ili{}(\ili{}\ref\ili{}{ex\ili{}:mweiness\ili{}:øyemedpass}\ili{})\ili{}.\ili{}
\ili{}
\ili{}\ea\ili{}\label\ili{}{ex\ili{}:mweiness\ili{}:øyemedpass}\ili{}
\ili{}\gll\ili{} De\ili{} var\ili{} derimot\ili{} ikke\ili{} klar\ili{} over\ili{} at\ili{} de\ili{} ble\ili{} \ili{}\textbf\ili{}{holdt}\ili{} \ili{}\textbf\ili{}{øye}\ili{} \ili{}\textbf\ili{}{med}\ili{}.\ili{} \ili{}\\ili{}\\ili{}
\ili{} \ili{} \ili{} \ili{} \ili{} they\ili{} were\ili{} on\ili{} the\ili{} other\ili{} hand\ili{} not\ili{} clear\ili{} over\ili{} that\ili{} they\ili{} became\ili{} held\ili{} eye\ili{} with\ili{}\\ili{}\\ili{}
\ili{}\glt\ili{} \ili{}`On\ili{} the\ili{} other\ili{} hand\ili{},\ili{} they\ili{} weren\ili{}'t\ili{} aware\ili{} that\ili{} someone\ili{} was\ili{} keeping\ili{} an\ili{} eye\ili{} on\ili{} them\ili{}.\ili{}'\ili{}
\ili{}\z\ili{}
\ili{}
The\ili{} second\ili{} alternative\ili{} in\ili{} the\ili{} main\ili{} disjunction\ili{} describes\ili{} the\ili{} impersonal\ili{} \ili{}(presentative\ili{})\ili{} passive\ili{} option\ili{} with\ili{} an\ili{} expletive\ili{} subject\ili{},\ili{} as\ili{} in\ili{} the\ili{} example\ili{} in\ili{} \ili{}(\ili{}\ref\ili{}{ex\ili{}:mweiness\ili{}:øyemedimpers}\ili{})\ili{}.\ili{}
\ili{}
\ili{}\ea\ili{}\label\ili{}{ex\ili{}:mweiness\ili{}:øyemedimpers}\ili{}
\ili{}\gll\ili{} Det\ili{} ble\ili{} \ili{}\textbf\ili{}{holdt}\ili{} \ili{}\textbf\ili{}{øye}\ili{} \ili{}\textbf\ili{}{med}\ili{} dem\ili{}.\ili{} \ili{}\\ili{}\\ili{}
\ili{} \ili{} \ili{} \ili{} \ili{} it\ili{} became\ili{} held\ili{} eye\ili{} with\ili{} them\ili{}\\ili{}\\ili{}
\ili{}\glt\ili{} \ili{}`Someone\ili{} was\ili{} keeping\ili{} an\ili{} eye\ili{} on\ili{} them\ili{}.\ili{}'\ili{}
\ili{}\z\ili{}
\ili{}
The\ili{} embedded\ili{} disjunction\ili{} of\ili{} two\ili{} predicate\ili{}-argument\ili{} structures\ili{} in\ili{} the\ili{} fourth\ili{} and\ili{} fifth\ili{} lines\ili{} of\ili{} the\ili{} \ili{}\isi\ili{}{template}\ili{} describes\ili{} the\ili{} possibility\ili{} of\ili{} including\ili{} an\ili{} OBL\ili{}-AG\ili{},\ili{} i\ili{}.e\ili{}.\ili{},\ili{} an\ili{} oblique\ili{} agent\ili{} in\ili{} a\ili{} prepositional\ili{} \ili{}\isi\ili{}{phrase}\ili{} with\ili{} \ili{}\textit\ili{}{av}\ili{} \ili{}`by\ili{}'\ili{}.\ili{}
Then\ili{} follow\ili{} equations\ili{} requiring\ili{} the\ili{} passive\ili{} form\ili{} of\ili{} the\ili{} verb\ili{} and\ili{} expletive\ili{} type\ili{} of\ili{} the\ili{} subject\ili{} pronoun\ili{}.\ili{}
\ili{}
\ili{}\section\ili{}{Complementation\ili{} patterns}\ili{}\label\ili{}{sec\ili{}:mweiness\ili{}:complementation}\ili{}
\ili{}
Verbal\ili{} MWEs\ili{} in\ili{} \ili{}\ili\ili{}{Norwegian}\ili{} show\ili{} considerable\ili{} variation\ili{} in\ili{} terms\ili{} of\ili{} subcategorizational\ili{} properties\ili{}.\ili{} \ili{}
Like\ili{} simple\ili{} verbs\ili{},\ili{} MWEs\ili{} can\ili{} have\ili{} transitivity\ili{} shifts\ili{},\ili{} take\ili{} different\ili{} types\ili{} of\ili{} arguments\ili{},\ili{} and\ili{} take\ili{} different\ili{} combinations\ili{} of\ili{} arguments\ili{}.\ili{} \ili{}
The\ili{} verb\ili{}-particle\ili{} construction\ili{} \ili{}\emph\ili{}{si\ili{} opp}\ili{},\ili{} for\ili{} instance\ili{},\ili{} has\ili{} both\ili{} an\ili{} intransitive\ili{} reading\ili{},\ili{} as\ili{} in\ili{} \ili{}(\ili{}\ref\ili{}{ex\ili{}:mweiness\ili{}:transitivity\ili{}-a}\ili{})\ili{},\ili{} and\ili{} a\ili{} transitive\ili{} reading\ili{},\ili{} as\ili{} in\ili{} \ili{}(\ili{}\ref\ili{}{ex\ili{}:mweiness\ili{}:transitivity\ili{}-b}\ili{})\ili{} and\ili{} \ili{}(\ili{}\ref\ili{}{ex\ili{}:mweiness\ili{}:transitivity\ili{}-c}\ili{})\ili{}.\ili{}
While\ili{} the\ili{} shift\ili{} in\ili{} transitivity\ili{} does\ili{} not\ili{} significantly\ili{} affect\ili{} the\ili{} semantics\ili{} of\ili{} the\ili{} expression\ili{} in\ili{} \ili{}(\ili{}\ref\ili{}{ex\ili{}:mweiness\ili{}:transitivity\ili{}-b}\ili{})\ili{},\ili{} the\ili{} shift\ili{} in\ili{} \ili{}(\ili{}\ref\ili{}{ex\ili{}:mweiness\ili{}:transitivity\ili{}-c}\ili{})\ili{} leads\ili{} to\ili{} a\ili{} change\ili{} in\ili{} meaning\ili{}.\ili{} \ili{} \ili{}
\ili{}
\ili{}\ea\ili{} \ili{}\label\ili{}{ex\ili{}:mweiness\ili{}:transitivity\ili{}-a}\ili{} \ili{}\gll\ili{} 150\ili{} befal\ili{} \ili{}\textbf\ili{}{sier}\ili{} \ili{}\textbf\ili{}{opp}\ili{}.\ili{} \ili{}\\ili{}\\ili{} \ili{}
\ili{} \ili{} \ili{} \ili{} \ili{} 150\ili{} officers\ili{} say\ili{} up\ili{}\\ili{}\\ili{}
\ili{}\glt\ili{} \ili{}`150\ili{} officers\ili{} resign\ili{}.\ili{}'\ili{} \ili{}\\ili{}\\ili{} \ili{}
\ili{}\z\ili{}
\ili{}
\ili{}\ea\ili{} \ili{}\label\ili{}{ex\ili{}:mweiness\ili{}:transitivity\ili{}-b}\ili{} \ili{}\gll\ili{} Hun\ili{} \ili{}\textbf\ili{}{sa}\ili{} \ili{}\textbf\ili{}{opp}\ili{} jobben\ili{}.\ili{} \ili{}\\ili{}\\ili{}
\ili{} \ili{} \ili{} \ili{} \ili{} she\ili{} said\ili{} up\ili{} \ili{}{the\ili{} job}\ili{} \ili{}\\ili{}\\ili{}
\ili{}\glt\ili{} \ili{}`She\ili{} resigned\ili{} from\ili{} her\ili{} job\ili{}.\ili{}'\ili{} \ili{}\\ili{}\\ili{}
\ili{}\z\ili{}
\ili{}
\ili{}\ea\ili{} \ili{}\label\ili{}{ex\ili{}:mweiness\ili{}:transitivity\ili{}-c}\ili{} \ili{}\gll\ili{} Man\ili{} må\ili{} \ili{}\textbf\ili{}{si}\ili{} \ili{}\textbf\ili{}{opp}\ili{} sjefen\ili{} for\ili{} Statkraft\ili{}.\ili{} \ili{}\\ili{}\\ili{}
\ili{} \ili{} \ili{} \ili{} \ili{} \ili{} one\ili{} must\ili{} say\ili{} up\ili{} \ili{}{the\ili{} boss}\ili{} for\ili{} Statkraft\ili{} \ili{}\\ili{}\\ili{}
\ili{}\glt\ili{} \ili{}`The\ili{} head\ili{} of\ili{} Statkraft\ili{} must\ili{} be\ili{} fired\ili{}.\ili{}'\ili{} \ili{}
\ili{}\z\ili{}
\ili{}
More\ili{} precisely\ili{},\ili{} the\ili{} theme\ili{} object\ili{} that\ili{} is\ili{} implicit\ili{} in\ili{} the\ili{} intransitive\ili{} usage\ili{} in\ili{} \ili{}(\ili{}\ref\ili{}{ex\ili{}:mweiness\ili{}:transitivity\ili{}-a}\ili{})\ili{} is\ili{} explicit\ili{} in\ili{} \ili{}(\ili{}\ref\ili{}{ex\ili{}:mweiness\ili{}:transitivity\ili{}-b}\ili{})\ili{},\ili{} while\ili{} in\ili{} \ili{}(\ili{}\ref\ili{}{ex\ili{}:mweiness\ili{}:transitivity\ili{}-c}\ili{})\ili{},\ili{} the\ili{} object\ili{} has\ili{} the\ili{} semantic\ili{} role\ili{} of\ili{} experiencer\ili{} instead\ili{} of\ili{} theme\ili{}.\ili{}
The\ili{} frames\ili{} V\ili{}-SUBJ\ili{}-PRT\ili{} and\ili{} V\ili{}-SUBJ\ili{}-PRT\ili{}-OBJ\ili{} represent\ili{} the\ili{} intransitive\ili{} and\ili{} the\ili{} transitive\ili{} usages\ili{} of\ili{} \ili{}\emph\ili{}{si\ili{} opp}\ili{}.\ili{}
NorGram\ili{},\ili{} being\ili{} mainly\ili{} a\ili{} syntactic\ili{} framework\ili{},\ili{} has\ili{} one\ili{} frame\ili{} for\ili{} both\ili{} transitive\ili{} readings\ili{},\ili{} leaving\ili{} semantic\ili{} roles\ili{} underspecified\ili{}.\ili{} \ili{}
\ili{}
Most\ili{} of\ili{} the\ili{} \ili{}\isi\ili{}{verbal\ili{} MWEs}\ili{} in\ili{} NorGram\ili{} are\ili{} \ili{}\isi\ili{}{phrasal\ili{} verbs}\ili{} or\ili{} VP\ili{} idioms\ili{}.\ili{}
Such\ili{} MWEs\ili{} have\ili{} free\ili{} subjects\ili{},\ili{} so\ili{} that\ili{} any\ili{} argument\ili{} variation\ili{} is\ili{} in\ili{} the\ili{} complements\ili{}.\ili{}\footnote\ili{}{The\ili{} exception\ili{} to\ili{} free\ili{} subjects\ili{} in\ili{} VP\ili{} idioms\ili{} is\ili{} expressions\ili{} with\ili{} the\ili{} expletive\ili{} subject\ili{} \ili{}\emph\ili{}{det}\ili{} \ili{}`it\ili{}'\ili{}.\ili{} However\ili{},\ili{} this\ili{} type\ili{} of\ili{} argument\ili{} variation\ili{} is\ili{} analyzed\ili{} as\ili{} a\ili{} grammatical\ili{} rather\ili{} than\ili{} a\ili{} lexical\ili{} selection\ili{} of\ili{} the\ili{} subject\ili{} and\ili{} is\ili{} thus\ili{} not\ili{} considered\ili{} here\ili{}.}\ili{}
The\ili{} lexical\ili{} entries\ili{} display\ili{} a\ili{} wide\ili{} range\ili{} of\ili{} complementation\ili{} patterns\ili{},\ili{} one\ili{} type\ili{} being\ili{} MWEs\ili{} where\ili{} the\ili{} verb\ili{} selects\ili{} all\ili{} of\ili{} its\ili{} complements\ili{}.\ili{} \ili{}
Table\ili{} \ili{}\ref\ili{}{tab\ili{}:mweiness\ili{}:selected}\ili{} presents\ili{} types\ili{} of\ili{} VP\ili{} idioms\ili{} in\ili{} NorGram\ili{} with\ili{} only\ili{} selected\ili{} complements\ili{}.\ili{} \ili{}
In\ili{} idioms\ili{} where\ili{} the\ili{} verb\ili{} subcategorizes\ili{} for\ili{} only\ili{} one\ili{} selected\ili{} complement\ili{},\ili{} the\ili{} selected\ili{} element\ili{} is\ili{} either\ili{} a\ili{} nominal\ili{} \ili{}(O\ili{})\ili{},\ili{} a\ili{} prepositional\ili{} \ili{}(\ili{}[P\ili{} \ili{}+\ili{} O\ili{}]\ili{})\ili{} or\ili{} a\ili{} predicative\ili{} \ili{}(PC\ili{})\ili{} complement\ili{}.\ili{} \ili{}
There\ili{} is\ili{} also\ili{} a\ili{} type\ili{} of\ili{} VP\ili{} idiom\ili{} with\ili{} two\ili{} selected\ili{} complements\ili{} \ili{}(O\ili{}~\ili{}+\ili{}~PRT\ili{})\ili{}.\ili{}
\ili{}
\ili{}\begin\ili{}{table}\ili{}
\ili{} \ili{} \ili{}\begin\ili{}{tabular}\ili{}{llll}\ili{}
\ili{} \ili{} \ili{} \ili{} \ili{}\lsptoprule\ili{}
\ili{} \ili{} \ili{} \ili{} Pattern\ili{} \ili{}&\ili{} Example\ili{} \ili{}&\ili{} Lit\ili{}.\ili{} translation\ili{} \ili{}&\ili{} Id\ili{}.\ili{} translation\ili{}\\ili{}\\ili{}
\ili{} \ili{} \ili{} \ili{} \ili{}\midrule\ili{}
\ili{} \ili{} \ili{} \ili{} V\ili{} \ili{}+\ili{} O\ili{} \ili{}&\ili{} \ili{}\emph\ili{}{slå\ili{} følge}\ili{} \ili{}&\ili{} \ili{} \ili{}`beat\ili{} company\ili{}'\ili{} \ili{}&\ili{} \ili{} \ili{}`accompany\ili{}'\ili{} \ili{}\\ili{}\\ili{}
\ili{} \ili{} \ili{} \ili{} \ili{}&\ili{} \ili{}\emph\ili{}{slå\ili{} leir}\ili{} \ili{}&\ili{} \ili{}`beat\ili{} camp\ili{}'\ili{} \ili{}&\ili{} \ili{} \ili{}`camp\ili{}'\ili{} \ili{}\\ili{}\\ili{}
\ili{} \ili{} \ili{} \ili{} \ili{}&\ili{} \ili{}\emph\ili{}{ta\ili{} feil}\ili{} \ili{}&\ili{} \ili{}`take\ili{} wrong\ili{}'\ili{} \ili{}&\ili{} \ili{}`be\ili{} wrong\ili{}'\ili{} \ili{} \ili{}\\ili{}\\ili{}
\ili{} \ili{} \ili{} \ili{} \ili{}&\ili{} \ili{}\emph\ili{}{ta\ili{} fyr}\ili{} \ili{}&\ili{} \ili{}`take\ili{} fire\ili{}'\ili{} \ili{}&\ili{} \ili{} \ili{}`catch\ili{} fire\ili{}'\ili{} \ili{}\\ili{}\\ili{} \ili{}\hline\ili{}
\ili{} \ili{} \ili{} \ili{} V\ili{} \ili{}+\ili{} \ili{}[P\ili{} \ili{}+\ili{} O\ili{}]\ili{} \ili{}&\ili{} \ili{}\emph\ili{}{gå\ili{} i\ili{} oppløsning}\ili{} \ili{}&\ili{} \ili{}`go\ili{} in\ili{} dissolution\ili{}'\ili{} \ili{}&\ili{} \ili{}`dissolve\ili{}'\ili{} \ili{}\\ili{}\\ili{} \ili{}
\ili{} \ili{} \ili{} \ili{} \ili{}&\ili{} \ili{}\emph\ili{}{komme\ili{} for\ili{} en\ili{} dag}\ili{} \ili{}&\ili{} \ili{}`come\ili{} for\ili{} a\ili{} day\ili{}'\ili{} \ili{}&\ili{} \ili{}`be\ili{} revealed\ili{}'\ili{} \ili{}\\ili{}\\ili{} \ili{} \ili{} \ili{} \ili{} \ili{}
\ili{} \ili{} \ili{} \ili{} \ili{}&\ili{} \ili{}\emph\ili{}{løfte\ili{} i\ili{} flokk}\ili{} \ili{}&\ili{} \ili{}`lift\ili{} in\ili{} flock\ili{}'\ili{} \ili{}&\ili{} \ili{}`join\ili{} forces\ili{}'\ili{} \ili{}\\ili{}\\ili{}
\ili{} \ili{} \ili{} \ili{} \ili{}&\ili{} \ili{}\emph\ili{}{legge\ili{} på\ili{} svøm}\ili{} \ili{}&\ili{} \ili{}`lay\ili{} on\ili{} swim\ili{}'\ili{} \ili{}&\ili{} \ili{} \ili{}`start\ili{} swimming\ili{}'\ili{} \ili{}\\ili{}\\ili{} \ili{}\hline\ili{}
\ili{} \ili{} \ili{} \ili{} V\ili{} \ili{}+\ili{} PC\ili{} \ili{}&\ili{} \ili{}\emph\ili{}{stå\ili{} brud}\ili{} \ili{}&\ili{} \ili{}`stand\ili{} bride\ili{}'\ili{} \ili{}&\ili{} \ili{}`get\ili{} married\ili{}'\ili{} \ili{}\\ili{}\\ili{} \ili{}\hline\ili{}
\ili{} \ili{} \ili{} \ili{} V\ili{} \ili{}+\ili{} O\ili{} \ili{}+\ili{} PRT\ili{} \ili{}&\ili{} \ili{}\emph\ili{}{sette\ili{} livet\ili{} til}\ili{} \ili{}&\ili{} \ili{}`put\ili{} the\ili{} life\ili{} to\ili{}'\ili{} \ili{}&\ili{} \ili{}`lose\ili{} one\ili{}'s\ili{} life\ili{}'\ili{} \ili{}\\ili{}\\ili{}
\ili{} \ili{} \ili{} \ili{} \ili{}\lspbottomrule\ili{}
\ili{} \ili{} \ili{}\end\ili{}{tabular}\ili{}
\ili{} \ili{} \ili{}\caption\ili{}{Verbal\ili{} MWEs\ili{} with\ili{} only\ili{} selected\ili{} complements}\ili{}
\ili{} \ili{} \ili{}\label\ili{}{tab\ili{}:mweiness\ili{}:selected}\ili{}
\ili{}\end\ili{}{table}\ili{}
\ili{}
\ili{}%Verbal\ili{} MWEs\ili{} with\ili{} simple\ili{} internal\ili{} structure\ili{} like\ili{} the\ili{} examples\ili{} in\ili{} Table\ili{} \ili{}(\ili{}\ref\ili{}{tab\ili{}:mweiness\ili{}:selected}\ili{})\ili{} are\ili{} particularly\ili{} suitable\ili{} for\ili{} NLP\ili{} related\ili{} tasks\ili{} such\ili{} as\ili{} MWE\ili{} identification\ili{},\ili{} \ili{}\isi\ili{}{MWE\ili{} extraction}\ili{} and\ili{} the\ili{} annotation\ili{} of\ili{} MWEs\ili{} in\ili{} corpora\ili{} and\ili{} \ili{}\isi\ili{}{treebanks}\ili{}.\ili{} \ili{}
\ili{}%\ili{}\citet\ili{}{Gibbs89\ili{},\ili{} Baldwin03\ili{},\ili{} Mcshane04\ili{},\ili{} Kay12\ili{},\ili{} Siyanova15}\ili{} are\ili{} examples\ili{} of\ili{} very\ili{} different\ili{} research\ili{} making\ili{} use\ili{} of\ili{} basic\ili{} MWE\ili{} structures\ili{} such\ili{} as\ili{} V\ili{} \ili{}+\ili{} OBJ\ili{} combinations\ili{} in\ili{} case\ili{} studies\ili{} and\ili{} experiments\ili{}.\ili{}
\ili{}%Many\ili{} MWEs\ili{},\ili{} however\ili{},\ili{} have\ili{} a\ili{} more\ili{} complex\ili{} build\ili{}-up\ili{} with\ili{} open\ili{} slots\ili{} that\ili{} must\ili{} be\ili{} filled\ili{}.\ili{} \ili{}
\ili{}%\ili{}
Most\ili{} \ili{}\isi\ili{}{verbal\ili{} MWEs}\ili{} in\ili{} NorGram\ili{} have\ili{} free\ili{} complements\ili{} in\ili{} addition\ili{} to\ili{} their\ili{} selected\ili{} complements\ili{}.\ili{} \ili{}
\ili{}%\ili{},\ili{} and\ili{} one\ili{} MWE\ili{} will\ili{} often\ili{} subcategorize\ili{} for\ili{} different\ili{} types\ili{} of\ili{} free\ili{} complements\ili{}.\ili{}
In\ili{} the\ili{} VP\ili{} idiom\ili{} \ili{}\emph\ili{}{legge\ili{} merke\ili{} til}\ili{} \ili{}`notice\ili{}'\ili{},\ili{} the\ili{} verb\ili{} \ili{}\emph\ili{}{legge}\ili{} \ili{}`lay\ili{}'\ili{} selects\ili{} the\ili{} object\ili{} \ili{}\emph\ili{}{merke}\ili{} \ili{}`mark\ili{}'\ili{} in\ili{} the\ili{} indefinite\ili{} form\ili{} and\ili{} a\ili{} prepositional\ili{} complement\ili{} which\ili{} is\ili{} either\ili{} nominal\ili{},\ili{} as\ili{} in\ili{} \ili{}(\ili{}\ref\ili{}{ex\ili{}:mweiness\ili{}:leggemerketil\ili{}-a}\ili{})\ili{},\ili{} clausal\ili{},\ili{} as\ili{} in\ili{} \ili{}(\ili{}\ref\ili{}{ex\ili{}:mweiness\ili{}:leggemerketil\ili{}-b}\ili{})\ili{},\ili{} or\ili{} an\ili{} interrogative\ili{} clausal\ili{} complement\ili{},\ili{} as\ili{} in\ili{} \ili{} \ili{}(\ili{}\ref\ili{}{ex\ili{}:mweiness\ili{}:leggemerketil\ili{}-c}\ili{})\ili{},\ili{} all\ili{} headed\ili{} by\ili{} the\ili{} selected\ili{} preposition\ili{} \ili{}\emph\ili{}{til}\ili{} \ili{}`to\ili{}'\ili{}.\ili{}
\ili{}
\ili{}\ea\ili{} \ili{}\label\ili{}{ex\ili{}:mweiness\ili{}:leggemerketil\ili{}-a}\ili{}
\ili{}\gll\ili{} Ingen\ili{} \ili{}\textbf\ili{}{legger}\ili{} \ili{}\textbf\ili{}{merke}\ili{} \ili{}\textbf\ili{}{til}\ili{} mannen\ili{} som\ili{} står\ili{} urørlig\ili{} og\ili{} venter\ili{}.\ili{} \ili{}\\ili{}\\ili{} \ili{}%\ili{} i\ili{} bakgrunnen\ili{}.\ili{} \ili{}\\ili{}\\ili{}
\ili{} \ili{} \ili{} \ili{} \ili{} \ili{}{no\ili{} one}\ili{} lays\ili{} mark\ili{} to\ili{} \ili{}{the\ili{} man}\ili{} who\ili{} stands\ili{} motionless\ili{} and\ili{} waits\ili{} \ili{}\\ili{}\\ili{} \ili{}%\ili{} in\ili{} \ili{}{the\ili{} background}\ili{} \ili{}\\ili{}\\ili{}
\ili{}\glt\ili{} \ili{}`No\ili{} one\ili{} notices\ili{} the\ili{} man\ili{} who\ili{} is\ili{} standing\ili{} motionless\ili{} waiting\ili{}.\ili{}'\ili{} \ili{}%\ili{} in\ili{} the\ili{} back\ili{}'\ili{}
\ili{}\z\ili{}
\ili{}
\ili{}\ea\ili{} \ili{}\label\ili{}{ex\ili{}:mweiness\ili{}:leggemerketil\ili{}-b}\ili{}
\ili{}\gll\ili{} Ingen\ili{} \ili{}\textbf\ili{}{legger}\ili{} \ili{}\textbf\ili{}{merke}\ili{} \ili{}\textbf\ili{}{til}\ili{} at\ili{} mannen\ili{} står\ili{} urørlig\ili{} og\ili{} venter\ili{}.\ili{} \ili{}\\ili{}\\ili{} \ili{}%\ili{} i\ili{} bakgrunnen\ili{}.\ili{} \ili{}\\ili{}\\ili{}
\ili{} \ili{} \ili{} \ili{} \ili{} \ili{}{no\ili{} one}\ili{} lays\ili{} mark\ili{} to\ili{} that\ili{} \ili{}{the\ili{} man}\ili{} stands\ili{} motionless\ili{} and\ili{} waits\ili{} \ili{}\\ili{}\\ili{} \ili{}%\ili{} in\ili{} \ili{}{the\ili{} background}\ili{}\\ili{}\\ili{}
\ili{}\glt\ili{} \ili{}`No\ili{} one\ili{} notices\ili{} that\ili{} the\ili{} man\ili{} is\ili{} standing\ili{} motionless\ili{} waiting\ili{}.\ili{}'\ili{} \ili{}%\ili{} in\ili{} the\ili{} back\ili{}'\ili{}
\ili{}\z\ili{}
\ili{}
\ili{}\ea\ili{} \ili{}\label\ili{}{ex\ili{}:mweiness\ili{}:leggemerketil\ili{}-c}\ili{}
\ili{}\gll\ili{} Ingen\ili{} \ili{}\textbf\ili{}{legger}\ili{} \ili{}\textbf\ili{}{merke}\ili{} \ili{}\textbf\ili{}{til}\ili{} om\ili{} mannen\ili{} står\ili{} urørlig\ili{} og\ili{} venter\ili{}.\ili{} \ili{}\\ili{}\\ili{} \ili{}%\ili{} i\ili{} bakgrunnen\ili{}.\ili{} \ili{}\\ili{}\\ili{}
\ili{} \ili{} \ili{} \ili{} \ili{} \ili{}{no\ili{} one}\ili{} lays\ili{} mark\ili{} to\ili{} if\ili{} \ili{}{the\ili{} man}\ili{} stands\ili{} motionless\ili{} and\ili{} waits\ili{} \ili{}\\ili{}\\ili{} \ili{}%\ili{} in\ili{} \ili{}{the\ili{} background}\ili{}\\ili{}\\ili{}
\ili{}\glt\ili{} \ili{}`No\ili{} one\ili{} notices\ili{} whether\ili{} the\ili{} man\ili{} is\ili{} standing\ili{} motionless\ili{} waiting\ili{}.\ili{}'\ili{} \ili{}%\ili{} in\ili{} the\ili{} back\ili{}'\ili{}
\ili{}\z\ili{}
\ili{}
\ili{}%\ili{}\ea\ili{}\label\ili{}{ex\ili{}:mweiness\ili{}:leggemerketil}\ili{}
\ili{}%\ili{}\begin\ili{}{xlist}\ili{}
\ili{}%\ili{}\ex\ili{} \ili{}\label\ili{}{ex\ili{}:mweiness\ili{}:leggemerketil\ili{}-a}\ili{}
\ili{}%\ili{}\gll\ili{} Ingen\ili{} \ili{}\textbf\ili{}{legger}\ili{} \ili{}\textbf\ili{}{merke}\ili{} \ili{}\textbf\ili{}{til}\ili{} \ili{}\emph\ili{}{mannen\ili{} som\ili{} står\ili{} urørlig\ili{} og\ili{} venter}\ili{}.\ili{} \ili{}\\ili{}\\ili{} \ili{}%\ili{} i\ili{} bakgrunnen\ili{}.\ili{} \ili{}\\ili{}\\ili{}
\ili{}%\ili{} \ili{} \ili{} \ili{} \ili{} \ili{}{no\ili{} one}\ili{} lays\ili{} mark\ili{} to\ili{} \ili{}{\ili{}{the\ili{} man}\ili{} who\ili{} stands\ili{} motionless\ili{} and\ili{} waits}\ili{} \ili{}\\ili{}\\ili{} \ili{}%\ili{} in\ili{} \ili{}{the\ili{} background}\ili{} \ili{}\\ili{}\\ili{}
\ili{}%\ili{}\glt\ili{} \ili{}`No\ili{} one\ili{} notices\ili{} the\ili{} man\ili{} who\ili{} stands\ili{} motionless\ili{} waiting\ili{}.\ili{}'\ili{} \ili{}%\ili{} in\ili{} the\ili{} back\ili{}'\ili{}
\ili{}%\ili{}\ex\ili{} \ili{}\label\ili{}{ex\ili{}:mweiness\ili{}:leggemerketil\ili{}-b}\ili{}
\ili{}%\ili{}\gll\ili{} Ingen\ili{} \ili{}\textbf\ili{}{legger}\ili{} \ili{}\textbf\ili{}{merke}\ili{} \ili{}\textbf\ili{}{til}\ili{} \ili{}\emph\ili{}{at\ili{} mannen\ili{} står\ili{} urørlig\ili{} og\ili{} venter}\ili{}.\ili{} \ili{}\\ili{}\\ili{} \ili{}%\ili{} i\ili{} bakgrunnen\ili{}.\ili{} \ili{}\\ili{}\\ili{}
\ili{}%\ili{} \ili{} \ili{} \ili{} \ili{} \ili{}{no\ili{} one}\ili{} lays\ili{} mark\ili{} to\ili{} \ili{}{that\ili{} \ili{}{the\ili{} man}\ili{} stands\ili{} motionless\ili{} and\ili{} waits}\ili{} \ili{}\\ili{}\\ili{} \ili{}%\ili{} in\ili{} \ili{}{the\ili{} background}\ili{}\\ili{}\\ili{}
\ili{}%\ili{}\glt\ili{} \ili{}`No\ili{} one\ili{} notices\ili{} that\ili{} the\ili{} man\ili{} stands\ili{} motionless\ili{} waiting\ili{}.\ili{}'\ili{} \ili{}%\ili{} in\ili{} the\ili{} back\ili{}'\ili{}
\ili{}%\ili{}\ex\ili{} \ili{}\label\ili{}{ex\ili{}:mweiness\ili{}:leggemerketil\ili{}-c}\ili{}
\ili{}%\ili{}\gll\ili{} Ingen\ili{} \ili{}\textbf\ili{}{legger}\ili{} \ili{}\textbf\ili{}{merke}\ili{} \ili{}\textbf\ili{}{til}\ili{} \ili{}\emph\ili{}{om\ili{} mannen\ili{} står\ili{} urørlig\ili{} og\ili{} venter}\ili{}.\ili{} \ili{}\\ili{}\\ili{} \ili{}%\ili{} i\ili{} bakgrunnen\ili{}.\ili{} \ili{}\\ili{}\\ili{}
\ili{}%\ili{} \ili{} \ili{} \ili{} \ili{} \ili{}{no\ili{} one}\ili{} lays\ili{} mark\ili{} to\ili{} \ili{}{if\ili{} \ili{}{the\ili{} man}\ili{} stands\ili{} motionless\ili{} and\ili{} waits}\ili{} \ili{}\\ili{}\\ili{} \ili{}%\ili{} in\ili{} \ili{}{the\ili{} background}\ili{}\\ili{}\\ili{}
\ili{}%\ili{}\glt\ili{} \ili{}`No\ili{} one\ili{} notices\ili{} if\ili{} the\ili{} man\ili{} stands\ili{} motionless\ili{} waiting\ili{}.\ili{}'\ili{} \ili{}%\ili{} in\ili{} the\ili{} back\ili{}'\ili{}
\ili{}%\ili{}\end\ili{}{xlist}\ili{}
\ili{}%\ili{}\z\ili{}
\ili{}
MWEs\ili{} that\ili{} subcategorize\ili{} for\ili{} different\ili{} types\ili{} of\ili{} complements\ili{} are\ili{} represented\ili{} in\ili{} the\ili{} lexicon\ili{} with\ili{} one\ili{} frame\ili{} for\ili{} each\ili{} \ili{}\isi\ili{}{subcategorization}\ili{} pattern\ili{}.\ili{}
In\ili{} the\ili{} \ili{}\isi\ili{}{template}\ili{} invocations\ili{} in\ili{} \ili{}(\ili{}\ref\ili{}{ex\ili{}:mweiness\ili{}:leggemerketilframes}\ili{})\ili{},\ili{} POBJ\ili{},\ili{} PCOMP\ili{} and\ili{} PCOMPint\ili{} represent\ili{} the\ili{} different\ili{} types\ili{} of\ili{} prepositional\ili{} complements\ili{} that\ili{} occur\ili{} with\ili{} \ili{}\emph\ili{}{legge\ili{} merke\ili{} til}\ili{} in\ili{} \ili{}(\ili{}\ref\ili{}{ex\ili{}:mweiness\ili{}:leggemerketil\ili{}-a}\ili{})\ili{},\ili{} \ili{}(\ili{}\ref\ili{}{ex\ili{}:mweiness\ili{}:leggemerketil\ili{}-b}\ili{})\ili{},\ili{} and\ili{} \ili{}(\ili{}\ref\ili{}{ex\ili{}:mweiness\ili{}:leggemerketil\ili{}-c}\ili{})\ili{} respectively\ili{}.\ili{}
\ili{}	\ili{}
\ili{}\ea\ili{}\label\ili{}{ex\ili{}:mweiness\ili{}:leggemerketilframes}\ili{}
\ili{}\begin\ili{}{xlist}\ili{}
\ili{}\ex\ili{} \ili{}\label\ili{}{ex\ili{}:mweiness\ili{}:leggemerketilframes\ili{}-a}\ili{}@\ili{}(VPIDIOM\ili{}-INDEFOBJ\ili{}-POBJ\ili{} legge\ili{} merke\ili{} til\ili{})\ili{} \ili{}\\ili{}\\ili{} \ili{}
\ili{}\ex\ili{} \ili{}\label\ili{}{ex\ili{}:mweiness\ili{}:leggemerketilframes\ili{}-b}\ili{}@\ili{}(VPIDIOM\ili{}-INDEFOBJ\ili{}-PCOMP\ili{} \ili{} legge\ili{} merke\ili{} til\ili{})\ili{} \ili{}\\ili{}\\ili{}
\ili{}\ex\ili{} \ili{}\label\ili{}{ex\ili{}:mweiness\ili{}:leggemerketilframes\ili{}-c}\ili{}@\ili{}(VPIDIOM\ili{}-INDEFOBJ\ili{}-PCOMPint\ili{} legge\ili{} merke\ili{} til\ili{})\ili{} \ili{}\\ili{}\\ili{}
\ili{}\end\ili{}{xlist}\ili{}
\ili{}\z\ili{}	\ili{}
\ili{}
\ili{}%Variation\ili{} in\ili{} the\ili{} complementation\ili{} is\ili{} limited\ili{} for\ili{} individual\ili{} MWEs\ili{},\ili{} as\ili{} is\ili{} the\ili{} case\ili{} for\ili{} \ili{}\emph\ili{}{legge\ili{} merke\ili{} til}\ili{} which\ili{} have\ili{} three\ili{} frames\ili{} in\ili{} which\ili{} only\ili{} one\ili{} of\ili{} the\ili{} complements\ili{} varies\ili{}.\ili{} \ili{}
\ili{}%Each\ili{} MWE\ili{} that\ili{} is\ili{} added\ili{} to\ili{} the\ili{} lexicon\ili{} thus\ili{} requires\ili{} a\ili{} relatively\ili{} low\ili{} number\ili{} of\ili{} \ili{}\isi\ili{}{subcategorization}\ili{} frames\ili{}.\ili{}
\ili{}%Variation\ili{} in\ili{} complementation\ili{} patterns\ili{} is\ili{} mainly\ili{} due\ili{} to\ili{} differences\ili{} between\ili{} MWEs\ili{},\ili{} and\ili{} this\ili{} is\ili{} reflected\ili{} in\ili{} the\ili{} total\ili{} number\ili{} of\ili{} unique\ili{} frames\ili{}.\ili{}
\ili{}%For\ili{} instance\ili{},\ili{} there\ili{} are\ili{} more\ili{} than\ili{} 80\ili{} templates\ili{} for\ili{} \ili{}\isi\ili{}{phrasal\ili{} verbs}\ili{},\ili{} which\ili{} may\ili{} be\ili{} grouped\ili{} into\ili{} seven\ili{} main\ili{} classes\ili{} according\ili{} to\ili{} the\ili{} types\ili{} and\ili{} number\ili{} of\ili{} complements\ili{} \ili{}(Table\ili{} \ili{}\ref\ili{}{tab\ili{}:mweiness\ili{}:phrasaltypes}\ili{})\ili{}.\ili{}\footnote\ili{}{This\ili{} number\ili{} does\ili{} not\ili{} include\ili{} particles\ili{} and\ili{} selected\ili{} prepositions\ili{} in\ili{} VP\ili{} idioms\ili{},\ili{} cf\ili{}.\ili{} the\ili{} numbers\ili{} given\ili{} in\ili{} Section\ili{}\ref\ili{}{sec\ili{}:mweiness\ili{}:prtprepverbs}\ili{}.}\ili{} \ili{}
\ili{}
While\ili{} Table\ili{} \ili{}\ref\ili{}{tab\ili{}:mweiness\ili{}:selected}\ili{} shows\ili{} different\ili{} types\ili{} of\ili{} selected\ili{} complements\ili{},\ili{} examples\ili{} \ili{}(\ili{}\ref\ili{}{ex\ili{}:mweiness\ili{}:leggemerketil\ili{}-a}\ili{})\ili{}-\ili{}(\ili{}\ref\ili{}{ex\ili{}:mweiness\ili{}:leggemerketil\ili{}-c}\ili{})\ili{} illustrate\ili{} how\ili{} one\ili{} MWE\ili{} may\ili{} take\ili{} different\ili{} types\ili{} of\ili{} free\ili{} complements\ili{}.\ili{}
In\ili{} both\ili{} cases\ili{},\ili{} we\ili{} see\ili{} that\ili{} variation\ili{} in\ili{} the\ili{} complementation\ili{} is\ili{} limited\ili{} for\ili{} individual\ili{} MWEs\ili{}.\ili{}
While\ili{} \ili{}\emph\ili{}{slå\ili{} følge}\ili{} \ili{}`accompany\ili{}'\ili{},\ili{} \ili{}\emph\ili{}{gå\ili{} i\ili{} oppløsning}\ili{} \ili{}`dissolve\ili{}'\ili{} and\ili{} the\ili{} other\ili{} examples\ili{} in\ili{} Table\ili{} \ili{}\ref\ili{}{tab\ili{}:mweiness\ili{}:selected}\ili{} all\ili{} have\ili{} fixed\ili{} complement\ili{} structures\ili{},\ili{} \ili{}\emph\ili{}{legge\ili{} merke\ili{} til}\ili{} \ili{}`notice\ili{}'\ili{} has\ili{} three\ili{} different\ili{} frames\ili{} in\ili{} which\ili{} only\ili{} one\ili{} of\ili{} the\ili{} complements\ili{} varies\ili{}.\ili{}
To\ili{} give\ili{} an\ili{} impression\ili{} of\ili{} the\ili{} variety\ili{} of\ili{} complementation\ili{} patterns\ili{} in\ili{} the\ili{} lexicon\ili{} it\ili{} is\ili{} thus\ili{} necessary\ili{} to\ili{} turn\ili{} to\ili{} the\ili{} inventory\ili{} of\ili{} unique\ili{} frames\ili{},\ili{} reflected\ili{} in\ili{} the\ili{} number\ili{} of\ili{} templates\ili{}.\ili{}
For\ili{} instance\ili{},\ili{} NorGram\ili{} has\ili{} more\ili{} than\ili{} 80\ili{} templates\ili{} for\ili{} \ili{}\isi\ili{}{phrasal\ili{} verbs}\ili{};\ili{} these\ili{} may\ili{} be\ili{} grouped\ili{} into\ili{} seven\ili{} main\ili{} classes\ili{} according\ili{} to\ili{} the\ili{} types\ili{} and\ili{} number\ili{} of\ili{} complements\ili{} \ili{}(Table\ili{} \ili{}\ref\ili{}{tab\ili{}:mweiness\ili{}:phrasaltypes}\ili{})\ili{}.\ili{} \ili{}%\ili{}\footnote\ili{}{This\ili{} number\ili{} does\ili{} not\ili{} include\ili{} particles\ili{} and\ili{} selected\ili{} prepositions\ili{} in\ili{} VP\ili{} idioms\ili{},\ili{} cf\ili{}.\ili{} the\ili{} numbers\ili{} given\ili{} in\ili{} Section\ili{} \ili{}\ref\ili{}{sec\ili{}:mweiness\ili{}:prtprepverbs}\ili{}.}\ili{} \ili{}
\ili{}
\ili{}\begin\ili{}{table}\ili{}
\ili{} \ili{} \ili{}\begin\ili{}{tabular}\ili{}{l\ili{}@\ili{}{\ili{}~}l\ili{}@\ili{}{\ili{}~}l}\ili{}
\ili{} \ili{} \ili{} \ili{} \ili{}\lsptoprule\ili{}
\ili{} \ili{} \ili{} \ili{} Type\ili{} \ili{}&\ili{} Example\ili{} frame\ili{} \ili{}&\ili{} Example\ili{} MWE\ili{} \ili{}\\ili{}\\ili{}
\ili{} \ili{} \ili{} \ili{} \ili{}\midrule\ili{}
\ili{} \ili{} \ili{} \ili{} V\ili{} \ili{}+\ili{} PRT\ili{} \ili{}&\ili{} V\ili{}-SUBJ\ili{}-PRT\ili{} \ili{}&\ili{} \ili{}\emph\ili{}{stryke\ili{} med}\ili{} \ili{}\\ili{}\\ili{}
\ili{} \ili{} \ili{} \ili{} V\ili{} \ili{}+\ili{} PRT\ili{} \ili{}+\ili{} 1\ili{} complement\ili{} \ili{}&\ili{} V\ili{}-SUBJ\ili{}-PRT\ili{}-XCOMP\ili{} \ili{}&\ili{} \ili{}\emph\ili{}{få\ili{} til}\ili{} \ili{}\\ili{}\\ili{}
\ili{} \ili{} \ili{} \ili{} V\ili{} \ili{}+\ili{} PRT\ili{} \ili{}+\ili{} 2\ili{} complements\ili{} \ili{}&\ili{} V\ili{}-SUBJ\ili{}-PRT\ili{}-OBJ\ili{}-OBJ\ili{} \ili{}&\ili{} \ili{}\emph\ili{}{gjøre\ili{} etter}\ili{}\\ili{}\\ili{} \ili{} \ili{} \ili{} \ili{} \ili{}\hline\ili{}
\ili{} \ili{} \ili{} \ili{} V\ili{} \ili{}+\ili{} PPsel\ili{} \ili{}&\ili{} V\ili{}-SUBJ\ili{}-POBJ\ili{} \ili{}&\ili{} \ili{}\emph\ili{}{advare\ili{} mot}\ili{} \ili{}\\ili{}\\ili{}
\ili{} \ili{} \ili{} \ili{} V\ili{} \ili{}+\ili{} PPsel\ili{} \ili{}+\ili{} 1\ili{} complement\ili{} \ili{} \ili{}&\ili{} V\ili{}-SUBJ\ili{}-OBJ\ili{}-PACOMP\ili{} \ili{}&\ili{} \ili{}\emph\ili{}{erklære\ili{} for}\ili{} \ili{}\\ili{}\\ili{}
\ili{} \ili{} \ili{} \ili{} V\ili{} \ili{}+\ili{} PPsel\ili{} \ili{}+\ili{} 2\ili{} complements\ili{} \ili{}&\ili{} V\ili{}-SUBJ\ili{}-OBJ\ili{}-POBJ\ili{}-PCOMP\ili{} \ili{}&\ili{} \ili{}\emph\ili{}{vedde\ili{} med\ili{} på}\ili{} \ili{}\\ili{}\\ili{} \ili{}\hline\ili{}
\ili{} \ili{} \ili{} \ili{} V\ili{} \ili{}+\ili{} PRT\ili{} \ili{}+\ili{} PPsel\ili{} \ili{}&\ili{} V\ili{}-SUBJ\ili{}-PRT\ili{}-POBJ\ili{} \ili{}&\ili{} \ili{}\emph\ili{}{gå\ili{} med\ili{} på}\ili{} \ili{}\\ili{}\\ili{}
\ili{} \ili{} \ili{} \ili{} V\ili{} \ili{}+\ili{} PRT\ili{} \ili{}+\ili{} PPsel\ili{} \ili{}+\ili{} 1\ili{} complement\ili{} \ili{}&\ili{} V\ili{}-SUBJ\ili{}-PRT\ili{}-OBJ\ili{}-POBJ\ili{} \ili{}&\ili{} \ili{}\emph\ili{}{venne\ili{} av\ili{} med}\ili{} \ili{}\\ili{}\\ili{} \ili{}%\ili{} gjøre\ili{} om\ili{} til\ili{}
\ili{} \ili{} \ili{} \ili{} \ili{}\lspbottomrule\ili{}
\ili{} \ili{} \ili{}\end\ili{}{tabular}\ili{}
\ili{} \ili{} \ili{}\caption\ili{}{Main\ili{} types\ili{} of\ili{} complementation\ili{} patterns\ili{} in\ili{} phrasal\ili{} verbs\ili{} in\ili{} NorGram}\ili{}
\ili{} \ili{} \ili{}\label\ili{}{tab\ili{}:mweiness\ili{}:phrasaltypes}\ili{}
\ili{}\end\ili{}{table}\ili{}
\ili{}
Table\ili{} \ili{}\ref\ili{}{tab\ili{}:mweiness\ili{}:phrasaltypes}\ili{} presents\ili{} the\ili{} different\ili{} types\ili{} of\ili{} complementation\ili{} patterns\ili{} for\ili{} \ili{}\isi\ili{}{verb\ili{}-particle\ili{} constructions}\ili{},\ili{} \ili{}\isi\ili{}{prepositional\ili{} verbs}\ili{},\ili{} and\ili{} \ili{}\isi\ili{}{verb\ili{}-particle\ili{} constructions}\ili{} with\ili{} selected\ili{} prepositions\ili{}.\ili{}
The\ili{} first\ili{} column\ili{} in\ili{} the\ili{} table\ili{} is\ili{} the\ili{} pattern\ili{} type\ili{},\ili{} represented\ili{} in\ili{} terms\ili{} of\ili{} the\ili{} main\ili{} complement\ili{}(s\ili{})\ili{},\ili{} which\ili{} may\ili{} be\ili{} a\ili{} particle\ili{} \ili{}(PRT\ili{})\ili{},\ili{} a\ili{} selected\ili{} prepositional\ili{} \ili{}\isi\ili{}{phrase}\ili{} \ili{}(PPsel\ili{})\ili{},\ili{} or\ili{} both\ili{},\ili{} and\ili{} the\ili{} number\ili{} of\ili{} additional\ili{} complements\ili{}.\ili{}\footnote\ili{}{\ili{}`\ili{}`Main\ili{} complement\ili{}'\ili{}'\ili{} in\ili{} this\ili{} context\ili{} refers\ili{} to\ili{} the\ili{} selected\ili{} complement\ili{} which\ili{} determines\ili{} the\ili{} type\ili{} of\ili{} the\ili{} overall\ili{} construction\ili{},\ili{} such\ili{} as\ili{} PRT\ili{} in\ili{} \ili{}\isi\ili{}{verb\ili{}-particle\ili{} constructions}\ili{}.}\ili{} \ili{}
\ili{}%The\ili{} first\ili{} column\ili{} in\ili{} the\ili{} table\ili{} is\ili{} the\ili{} pattern\ili{} type\ili{},\ili{} represented\ili{} in\ili{} terms\ili{} of\ili{} the\ili{} main\ili{} complement\ili{} \ili{}(PRT\ili{},\ili{} PPsel\ili{} or\ili{} both\ili{})\ili{} and\ili{} the\ili{} number\ili{} of\ili{} additional\ili{} complements\ili{}.\ili{}\footnote\ili{}{\ili{}`\ili{}`Main\ili{} complement\ili{}'\ili{}'\ili{} in\ili{} this\ili{} context\ili{} refers\ili{} to\ili{} the\ili{} selected\ili{} complement\ili{} which\ili{} determines\ili{} the\ili{} type\ili{} of\ili{} the\ili{} overall\ili{} construction\ili{},\ili{} such\ili{} as\ili{} PRT\ili{} in\ili{} \ili{}\isi\ili{}{verb\ili{}-particle\ili{} constructions}\ili{}.}\ili{} \ili{}
Examples\ili{} of\ili{} \ili{}\isi\ili{}{subcategorization}\ili{} frames\ili{} for\ili{} each\ili{} type\ili{} are\ili{} given\ili{} in\ili{} the\ili{} second\ili{} column\ili{} using\ili{} \ili{}\isi\ili{}{template}\ili{} names\ili{}.\ili{}
The\ili{} example\ili{} MWEs\ili{},\ili{} represented\ili{} in\ili{} the\ili{} table\ili{} with\ili{} only\ili{} their\ili{} fixed\ili{} components\ili{},\ili{} are\ili{} instances\ili{} of\ili{} the\ili{} example\ili{} frames\ili{} and\ili{} are\ili{} discussed\ili{} in\ili{} more\ili{} detail\ili{} in\ili{} \ili{}(\ili{}\ref\ili{}{ex\ili{}:mweiness\ili{}:strykemed}\ili{})\ili{}-\ili{}(\ili{}\ref\ili{}{ex\ili{}:mweiness\ili{}:venneavmed}\ili{})\ili{}.\ili{}
\ili{}
\ili{} As\ili{} Table\ili{} \ili{}\ref\ili{}{tab\ili{}:mweiness\ili{}:phrasaltypes}\ili{} shows\ili{},\ili{} \ili{}\isi\ili{}{verb\ili{}-particle\ili{} constructions}\ili{} in\ili{} NorGram\ili{} may\ili{} either\ili{} be\ili{} intransitive\ili{},\ili{} such\ili{} as\ili{} \ili{}\emph\ili{}{stryke\ili{} med}\ili{} \ili{}`die\ili{}'\ili{} in\ili{} \ili{}(\ili{}\ref\ili{}{ex\ili{}:mweiness\ili{}:strykemed}\ili{})\ili{},\ili{} or\ili{} have\ili{} one\ili{} or\ili{} two\ili{} free\ili{} complements\ili{},\ili{} such\ili{} as\ili{} \ili{}\emph\ili{}{få\ili{} noe\ili{} til}\ili{} \ili{}`accomplish\ili{} something\ili{}'\ili{} in\ili{} \ili{}(\ili{}\ref\ili{}{ex\ili{}:mweiness\ili{}:fåtil\ili{}-1}\ili{})\ili{} and\ili{} \ili{}\emph\ili{}{gjøre\ili{} noen\ili{} noe\ili{} etter}\ili{} \ili{}`repeat\ili{} something\ili{} after\ili{} someone\ili{}'\ili{} in\ili{} \ili{}(\ili{}\ref\ili{}{ex\ili{}:mweiness\ili{}:gjøreetter}\ili{})\ili{}.\ili{}
\ili{}
\ili{}%As\ili{} Table\ili{} \ili{}\ref\ili{}{tab\ili{}:mweiness\ili{}:phrasaltypes}\ili{} shows\ili{},\ili{} particle\ili{}-verb\ili{} constructions\ili{} in\ili{} NorGram\ili{} are\ili{} either\ili{} intransitive\ili{},\ili{} such\ili{} as\ili{} \ili{}\emph\ili{}{stryke\ili{} med}\ili{} \ili{}`die\ili{}'\ili{} in\ili{} \ili{}(\ili{}\ref\ili{}{ex\ili{}:mweiness\ili{}:strykemed}\ili{})\ili{} which\ili{} only\ili{} requires\ili{} a\ili{} subject\ili{},\ili{} transitive\ili{} like\ili{} \ili{}\emph\ili{}{få\ili{} noe\ili{} til}\ili{} \ili{}`accomplish\ili{} something\ili{}'\ili{} in\ili{} \ili{}(\ili{}\ref\ili{}{ex\ili{}:mweiness\ili{}:fåtil}\ili{})\ili{},\ili{} \ili{} or\ili{} \ili{}`\ili{}`ditransitive\ili{}'\ili{}'\ili{} with\ili{} two\ili{} free\ili{} complements\ili{} such\ili{} as\ili{} \ili{} \ili{}\emph\ili{}{gjøre\ili{} noen\ili{} noe\ili{} etter}\ili{} \ili{}`do\ili{} someone\ili{} something\ili{} after\ili{}'\ili{} \ili{} in\ili{} \ili{}(\ili{}\ref\ili{}{ex\ili{}:mweiness\ili{}:gjøreetter}\ili{})\ili{}.\ili{}
\ili{}%As\ili{} Table\ili{} \ili{}\ref\ili{}{tab\ili{}:mweiness\ili{}:phrasaltypes}\ili{} shows\ili{},\ili{} \ili{}\isi\ili{}{verb\ili{}-particle\ili{} constructions}\ili{} in\ili{} NorGram\ili{} are\ili{} either\ili{} intransitive\ili{},\ili{} such\ili{} as\ili{} \ili{}\emph\ili{}{stryke\ili{} med}\ili{} \ili{}`die\ili{}'\ili{} in\ili{} \ili{}(\ili{}\ref\ili{}{ex\ili{}:mweiness\ili{}:strykemed}\ili{})\ili{} which\ili{} only\ili{} requires\ili{} a\ili{} subject\ili{},\ili{} or\ili{} may\ili{} have\ili{} one\ili{} or\ili{} two\ili{} free\ili{} complements\ili{},\ili{} such\ili{} as\ili{} \ili{}\emph\ili{}{få\ili{} noe\ili{} til}\ili{} \ili{}`accomplish\ili{} something\ili{}'\ili{} in\ili{} \ili{}(\ili{}\ref\ili{}{ex\ili{}:mweiness\ili{}:fåtil\ili{}-1}\ili{})\ili{} and\ili{} \ili{}\emph\ili{}{gjøre\ili{} noen\ili{} noe\ili{} etter}\ili{} \ili{}`repeat\ili{} something\ili{} after\ili{} someone\ili{}'\ili{} in\ili{} \ili{}(\ili{}\ref\ili{}{ex\ili{}:mweiness\ili{}:gjøreetter}\ili{})\ili{}.\ili{}
\ili{}
\ili{}%Og\ili{} vi\ili{} fortsetter\ili{} å\ili{} banke\ili{} deg\ili{} til\ili{} du\ili{} stryker\ili{} med\ili{}!\ili{}»\ili{}
\ili{}%Treebank\ili{}:\ili{} nob\ili{}-novel_2\ili{} version\ili{}:\ili{} 2016\ili{}-05\ili{}-17\ili{};\ili{} Document\ili{}:\ili{} Brandstadmoen\ili{},\ili{} Geir\ili{}:\ili{} På\ili{} barndommens\ili{} solbrente\ili{} enger\ili{};\ili{} grammar\ili{}:\ili{} \ili{}\ili\ili{}{Norwegian}\ili{} Bokmål\ili{}
\ili{}%Sentence\ili{} \ili{}#4024\ili{}
\ili{}\ea\ili{}\label\ili{}{ex\ili{}:mweiness\ili{}:strykemed}\ili{}
\ili{}\gll\ili{} Og\ili{} vi\ili{} fortsetter\ili{} å\ili{} banke\ili{} deg\ili{} til\ili{} du\ili{} \ili{}\textbf\ili{}{stryker}\ili{} \ili{}\textbf\ili{}{med}\ili{}!\ili{} \ili{}\\ili{}\\ili{} \ili{}
\ili{} and\ili{} we\ili{} continue\ili{} to\ili{} beat\ili{} you\ili{} until\ili{} you\ili{} stroke\ili{} with\ili{} \ili{}\\ili{}\\ili{}
\ili{}\glt\ili{} \ili{}`And\ili{} we\ili{} will\ili{} continue\ili{} to\ili{} beat\ili{} you\ili{} until\ili{} you\ili{}'re\ili{} dead\ili{}!\ili{}'\ili{} \ili{}\\ili{}\\ili{} \ili{}
\ili{}\z\ili{}
\ili{}
\ili{}\ea\ili{}\label\ili{}{ex\ili{}:mweiness\ili{}:fåtil\ili{}-1}\ili{}
\ili{}\gll\ili{} Nå\ili{} \ili{}\textbf\ili{}{fikk}\ili{} han\ili{} \ili{}\textbf\ili{}{til}\ili{} å\ili{} tenke\ili{} igjen\ili{}.\ili{} \ili{}\\ili{}\\ili{}
\ili{} now\ili{} got\ili{} he\ili{} to\ili{} to\ili{} think\ili{} again\ili{} \ili{}\\ili{}\\ili{}
\ili{}\glt\ili{} \ili{}`Now\ili{} he\ili{} managed\ili{} to\ili{} think\ili{} again\ili{}.\ili{}'\ili{} \ili{}\\ili{}\\ili{}
\ili{}\z\ili{}
\ili{}
\ili{}%\ili{}%Ikke\ili{} mange\ili{} kunne\ili{} ha\ili{} gjort\ili{} ham\ili{} noe\ili{} slikt\ili{} etter\ili{}!\ili{}
\ili{}%\ili{}%Treebank\ili{}:\ili{} nob\ili{}-child\ili{} version\ili{}:\ili{} 2016\ili{}-05\ili{}-17\ili{};\ili{} Document\ili{}:\ili{} Pedersen\ili{},\ili{} Erling\ili{}:\ili{} Hvile\ili{} i\ili{} grønne\ili{} enger\ili{};\ili{} grammar\ili{}:\ili{} \ili{}\ili\ili{}{Norwegian}\ili{} Bokmål\ili{}
\ili{}%\ili{}%Sentence\ili{} \ili{}#2774\ili{}
\ili{}%\ili{}
\ili{}\ea\ili{}\label\ili{}{ex\ili{}:mweiness\ili{}:gjøreetter}\ili{}
\ili{}\gll\ili{} Ikke\ili{} mange\ili{} kunne\ili{} ha\ili{} \ili{}\textbf\ili{}{gjort}\ili{} ham\ili{} noe\ili{} slikt\ili{} \ili{}\textbf\ili{}{etter}\ili{}!\ili{} \ili{}\\ili{}\\ili{} \ili{}
\ili{} \ili{} not\ili{} many\ili{} could\ili{} have\ili{} done\ili{} him\ili{} something\ili{} \ili{}{like\ili{} that}\ili{} after\ili{} \ili{}\\ili{}\\ili{}
\ili{}\glt\ili{} \ili{}`Not\ili{} many\ili{} people\ili{} could\ili{} have\ili{} done\ili{} what\ili{} he\ili{} did\ili{}!\ili{}'\ili{} \ili{}\\ili{}\\ili{} \ili{}
\ili{}\z\ili{}
\ili{}
\ili{}
The\ili{} example\ili{} \ili{}\emph\ili{}{få\ili{} noe\ili{} til}\ili{} in\ili{} \ili{}(\ili{}\ref\ili{}{ex\ili{}:mweiness\ili{}:fåtil\ili{}-1}\ili{})\ili{} is\ili{} an\ili{} instantiation\ili{} of\ili{} the\ili{} frame\ili{} V\ili{}-SUBJ\ili{}-PRT\ili{}-XCOMP\ili{},\ili{} with\ili{} one\ili{} free\ili{} complement\ili{} in\ili{} the\ili{} form\ili{} of\ili{} the\ili{} infinitival\ili{} complement\ili{} \ili{}\emph\ili{}{å\ili{} tenke\ili{} igjen}\ili{} \ili{}`to\ili{} think\ili{} again\ili{}'\ili{}.\ili{} \ili{} \ili{}
There\ili{} is\ili{} one\ili{} other\ili{} frame\ili{} for\ili{} this\ili{} particular\ili{} MWE\ili{} in\ili{} the\ili{} lexicon\ili{},\ili{} with\ili{} a\ili{} nominal\ili{} object\ili{} instead\ili{} of\ili{} the\ili{} infinitival\ili{} complement\ili{} \ili{}(V\ili{}-SUBJ\ili{}-PRT\ili{}-OBJ\ili{})\ili{}.\ili{} \ili{}
Example\ili{} \ili{}(\ili{}\ref\ili{}{ex\ili{}:mweiness\ili{}:fåtil\ili{}-2}\ili{})\ili{} illustrates\ili{} this\ili{} complement\ili{} structure\ili{}.\ili{}
\ili{}
\ili{}%\ili{}«Dette\ili{} er\ili{} hva\ili{} du\ili{} fikk\ili{} til\ili{}.\ili{}»\ili{}
\ili{}%Treebank\ili{}:\ili{} nob\ili{}-avis\ili{} version\ili{}:\ili{} 2016\ili{}-05\ili{}-17\ili{};\ili{} Document\ili{}:\ili{} Østli\ili{},\ili{} Kjetil\ili{}:\ili{} \ili{};\ili{} grammar\ili{}:\ili{} \ili{}\ili\ili{}{Norwegian}\ili{} Bokmål\ili{}
\ili{}%Sentence\ili{} \ili{}#6\ili{}
\ili{}\ea\ili{}\label\ili{}{ex\ili{}:mweiness\ili{}:fåtil\ili{}-2}\ili{}
\ili{}\gll\ili{} Dette\ili{} er\ili{} hva\ili{} du\ili{} \ili{}\textbf\ili{}{fikk}\ili{} \ili{}\textbf\ili{}{til}\ili{}.\ili{} \ili{}\\ili{}\\ili{} \ili{}
\ili{} this\ili{} is\ili{} what\ili{} you\ili{} got\ili{} to\ili{} \ili{}\\ili{}\\ili{}
\ili{}\glt\ili{} \ili{}`This\ili{} is\ili{} what\ili{} you\ili{} accomplished\ili{}.\ili{}'\ili{} \ili{}\\ili{}\\ili{} \ili{}
\ili{}\z\ili{}
\ili{}
The\ili{} lexicon\ili{} also\ili{} has\ili{} a\ili{} frame\ili{} for\ili{} \ili{}\emph\ili{}{få\ili{} til}\ili{} with\ili{} two\ili{} free\ili{} complements\ili{},\ili{} in\ili{} the\ili{} form\ili{} of\ili{} an\ili{} object\ili{} and\ili{} an\ili{} infinitival\ili{} complement\ili{}.\ili{} \ili{}
The\ili{} difference\ili{} in\ili{} the\ili{} number\ili{} of\ili{} complements\ili{} also\ili{} yields\ili{} a\ili{} difference\ili{} in\ili{} meaning\ili{},\ili{} as\ili{} shown\ili{} in\ili{} \ili{}(\ili{}\ref\ili{}{ex\ili{}:mweiness\ili{}:fåtil\ili{}-3}\ili{})\ili{}.\ili{} \ili{}
These\ili{} should\ili{} thus\ili{} be\ili{} considered\ili{} different\ili{} MWEs\ili{}.\ili{}
\ili{}
\ili{}%http\ili{}:\ili{}/\ili{}/clarino\ili{}.uib\ili{}.no\ili{}/iness\ili{}/lfg\ili{}-sentence\ili{}?treebank\ili{}=nob\ili{}-avis\ili{}&version\ili{}=2016\ili{}-05\ili{}-17\ili{}&unique\ili{}-id\ili{}=3856989\ili{}&solution\ili{}-nr\ili{}=2\ili{}&session\ili{}-id\ili{}=241982373627117\ili{}
\ili{}\ea\ili{}\label\ili{}{ex\ili{}:mweiness\ili{}:fåtil\ili{}-3}\ili{}
\ili{}\gll\ili{} Hvorfor\ili{} \ili{}\textbf\ili{}{får}\ili{} vi\ili{} ikke\ili{} dem\ili{} \ili{}\textbf\ili{}{til}\ili{} å\ili{} bli\ili{}?\ili{} \ili{}\\ili{}\\ili{}
\ili{} why\ili{} get\ili{} we\ili{} not\ili{} them\ili{} to\ili{} to\ili{} stay\ili{} \ili{}\\ili{}\\ili{}
\ili{}\glt\ili{} \ili{}`Why\ili{} can\ili{}'t\ili{} we\ili{} make\ili{} them\ili{} stay\ili{}?\ili{}'\ili{} \ili{}\\ili{}\\ili{} \ili{}
\ili{}\z\ili{}
\ili{}
The\ili{} last\ili{} type\ili{} of\ili{} verb\ili{}-particle\ili{} construction\ili{} in\ili{} Table\ili{} \ili{}\ref\ili{}{tab\ili{}:mweiness\ili{}:phrasaltypes}\ili{},\ili{} with\ili{} two\ili{} free\ili{} complements\ili{} in\ili{} addition\ili{} to\ili{} the\ili{} particle\ili{},\ili{} is\ili{} exemplified\ili{} with\ili{} the\ili{} frame\ili{} V\ili{}-SUBJ\ili{}-PRT\ili{}-OBJ\ili{}-OBJ\ili{}.\ili{}
This\ili{} argument\ili{} structure\ili{},\ili{} illustrated\ili{} in\ili{} \ili{}(\ili{}\ref\ili{}{ex\ili{}:mweiness\ili{}:gjøreetter}\ili{})\ili{} for\ili{} \ili{}\emph\ili{}{gjøre\ili{} noen\ili{} noe\ili{} etter}\ili{},\ili{} involves\ili{} both\ili{} an\ili{} indirect\ili{} object\ili{} \ili{}(OBJ\ili{}-BEN\ili{})\ili{},\ili{} \ili{}\emph\ili{}{ham}\ili{} \ili{}`him\ili{}'\ili{},\ili{} and\ili{} a\ili{} direct\ili{} object\ili{} \ili{}(OBJ\ili{})\ili{},\ili{} \ili{}\emph\ili{}{noe\ili{} slikt}\ili{} \ili{}`something\ili{} like\ili{} that\ili{}'\ili{} \ili{}(OBJ\ili{}-BEN\ili{} is\ili{} shortened\ili{} to\ili{} OBJ\ili{} in\ili{} the\ili{} name\ili{} of\ili{} the\ili{} \ili{}\isi\ili{}{template}\ili{})\ili{}.\ili{}
\ili{}%This\ili{} argument\ili{} structure\ili{},\ili{} illustrated\ili{} in\ili{} \ili{}(\ili{}\ref\ili{}{ex\ili{}:mweiness\ili{}:gjøreetter}\ili{})\ili{} for\ili{} \ili{}\emph\ili{}{gjøre\ili{} noen\ili{} noe\ili{} etter}\ili{},\ili{} is\ili{} idiosyncratic\ili{} because\ili{} verbs\ili{} normally\ili{} do\ili{} not\ili{} take\ili{} two\ili{} objects\ili{}.\ili{}
\ili{}%Other\ili{} frames\ili{} with\ili{} two\ili{} free\ili{} complements\ili{} are\ili{} V\ili{}-SUBJ\ili{}-PRT\ili{}-POBJrefl\ili{}-COMPat\ili{} and\ili{} V\ili{}-SUBJexpl\ili{}-PRT\ili{}-OBJ\ili{}-COMP\ili{}.\ili{}
A\ili{} second\ili{} frame\ili{} of\ili{} this\ili{} type\ili{},\ili{} which\ili{} is\ili{} slightly\ili{} more\ili{} complex\ili{} with\ili{} a\ili{} nominal\ili{} object\ili{} and\ili{} a\ili{} clausal\ili{} complement\ili{} \ili{}(COMP\ili{})\ili{} as\ili{} well\ili{} as\ili{} an\ili{} expletive\ili{} subject\ili{},\ili{} is\ili{} V\ili{}-SUBJexpl\ili{}-PRT\ili{}-OBJ\ili{}-COMP\ili{}.\ili{} \ili{}
The\ili{} MWE\ili{} \ili{}\emph\ili{}{det\ili{} faller\ili{} noen\ili{} noe\ili{} inn}\ili{} \ili{}`something\ili{} occurs\ili{} to\ili{} someone\ili{}'\ili{} in\ili{} \ili{}(\ili{}\ref\ili{}{ex\ili{}:mweiness\ili{}:falleinn}\ili{})\ili{} is\ili{} an\ili{} example\ili{} of\ili{} this\ili{} frame\ili{},\ili{} literally\ili{} translating\ili{} into\ili{} \ili{}`it\ili{} falls\ili{} someone\ili{} something\ili{} in\ili{}'\ili{}.\ili{}
Except\ili{} for\ili{} the\ili{} expletive\ili{} subject\ili{} and\ili{} the\ili{} particle\ili{},\ili{} the\ili{} frame\ili{} has\ili{} the\ili{} same\ili{} arguments\ili{} as\ili{} V\ili{}-SUBJ\ili{}-OBJ\ili{}-COMP\ili{} for\ili{} single\ili{} verbs\ili{} such\ili{} as\ili{} \ili{}\emph\ili{}{forklare}\ili{} \ili{}`explain\ili{}'\ili{}.\ili{}
The\ili{} frame\ili{} is\ili{} thus\ili{} regular\ili{} in\ili{} terms\ili{} of\ili{} argument\ili{} structure\ili{}.\ili{} \ili{} \ili{} \ili{}
\ili{}%\ili{}\footnote\ili{}{Compare\ili{} to\ili{} the\ili{} frame\ili{} V\ili{}-SUBJ\ili{}-OBJ\ili{}-COMP\ili{} for\ili{} the\ili{} verb\ili{} \ili{}\emph\ili{}{forklare}\ili{} \ili{}`explain\ili{}'\ili{}.}\ili{} \ili{}
\ili{}%http\ili{}:\ili{}/\ili{}/clarino\ili{}.uib\ili{}.no\ili{}/iness\ili{}/lfg\ili{}-sentence\ili{}?treebank\ili{}=nob\ili{}-avis\ili{}&version\ili{}=2016\ili{}-05\ili{}-17\ili{}&unique\ili{}-id\ili{}=3789406\ili{}&solution\ili{}-nr\ili{}=0\ili{}&session\ili{}-id\ili{}=241982373627117\ili{}
\ili{}%Det\ili{} falt\ili{} ikke\ili{} britene\ili{} inn\ili{} at\ili{} særlig\ili{} mange\ili{} hadde\ili{} lyst\ili{}.\ili{}
\ili{}
\ili{}\ea\ili{}\label\ili{}{ex\ili{}:mweiness\ili{}:falleinn}\ili{}
\ili{}\gll\ili{} \ili{} \ili{} Det\ili{} \ili{}\textbf\ili{}{falt}\ili{} ikke\ili{} britene\ili{} \ili{}\textbf\ili{}{inn}\ili{} at\ili{} særlig\ili{} mange\ili{} hadde\ili{} lyst\ili{}.\ili{} \ili{}\\ili{}\\ili{}
\ili{} \ili{} \ili{} \ili{} \ili{} \ili{} \ili{} \ili{} it\ili{} fell\ili{} not\ili{} \ili{}{the\ili{} Brits}\ili{} in\ili{} that\ili{} particularly\ili{} many\ili{} had\ili{} desire\ili{} \ili{}\\ili{}\\ili{}
\ili{}\glt\ili{} \ili{} \ili{}`It\ili{} did\ili{} not\ili{} occur\ili{} to\ili{} the\ili{} Brits\ili{} that\ili{} more\ili{} than\ili{} a\ili{} few\ili{} should\ili{} want\ili{} to\ili{}.\ili{}'\ili{} \ili{}\\ili{}\\ili{} \ili{}
\ili{}\z\ili{}
\ili{}
In\ili{} contrast\ili{} to\ili{} \ili{}\isi\ili{}{verb\ili{}-particle\ili{} constructions}\ili{} which\ili{} may\ili{} be\ili{} intransitive\ili{},\ili{} \ili{}\isi\ili{}{prepositional\ili{} verbs}\ili{} will\ili{} always\ili{} have\ili{} a\ili{} free\ili{} complement\ili{},\ili{} introduced\ili{} by\ili{} the\ili{} selected\ili{} preposition\ili{}.\ili{}
Prepositional\ili{} verbs\ili{} can\ili{} subcategorize\ili{} for\ili{} exactly\ili{} one\ili{} prepositional\ili{} \ili{}\isi\ili{}{phrase}\ili{},\ili{} as\ili{} in\ili{}
\ili{}\emph\ili{}{advare\ili{} mot\ili{} noe}\ili{} \ili{}`warn\ili{} against\ili{} something\ili{}'\ili{} in\ili{} \ili{}(\ili{}\ref\ili{}{ex\ili{}:mweiness\ili{}:advaremot}\ili{})\ili{},\ili{} where\ili{} \ili{}\emph\ili{}{mot\ili{} segregering}\ili{} \ili{}`against\ili{} segregation\ili{}'\ili{} is\ili{} a\ili{} PPsel\ili{}.\ili{} \ili{}
\ili{}
\ili{}%Treebank\ili{}:\ili{} nob\ili{}-avis\ili{} version\ili{}:\ili{} 2016\ili{}-05\ili{}-17\ili{};\ili{} Document\ili{}:\ili{} Stærk\ili{},\ili{} Bjørn\ili{}:\ili{} \ili{};\ili{} grammar\ili{}:\ili{} \ili{}\ili\ili{}{Norwegian}\ili{} Bokmål\ili{}
\ili{}%Sentence\ili{} \ili{}#62\ili{}:\ili{} Han\ili{} advarer\ili{} mot\ili{} segregering\ili{}.\ili{}
\ili{}%http\ili{}:\ili{}/\ili{}/clarino\ili{}.uib\ili{}.no\ili{}/iness\ili{}/lfg\ili{}-sentence\ili{}?treebank\ili{}=nob\ili{}-avis\ili{}&version\ili{}=2016\ili{}-05\ili{}-17\ili{}&unique\ili{}-id\ili{}=3789421\ili{}&solution\ili{}-nr\ili{}=0\ili{}
\ili{}\ea\ili{}\label\ili{}{ex\ili{}:mweiness\ili{}:advaremot}\ili{}
\ili{}\gll\ili{} \ili{} \ili{} Han\ili{} \ili{}\textbf\ili{}{advarer}\ili{} \ili{}\textbf\ili{}{mot}\ili{} segregering\ili{}.\ili{} \ili{}\\ili{}\\ili{}
\ili{} \ili{} \ili{} \ili{} \ili{} \ili{} \ili{} \ili{} he\ili{} warns\ili{} against\ili{} segregation\ili{}\\ili{}\\ili{}
\ili{}\glt\ili{} \ili{} \ili{}`He\ili{} warns\ili{} against\ili{} segregation\ili{}.\ili{}'\ili{} \ili{}\\ili{}\\ili{} \ili{}
\ili{}\z\ili{}
\ili{}
Similar\ili{} to\ili{} \ili{}\isi\ili{}{verb\ili{}-particle\ili{} constructions}\ili{},\ili{} the\ili{} \ili{}\isi\ili{}{prepositional\ili{} verbs}\ili{} in\ili{} NorGram\ili{} can\ili{} take\ili{} one\ili{} or\ili{} two\ili{} complements\ili{} in\ili{} addition\ili{} to\ili{} the\ili{} selected\ili{} complement\ili{}.\ili{} \ili{}
In\ili{} \ili{}(\ili{}\ref\ili{}{ex\ili{}:mweiness\ili{}:erklærefor}\ili{})\ili{},\ili{} \ili{}\emph\ili{}{erklære\ili{} noen\ili{} for\ili{} noe}\ili{} \ili{}`declare\ili{} someone\ili{} something\ili{}'\ili{} has\ili{} one\ili{} complement\ili{},\ili{} the\ili{} free\ili{} object\ili{} \ili{}\emph\ili{}{marken}\ili{} \ili{}`the\ili{} mark\ili{}'\ili{},\ili{} \ili{} in\ili{} addition\ili{} to\ili{} the\ili{} selected\ili{} prepositional\ili{} \ili{}\isi\ili{}{phrase}\ili{} \ili{} \ili{}\emph\ili{}{for\ili{} død}\ili{} \ili{}`for\ili{} dead\ili{}'\ili{}.\ili{} \ili{}
\ili{}
\ili{}%Treebank\ili{}:\ili{} nob\ili{}-fn\ili{} version\ili{}:\ili{} 2016\ili{}-05\ili{}-17\ili{};\ili{} Document\ili{}:\ili{} Spilde\ili{},\ili{} Ingrid\ili{}:\ili{} Økonomisk\ili{} oppstandelse\ili{} i\ili{} virtuell\ili{} verden\ili{};\ili{} grammar\ili{}:\ili{} \ili{}\ili\ili{}{Norwegian}\ili{} Bokmål\ili{}
\ili{}%Sentence\ili{} \ili{}#24\ili{}:\ili{} \ili{}–\ili{} Der\ili{} måtte\ili{} myndighetene\ili{} erklære\ili{} marken\ili{} for\ili{} død\ili{}.\ili{}
\ili{}\ea\ili{}\label\ili{}{ex\ili{}:mweiness\ili{}:erklærefor}\ili{}
\ili{}\gll\ili{} \ili{} \ili{} Der\ili{} måtte\ili{} myndighetene\ili{} \ili{}\textbf\ili{}{erklære}\ili{} marken\ili{} \ili{}\textbf\ili{}{for}\ili{} død\ili{}.\ili{} \ili{}\\ili{}\\ili{}
\ili{} \ili{} \ili{} \ili{} \ili{} \ili{} \ili{} \ili{} there\ili{} \ili{}{had\ili{} to}\ili{} \ili{}{the\ili{} government}\ili{} declare\ili{} \ili{}{the\ili{} mark}\ili{} for\ili{} dead\ili{} \ili{}\\ili{}\\ili{}
\ili{}\glt\ili{} \ili{}`There\ili{} the\ili{} authorities\ili{} had\ili{} to\ili{} declare\ili{} the\ili{} \ili{}(\ili{}\ili\ili{}{German}\ili{})\ili{} mark\ili{} dead\ili{}.\ili{}'\ili{}\\ili{}\\ili{} \ili{}
\ili{}\z\ili{}
\ili{}
\ili{}%The\ili{} MWE\ili{} in\ili{} \ili{}(\ili{}\ref\ili{}{ex\ili{}:mweiness\ili{}:erklærefor}\ili{})\ili{} examplifies\ili{} the\ili{} pattern\ili{} type\ili{} V\ili{} \ili{}+\ili{} PPsel\ili{} \ili{}+\ili{} 1\ili{} complement\ili{} in\ili{} Table\ili{} \ili{}\ref\ili{}{tab\ili{}:mweiness\ili{}:phrasaltypes}\ili{},\ili{} and\ili{} has\ili{} the\ili{} \ili{}\isi\ili{}{subcategorization}\ili{} frame\ili{} V\ili{}-SUBJ\ili{}-OBJ\ili{}-PACOMP\ili{}.\ili{} \ili{} \ili{} \ili{}
The\ili{} relevant\ili{} frame\ili{} in\ili{} \ili{}(\ili{}\ref\ili{}{ex\ili{}:mweiness\ili{}:erklærefor}\ili{})\ili{} is\ili{} V\ili{}-SUBJ\ili{}-OBJ\ili{}-PACOMP\ili{},\ili{} where\ili{} PACOMP\ili{} is\ili{} the\ili{} selected\ili{} prepositional\ili{} \ili{}\isi\ili{}{phrase}\ili{}.\ili{}
While\ili{} PPsel\ili{} is\ili{} the\ili{} c\ili{}-structure\ili{} category\ili{} for\ili{} constituents\ili{} headed\ili{} by\ili{} selected\ili{} prepositions\ili{} and\ili{} may\ili{} refer\ili{} to\ili{} any\ili{} type\ili{} of\ili{} prepositional\ili{} complement\ili{},\ili{} PACOMP\ili{} is\ili{} a\ili{} syntactic\ili{} variable\ili{} that\ili{} reflects\ili{} the\ili{} type\ili{} of\ili{} complement\ili{}.\ili{} \ili{}%\ili{},\ili{} cf\ili{}.\ili{} POBJ\ili{},\ili{} PCOMP\ili{} and\ili{} PXCOMP\ili{} in\ili{} \ili{}(\ili{}\ref\ili{}{ex\ili{}:mweiness\ili{}:leggemerketilframes}\ili{})\ili{}.\ili{} \ili{}
In\ili{} this\ili{} case\ili{},\ili{} the\ili{} preposition\ili{} \ili{}\emph\ili{}{for}\ili{} takes\ili{} the\ili{} adjectival\ili{} predicative\ili{} complement\ili{} \ili{}\emph\ili{}{død}\ili{} \ili{}`dead\ili{}'\ili{}.\ili{} \ili{}%\ili{}\footnote\ili{}{A\ili{} nominal\ili{} predicative\ili{} complement\ili{} \ili{}(as\ili{} in\ili{} \ili{}\emph\ili{}{declare\ili{} someone\ili{} king}\ili{})\ili{} would\ili{} be\ili{} represented\ili{} with\ili{} the\ili{} category\ili{} NCOMP\ili{} in\ili{} the\ili{} \ili{}\isi\ili{}{subcategorization}\ili{} frame\ili{}.}\ili{}
\ili{}
The\ili{} final\ili{} type\ili{} of\ili{} prepositional\ili{} verb\ili{} in\ili{} Table\ili{} \ili{}\ref\ili{}{tab\ili{}:mweiness\ili{}:phrasaltypes}\ili{} is\ili{} illustrated\ili{} in\ili{} \ili{}(\ili{}\ref\ili{}{ex\ili{}:mweiness\ili{}:veddemedpå}\ili{})\ili{} with\ili{} the\ili{} MWE\ili{} \ili{}\emph\ili{}{vedde\ili{} noe\ili{} med\ili{} noen\ili{} på\ili{} noe}\ili{} \ili{}`bet\ili{} something\ili{} with\ili{} someone\ili{} on\ili{} something\ili{}'\ili{}.\ili{} \ili{}
\ili{}
\ili{}\ea\ili{}\label\ili{}{ex\ili{}:mweiness\ili{}:veddemedpå}\ili{}
\ili{}\gll\ili{} \ili{} \ili{} Abrams\ili{} \ili{}\textbf\ili{}{veddet}\ili{} en\ili{} sigarett\ili{} \ili{}\textbf\ili{}{med}\ili{} Browne\ili{} \ili{}\textbf\ili{}{på}\ili{} at\ili{} det\ili{} regnet\ili{}.\ili{} \ili{}\\ili{}\\ili{}
\ili{} \ili{} \ili{} \ili{} \ili{} \ili{} \ili{} \ili{} Abrams\ili{} bet\ili{} a\ili{} cigarette\ili{} with\ili{} Brown\ili{} on\ili{} that\ili{} it\ili{} rained\ili{} \ili{}\\ili{}\\ili{}
\ili{}\glt\ili{} \ili{}`Abrams\ili{} bet\ili{} a\ili{} cigarette\ili{} with\ili{} Brown\ili{} that\ili{} it\ili{} \ili{}{was\ili{} raining}\ili{}.\ili{}'\ili{} \ili{}\\ili{}\\ili{} \ili{}
\ili{}\z\ili{}
\ili{}
This\ili{} example\ili{} is\ili{} an\ili{} instance\ili{} of\ili{} the\ili{} frame\ili{} V\ili{}-SUBJ\ili{}-OBJ\ili{}-POBJ\ili{}-PCOMP\ili{},\ili{} which\ili{} has\ili{} two\ili{} complements\ili{} in\ili{} addition\ili{} to\ili{} a\ili{} PPsel\ili{},\ili{} in\ili{} this\ili{} case\ili{} a\ili{} free\ili{} object\ili{} and\ili{} a\ili{} second\ili{} PPsel\ili{}.\ili{}
The\ili{} free\ili{} object\ili{} is\ili{} \ili{}\emph\ili{}{en\ili{} sigarett}\ili{} \ili{}`a\ili{} cigarette\ili{}'\ili{}.\ili{} \ili{}
In\ili{} the\ili{} first\ili{} PPsel\ili{},\ili{} which\ili{} corresponds\ili{} to\ili{} POBJ\ili{} in\ili{} the\ili{} \ili{}\isi\ili{}{subcategorization}\ili{} frame\ili{},\ili{} the\ili{} selected\ili{} preposition\ili{} \ili{}\emph\ili{}{med}\ili{} \ili{}`with\ili{}'\ili{} takes\ili{} the\ili{} nominal\ili{} object\ili{} \ili{}\emph\ili{}{Brown}\ili{}.\ili{} \ili{}
In\ili{} the\ili{} second\ili{} PPsel\ili{},\ili{} corresponding\ili{} to\ili{} PCOMP\ili{},\ili{} \ili{} the\ili{} preposition\ili{} \ili{}\emph\ili{}{på}\ili{} \ili{}`on\ili{}'\ili{} takes\ili{} the\ili{} clausal\ili{} complement\ili{} \ili{}\emph\ili{}{at\ili{} det\ili{} regnet}\ili{} \ili{}`that\ili{} it\ili{} was\ili{} raining\ili{}'\ili{}.\ili{}
\ili{}
Like\ili{} \ili{}\isi\ili{}{prepositional\ili{} verbs}\ili{},\ili{} \ili{}\isi\ili{}{verb\ili{}-particle\ili{} constructions}\ili{} with\ili{} selected\ili{} prepositions\ili{} always\ili{} subcategorize\ili{} for\ili{} at\ili{} least\ili{} one\ili{} free\ili{} complement\ili{}.\ili{} \ili{}
Such\ili{} constructions\ili{} can\ili{} have\ili{} one\ili{} complement\ili{},\ili{} as\ili{} in\ili{} \ili{}\emph\ili{}{gå\ili{} med\ili{} på\ili{} noe}\ili{} \ili{}`go\ili{} along\ili{} with\ili{} something\ili{}'\ili{} in\ili{} \ili{}(\ili{}\ref\ili{}{ex\ili{}:mweiness\ili{}:gåmedpå}\ili{})\ili{},\ili{} which\ili{} is\ili{} an\ili{} instance\ili{} of\ili{} the\ili{} frame\ili{} V\ili{}-SUBJ\ili{}-PRT\ili{}-POBJ\ili{}.\ili{}
\ili{} \ili{}
\ili{}%Treebank\ili{}:\ili{} nob\ili{}-avis\ili{} version\ili{}:\ili{} 2016\ili{}-05\ili{}-17\ili{};\ili{} Document\ili{}:\ili{} Melgård\ili{},\ili{} Marie\ili{};\ili{} Kagge\ili{},\ili{} Gunnar\ili{}:\ili{} \ili{};\ili{} grammar\ili{}:\ili{} \ili{}\ili\ili{}{Norwegian}\ili{} Bokmål\ili{}
\ili{}%Sentence\ili{} \ili{}#27\ili{}:\ili{} I\ili{} land\ili{} som\ili{} Sverige\ili{} gikk\ili{} fagbevegelsen\ili{} med\ili{} på\ili{} de\ili{} nye\ili{} tankene\ili{}.\ili{}
\ili{}%http\ili{}:\ili{}/\ili{}/clarino\ili{}.uib\ili{}.no\ili{}/iness\ili{}/lfg\ili{}-sentence\ili{}?treebank\ili{}=nob\ili{}-avis\ili{}&version\ili{}=2016\ili{}-05\ili{}-17\ili{}&unique\ili{}-id\ili{}=3935000\ili{}&solution\ili{}-nr\ili{}=36\ili{}
\ili{}\ea\ili{}\label\ili{}{ex\ili{}:mweiness\ili{}:gåmedpå}\ili{}
\ili{}\gll\ili{} \ili{} \ili{} I\ili{} land\ili{} som\ili{} Sverige\ili{} \ili{}\textbf\ili{}{gikk}\ili{} fagbevegelsen\ili{} \ili{}\textbf\ili{}{med}\ili{} \ili{}\textbf\ili{}{på}\ili{} de\ili{} nye\ili{} tankene\ili{}.\ili{} \ili{}\\ili{}\\ili{}
\ili{} \ili{} \ili{} \ili{} \ili{} \ili{} \ili{} \ili{} in\ili{} countries\ili{} like\ili{} Sweden\ili{} went\ili{} \ili{}{the\ili{} unions}\ili{} with\ili{} on\ili{} the\ili{} new\ili{} thoughts\ili{} \ili{}\\ili{}\\ili{}
\ili{}\glt\ili{} \ili{}`In\ili{} countries\ili{} like\ili{} Sweden\ili{} the\ili{} unions\ili{} went\ili{} along\ili{} with\ili{} the\ili{} new\ili{} ideas\ili{}.\ili{}'\ili{} \ili{}\\ili{}\\ili{}
\ili{}\z\ili{}
\ili{}
In\ili{} \ili{}(\ili{}\ref\ili{}{ex\ili{}:mweiness\ili{}:gåmedpå}\ili{})\ili{},\ili{} the\ili{} particle\ili{} is\ili{} \ili{}\emph\ili{}{med}\ili{} \ili{}`with\ili{}'\ili{},\ili{} and\ili{} the\ili{} free\ili{} argument\ili{} \ili{} \ili{}\emph\ili{}{de\ili{} nye\ili{} tankene}\ili{} \ili{}`the\ili{} new\ili{} thoughts\ili{}'\ili{} is\ili{} the\ili{} complement\ili{} of\ili{} the\ili{} selected\ili{} preposition\ili{} \ili{}\emph\ili{}{på}\ili{} \ili{}`on\ili{}'\ili{}.\ili{} \ili{} \ili{}
The\ili{} prepositional\ili{} complement\ili{} could\ili{},\ili{} however\ili{},\ili{} also\ili{} be\ili{} clausal\ili{},\ili{} as\ili{} in\ili{} \ili{}(\ili{}\ref\ili{}{ex\ili{}:mweiness\ili{}:gåmedpå\ili{}-2}\ili{})\ili{},\ili{} or\ili{} infinitival\ili{},\ili{} as\ili{} in\ili{} \ili{}(\ili{}\ref\ili{}{ex\ili{}:mweiness\ili{}:gåmedpå\ili{}-3}\ili{})\ili{}.\ili{}
With\ili{} the\ili{} infinitival\ili{} complement\ili{},\ili{} there\ili{} is\ili{} a\ili{} shift\ili{} in\ili{} meaning\ili{} from\ili{} \ili{}`go\ili{} along\ili{} with\ili{}'\ili{}/\ili{}`admit\ili{}'\ili{} to\ili{} \ili{}`agree\ili{}'\ili{}.\ili{} \ili{}
\ili{}
\ili{}\ea\ili{}\label\ili{}{ex\ili{}:mweiness\ili{}:gåmedpå\ili{}-2}\ili{}
\ili{}\gll\ili{} \ili{} \ili{} Han\ili{} vil\ili{} ikke\ili{} \ili{}\textbf\ili{}{gå}\ili{} \ili{}\textbf\ili{}{med}\ili{} \ili{}\textbf\ili{}{på}\ili{} at\ili{} hun\ili{} er\ili{} utpreget\ili{} modig\ili{}.\ili{} \ili{}\\ili{}\\ili{}
\ili{} \ili{} \ili{} \ili{} \ili{} \ili{} \ili{} \ili{} he\ili{} will\ili{} not\ili{} go\ili{} with\ili{} on\ili{} that\ili{} she\ili{} is\ili{} exceptionally\ili{} brave\ili{} \ili{}\\ili{}\\ili{}
\ili{}\glt\ili{} \ili{}`He\ili{} will\ili{} not\ili{} admit\ili{} that\ili{} she\ili{} is\ili{} exceptionally\ili{} brave\ili{}.\ili{}'\ili{} \ili{}\\ili{}\\ili{}
\ili{}\z\ili{}
\ili{}
\ili{}\ea\ili{}\label\ili{}{ex\ili{}:mweiness\ili{}:gåmedpå\ili{}-3}\ili{}
\ili{}\gll\ili{} \ili{} \ili{} Til\ili{} Libbys\ili{} forbauselse\ili{} hadde\ili{} Jerry\ili{} \ili{}\textbf\ili{}{gått}\ili{} \ili{}\textbf\ili{}{med}\ili{} \ili{}\textbf\ili{}{på}\ili{} å\ili{} prøve\ili{}.\ili{} \ili{}\\ili{}\\ili{}
\ili{} \ili{} \ili{} \ili{} \ili{} \ili{} \ili{} \ili{} to\ili{} Libby\ili{}'s\ili{} surprise\ili{} had\ili{} Jerry\ili{} gone\ili{} with\ili{} on\ili{} to\ili{} try\ili{} \ili{}\\ili{}\\ili{}
\ili{}\glt\ili{} \ili{}`To\ili{} Libby\ili{}'s\ili{} surprise\ili{},\ili{} Jerry\ili{} had\ili{} agreed\ili{} to\ili{} try\ili{}.\ili{}'\ili{} \ili{}\\ili{}\\ili{}
\ili{}\z\ili{}
\ili{}
\ili{}
\ili{}%http\ili{}:\ili{}/\ili{}/clarino\ili{}.uib\ili{}.no\ili{}/iness\ili{}/lfg\ili{}-sentence\ili{}?treebank\ili{}=nob\ili{}-avis\ili{}&version\ili{}=2016\ili{}-05\ili{}-17\ili{}&unique\ili{}-id\ili{}=3857732\ili{}&solution\ili{}-nr\ili{}=1\ili{}&session\ili{}-id\ili{}=241982373627117\ili{}
\ili{}%\ili{}–\ili{} Jeg\ili{} går\ili{} ut\ili{} fra\ili{} at\ili{} de\ili{} var\ili{} under\ili{} sterkt\ili{} politisk\ili{} press\ili{}.\ili{}
\ili{}
\ili{}%\ili{}(V\ili{}-SUBJ\ili{}-PRT\ili{}-PXCOMP\ili{} ende\ili{} ende\ili{} opp\ili{} med\ili{})\ili{}
\ili{}%http\ili{}:\ili{}/\ili{}/clarino\ili{}.uib\ili{}.no\ili{}/iness\ili{}/lfg\ili{}-sentence\ili{}?treebank\ili{}=nob\ili{}-lbk\ili{}-sa\ili{}&version\ili{}=2016\ili{}-05\ili{}-17\ili{}&unique\ili{}-id\ili{}=4393751\ili{}&solution\ili{}-nr\ili{}=0\ili{}&session\ili{}-id\ili{}=241982373627117\ili{}
\ili{}%Jeg\ili{} ville\ili{} ikke\ili{} ende\ili{} opp\ili{} med\ili{} et\ili{} kulehull\ili{} i\ili{} brystet\ili{}.\ili{}
\ili{}
Verb\ili{}-particle\ili{} constructions\ili{} with\ili{} selected\ili{} prepositions\ili{} may\ili{} also\ili{} have\ili{} two\ili{} free\ili{} complements\ili{}.\ili{} \ili{}
This\ili{} is\ili{} the\ili{} case\ili{} for\ili{} \ili{}\emph\ili{}{venne\ili{} noen\ili{} av\ili{} med\ili{} noe}\ili{} \ili{}`wean\ili{} someone\ili{} of\ili{} something\ili{}'\ili{} in\ili{} \ili{}(\ili{}\ref\ili{}{ex\ili{}:mweiness\ili{}:venneavmed}\ili{})\ili{}.\ili{}
This\ili{} example\ili{},\ili{} instantiating\ili{} the\ili{} frame\ili{} V\ili{}-SUBJ\ili{}-PRT\ili{}-OBJ\ili{}-POBJ\ili{},\ili{} has\ili{} the\ili{} particle\ili{} \ili{}\emph\ili{}{av}\ili{} \ili{}`off\ili{}'\ili{},\ili{} the\ili{} free\ili{} pronominal\ili{} object\ili{} \ili{}\emph\ili{}{meg}\ili{} \ili{}`me\ili{}'\ili{},\ili{} and\ili{} the\ili{} selected\ili{} prepositional\ili{} object\ili{} \ili{}\emph\ili{}{med\ili{} det}\ili{} \ili{}`with\ili{} that\ili{}'\ili{}.\ili{}
Also\ili{} here\ili{},\ili{} the\ili{} prepositional\ili{} complement\ili{} may\ili{} vary\ili{}.\ili{}
The\ili{} alternative\ili{} frame\ili{} is\ili{} V\ili{}-SUBJ\ili{}-PRT\ili{}-OBJ\ili{}-PXCOMP\ili{},\ili{} allowing\ili{} an\ili{} infinitival\ili{} prepositional\ili{} complement\ili{},\ili{} but\ili{} in\ili{} this\ili{} case\ili{} yielding\ili{} no\ili{} difference\ili{} in\ili{} meaning\ili{}.\ili{}
\ili{}
\ili{}%Mor\ili{} klarte\ili{} aldri\ili{} å\ili{} venne\ili{} meg\ili{} av\ili{} med\ili{} det\ili{}.\ili{}
\ili{}%http\ili{}:\ili{}/\ili{}/clarino\ili{}.uib\ili{}.no\ili{}/iness\ili{}/lfg\ili{}-sentence\ili{}?treebank\ili{}=nob\ili{}-novel_2\ili{}&version\ili{}=2016\ili{}-05\ili{}-17\ili{}&unique\ili{}-id\ili{}=2170917\ili{}&solution\ili{}-nr\ili{}=0\ili{}
\ili{}\ea\ili{}\label\ili{}{ex\ili{}:mweiness\ili{}:venneavmed}\ili{}
\ili{}\gll\ili{} \ili{} \ili{} Mor\ili{} klarte\ili{} aldri\ili{} å\ili{} \ili{}\textbf\ili{}{venne}\ili{} meg\ili{} \ili{}\textbf\ili{}{av}\ili{} \ili{}\textbf\ili{}{med}\ili{} det\ili{}.\ili{} \ili{}\\ili{}\\ili{}
\ili{} \ili{} \ili{} \ili{} \ili{} \ili{} \ili{} \ili{} \ili{} mother\ili{} managed\ili{} never\ili{} to\ili{} accustom\ili{} me\ili{} off\ili{} with\ili{} that\ili{} \ili{}\\ili{}\\ili{}
\ili{}\glt\ili{} \ili{}`\ili{}{Mother}\ili{} never\ili{} managed\ili{} to\ili{} wean\ili{} me\ili{} of\ili{} that\ili{} habit\ili{}.\ili{}'\ili{} \ili{}\\ili{}\\ili{}
\ili{}\z\ili{}
\ili{}
\ili{}%A\ili{} slightly\ili{} more\ili{} complex\ili{} example\ili{} of\ili{} this\ili{} pattern\ili{} is\ili{} the\ili{} frame\ili{} V\ili{}-SUBJ\ili{}-PRT\ili{}-POBJrefl\ili{}-COMPat\ili{}.\ili{}
\ili{}%This\ili{} is\ili{} the\ili{} \ili{}\isi\ili{}{template}\ili{} for\ili{} MWEs\ili{} with\ili{} a\ili{} PPsel\ili{} whose\ili{} complement\ili{} is\ili{} a\ili{} reflexive\ili{} object\ili{} \ili{}(POBJrefl\ili{})\ili{},\ili{} and\ili{} which\ili{} also\ili{} takes\ili{} a\ili{} clausal\ili{} complement\ili{} starting\ili{} with\ili{} the\ili{} subjunction\ili{} \ili{}\emph\ili{}{at}\ili{} \ili{}`that\ili{}'\ili{} \ili{}(COMPat\ili{})\ili{}.\ili{}\footnote\ili{}{In\ili{} contrast\ili{},\ili{} clausal\ili{} complements\ili{} of\ili{} the\ili{} type\ili{} COMP\ili{} do\ili{} not\ili{} require\ili{} the\ili{} subjunction\ili{}.}\ili{} \ili{} \ili{} \ili{}
\ili{}%The\ili{} MWE\ili{} \ili{}\emph\ili{}{ta\ili{} noe\ili{} inn\ili{} over\ili{} seg}\ili{} \ili{}`realize\ili{}/admit\ili{} something\ili{}'\ili{} has\ili{} this\ili{} structure\ili{}.\ili{} \ili{}
\ili{}%The\ili{} literal\ili{} translation\ili{} is\ili{} \ili{}`take\ili{} something\ili{} in\ili{} over\ili{} oneself\ili{}'\ili{}.\ili{}
\ili{}
The\ili{} examples\ili{} of\ili{} complementation\ili{} patterns\ili{} for\ili{} \ili{}\isi\ili{}{phrasal\ili{} verbs}\ili{} in\ili{} NorGram\ili{} show\ili{} that\ili{} the\ili{} subcategorizational\ili{} properties\ili{} of\ili{} MWEs\ili{} can\ili{} be\ili{} the\ili{} source\ili{} of\ili{} variation\ili{} both\ili{} at\ili{} the\ili{} syntactic\ili{} and\ili{} the\ili{} semantic\ili{} levels\ili{}.\ili{} \ili{}
We\ili{} have\ili{} seen\ili{} that\ili{} the\ili{} main\ili{} types\ili{} of\ili{} complementation\ili{} patterns\ili{} in\ili{} Table\ili{} \ili{}\ref\ili{}{tab\ili{}:mweiness\ili{}:phrasaltypes}\ili{} are\ili{} shared\ili{} by\ili{} a\ili{} number\ili{} of\ili{} \ili{}\isi\ili{}{subcategorization}\ili{} frames\ili{}.\ili{} \ili{}%\ili{},\ili{} which\ili{} indicates\ili{} a\ili{} certain\ili{} variance\ili{} within\ili{} each\ili{} type\ili{}.\ili{} \ili{}
Table\ili{} \ili{}\ref\ili{}{tab\ili{}:mweiness\ili{}:ppsel1comp}\ili{} presents\ili{} some\ili{} of\ili{} the\ili{} frames\ili{} that\ili{} are\ili{} variants\ili{} of\ili{} the\ili{} type\ili{} V\ili{} \ili{}+\ili{} PPsel\ili{} \ili{}+\ili{} 1\ili{} complement\ili{} in\ili{} Table\ili{} \ili{}\ref\ili{}{tab\ili{}:mweiness\ili{}:phrasaltypes}\ili{} \ili{}(\ili{}\isi\ili{}{prepositional\ili{} verbs}\ili{} with\ili{} one\ili{} free\ili{} complement\ili{})\ili{}.\ili{} \ili{}
The\ili{} frames\ili{} are\ili{} divided\ili{} into\ili{} groups\ili{} of\ili{} MWEs\ili{} that\ili{} share\ili{} the\ili{} same\ili{} or\ili{} similar\ili{} types\ili{} of\ili{} arguments\ili{},\ili{} resulting\ili{} in\ili{} five\ili{} categories\ili{} of\ili{} argument\ili{} patterning\ili{} for\ili{} this\ili{} type\ili{}.\ili{}\footnote\ili{}{Several\ili{} frames\ili{} of\ili{} this\ili{} type\ili{} are\ili{} not\ili{} listed\ili{} here\ili{},\ili{} including\ili{} frames\ili{} with\ili{} expletive\ili{} subjects\ili{} and\ili{} objects\ili{} and\ili{} subtypes\ili{} of\ili{} clausal\ili{} complements\ili{}.}\ili{} \ili{}
\ili{}%Presenting\ili{} these\ili{} argument\ili{} types\ili{} would\ili{} require\ili{} some\ili{} elaboration\ili{} and\ili{} they\ili{} are\ili{} thus\ili{} left\ili{} out\ili{} of\ili{} this\ili{} overview\ili{}.}\ili{}
\ili{}%Subcategorization\ili{} frames\ili{} for\ili{} new\ili{} MWEs\ili{} have\ili{} so\ili{} far\ili{} been\ili{} added\ili{} to\ili{} the\ili{} lexicon\ili{} under\ili{} the\ili{} relevant\ili{} \ili{}\isi\ili{}{lexical\ili{} entry}\ili{},\ili{} ensuring\ili{} that\ili{} the\ili{} MWE\ili{} will\ili{} receive\ili{} an\ili{} idiomatic\ili{} analysis\ili{} during\ili{} parsing\ili{}.\ili{} \ili{}
While\ili{} the\ili{} current\ili{} section\ili{} provides\ili{} only\ili{} superficial\ili{} observations\ili{} about\ili{} the\ili{} types\ili{} of\ili{} MWE\ili{} argument\ili{} patterns\ili{} in\ili{} the\ili{} NorGram\ili{} lexicon\ili{},\ili{} it\ili{} seems\ili{} that\ili{} a\ili{} more\ili{} systematic\ili{} study\ili{} of\ili{} their\ili{} subcategorizational\ili{} properties\ili{} could\ili{} provide\ili{} useful\ili{} information\ili{} about\ili{} MWE\ili{} types\ili{} and\ili{} tokens\ili{} and\ili{} perhaps\ili{} also\ili{} new\ili{} insights\ili{} into\ili{} the\ili{} relationship\ili{} between\ili{} argument\ili{} patterns\ili{} and\ili{} the\ili{} semantics\ili{} of\ili{} MWEs\ili{}.\ili{}
\ili{} \ili{}
\ili{}\begin\ili{}{table}\ili{}
\ili{} \ili{} \ili{}\begin\ili{}{tabular}\ili{}{ll}\ili{}
\ili{} \ili{} \ili{} \ili{} \ili{}\lsptoprule\ili{}
Complementation\ili{} type\ili{} \ili{}&\ili{} Subcategorization\ili{} frame\ili{} \ili{}\\ili{}\\ili{}
\ili{} \ili{} \ili{} \ili{} \ili{}\midrule\ili{}
Free\ili{} object\ili{} \ili{}&\ili{} V\ili{}-SUBJ\ili{}-OBJ\ili{}-PACOMP\ili{} \ili{}\\ili{}\\ili{}
and\ili{} PPsel\ili{} \ili{}&\ili{} V\ili{}-SUBJ\ili{}-OBJ\ili{}-POBJACOMP\ili{} \ili{}\\ili{}\\ili{}
\ili{}&\ili{} V\ili{}-SUBJ\ili{}-OBJ\ili{}-POBJNCOMP\ili{} \ili{}\\ili{}\\ili{}
\ili{}&\ili{} V\ili{}-SUBJ\ili{}-OBJ\ili{}-POBJ\ili{} \ili{}\\ili{}\\ili{}
\ili{}&\ili{} V\ili{}-SUBJ\ili{}-OBJ\ili{}-PCOMP\ili{} \ili{}\\ili{}\\ili{}
\ili{}&\ili{} V\ili{}-SUBJ\ili{}-OBJ\ili{}-PCOMPinf\ili{} \ili{}\\ili{}\\ili{}
\ili{}&\ili{} V\ili{}-SUBJ\ili{}-OBJ\ili{}-PCOMPint\ili{} \ili{}\\ili{}\\ili{}
\ili{}&\ili{} V\ili{}-SUBJ\ili{}-OBJ\ili{}-PXCOMP\ili{} \ili{}\\ili{}\\ili{} \ili{}
\ili{}&\ili{} V\ili{}-SUBJ\ili{}-INDOBJ\ili{}-POBJ\ili{} \ili{}\\ili{}\\ili{}
\ili{}%\ili{}&\ili{} V\ili{}-SUBJ\ili{}-OBJexpl\ili{}-POBJ\ili{} \ili{}\\ili{}\\ili{}
\ili{}&\ili{} V\ili{}-SUBJexpl\ili{}-OBJ\ili{}-POBJ\ili{} \ili{}\\ili{}\\ili{} \ili{}\hline\ili{}
Reflexive\ili{} object\ili{} \ili{} \ili{}&\ili{} V\ili{}-SUBJ\ili{}-OBJrefl\ili{}-POBJ\ili{} \ili{}\\ili{}\\ili{}
\ili{} and\ili{} PPsel\ili{} \ili{} \ili{}&\ili{} V\ili{}-SUBJ\ili{}-OBJrefl\ili{}-PCOMP\ili{} \ili{}\\ili{}\\ili{}
\ili{}&\ili{} V\ili{}-SUBJ\ili{}-OBJrefl\ili{}-PCOMPat\ili{} \ili{}\\ili{}\\ili{}
\ili{}&\ili{} V\ili{}-SUBJ\ili{}-OBJrefl\ili{}-PCOMPint\ili{} \ili{}\\ili{}\\ili{}
\ili{}&\ili{} V\ili{}-SUBJ\ili{}-OBJrefl\ili{}-PXCOMP\ili{} \ili{}\\ili{}\\ili{} \ili{}\hline\ili{}
PPsel\ili{} \ili{}&\ili{} V\ili{}-SUBJ\ili{}-POBJ\ili{}-COMP\ili{} \ili{}\\ili{}\\ili{}
and\ili{} free\ili{} \ili{} \ili{}&\ili{} V\ili{}-SUBJ\ili{}-POBJ\ili{}-XCOMP\ili{} \ili{}\\ili{}\\ili{}
nominal\ili{} \ili{} \ili{}&\ili{} V\ili{}-SUBJ\ili{}-POBJ\ili{}-OBL\ili{} \ili{}\\ili{}\\ili{}
complement\ili{} \ili{}&\ili{} V\ili{}-SUBJ\ili{}-POBJ\ili{}-OBLBEN\ili{} \ili{}\\ili{}\\ili{} \ili{}\hline\ili{}
Prepositional\ili{} reflexive\ili{} object\ili{} \ili{}&\ili{} V\ili{}-SUBJ\ili{}-POBJrefl\ili{}-OBJ\ili{} \ili{}\\ili{}\\ili{}
and\ili{} free\ili{} nominal\ili{} complement\ili{} \ili{}&\ili{} V\ili{}-SUBJ\ili{}-POBJrefl\ili{}-COMP\ili{} \ili{}\\ili{}\\ili{} \ili{}\hline\ili{}
PPsel\ili{} \ili{}&\ili{} V\ili{}-SUBJ\ili{}-POBJ\ili{}-PXCOMP\ili{} \ili{}\\ili{}\\ili{}
and\ili{} PPsel\ili{} \ili{}&\ili{} V\ili{}-SUBJ\ili{}-POBJrefl\ili{}-POBJ\ili{} \ili{}\\ili{}\\ili{} \ili{}\hline\ili{}
\ili{} \ili{} \ili{} \ili{} \ili{} \ili{} \ili{} \ili{} \ili{}\lspbottomrule\ili{}
\ili{} \ili{} \ili{}\end\ili{}{tabular}\ili{}
\ili{} \ili{} \ili{}\caption\ili{}{Some\ili{} variants\ili{} of\ili{} \ili{} V\ili{} \ili{}+\ili{} PPsel\ili{} \ili{}+\ili{} 1\ili{} complement}\ili{}
\ili{} \ili{} \ili{}\label\ili{}{tab\ili{}:mweiness\ili{}:ppsel1comp}\ili{}
\ili{}\end\ili{}{table}\ili{}
\ili{} \ili{}
\ili{} \ili{}%V\ili{} \ili{}+\ili{} PPsel\ili{} \ili{}+\ili{} 1\ili{} complement\ili{}
\ili{}%V\ili{}-SUBJ\ili{}-OBJ\ili{}-PACOMP\ili{}:\ili{} 2\ili{}
\ili{}%V\ili{}-SUBJ\ili{}-OBJ\ili{}-POBJACOMP\ili{}:\ili{} 9\ili{}
\ili{}%V\ili{}-SUBJ\ili{}-OBJ\ili{}-POBJACOMPsubj\ili{}:\ili{} 1\ili{}
\ili{}%V\ili{}-SUBJ\ili{}-OBJ\ili{}-POBJNCOMP\ili{}:\ili{} 15\ili{}
\ili{}%V\ili{}-SUBJ\ili{}-OBJ\ili{}-POBJ\ili{}:\ili{} 124\ili{}
\ili{}%V\ili{}-SUBJ\ili{}-OBJ\ili{}-PCOMP\ili{}:\ili{} 16\ili{}
\ili{}%V\ili{}-SUBJ\ili{}-OBJ\ili{}-PCOMPinf\ili{}:\ili{} 3\ili{}
\ili{}%V\ili{}-SUBJ\ili{}-OBJ\ili{}-PCOMPint\ili{}:\ili{} 4\ili{}
\ili{}%V\ili{}-SUBJ\ili{}-OBJ\ili{}-PXCOMP\ili{}:\ili{} 43\ili{}
\ili{}%V\ili{}-SUBJ\ili{}-OBJ\ili{}-PXCOMPfå\ili{}:\ili{} 1\ili{}
\ili{}%V\ili{}-SUBJ\ili{}-OBJ\ili{}-PXCOMPobjcont\ili{}:\ili{} 3\ili{}
\ili{}%V\ili{}-SUBJ\ili{}-OBJ\ili{}-PXCOMPsubjcont\ili{}:\ili{} 7\ili{}
\ili{}%\ili{}
\ili{}%V\ili{}-SUBJ\ili{}-OBJrefl\ili{}-POBJ\ili{}:\ili{} 115\ili{}
\ili{}%V\ili{}-SUBJ\ili{}-OBJrefl\ili{}-PCOMP\ili{}:\ili{} 3\ili{}
\ili{}%V\ili{}-SUBJ\ili{}-OBJrefl\ili{}-PCOMPat\ili{}:\ili{} 7\ili{}
\ili{}%V\ili{}-SUBJ\ili{}-OBJrefl\ili{}-PCOMPint\ili{}:\ili{} 3\ili{}
\ili{}%V\ili{}-SUBJ\ili{}-OBJrefl\ili{}-PXCOMP\ili{}:\ili{} 31\ili{}
\ili{}%\ili{}
\ili{}%V\ili{}-SUBJ\ili{}-POBJ\ili{}-COMP\ili{}:\ili{} 1\ili{}
\ili{}%V\ili{}-SUBJ\ili{}-POBJ\ili{}-XCOMP\ili{}:\ili{} 1\ili{}
\ili{}%V\ili{}-SUBJ\ili{}-POBJ\ili{}-OBL\ili{}:\ili{} 2\ili{}
\ili{}%V\ili{}-SUBJ\ili{}-POBJ\ili{}-OBLBEN\ili{}:\ili{} 1\ili{}
\ili{}%\ili{}
\ili{}%V\ili{}-SUBJ\ili{}-POBJrefl\ili{}-OBJ\ili{}:\ili{} 68\ili{}
\ili{}%V\ili{}-SUBJ\ili{}-POBJrefl\ili{}-COMP\ili{}:\ili{} 2\ili{}
\ili{}%\ili{}
\ili{}%V\ili{}-SUBJ\ili{}-POBJ\ili{}-PXCOMP\ili{}:\ili{} 1\ili{}
\ili{}%V\ili{}-SUBJ\ili{}-POBJrefl\ili{}-POBJ\ili{}:\ili{} 1\ili{}
\ili{}%\ili{}
\ili{}%V\ili{}-SUBJ\ili{}-INDOBJ\ili{}-POBJ\ili{}:\ili{} 2\ili{}
\ili{}%V\ili{}-SUBJ\ili{}-OBJexpl\ili{}-POBJ\ili{}:\ili{} 1\ili{}
\ili{}%V\ili{}-SUBJexpl\ili{}-OBJ\ili{}-POBJ\ili{}:\ili{} 1\ili{}
\ili{}
\ili{}
\ili{}%\ili{}\begin\ili{}{table}\ili{}
\ili{}%\ili{} \ili{} \ili{}\begin\ili{}{tabular}\ili{}{ll}\ili{}
\ili{}%\ili{} \ili{} \ili{} \ili{} \ili{}\lsptoprule\ili{}
\ili{}%\ili{} \ili{} \ili{} \ili{} Pattern\ili{} \ili{}&\ili{} Instances\ili{} \ili{}\\ili{}\\ili{}
\ili{}%\ili{} \ili{} \ili{} \ili{} \ili{}\midrule\ili{}
\ili{}%V\ili{}-SUBJ\ili{}-PRT\ili{}-OBJ\ili{} \ili{}&\ili{} 621\ili{} \ili{}\\ili{}\\ili{}
\ili{}%V\ili{}-SUBJ\ili{}-POBJ\ili{} \ili{}&\ili{} 423\ili{} \ili{}\\ili{}\\ili{}
\ili{}%V\ili{}-SUBJ\ili{}-PRT\ili{} \ili{}&\ili{} 315\ili{} \ili{}\\ili{}\\ili{}
\ili{}%V\ili{}-SUBJ\ili{}-OBJ\ili{}-POBJ\ili{} \ili{}&\ili{} 124\ili{} \ili{}\\ili{}\\ili{}
\ili{}%V\ili{}-SUBJ\ili{}-OBJrefl\ili{}-POBJ\ili{} \ili{}&\ili{} 115\ili{} \ili{}\\ili{}\\ili{}
\ili{}%V\ili{}-SUBJ\ili{}-OBJrefl\ili{}-PRT\ili{} \ili{}&\ili{} 105\ili{} \ili{}\\ili{}\\ili{}
\ili{}%\ili{}%V\ili{}-SUBJ\ili{}-PRT\ili{}-POBJ\ili{} \ili{}&\ili{} 74\ili{}
\ili{}%\ili{}%V\ili{}-SUBJ\ili{}-POBJrefl\ili{}-OBJ\ili{} \ili{}&\ili{} 68\ili{}
\ili{}%\ili{} \ili{} \ili{} \ili{} \ili{} \ili{} \ili{} \ili{} \ili{}\lspbottomrule\ili{}
\ili{}%\ili{} \ili{} \ili{}\end\ili{}{tabular}\ili{}
\ili{}%\ili{} \ili{} \ili{}\caption\ili{}{The\ili{} most\ili{} frequent\ili{} patterns\ili{} \ili{}(\ili{}>100\ili{})}\ili{}
\ili{}%\ili{} \ili{} \ili{}\label\ili{}{tab\ili{}:mweiness\ili{}:frequentpatterns}\ili{}
\ili{}%\ili{}\end\ili{}{table}\ili{}
\ili{}
\ili{}\section\ili{}{Conclusion}\ili{}\label\ili{}{sec\ili{}:mweiness\ili{}:conc}\ili{}
\ili{}
In\ili{} this\ili{} chapter\ili{} we\ili{} have\ili{} shown\ili{} how\ili{} the\ili{} modularization\ili{} of\ili{} NorGram\ili{} makes\ili{} it\ili{} possible\ili{} to\ili{} integrate\ili{} MWEs\ili{} into\ili{} the\ili{} LFG\ili{} analyses\ili{} in\ili{} a\ili{} way\ili{} that\ili{} does\ili{} justice\ili{} to\ili{} the\ili{} proper\ili{} division\ili{} of\ili{} labor\ili{} between\ili{} the\ili{} lexicon\ili{} and\ili{} the\ili{} grammar\ili{}.\ili{}
On\ili{} the\ili{} one\ili{} hand\ili{},\ili{} each\ili{} MWE\ili{} is\ili{} entered\ili{} into\ili{} the\ili{} lexicon\ili{} with\ili{} the\ili{} information\ili{} necessary\ili{} for\ili{} its\ili{} idiomatic\ili{} meaning\ili{}.\ili{}
On\ili{} the\ili{} other\ili{} hand\ili{},\ili{} the\ili{} syntactic\ili{} treatment\ili{} uses\ili{} ordinary\ili{} syntactic\ili{} rules\ili{} to\ili{} the\ili{} extent\ili{} that\ili{} the\ili{} flexibility\ili{} of\ili{} the\ili{} individual\ili{} MWE\ili{} allows\ili{}.\ili{}
\ili{}
Up\ili{} until\ili{} now\ili{} MWEs\ili{} have\ili{} been\ili{} severely\ili{} underrepresented\ili{} in\ili{} lexical\ili{} resources\ili{} for\ili{} \ili{}\ili\ili{}{Norwegian}\ili{},\ili{} as\ili{} they\ili{} have\ili{} been\ili{} for\ili{} many\ili{} other\ili{} languages\ili{}.\ili{}
The\ili{} main\ili{} strategy\ili{} for\ili{} NorGram\ili{} has\ili{} been\ili{} to\ili{} incorporate\ili{} them\ili{} into\ili{} the\ili{} lexicon\ili{} and\ili{} grammar\ili{} when\ili{} they\ili{} are\ili{} encountered\ili{} during\ili{} the\ili{} construction\ili{} of\ili{} NorGramBank\ili{}.\ili{}
MWEs\ili{} have\ili{} thus\ili{} been\ili{} added\ili{} to\ili{} NorGram\ili{} in\ili{} tandem\ili{} with\ili{} the\ili{} development\ili{} of\ili{} the\ili{} treebank\ili{}.\ili{}
As\ili{} a\ili{} natural\ili{} consequence\ili{} of\ili{} the\ili{} way\ili{} in\ili{} which\ili{} the\ili{} MWEs\ili{} are\ili{} represented\ili{} in\ili{} the\ili{} grammar\ili{} and\ili{} lexicon\ili{},\ili{} it\ili{} is\ili{} possible\ili{} to\ili{} search\ili{} for\ili{} the\ili{} various\ili{} MWE\ili{} types\ili{} in\ili{} the\ili{} treebank\ili{}.\ili{}
The\ili{} wealth\ili{} of\ili{} information\ili{} provided\ili{} by\ili{} the\ili{} LFG\ili{} representations\ili{} enables\ili{} search\ili{} for\ili{} many\ili{} different\ili{} properties\ili{} of\ili{} the\ili{} MWEs\ili{},\ili{} and\ili{} the\ili{} MWEs\ili{} may\ili{} be\ili{} recovered\ili{} in\ili{} all\ili{} the\ili{} syntactic\ili{} \ili{}\isi\ili{}{variations}\ili{} they\ili{} occur\ili{} in\ili{}.\ili{}
As\ili{} a\ili{} result\ili{},\ili{} NorGramBank\ili{} is\ili{} now\ili{} an\ili{} important\ili{} resource\ili{} for\ili{} studying\ili{} \ili{}\ili\ili{}{Norwegian}\ili{} MWEs\ili{} in\ili{} context\ili{}.\ili{}
\ili{}
\ili{}%Another\ili{} strategy\ili{} could\ili{} be\ili{} to\ili{} add\ili{} MWEs\ili{} found\ili{} through\ili{} other\ili{} studies\ili{}.\ili{}
\ili{}%\ili{}\cite\ili{}{Losnegaard\ili{}:2015}\ili{} have\ili{} examined\ili{} how\ili{} 859\ili{} \ili{}\ili\ili{}{Norwegian}\ili{} \ili{}\isi\ili{}{verbal\ili{} MWEs}\ili{},\ili{} idiomatic\ili{} expressions\ili{},\ili{} and\ili{} support\ili{} verb\ili{} constructions\ili{} distribute\ili{} over\ili{} syntactic\ili{} subtypes\ili{}.\ili{}
\ili{}%Systematic\ili{} corpus\ili{} studies\ili{} of\ili{} this\ili{} type\ili{} can\ili{} also\ili{} lead\ili{} to\ili{} additions\ili{} that\ili{} can\ili{} improve\ili{} the\ili{} MWE\ili{} coverage\ili{} of\ili{} NorGram\ili{}.\ili{}
\ili{}
\ili{}\section\ili{}*\ili{}{Acknowledgments}\ili{}
We\ili{} thank\ili{} Koenraad\ili{} De\ili{} Smedt\ili{} and\ili{} two\ili{} anonymous\ili{} reviewers\ili{} for\ili{} valuable\ili{} comments\ili{} and\ili{} suggestions\ili{} for\ili{} improvements\ili{}.\ili{}
This\ili{} work\ili{} was\ili{} partially\ili{} financed\ili{} by\ili{} the\ili{} \ili{}\ili\ili{}{Norwegian}\ili{} Research\ili{} Council\ili{} and\ili{} the\ili{} University\ili{} of\ili{} Bergen\ili{} through\ili{} the\ili{} INESS\ili{} project\ili{}.\ili{}
\ili{}
\ili{}%\ili{}\section\ili{}*\ili{}{Abbreviations}\ili{}
\ili{}%SUBJ\ili{}	subject\ili{}
\ili{}%OBJ\ili{}	object\ili{}
\ili{}%OBL\ili{}-TH\ili{}	oblique\ili{} argument\ili{} which\ili{} expresses\ili{} a\ili{} theta\ili{} role\ili{}
\ili{} \ili{}
\ili{}
\ili{}
\ili{}\printbibliography\ili{}[heading\ili{}=subbibliography\ili{},notkeyword\ili{}=this\ili{}]\ili{}
\ili{}
\ili{}\end\ili{}{document}\ili{}
\ili{}