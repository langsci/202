\documentclass[output=paper
,modfonts
,nonflat
,biblatexbackend=biber
]{langsci/langscibook}

\author{
  Timm Lichte\affiliation{University of Düsseldorf}\and 
  Simon Petitjean\affiliation{University of Düsseldorf}\and 
  Agata Savary\affiliation{Université François Rabelais Tours}\lastand 
  Jakub Waszczuk\affiliation{Université François Rabelais Tours\\Université d'Orléans}
}
\title{Lexical encoding formats for multi-word expressions: The challenge of ``irregular'' regularities}

%\lehead{}
\shorttitlerunninghead{Lexical encoding formats for multi-word expressions}

\abstract{
  This chapter contributes a general overview and discussion of lexical encoding formats for multi-word expressions (MWEs) that can be used in NLP systems, in particular with large-scale grammars. The presentation is kept general in the sense that we will try to elicit basic aspects of lexical encoding and then elaborate on the specific sorts of challenges encountered when dealing with MWEs, in particular the ``irregular'' regularities mentioned in the title. These insights will eventually be used to classify and evaluate different approaches to encoding. Even though this kind of evaluation cannot be conclusive given the diversity of languages and tastes, we will nevertheless argue in favor of fully flexible encoding formats exemplified with PATR-II and XMG, as opposed to the fixed encoding formats of DuELME and Walenty.  
}

 
\begin{document}\ili{}
\ili{}
\ili{}\maketitle\ili{}
\ili{}
\ili{}
\ili{}%\ili{}\input\ili{}{01\ili{}-intro}\ili{}
\ili{}\section\ili{}{Introduction}\ili{}
\ili{}
In\ili{} this\ili{} chapter\ili{},\ili{} we\ili{} seek\ili{} to\ili{} answer\ili{} a\ili{} seemingly\ili{} simple\ili{} question\ili{}:\ili{} what\ili{} is\ili{} it\ili{} that\ili{} makes\ili{} an\ili{} encoding\ili{} format\ili{} suitable\ili{} for\ili{} encoding\ili{} multi\ili{}-word\ili{} expressions\ili{} \ili{}(MWEs\ili{})\ili{} as\ili{} part\ili{} of\ili{} an\ili{} electronic\ili{} resource\ili{}?\ili{} One\ili{} quick\ili{} answer\ili{} could\ili{} be\ili{}:\ili{} the\ili{} encoding\ili{} must\ili{} be\ili{} machine\ili{}-\ili{} and\ili{} human\ili{}-\ili{} readable\ili{},\ili{} it\ili{} must\ili{} be\ili{} factorized\ili{},\ili{} and\ili{},\ili{} last\ili{} but\ili{} not\ili{} least\ili{},\ili{} it\ili{} must\ili{} be\ili{} able\ili{} to\ili{} cope\ili{} with\ili{} the\ili{} specific\ili{} irregularities\ili{} of\ili{} these\ili{} objects\ili{}.\ili{} But\ili{} what\ili{} does\ili{} this\ili{} exactly\ili{} mean\ili{}?\ili{} In\ili{} fact\ili{},\ili{} we\ili{} claim\ili{} that\ili{} the\ili{} casual\ili{} use\ili{} of\ili{} \ili{}`\ili{}`irregularity\ili{}'\ili{}'\ili{} actually\ili{} threatens\ili{} to\ili{} cover\ili{} a\ili{} great\ili{} deal\ili{} of\ili{} regularity\ili{},\ili{} even\ili{} though\ili{} it\ili{} is\ili{} often\ili{} a\ili{} regularity\ili{} that\ili{} might\ili{} look\ili{} uncommon\ili{}.\ili{} In\ili{} this\ili{} chapter\ili{},\ili{} we\ili{} therefore\ili{} aim\ili{} to\ili{} provide\ili{} a\ili{} more\ili{} precise\ili{} understanding\ili{} of\ili{} the\ili{} underlying\ili{} notions\ili{} and\ili{} concepts\ili{},\ili{} and\ili{} to\ili{} apply\ili{} this\ili{} to\ili{} a\ili{} selection\ili{} of\ili{} formats\ili{} which\ili{} have\ili{} a\ili{} potential\ili{} of\ili{} encoding\ili{} large\ili{} classes\ili{} of\ili{} MWEs\ili{},\ili{} including\ili{} notably\ili{} verbal\ili{} ones\ili{},\ili{} namely\ili{} DuELME\ili{},\ili{} Walenty\ili{},\ili{} PATR\ili{}-II\ili{} and\ili{} XMG\ili{}.\ili{} Thus\ili{} we\ili{} are\ili{} not\ili{} aiming\ili{} at\ili{} the\ili{} presentation\ili{} of\ili{} a\ili{} comprehensive\ili{} list\ili{} of\ili{} encoding\ili{} formats\ili{} ever\ili{} proposed\ili{} for\ili{} MWEs\ili{},\ili{} but\ili{} rather\ili{} want\ili{} to\ili{} elicit\ili{} general\ili{} aspects\ili{} and\ili{} typical\ili{} examples\ili{} thereof\ili{}.\ili{} \ili{}
\ili{}
The\ili{} chapter\ili{} is\ili{} structured\ili{} as\ili{} follows\ili{}.\ili{} We\ili{} will\ili{} first\ili{} sort\ili{} out\ili{} general\ili{} notions\ili{} and\ili{} principles\ili{} of\ili{} lexical\ili{} encoding\ili{},\ili{} starting\ili{} with\ili{} the\ili{} notion\ili{} of\ili{} regularity\ili{} in\ili{} Section\ili{}~\ili{}\ref\ili{}{sec\ili{}:notion\ili{}-regularity}\ili{} and\ili{} the\ili{} notion\ili{} of\ili{} encoding\ili{} in\ili{} Section\ili{}~\ili{}\ref\ili{}{sec\ili{}:simplest\ili{}-encoding}\ili{},\ili{} and\ili{} then\ili{} turn\ili{} to\ili{} general\ili{} virtues\ili{} of\ili{} lexical\ili{} encoding\ili{} formats\ili{} in\ili{} Section\ili{}~\ili{}\ref\ili{}{sec\ili{}:general\ili{}-virtues}\ili{}.\ili{} Following\ili{} this\ili{},\ili{} in\ili{} Section\ili{}~\ili{}\ref\ili{}{sec\ili{}:challenges}\ili{},\ili{} we\ili{} will\ili{} go\ili{} into\ili{} more\ili{} specific\ili{} aspects\ili{},\ili{} or\ili{} rather\ili{} challenges\ili{},\ili{} that\ili{} are\ili{} to\ili{} be\ili{} dealt\ili{} with\ili{} when\ili{} encoding\ili{} MWEs\ili{}.\ili{} With\ili{} this\ili{} in\ili{} view\ili{},\ili{} we\ili{} will\ili{} then\ili{} analyze\ili{} existing\ili{} formats\ili{} by\ili{} dividing\ili{} them\ili{} into\ili{} two\ili{} groups\ili{}:\ili{} fixed\ili{} encoding\ili{} formats\ili{} will\ili{} be\ili{} treated\ili{} in\ili{} Section\ili{}~\ili{}\ref\ili{}{sec\ili{}:fixed}\ili{},\ili{} and\ili{} fully\ili{} flexible\ili{} ones\ili{} in\ili{} Section\ili{}~\ili{}\ref\ili{}{sec\ili{}:fullyflexible}\ili{}.\ili{} In\ili{} Section\ili{}~\ili{}\ref\ili{}{sec\ili{}:summary}\ili{},\ili{} we\ili{} will\ili{} finally\ili{} compare\ili{} the\ili{} encoding\ili{} formats\ili{} and\ili{} summarize\ili{} the\ili{} chapter\ili{}.\ili{}
\ili{}
\ili{}\section\ili{}{\ili{} On\ili{} the\ili{} notion\ili{} of\ili{} regularity}\ili{}
\ili{}\label\ili{}{sec\ili{}:notion\ili{}-regularity}\ili{}
\ili{}
Regularity\ili{} in\ili{} the\ili{} sense\ili{} we\ili{} are\ili{} concerned\ili{} with\ili{} refers\ili{} to\ili{} the\ili{} way\ili{} properties\ili{} are\ili{} shared\ili{} between\ili{} the\ili{} members\ili{} of\ili{} a\ili{} set\ili{} of\ili{} objects\ili{}.\ili{} For\ili{} now\ili{},\ili{} we\ili{} take\ili{} a\ili{} property\ili{} to\ili{} be\ili{} just\ili{} some\ili{} atomic\ili{} name\ili{} and\ili{} assume\ili{} that\ili{} every\ili{} object\ili{} is\ili{} assigned\ili{} exactly\ili{} one\ili{} subset\ili{} of\ili{} a\ili{} given\ili{} set\ili{} of\ili{} properties\ili{}.\ili{} We\ili{} then\ili{} say\ili{} that\ili{} a\ili{} property\ili{} \ili{}$p\ili{}$\ili{} \ili{} is\ili{} \ili{}\textsc\ili{}{regular}\ili{} with\ili{} respect\ili{} to\ili{} a\ili{} set\ili{} of\ili{} objects\ili{} \ili{}$E\ili{}$\ili{},\ili{} iff\ili{} \ili{}$p\ili{}$\ili{} is\ili{} shared\ili{} by\ili{} at\ili{} least\ili{} two\ili{} members\ili{} in\ili{} \ili{}$E\ili{}$\ili{}.\ili{} Otherwise\ili{} \ili{}$p\ili{}$\ili{} is\ili{} \ili{}\textsc\ili{}{irregular}\ili{} \ili{}(or\ili{} \ili{}\textsc\ili{}{idiosyncratic}\ili{})\ili{}.\ili{} If\ili{} \ili{}$p\ili{}$\ili{} is\ili{} regular\ili{} but\ili{} is\ili{} shared\ili{} only\ili{} by\ili{} a\ili{} proper\ili{} subset\ili{} of\ili{} \ili{}$E\ili{}$\ili{},\ili{} we\ili{} call\ili{} \ili{}$p\ili{}$\ili{} \ili{}\textsc\ili{}{non\ili{}-trivially\ili{} regular}\ili{}.\ili{} By\ili{} contrast\ili{},\ili{} in\ili{} the\ili{} \ili{}\textsc\ili{}{trivially\ili{} regular}\ili{} case\ili{},\ili{} \ili{}$p\ili{}$\ili{} is\ili{} regular\ili{} and\ili{} shared\ili{} by\ili{} all\ili{} the\ili{} objects\ili{} in\ili{} \ili{}$E\ili{}$\ili{}.\ili{} Here\ili{},\ili{} \ili{}$p\ili{}$\ili{} can\ili{} be\ili{} removed\ili{} without\ili{} harm\ili{} because\ili{} it\ili{} does\ili{} not\ili{} distinguish\ili{} any\ili{} two\ili{} objects\ili{} in\ili{} \ili{}$E\ili{}$\ili{}.\ili{} Sets\ili{} of\ili{} properties\ili{} can\ili{} be\ili{} treated\ili{} accordingly\ili{},\ili{} hence\ili{} a\ili{} property\ili{} set\ili{} \ili{}$P\ili{}$\ili{} is\ili{} regular\ili{},\ili{} if\ili{} it\ili{} is\ili{} a\ili{} subset\ili{} of\ili{} property\ili{} sets\ili{} of\ili{} at\ili{} least\ili{} two\ili{} objects\ili{} in\ili{} \ili{}$E\ili{}$\ili{}.\ili{} We\ili{} then\ili{} extend\ili{} the\ili{} notion\ili{} of\ili{} regularity\ili{} to\ili{} objects\ili{} by\ili{} calling\ili{} an\ili{} object\ili{} regular\ili{},\ili{} if\ili{} it\ili{} only\ili{} has\ili{} regular\ili{} properties\ili{} and\ili{} property\ili{} sets\ili{},\ili{} and\ili{} otherwise\ili{} irregular\ili{}.\ili{} Finally\ili{},\ili{} this\ili{} simplistic\ili{} formalization\ili{} allows\ili{} for\ili{} a\ili{} straightforward\ili{} characterization\ili{} of\ili{} the\ili{} \ili{}\textsc\ili{}{degree\ili{} of\ili{} regularity}\ili{},\ili{} for\ili{} example\ili{},\ili{} in\ili{} terms\ili{} of\ili{} likelihood\ili{} \ili{}(how\ili{} likely\ili{} is\ili{} the\ili{} property\ili{} set\ili{} of\ili{} an\ili{} object\ili{} given\ili{} a\ili{} property\ili{} distribution\ili{} in\ili{} the\ili{} underlying\ili{} object\ili{} set\ili{})\ili{} and\ili{} diversity\ili{} \ili{}(how\ili{} many\ili{} property\ili{} sets\ili{} are\ili{} found\ili{} in\ili{} an\ili{} object\ili{} set\ili{})\ili{}.\ili{}
\ili{}
This\ili{} notion\ili{} of\ili{} \ili{}(ir\ili{})regularity\ili{} implies\ili{} that\ili{} it\ili{} is\ili{} impossible\ili{} to\ili{} determine\ili{} once\ili{} and\ili{} for\ili{} all\ili{} whether\ili{} the\ili{} properties\ili{} of\ili{} certain\ili{} objects\ili{} are\ili{} regular\ili{} or\ili{} irregular\ili{},\ili{} simply\ili{} because\ili{} the\ili{} set\ili{} of\ili{} conceivable\ili{} properties\ili{} and\ili{} objects\ili{} is\ili{} unbounded\ili{}.\ili{} In\ili{} other\ili{} words\ili{},\ili{} the\ili{} whole\ili{} business\ili{} of\ili{} telling\ili{} apart\ili{} regularity\ili{} from\ili{} irregularity\ili{} hinges\ili{} on\ili{} the\ili{} selection\ili{} of\ili{} properties\ili{} along\ili{} with\ili{} a\ili{} specific\ili{} set\ili{} of\ili{} objects\ili{}.\ili{} \ili{}
\ili{}
Applying\ili{} this\ili{} to\ili{} linguistics\ili{},\ili{} the\ili{} traditional\ili{} view\ili{} on\ili{} the\ili{} division\ili{} of\ili{} labor\ili{} between\ili{} syntax\ili{} and\ili{} lexicon\ili{} is\ili{} only\ili{} valid\ili{} for\ili{} a\ili{} specific\ili{} set\ili{} of\ili{} linguistic\ili{} objects\ili{},\ili{} namely\ili{} words\ili{},\ili{} phrases\ili{} and\ili{} sentences\ili{},\ili{} and\ili{} a\ili{} specific\ili{} set\ili{} of\ili{} \ili{}`\ili{}`syntactic\ili{}'\ili{}'\ili{} properties\ili{}.\ili{} Only\ili{} on\ili{} these\ili{} premises\ili{} is\ili{} it\ili{} valid\ili{} to\ili{} say\ili{} that\ili{} syntax\ili{} is\ili{} the\ili{} realm\ili{} of\ili{} regularity\ili{} whereas\ili{} the\ili{} lexicon\ili{} is\ili{} the\ili{} collecting\ili{} point\ili{} for\ili{} \ili{} irregular\ili{} aspects\ili{}.\ili{} To\ili{} give\ili{} an\ili{} example\ili{},\ili{} one\ili{} could\ili{} consider\ili{} \ili{}\isi\ili{}{phrase}\ili{} structure\ili{} rules\ili{} as\ili{} properties\ili{} of\ili{} words\ili{},\ili{} phrases\ili{} and\ili{} sentences\ili{},\ili{} depending\ili{} on\ili{} whether\ili{} the\ili{} \ili{}\isi\ili{}{phrase}\ili{} structure\ili{} rules\ili{} can\ili{} be\ili{} used\ili{} to\ili{} derive\ili{} them\ili{}.\ili{} According\ili{} to\ili{} this\ili{} set\ili{} of\ili{} properties\ili{},\ili{} the\ili{} words\ili{} would\ili{} be\ili{} derived\ili{} only\ili{} by\ili{} idiosyncratic\ili{} rules\ili{} that\ili{} cannot\ili{} be\ili{} used\ili{} to\ili{} derive\ili{} any\ili{} other\ili{} word\ili{}.\ili{} Hence\ili{},\ili{} the\ili{} set\ili{} of\ili{} words\ili{} \ili{}(\ili{}=\ili{} the\ili{} lexicon\ili{})\ili{} would\ili{} not\ili{} be\ili{} fully\ili{} regular\ili{},\ili{} other\ili{} than\ili{} the\ili{} sets\ili{} of\ili{} phrases\ili{} and\ili{} sentences\ili{} \ili{}(\ili{}=\ili{} the\ili{} syntax\ili{})\ili{}.\ili{} However\ili{},\ili{} when\ili{} taking\ili{} other\ili{} properties\ili{} into\ili{} account\ili{} such\ili{} as\ili{} semantic\ili{},\ili{} morphological\ili{} and\ili{} phonological\ili{} ones\ili{},\ili{} this\ili{} division\ili{} gets\ili{} blurred\ili{} quite\ili{} easily\ili{}.\ili{}
\ili{}
Similarly\ili{},\ili{} if\ili{} an\ili{} MWE\ili{} \ili{}(or\ili{} some\ili{} property\ili{} of\ili{} it\ili{})\ili{} is\ili{} called\ili{} \ili{}`\ili{}`irregular\ili{}'\ili{}'\ili{},\ili{} this\ili{} can\ili{} have\ili{} at\ili{} least\ili{} one\ili{} of\ili{} three\ili{} possible\ili{} reasons\ili{}:\ili{} \ili{}(i\ili{})\ili{} the\ili{} set\ili{} of\ili{} objects\ili{} is\ili{} sufficiently\ili{} restricted\ili{} \ili{}(e\ili{}.g\ili{}.\ili{},\ili{} by\ili{} contrasting\ili{} the\ili{} MWE\ili{} with\ili{} non\ili{}-MWEs\ili{} only\ili{})\ili{},\ili{} or\ili{} \ili{}(ii\ili{})\ili{} the\ili{} set\ili{} of\ili{} properties\ili{} is\ili{} sufficiently\ili{} extended\ili{} \ili{}(e\ili{}.g\ili{}.\ili{},\ili{} by\ili{} taking\ili{} into\ili{} account\ili{} very\ili{} specific\ili{} properties\ili{} of\ili{} the\ili{} MWE\ili{})\ili{},\ili{} or\ili{} \ili{}(iii\ili{})\ili{} the\ili{} property\ili{} set\ili{} of\ili{} the\ili{} \ili{} MWE\ili{} is\ili{} relatively\ili{} unlikely\ili{} and\ili{} \ili{}`\ili{}`irregular\ili{}'\ili{}'\ili{} is\ili{} assigned\ili{} a\ili{} likelihood\ili{} related\ili{} meaning\ili{}.\ili{} \ili{}
In\ili{} all\ili{} three\ili{} cases\ili{},\ili{} there\ili{} is\ili{} actually\ili{} a\ili{} high\ili{} risk\ili{} of\ili{} overlooking\ili{} or\ili{} neglecting\ili{} some\ili{} regularities\ili{},\ili{} even\ili{} more\ili{} since\ili{} we\ili{} are\ili{} dealing\ili{} with\ili{} objects\ili{} that\ili{} have\ili{} not\ili{} been\ili{} in\ili{} the\ili{} center\ili{} of\ili{} interest\ili{} in\ili{} most\ili{} of\ili{} the\ili{} mainstream\ili{} grammar\ili{} theories\ili{}.\ili{} This\ili{} gives\ili{} a\ili{} hint\ili{} of\ili{} how\ili{} we\ili{} want\ili{} \ili{}`\ili{}`irregular\ili{} regularities\ili{}'\ili{}'\ili{} from\ili{} the\ili{} title\ili{} to\ili{} be\ili{} understood\ili{}:\ili{} as\ili{} regularities\ili{} that\ili{} concern\ili{} unusual\ili{} properties\ili{}.\ili{} The\ili{} assumption\ili{} throughout\ili{} this\ili{} chapter\ili{} will\ili{} be\ili{} that\ili{} the\ili{} irregularity\ili{} of\ili{} MWEs\ili{} can\ili{} be\ili{} attributed\ili{} to\ili{} very\ili{} few\ili{} properties\ili{} concerning\ili{} the\ili{} syntax\ili{}-semantics\ili{} interface\ili{},\ili{} while\ili{} there\ili{} is\ili{} a\ili{} great\ili{} deal\ili{} of\ili{} non\ili{}-trivially\ili{} regular\ili{} properties\ili{} that\ili{} are\ili{} shared\ili{} across\ili{} MWEs\ili{} and\ili{} that\ili{} permeate\ili{} all\ili{} levels\ili{} of\ili{} linguistic\ili{} descriptions\ili{}.\ili{}
\ili{} \ili{}
\ili{}
\ili{}\section\ili{}{The\ili{} most\ili{} basic\ili{} encoding\ili{} format}\ili{}
\ili{}\label\ili{}{sec\ili{}:simplest\ili{}-encoding}\ili{}
\ili{}
Given\ili{} what\ili{} has\ili{} been\ili{} said\ili{} in\ili{} the\ili{} last\ili{} section\ili{},\ili{} it\ili{} should\ili{} be\ili{} fairly\ili{} easy\ili{} to\ili{} see\ili{} \ili{} that\ili{} the\ili{} most\ili{} basic\ili{} encoding\ili{} format\ili{} of\ili{} the\ili{} properties\ili{} of\ili{} an\ili{} MWE\ili{} is\ili{} via\ili{} \ili{}\textsc\ili{}{property\ili{} name\ili{} sets}\ili{}.\ili{} Two\ili{} examples\ili{} for\ili{} \ili{}\textit\ili{}{kick\ili{} the\ili{} bucket}\ili{} and\ili{} \ili{}\textit\ili{}{spill\ili{} beans}\ili{} are\ili{} shown\ili{} in\ili{} \ili{}(\ili{}\ref\ili{}{ex\ili{}:property\ili{}:sets}\ili{})\ili{}:\ili{}
\ili{}
\ili{}\eal\ili{} \ili{}\label\ili{}{ex\ili{}:property\ili{}:sets}\ili{}
\ili{}\ex\ili{}[\ili{}]\ili{}{\ili{} \ili{}\label\ili{}{ex\ili{}:property\ili{}:sets\ili{}:a}kick\ili{}-the\ili{}-bucket\ili{} \ili{}$\ili{}:\ili{}=\ili{}$\ili{} \ili{}\\ili{}\\ili{} \ili{}$\ili{}\\ili{}{\ili{}$NP\ili{}$_0\ili{}$\ili{} V\ili{} NP\ili{}$_1\ili{}$\ili{},\ili{} NP\ili{}$_1\ili{}$\ili{}.Det\ili{}.the\ili{},\ili{} \ili{} NP\ili{}$_1\ili{}$\ili{}.N\ili{}.bucket\ili{},\ili{} V\ili{}.kick\ili{},\ili{} meaning\ili{}=die\ili{}$\ili{}\}\ili{}$\ili{} }\ili{}
\ili{}\ex\ili{}[\ili{}]\ili{}{\ili{} \ili{}\label\ili{}{ex\ili{}:property\ili{}:sets\ili{}:b}spill\ili{}-beans\ili{} \ili{}$\ili{}:\ili{}=\ili{}$\ili{} \ili{}\\ili{}\\ili{} \ili{} \ili{}$\ili{}\\ili{}{\ili{}$NP\ili{}$_0\ili{}$\ili{} V\ili{} NP\ili{}$_1\ili{}$\ili{},\ili{} \ili{} NP\ili{}$_1\ili{}$\ili{}.N\ili{}.beans\ili{},\ili{} V\ili{}.spill\ili{},\ili{} passive\ili{},\ili{} meaning\ili{}=divulge\ili{}$\ili{}\}\ili{}$\ili{} }\ili{}
\ili{}\zl\ili{}
\ili{}
Even\ili{} if\ili{} the\ili{} property\ili{} names\ili{} seem\ili{} to\ili{} have\ili{} some\ili{} compositional\ili{} structure\ili{} \ili{}(NP\ili{}$_1\ili{}$\ili{}.\ili{} Det\ili{}.the\ili{} means\ili{} that\ili{} the\ili{} determiner\ili{} of\ili{} the\ili{} object\ili{} NP\ili{} is\ili{} \ili{}\textit\ili{}{the}\ili{})\ili{},\ili{} they\ili{} are\ili{} chosen\ili{} here\ili{} for\ili{} purely\ili{} mnemonic\ili{} reasons\ili{} \ili{}-\ili{}-\ili{} one\ili{} could\ili{} have\ili{} equally\ili{} written\ili{} something\ili{} alphabetically\ili{} innocent\ili{} like\ili{} \ili{}$p_\ili{}{23}\ili{}$\ili{}.\ili{} So\ili{},\ili{} in\ili{} order\ili{} to\ili{} proceed\ili{},\ili{} what\ili{} is\ili{} needed\ili{} is\ili{} an\ili{} \ili{}\textsc\ili{}{interpretation\ili{} function}\ili{} from\ili{} property\ili{} names\ili{} to\ili{} objects\ili{} of\ili{} whatever\ili{} target\ili{} formalism\ili{} is\ili{} chosen\ili{}.\ili{} Essentially\ili{},\ili{} this\ili{} is\ili{} characteristic\ili{} of\ili{} any\ili{} encoding\ili{} format\ili{},\ili{} even\ili{} the\ili{} more\ili{} sophisticated\ili{} ones\ili{}.\ili{} Of\ili{} course\ili{},\ili{} there\ili{} is\ili{} some\ili{} variance\ili{} as\ili{} to\ili{} how\ili{} close\ili{} the\ili{} encoding\ili{} format\ili{} is\ili{} related\ili{} to\ili{} the\ili{} target\ili{} formalism\ili{}.\ili{} \ili{}\cite\ili{}{Daelemans\ili{}:VanDerLinden\ili{}:92}\ili{} refer\ili{} to\ili{} this\ili{} aspect\ili{} as\ili{} notational\ili{} adequacy\ili{}.\ili{} But\ili{} be\ili{} aware\ili{} that\ili{},\ili{} in\ili{} our\ili{} view\ili{},\ili{} the\ili{} adequacy\ili{} of\ili{} a\ili{} lexical\ili{} encoding\ili{} format\ili{} is\ili{} multi\ili{}-aspectual\ili{} \ili{}(see\ili{} Figure\ili{}~\ili{}\ref\ili{}{fig\ili{}:encoding\ili{}-aspects}\ili{} on\ili{} page\ili{} \ili{}\pageref\ili{}{fig\ili{}:encoding\ili{}-aspects}\ili{})\ili{} and\ili{} ultimately\ili{} \ili{}\textit\ili{}{user\ili{}-oriented}\ili{}.\ili{} We\ili{} will\ili{} elaborate\ili{} more\ili{} on\ili{} this\ili{} in\ili{} Section\ili{}~\ili{}\ref\ili{}{sec\ili{}:general\ili{}-virtues}\ili{}.\ili{}
\ili{}
Speaking\ili{} of\ili{} the\ili{} adequacy\ili{} of\ili{} property\ili{} name\ili{} sets\ili{},\ili{} there\ili{} are\ili{},\ili{} in\ili{} fact\ili{},\ili{} some\ili{} attractive\ili{} properties\ili{} of\ili{} this\ili{} very\ili{} simple\ili{} way\ili{} of\ili{} encoding\ili{}:\ili{} \ili{}(i\ili{})\ili{} it\ili{} is\ili{} very\ili{} flexible\ili{} in\ili{} terms\ili{} of\ili{} adding\ili{} and\ili{} removing\ili{} property\ili{} names\ili{} and\ili{} adapting\ili{} the\ili{} interpretation\ili{} function\ili{} to\ili{} some\ili{} target\ili{} formalism\ili{};\ili{} \ili{}(ii\ili{})\ili{} it\ili{} makes\ili{} empirically\ili{} largely\ili{} neutral\ili{} descriptions\ili{} available\ili{};\ili{} \ili{}(iii\ili{})\ili{} it\ili{} is\ili{} conceptually\ili{} lean\ili{} and\ili{} inviting\ili{} for\ili{} formal\ili{} novices\ili{} because\ili{} the\ili{} main\ili{} data\ili{} structures\ili{} are\ili{} just\ili{} ordinary\ili{} sets\ili{}.\ili{} On\ili{} the\ili{} other\ili{} hand\ili{},\ili{} it\ili{} is\ili{} obvious\ili{} that\ili{} nobody\ili{} would\ili{} seriously\ili{} make\ili{} use\ili{} of\ili{} property\ili{} name\ili{} sets\ili{} when\ili{} encoding\ili{} a\ili{} large\ili{} electronic\ili{} lexicon\ili{} \ili{}-\ili{}-\ili{} at\ili{} least\ili{} not\ili{} without\ili{} a\ili{} tool\ili{} that\ili{} helps\ili{} to\ili{} ensure\ili{} correctness\ili{} by\ili{} accounting\ili{} for\ili{},\ili{} and\ili{} therefore\ili{} encoding\ili{},\ili{} underlying\ili{} generalizations\ili{},\ili{} that\ili{} is\ili{},\ili{} patterns\ili{} of\ili{} co\ili{}-occurrence\ili{} among\ili{} properties\ili{}.\ili{} Furthermore\ili{},\ili{} one\ili{} would\ili{} need\ili{} tools\ili{} to\ili{} specify\ili{} and\ili{} carry\ili{} out\ili{} the\ili{} interpretation\ili{} function\ili{}.\ili{} In\ili{} our\ili{} view\ili{},\ili{} this\ili{} does\ili{} not\ili{} only\ili{} hold\ili{} for\ili{} pure\ili{} property\ili{} name\ili{} sets\ili{};\ili{} the\ili{} actual\ili{} encoding\ili{} format\ili{} is\ili{} \ili{}\textit\ili{}{always}\ili{} surrounded\ili{} by\ili{} tools\ili{} mediating\ili{} towards\ili{} the\ili{} human\ili{} user\ili{},\ili{} the\ili{} target\ili{} formalism\ili{} or\ili{} the\ili{} electronic\ili{} resource\ili{} \ili{}-\ili{}-\ili{} to\ili{} what\ili{} degree\ili{} depends\ili{} on\ili{} the\ili{} encoding\ili{} format\ili{} in\ili{} question\ili{} \ili{}(see\ili{} Section\ili{}~\ili{}\ref\ili{}{sec\ili{}:general\ili{}-virtues}\ili{})\ili{}.\ili{}
\ili{}
A\ili{} closely\ili{} related\ili{} but\ili{} more\ili{} transparent\ili{} encoding\ili{} format\ili{} is\ili{} based\ili{} on\ili{} tables\ili{} in\ili{} which\ili{} the\ili{} rows\ili{} correspond\ili{} to\ili{} lexical\ili{} entries\ili{},\ili{} or\ili{} any\ili{} other\ili{} sort\ili{} of\ili{} object\ili{},\ili{} and\ili{} the\ili{} columns\ili{} to\ili{} properties\ili{}.\ili{} Binary\ili{} cell\ili{} values\ili{} then\ili{} indicate\ili{} whether\ili{} a\ili{} property\ili{} holds\ili{} for\ili{} an\ili{} object\ili{} or\ili{} not\ili{}.\ili{} This\ili{} format\ili{} has\ili{} gained\ili{} some\ili{} popularity\ili{},\ili{} for\ili{} example\ili{},\ili{} through\ili{} the\ili{} extensive\ili{} work\ili{} of\ili{} Maurice\ili{} Gross\ili{} \ili{}(and\ili{} colleagues\ili{})\ili{} within\ili{} his\ili{} lexicon\ili{}-grammar\ili{} framework\ili{} \ili{}\citep\ili{}{Gross\ili{}:94}\ili{}.\ili{} While\ili{} lexicon\ili{}-grammar\ili{} matrices\ili{} are\ili{} binary\ili{},\ili{} at\ili{} least\ili{} for\ili{} the\ili{} most\ili{} part\ili{},\ili{} a\ili{} larger\ili{} range\ili{} of\ili{} cell\ili{} values\ili{} helps\ili{} to\ili{} yield\ili{} a\ili{} more\ili{} succinct\ili{} matrix\ili{}.\ili{} This\ili{} is\ili{} shown\ili{} in\ili{} Table\ili{}~\ili{}\ref\ili{}{tab\ili{}:table\ili{}-encoding}\ili{} which\ili{} translates\ili{} the\ili{} property\ili{} sets\ili{} from\ili{} \ili{}(\ili{}\ref\ili{}{ex\ili{}:property\ili{}:sets}\ili{})\ili{}.\ili{} \ili{}
\ili{}\begin\ili{}{table}\ili{}[tp\ili{}]\ili{}
\ili{} \ili{} \ili{}\small\ili{}
\ili{} \ili{} \ili{}\begin\ili{}{tabular}\ili{}{lcccccc}\ili{}
\ili{} \ili{} \ili{} \ili{} ID\ili{} \ili{} \ili{} \ili{} \ili{} \ili{} \ili{} \ili{} \ili{} \ili{} \ili{} \ili{} \ili{} \ili{} \ili{}&\ili{} NP\ili{}$_\ili{}{\ili{}\text\ili{}{0}}\ili{}$\ili{} V\ili{} NP\ili{}$_\ili{}{\ili{}\text\ili{}{1}}\ili{}$\ili{} \ili{}&\ili{} NP\ili{}$_\ili{}{\ili{}\text\ili{}{1}}\ili{}$\ili{}.det\ili{} \ili{}&\ili{} NP\ili{}$_\ili{}{\ili{}\text\ili{}{1}}\ili{}$\ili{}.N\ili{} \ili{}&\ili{} V\ili{} \ili{} \ili{} \ili{} \ili{} \ili{}&\ili{} passive\ili{} \ili{}&\ili{} meaning\ili{} \ili{}\\ili{}\\ili{} \ili{} \ili{}
\ili{} \ili{} \ili{} \ili{} \ili{}\hline\ili{}
\ili{} \ili{} \ili{} \ili{} kick\ili{}-the\ili{}-bucket\ili{} \ili{}&\ili{} \ili{}+\ili{} \ili{} \ili{} \ili{} \ili{} \ili{} \ili{} \ili{} \ili{} \ili{} \ili{} \ili{} \ili{} \ili{} \ili{} \ili{} \ili{} \ili{} \ili{} \ili{} \ili{} \ili{} \ili{} \ili{} \ili{} \ili{} \ili{} \ili{} \ili{} \ili{} \ili{} \ili{} \ili{} \ili{}&\ili{} the\ili{} \ili{} \ili{} \ili{} \ili{} \ili{} \ili{} \ili{} \ili{} \ili{} \ili{} \ili{} \ili{} \ili{} \ili{} \ili{} \ili{} \ili{}&\ili{} bucket\ili{} \ili{} \ili{} \ili{} \ili{} \ili{} \ili{} \ili{} \ili{} \ili{} \ili{} \ili{} \ili{}&\ili{} kick\ili{} \ili{} \ili{}&\ili{} \ili{}-\ili{} \ili{} \ili{} \ili{} \ili{} \ili{} \ili{} \ili{}&\ili{} die\ili{} \ili{} \ili{} \ili{} \ili{} \ili{}\\ili{}\\ili{}
\ili{} \ili{} \ili{} \ili{} spill\ili{}-beans\ili{} \ili{} \ili{} \ili{} \ili{} \ili{}&\ili{} \ili{}+\ili{} \ili{} \ili{} \ili{} \ili{} \ili{} \ili{} \ili{} \ili{} \ili{} \ili{} \ili{} \ili{} \ili{} \ili{} \ili{} \ili{} \ili{} \ili{} \ili{} \ili{} \ili{} \ili{} \ili{} \ili{} \ili{} \ili{} \ili{} \ili{} \ili{} \ili{} \ili{} \ili{} \ili{}&\ili{} \ili{} \ili{} \ili{} \ili{} \ili{} \ili{} \ili{} \ili{} \ili{} \ili{} \ili{} \ili{} \ili{} \ili{} \ili{} \ili{} \ili{} \ili{} \ili{} \ili{} \ili{}&\ili{} bean\ili{} \ili{} \ili{} \ili{} \ili{} \ili{} \ili{} \ili{} \ili{} \ili{} \ili{} \ili{} \ili{} \ili{} \ili{}&\ili{} spill\ili{} \ili{}&\ili{} \ili{}+\ili{} \ili{} \ili{} \ili{} \ili{} \ili{} \ili{} \ili{}&\ili{} divulge\ili{} \ili{}\\ili{}\\ili{}
\ili{} \ili{} \ili{}\end\ili{}{tabular}\ili{}
\ili{} \ili{} \ili{}\caption\ili{}{Table\ili{} encoding\ili{} of\ili{} the\ili{} property\ili{} name\ili{} sets\ili{} in\ili{} \ili{}(\ili{}\ref\ili{}{ex\ili{}:property\ili{}:sets}\ili{})}\ili{}
\ili{} \ili{} \ili{}\label\ili{}{tab\ili{}:table\ili{}-encoding}\ili{}
\ili{}\end\ili{}{table}\ili{}
Needless\ili{} to\ili{} say\ili{},\ili{} for\ili{} any\ili{} such\ili{} non\ili{}-binary\ili{} matrix\ili{},\ili{} there\ili{} is\ili{} a\ili{} equivalent\ili{} binary\ili{} one\ili{} with\ili{} a\ili{} larger\ili{} number\ili{} of\ili{} columns\ili{},\ili{} or\ili{} properties\ili{}.\ili{}
\ili{}
The\ili{} table\ili{} format\ili{} makes\ili{} the\ili{} presentation\ili{} of\ili{} property\ili{} name\ili{} sets\ili{} more\ili{} readable\ili{},\ili{} but\ili{} apart\ili{} from\ili{} this\ili{},\ili{} it\ili{} comes\ili{} with\ili{} very\ili{} similar\ili{} methodological\ili{} implications\ili{}:\ili{} it\ili{} is\ili{} suitable\ili{} for\ili{} collecting\ili{} observations\ili{},\ili{} but\ili{} it\ili{} cannot\ili{} express\ili{} recurring\ili{} patterns\ili{} within\ili{} these\ili{} observations\ili{},\ili{} that\ili{} is\ili{},\ili{} a\ili{} theory\ili{}.\ili{} For\ili{} this\ili{},\ili{} and\ili{} thus\ili{} also\ili{} for\ili{} ensuring\ili{} correctness\ili{} and\ili{} completeness\ili{},\ili{} additional\ili{} tools\ili{} are\ili{} needed\ili{}.\ili{}
\ili{}
\ili{}
\ili{}%\ili{}\input\ili{}{02\ili{}-virtues}\ili{}
\ili{}\section\ili{}{General\ili{} virtues\ili{} of\ili{} lexical\ili{} encoding\ili{} formats}\ili{}
\ili{}\label\ili{}{sec\ili{}:general\ili{}-virtues}\ili{}
\ili{}
The\ili{} preceding\ili{} section\ili{} showed\ili{} that\ili{} certain\ili{} encoding\ili{} formats\ili{} stand\ili{} out\ili{} in\ili{} terms\ili{} of\ili{} simplicity\ili{} and\ili{} accessibility\ili{},\ili{} but\ili{} also\ili{} manifest\ili{} critical\ili{} drawbacks\ili{} as\ili{} to\ili{} usability\ili{} and\ili{} expressivity\ili{}.\ili{} This\ili{} section\ili{} tries\ili{} to\ili{} sort\ili{} out\ili{} more\ili{} systematically\ili{} the\ili{} diverse\ili{} and\ili{} sometimes\ili{} contradicting\ili{} virtues\ili{} an\ili{} encoding\ili{} format\ili{} can\ili{} have\ili{}.\ili{} The\ili{} cause\ili{} of\ili{} diversity\ili{} is\ili{} not\ili{} hard\ili{} to\ili{} pinpoint\ili{}:\ili{} it\ili{} is\ili{} the\ili{} interface\ili{} status\ili{} of\ili{} encoding\ili{} formats\ili{},\ili{} as\ili{} illustrated\ili{} in\ili{} Figure\ili{}~\ili{}\ref\ili{}{fig\ili{}:encoding\ili{}-aspects}\ili{},\ili{} with\ili{} similarly\ili{} diverse\ili{} conjugates\ili{},\ili{} namely\ili{} a\ili{} human\ili{} user\ili{},\ili{} a\ili{} lexical\ili{} object\ili{} and\ili{} a\ili{} lexical\ili{} resource\ili{}.\ili{}
\ili{}\begin\ili{}{figure}\ili{}[tp\ili{}]\ili{}
\ili{} \ili{} \ili{}\centering\ili{} \ili{}
\ili{} \ili{} \ili{}\begin\ili{}{tikzpicture}\ili{}
\ili{} \ili{} \ili{} \ili{} \ili{}\draw\ili{} \ili{}
\ili{} \ili{} \ili{} \ili{} \ili{}(0\ili{}.2\ili{},0\ili{})\ili{} node\ili{}[align\ili{}=center\ili{}]\ili{} \ili{}{lexical\ili{}\\ili{}\object}\ili{}
\ili{} \ili{} \ili{} \ili{} \ili{}(3\ili{}.8\ili{},0\ili{})\ili{} node\ili{}[align\ili{}=center\ili{}]\ili{} \ili{}{human\ili{}\\ili{}\user}\ili{}
\ili{} \ili{} \ili{} \ili{} \ili{}(2\ili{},\ili{}-0\ili{}.5\ili{})\ili{} node\ili{}[regular\ili{} polygon\ili{},\ili{} regular\ili{} polygon\ili{} sides\ili{}=3\ili{},draw\ili{}=black\ili{},align\ili{}=center\ili{},inner\ili{} sep\ili{}=\ili{}-5pt\ili{}]\ili{} \ili{}{lexical\ili{}\\ili{}\encoding}\ili{}
\ili{} \ili{} \ili{} \ili{} \ili{}(2\ili{},\ili{}-2\ili{})\ili{} node\ili{}[align\ili{}=center\ili{}]\ili{} \ili{}{lexical\ili{}\\ili{}\resource}\ili{};\ili{}
\ili{} \ili{} \ili{}\end\ili{}{tikzpicture}\ili{}
\ili{} \ili{} \ili{}\caption\ili{}{Interface\ili{} aspects\ili{} of\ili{} lexical\ili{} encoding}\ili{}
\ili{} \ili{} \ili{}\label\ili{}{fig\ili{}:encoding\ili{}-aspects}\ili{}
\ili{}\end\ili{}{figure}\ili{}
\ili{}
\ili{}\subsection\ili{}{Encoding\ili{} virtues\ili{} with\ili{} respect\ili{} to\ili{} a\ili{} lexical\ili{} object}\ili{}
\ili{}\label\ili{}{sec\ili{}:virtues\ili{}-object}\ili{}
\ili{}
We\ili{} already\ili{} learned\ili{} in\ili{} Sections\ili{}~\ili{}\ref\ili{}{sec\ili{}:notion\ili{}-regularity}\ili{} and\ili{} \ili{}~\ili{}\ref\ili{}{sec\ili{}:simplest\ili{}-encoding}\ili{} that\ili{} the\ili{} simplest\ili{} conception\ili{} of\ili{} a\ili{} lexical\ili{} object\ili{} and\ili{} an\ili{} encoding\ili{} format\ili{} is\ili{} being\ili{} a\ili{} set\ili{} of\ili{} properties\ili{} or\ili{} property\ili{} names\ili{}.\ili{} Let\ili{} \ili{}$P_i\ili{}$\ili{} be\ili{} the\ili{} property\ili{} set\ili{} of\ili{} a\ili{} lexical\ili{} object\ili{}.\ili{} An\ili{} encoding\ili{} of\ili{} \ili{}$P_i\ili{}$\ili{} is\ili{} a\ili{} property\ili{} name\ili{} set\ili{} \ili{}$P\ili{}^e_i\ili{}$\ili{} together\ili{} with\ili{} an\ili{} encoding\ili{} function\ili{} which\ili{} maps\ili{} \ili{}$P_i\ili{}$\ili{} onto\ili{} \ili{}$P\ili{}^e_i\ili{}$\ili{}.\ili{} Hence\ili{},\ili{} the\ili{} encoding\ili{} examples\ili{} given\ili{} in\ili{} \ili{}(\ili{}\ref\ili{}{ex\ili{}:property\ili{}:sets}\ili{})\ili{} on\ili{} page\ili{}~\ili{}\pageref\ili{}{ex\ili{}:property\ili{}:sets}\ili{} are\ili{} actually\ili{} accompanied\ili{} by\ili{} an\ili{} imagined\ili{} lexical\ili{} object\ili{} and\ili{} an\ili{} encoding\ili{} function\ili{}.\ili{} It\ili{} is\ili{} furthermore\ili{} important\ili{} to\ili{} keep\ili{} in\ili{} mind\ili{} that\ili{},\ili{} for\ili{} now\ili{},\ili{} we\ili{} ignore\ili{} inferential\ili{} means\ili{} of\ili{} encoding\ili{} formats\ili{} that\ili{} help\ili{} to\ili{} express\ili{} generalizations\ili{},\ili{} that\ili{} is\ili{},\ili{} we\ili{} assume\ili{} that\ili{} encodings\ili{} are\ili{} fully\ili{} resolved\ili{}.\ili{}
\ili{}
Based\ili{} on\ili{} this\ili{} understanding\ili{} of\ili{} encoding\ili{},\ili{} the\ili{} encoding\ili{} virtues\ili{} are\ili{} easy\ili{} to\ili{} see\ili{} and\ili{} capture\ili{},\ili{} namely\ili{},\ili{} the\ili{} encoding\ili{} of\ili{} a\ili{} property\ili{} set\ili{} \ili{}$P_i\ili{}$\ili{} should\ili{} be\ili{} complete\ili{} and\ili{} concise\ili{}.\ili{} An\ili{} encoding\ili{} \ili{}(function\ili{})\ili{} is\ili{} \ili{}\textsc\ili{}{complete}\ili{} \ili{}\textit\ili{}{iff}\ili{} every\ili{} property\ili{} of\ili{} \ili{}$P_i\ili{}$\ili{} is\ili{} mapped\ili{} onto\ili{} a\ili{} property\ili{} name\ili{} of\ili{} \ili{}$P\ili{}^e_i\ili{}$\ili{}.\ili{} Thus\ili{} the\ili{} encoding\ili{} function\ili{} is\ili{} injective\ili{}.\ili{} On\ili{} the\ili{} other\ili{} hand\ili{},\ili{} an\ili{} encoding\ili{} is\ili{} \ili{}\textsc\ili{}{concise}\ili{} \ili{}\textit\ili{}{iff}\ili{} for\ili{} every\ili{} encoding\ili{} property\ili{} \ili{}$p\ili{}^e_i\ili{}$\ili{} there\ili{} is\ili{} a\ili{} source\ili{} property\ili{} \ili{}$p_i\ili{}$\ili{} such\ili{} that\ili{} \ili{}$p\ili{}^e_i\ili{}$\ili{} is\ili{} the\ili{} encoding\ili{} of\ili{} \ili{}$p_i\ili{}$\ili{}.\ili{} Here\ili{},\ili{} the\ili{} encoding\ili{} is\ili{} surjective\ili{}.\ili{} In\ili{} other\ili{} words\ili{},\ili{} no\ili{} property\ili{} name\ili{} is\ili{} added\ili{} unmotivatedly\ili{}.\ili{} Of\ili{} course\ili{},\ili{} an\ili{} encoding\ili{} should\ili{} be\ili{} both\ili{} complete\ili{} and\ili{} concise\ili{},\ili{} and\ili{} consequently\ili{} the\ili{} encoding\ili{} function\ili{} should\ili{} be\ili{} bijective\ili{}.\ili{} This\ili{} implies\ili{} that\ili{} distinctions\ili{} made\ili{} in\ili{} \ili{}$P_i\ili{}$\ili{} are\ili{} minimally\ili{} preserved\ili{} in\ili{} the\ili{} encoding\ili{} of\ili{} \ili{}$P_i\ili{}$\ili{}.\ili{}
\ili{}
To\ili{} give\ili{} an\ili{} example\ili{},\ili{} Table\ili{}~\ili{}\ref\ili{}{tab\ili{}:table\ili{}-encoding}\ili{} is\ili{} a\ili{} complete\ili{} encoding\ili{} of\ili{} the\ili{} property\ili{} sets\ili{} in\ili{} \ili{}(\ili{}\ref\ili{}{ex\ili{}:property\ili{}:sets}\ili{})\ili{}.\ili{} Yet\ili{} it\ili{} is\ili{} not\ili{} perfectly\ili{} concise\ili{}:\ili{} the\ili{} property\ili{} set\ili{} of\ili{} kick\ili{}-the\ili{}-bucket\ili{} does\ili{} not\ili{} have\ili{} a\ili{} passive\ili{} feature\ili{},\ili{} while\ili{} there\ili{} is\ili{} a\ili{} passive\ili{} cell\ili{} in\ili{} the\ili{} table\ili{} encoding\ili{}.\ili{} Similarly\ili{},\ili{} the\ili{} NP\ili{}$_1\ili{}$\ili{}.det\ili{} cell\ili{} in\ili{} the\ili{} encoding\ili{} of\ili{} spill\ili{}-beans\ili{} does\ili{} not\ili{} have\ili{} a\ili{} corresponding\ili{} property\ili{} in\ili{} the\ili{} source\ili{} set\ili{}.\ili{} Still\ili{},\ili{} the\ili{} encoding\ili{} in\ili{} Table\ili{}~\ili{}\ref\ili{}{tab\ili{}:table\ili{}-encoding}\ili{} appears\ili{} to\ili{} be\ili{} only\ili{} slightly\ili{} less\ili{} concise\ili{} than\ili{} the\ili{} original\ili{} property\ili{} sets\ili{} in\ili{} \ili{}(\ili{}\ref\ili{}{ex\ili{}:property\ili{}:sets}\ili{})\ili{},\ili{} and\ili{} moreover\ili{} the\ili{} table\ili{} encoding\ili{} is\ili{} \ili{}(in\ili{} most\ili{} cases\ili{})\ili{} more\ili{} accessible\ili{} for\ili{} the\ili{} human\ili{} eye\ili{}.\ili{} This\ili{} teaches\ili{} us\ili{} two\ili{} things\ili{}:\ili{} \ili{}(i\ili{})\ili{} the\ili{} validity\ili{} of\ili{} some\ili{} encoding\ili{} virtues\ili{} can\ili{} be\ili{} a\ili{} matter\ili{} of\ili{} degree\ili{},\ili{} and\ili{} \ili{}(ii\ili{})\ili{} they\ili{} may\ili{} conflict\ili{} with\ili{} other\ili{} encoding\ili{} virtues\ili{}.\ili{}
\ili{}
But\ili{} before\ili{} turning\ili{} to\ili{} possibly\ili{} conflicting\ili{} encoding\ili{} virtues\ili{} having\ili{} to\ili{} do\ili{} with\ili{} other\ili{} aspects\ili{} of\ili{} encoding\ili{},\ili{} let\ili{} us\ili{} finally\ili{} have\ili{} a\ili{} look\ili{} at\ili{} the\ili{} encoding\ili{} of\ili{} \ili{}\textit\ili{}{sets}\ili{} of\ili{} lexical\ili{} objects\ili{}.\ili{} Here\ili{},\ili{} it\ili{} is\ili{} clearly\ili{} desirable\ili{} for\ili{} an\ili{} encoding\ili{} to\ili{} be\ili{} \ili{}\textsc\ili{}{consistent}\ili{},\ili{} simply\ili{} meaning\ili{} that\ili{} the\ili{} relation\ili{} between\ili{} the\ili{} properties\ili{} appearing\ili{} in\ili{} all\ili{} the\ili{} lexical\ili{} objects\ili{} under\ili{} consideration\ili{} and\ili{} the\ili{} target\ili{} properties\ili{} of\ili{} the\ili{} encoding\ili{} is\ili{} functional\ili{} as\ili{} well\ili{}.\ili{} This\ili{} clearly\ili{} holds\ili{} for\ili{} the\ili{} encoding\ili{} in\ili{} Table\ili{}~\ili{}\ref\ili{}{tab\ili{}:table\ili{}-encoding}\ili{} where\ili{} identical\ili{} properties\ili{} are\ili{} encoded\ili{} as\ili{} identical\ili{} cell\ili{} values\ili{} within\ili{} the\ili{} same\ili{} row\ili{}.\ili{}
\ili{}
\ili{} \ili{} \ili{}
\ili{}\subsection\ili{}{Encoding\ili{} virtues\ili{} with\ili{} respect\ili{} to\ili{} a\ili{} human\ili{} user}\ili{}
\ili{}\label\ili{}{sec\ili{}:virtues\ili{}-human}\ili{}
\ili{}
When\ili{} it\ili{} comes\ili{} to\ili{} the\ili{} human\ili{} user\ili{},\ili{} a\ili{} lexical\ili{} encoding\ili{} should\ili{} be\ili{} transparent\ili{},\ili{} flexible\ili{} and\ili{} sufficiently\ili{} powerful\ili{} to\ili{} capture\ili{} generalizations\ili{}.\ili{}
\ili{}
By\ili{} \ili{}\textsc\ili{}{transparent}\ili{} we\ili{} mean\ili{} that\ili{} the\ili{} human\ili{} user\ili{} should\ili{} be\ili{} able\ili{} to\ili{} map\ili{} the\ili{} encoding\ili{} back\ili{} to\ili{} the\ili{} source\ili{} set\ili{} of\ili{} lexical\ili{} properties\ili{}.\ili{} Needless\ili{} to\ili{} say\ili{},\ili{} the\ili{} degree\ili{} of\ili{} \ili{}\isi\ili{}{transparency}\ili{} very\ili{} much\ili{} depends\ili{} on\ili{} the\ili{} taste\ili{} and\ili{} reading\ili{} habits\ili{} of\ili{} the\ili{} user\ili{} in\ili{} question\ili{}.\ili{} It\ili{} could\ili{} well\ili{} be\ili{},\ili{} although\ili{} it\ili{} is\ili{} rather\ili{} unlikely\ili{},\ili{} that\ili{} some\ili{} users\ili{} will\ili{} feel\ili{} more\ili{} comfortable\ili{} with\ili{} plain\ili{} property\ili{} sets\ili{} also\ili{} when\ili{} dealing\ili{} with\ili{} larger\ili{} lexicons\ili{}.\ili{} Depending\ili{} on\ili{} the\ili{} degree\ili{} of\ili{} training\ili{},\ili{} it\ili{} is\ili{} even\ili{} imaginable\ili{} that\ili{} users\ili{} become\ili{} fluent\ili{} in\ili{} rather\ili{} opaque\ili{} encoding\ili{} languages\ili{} that\ili{} make\ili{} use\ili{} of\ili{} property\ili{} names\ili{} such\ili{} as\ili{} \ili{}$p_\ili{}{23}\ili{}$\ili{}.\ili{} This\ili{} is\ili{},\ili{} of\ili{} course\ili{},\ili{} not\ili{} what\ili{} we\ili{} consider\ili{} desirable\ili{}:\ili{} lexical\ili{} encodings\ili{} should\ili{} not\ili{} come\ili{} with\ili{} notational\ili{} idiosyncrasies\ili{} that\ili{} keep\ili{} novices\ili{} away\ili{} or\ili{} are\ili{} prone\ili{} to\ili{} lead\ili{} to\ili{} encoding\ili{} errors\ili{} \ili{}(e\ili{}.g\ili{}.\ili{},\ili{} by\ili{} misremembering\ili{} \ili{}$p_\ili{}{23}\ili{}$\ili{})\ili{}.\ili{} Thus\ili{},\ili{} since\ili{} we\ili{} are\ili{} dealing\ili{} with\ili{} computational\ili{} lexicons\ili{},\ili{} we\ili{} conceive\ili{} an\ili{} encoding\ili{} language\ili{} as\ili{} transparent\ili{} \ili{}\textit\ili{}{iff}\ili{} it\ili{} is\ili{} \ili{}(i\ili{})\ili{} mnemonic\ili{} as\ili{} to\ili{} the\ili{} property\ili{} names\ili{} and\ili{} their\ili{} denoted\ili{} properties\ili{} and\ili{} \ili{}(ii\ili{})\ili{} precise\ili{} by\ili{} means\ili{} of\ili{} a\ili{} rigorous\ili{} denotational\ili{} semantics\ili{} to\ili{} avoid\ili{} vagueness\ili{} and\ili{} thus\ili{} inconsistencies\ili{}.\ili{} \ili{}
\ili{}
Since\ili{} \ili{}\isi\ili{}{transparency}\ili{} is\ili{} so\ili{} important\ili{} to\ili{} the\ili{} human\ili{} user\ili{},\ili{} but\ili{} at\ili{} the\ili{} same\ili{} time\ili{} human\ili{} users\ili{} and\ili{} also\ili{} lexical\ili{} objects\ili{} can\ili{} differ\ili{} to\ili{} a\ili{} great\ili{} deal\ili{},\ili{} another\ili{} crucial\ili{} virtue\ili{} of\ili{} encoding\ili{} formats\ili{} is\ili{} \ili{}\textsc\ili{}{\ili{}\isi\ili{}{flexibility}}\ili{}.\ili{} Lexical\ili{} encoding\ili{} usually\ili{} is\ili{} an\ili{} incremental\ili{} process\ili{} where\ili{} unforeseen\ili{} properties\ili{} can\ili{} be\ili{} encountered\ili{} or\ili{} the\ili{} denotation\ili{} of\ili{} a\ili{} property\ili{} may\ili{} change\ili{} over\ili{} time\ili{}.\ili{} A\ili{} flexible\ili{} encoding\ili{} format\ili{} allows\ili{} the\ili{} user\ili{} to\ili{} freely\ili{} choose\ili{} property\ili{} names\ili{} and\ili{} to\ili{} include\ili{} new\ili{} properties\ili{} on\ili{} the\ili{} fly\ili{}.\ili{}\footnote\ili{}{Of\ili{} course\ili{},\ili{} flexibility\ili{} also\ili{} helps\ili{} to\ili{} keep\ili{} the\ili{} encoding\ili{} complete\ili{} in\ili{} the\ili{} sense\ili{} of\ili{} Section\ili{}~\ili{}\ref\ili{}{sec\ili{}:virtues\ili{}-object}\ili{}.}\ili{} \ili{}
\ili{}
Closely\ili{} related\ili{} to\ili{} flexibility\ili{} is\ili{} the\ili{} \ili{}\textsc\ili{}{power\ili{} to\ili{} generalize}\ili{}.\ili{} With\ili{} the\ili{} increase\ili{} of\ili{} the\ili{} number\ili{} of\ili{} lexical\ili{} objects\ili{} that\ili{} are\ili{} encoded\ili{} in\ili{} a\ili{} lexicon\ili{},\ili{} usually\ili{} also\ili{} the\ili{} desire\ili{} to\ili{} factorize\ili{} the\ili{} property\ili{} sets\ili{} increases\ili{} in\ili{} order\ili{} to\ili{} avoid\ili{} redundancy\ili{}.\ili{} In\ili{} other\ili{} words\ili{},\ili{} one\ili{} would\ili{} like\ili{} to\ili{} group\ili{} properties\ili{} and\ili{} assign\ili{} them\ili{} collectively\ili{}.\ili{} Again\ili{},\ili{} the\ili{} human\ili{} encoder\ili{} should\ili{} be\ili{} free\ili{} to\ili{} choose\ili{} the\ili{} content\ili{} and\ili{} name\ili{} of\ili{} property\ili{} subsets\ili{},\ili{} or\ili{},\ili{} more\ili{} technically\ili{} speaking\ili{},\ili{} the\ili{} parts\ili{} of\ili{} encodings\ili{} should\ili{} be\ili{} reusable\ili{} at\ili{} any\ili{} level\ili{} of\ili{} representation\ili{} and\ili{} detail\ili{}.\ili{} What\ili{} maybe\ili{} sounds\ili{} like\ili{} a\ili{} nice\ili{} add\ili{}-on\ili{} is\ili{} in\ili{} fact\ili{} a\ili{} necessary\ili{} prerequisite\ili{} to\ili{} express\ili{} any\ili{} non\ili{}-trivial\ili{} lexical\ili{} generalization\ili{},\ili{} such\ili{} as\ili{} that\ili{} a\ili{} passive\ili{} construction\ili{} does\ili{} not\ili{} include\ili{} an\ili{} accusative\ili{} object\ili{}.\ili{}
\ili{}
Finally\ili{},\ili{} an\ili{} encoding\ili{} format\ili{} furthermore\ili{} is\ili{} \ili{}\textsc\ili{}{implementation\ili{}-friendly}\ili{} \ili{}\textit\ili{}{iff}\ili{} the\ili{}\\ili{}-re\ili{} exist\ili{} tools\ili{} that\ili{} assist\ili{} a\ili{} human\ili{} user\ili{} with\ili{} encoding\ili{} large\ili{} sets\ili{} of\ili{} lexical\ili{} objects\ili{},\ili{} or\ili{} with\ili{} verifying\ili{} these\ili{} encodings\ili{}.\ili{} This\ili{} virtue\ili{} already\ili{} touches\ili{} upon\ili{} one\ili{} aspect\ili{} that\ili{} will\ili{} be\ili{} also\ili{} dealt\ili{} with\ili{} in\ili{} the\ili{} next\ili{} section\ili{},\ili{} namely\ili{} the\ili{} existence\ili{} of\ili{} software\ili{} tools\ili{} that\ili{} help\ili{} to\ili{} convert\ili{} lexical\ili{} encodings\ili{} into\ili{} a\ili{} lexical\ili{} resource\ili{}.\ili{} \ili{}
\ili{}
\ili{}\subsection\ili{}{Encoding\ili{} virtues\ili{} with\ili{} respect\ili{} to\ili{} a\ili{} lexical\ili{} resource}\ili{}
\ili{}\label\ili{}{sec\ili{}:virtues\ili{}-resource}\ili{}
\ili{}
\ili{} \ili{} A\ili{} lexical\ili{} resource\ili{} is\ili{} an\ili{} electronic\ili{} representation\ili{} of\ili{} lexical\ili{} encodings\ili{} that\ili{} can\ili{} be\ili{} \ili{}(more\ili{} or\ili{} less\ili{})\ili{} directly\ili{} used\ili{} in\ili{} NLP\ili{} applications\ili{}.\ili{} Accordingly\ili{},\ili{} the\ili{} virtue\ili{} of\ili{} \ili{}\textsc\ili{}{electronic\ili{} versatility}\ili{} assigned\ili{} to\ili{} lexical\ili{} encoding\ili{} formats\ili{} describes\ili{} the\ili{} relative\ili{} ease\ili{} with\ili{} which\ili{} a\ili{} corresponding\ili{} lexical\ili{} encoding\ili{} can\ili{} be\ili{} converted\ili{} into\ili{} a\ili{} lexical\ili{} resource\ili{}.\ili{} This\ili{} easiness\ili{} can\ili{} allude\ili{} to\ili{} at\ili{} least\ili{} two\ili{} different\ili{} aspects\ili{},\ili{} namely\ili{} either\ili{} the\ili{} properties\ili{} of\ili{} existing\ili{} conversion\ili{} tools\ili{} or\ili{} the\ili{} engineering\ili{} task\ili{} to\ili{} produce\ili{} them\ili{}.\ili{} Ultimately\ili{},\ili{} what\ili{} really\ili{} matters\ili{} when\ili{} mapping\ili{} a\ili{} lexical\ili{} encoding\ili{} to\ili{} an\ili{} electronic\ili{} resource\ili{} is\ili{} the\ili{} mere\ili{} existence\ili{} of\ili{} software\ili{} tools\ili{} to\ili{} achieve\ili{} this\ili{}.\ili{} Obviously\ili{},\ili{} this\ili{} is\ili{} not\ili{} a\ili{} property\ili{} of\ili{} the\ili{} encoding\ili{} format\ili{} itself\ili{},\ili{} but\ili{} a\ili{} property\ili{} of\ili{} its\ili{} interface\ili{} with\ili{} the\ili{} specific\ili{} format\ili{} of\ili{} an\ili{} intended\ili{} lexical\ili{} resource\ili{}.\ili{} Thus\ili{},\ili{} in\ili{} this\ili{} view\ili{},\ili{} an\ili{} encoding\ili{} format\ili{} would\ili{} be\ili{} electronically\ili{} versatile\ili{} whenever\ili{} there\ili{} exist\ili{} many\ili{} \ili{}(and\ili{} among\ili{} them\ili{} the\ili{} desired\ili{})\ili{} conversion\ili{} tools\ili{}.\ili{} From\ili{} the\ili{} perspective\ili{} of\ili{} the\ili{} programmer\ili{},\ili{} however\ili{},\ili{} electronic\ili{} versatility\ili{} has\ili{} a\ili{} different\ili{} implication\ili{}.\ili{} There\ili{} it\ili{} is\ili{} rather\ili{} determined\ili{} by\ili{} the\ili{} effort\ili{} that\ili{} it\ili{} takes\ili{} to\ili{} implement\ili{} such\ili{} a\ili{} conversion\ili{} tool\ili{} from\ili{} scratch\ili{}.\ili{}
\ili{}
\ili{} \ili{} Even\ili{} worse\ili{},\ili{} it\ili{}'s\ili{} certainly\ili{} hard\ili{} to\ili{} say\ili{} something\ili{} conclusive\ili{} about\ili{} electronic\ili{} versatility\ili{} in\ili{} global\ili{} terms\ili{},\ili{} as\ili{} there\ili{} is\ili{} no\ili{} true\ili{} one\ili{}-to\ili{}-one\ili{} relation\ili{}.\ili{} NLP\ili{} applications\ili{} can\ili{} vary\ili{} distinctively\ili{} in\ili{} their\ili{} interface\ili{} specifications\ili{},\ili{} and\ili{} therefore\ili{} there\ili{} is\ili{} rather\ili{} a\ili{} one\ili{}-to\ili{}-many\ili{} relation\ili{} between\ili{} a\ili{} particular\ili{} lexical\ili{} encoding\ili{} and\ili{} the\ili{} lexical\ili{} resources\ili{} that\ili{} one\ili{} might\ili{} wish\ili{} to\ili{} derive\ili{} from\ili{} it\ili{}.\ili{} In\ili{} the\ili{} simplest\ili{} case\ili{},\ili{} the\ili{} lexical\ili{} encoding\ili{} can\ili{} act\ili{} as\ili{} the\ili{} lexical\ili{} resource\ili{} proper\ili{}.\ili{} Yet\ili{} presumably\ili{} in\ili{} the\ili{} majority\ili{} of\ili{} cases\ili{},\ili{} the\ili{} lexical\ili{} encoding\ili{} will\ili{} be\ili{} preprocessed\ili{} and\ili{} converted\ili{} into\ili{} something\ili{} \ili{}\textit\ili{}{less}\ili{} user\ili{}-friendly\ili{}.\ili{} This\ili{} is\ili{} most\ili{} obvious\ili{} in\ili{} graphically\ili{} enhanced\ili{} encoding\ili{} methods\ili{} where\ili{} the\ili{} lexical\ili{} resource\ili{} is\ili{} derived\ili{} from\ili{} the\ili{} underlying\ili{},\ili{} non\ili{}-graphical\ili{} representation\ili{}.\ili{} But\ili{},\ili{} of\ili{} course\ili{},\ili{} this\ili{} also\ili{} holds\ili{} for\ili{} interchange\ili{} formats\ili{} such\ili{} as\ili{} LMF\ili{} \ili{}\citep\ili{}{Francopoulo\ili{}:etal\ili{}:06}\ili{},\ili{} which\ili{} are\ili{} meant\ili{} to\ili{} provide\ili{} a\ili{} mediating\ili{} standard\ili{} and\ili{} rely\ili{} on\ili{} cumbersome\ili{} XML\ili{} or\ili{} the\ili{} like\ili{}.\ili{} \ili{}
\ili{}
Another\ili{} relevant\ili{} property\ili{} of\ili{} the\ili{} interface\ili{} between\ili{} the\ili{} lexical\ili{} encoding\ili{} and\ili{} the\ili{} lexical\ili{} resource\ili{} seems\ili{} to\ili{} be\ili{} whether\ili{} the\ili{} generalizations\ili{} expressed\ili{} in\ili{} the\ili{} lexical\ili{} encoding\ili{} are\ili{} preserved\ili{} during\ili{} conversion\ili{},\ili{} or\ili{} whether\ili{} only\ili{} fully\ili{} resolved\ili{} entries\ili{} are\ili{} included\ili{}.\ili{} From\ili{} the\ili{} point\ili{} of\ili{} view\ili{} of\ili{} the\ili{} encoder\ili{},\ili{} the\ili{} availability\ili{} of\ili{} generalizations\ili{} seems\ili{} to\ili{} be\ili{} preferred\ili{},\ili{} but\ili{} this\ili{} is\ili{} a\ili{} virtue\ili{} of\ili{} the\ili{} lexical\ili{} resource\ili{} proper\ili{},\ili{} and\ili{} also\ili{} depends\ili{} on\ili{} the\ili{} targeted\ili{} NLP\ili{} application\ili{}.\ili{}
\ili{}
Summing\ili{} up\ili{},\ili{} electronic\ili{} versatility\ili{} is\ili{} an\ili{} important\ili{} but\ili{} also\ili{} complex\ili{} virtue\ili{} that\ili{} covers\ili{} orthogonal\ili{},\ili{} or\ili{} even\ili{} conflicting\ili{},\ili{} aspects\ili{} of\ili{} the\ili{} interface\ili{} between\ili{} lexical\ili{} encodings\ili{} and\ili{} lexical\ili{} resources\ili{}.\ili{} Moreover\ili{},\ili{} given\ili{} the\ili{} heterogeneity\ili{} of\ili{} the\ili{} latter\ili{} ones\ili{},\ili{} a\ili{} general\ili{} verdict\ili{} is\ili{} often\ili{} difficult\ili{} to\ili{} obtain\ili{}.\ili{} \ili{} \ili{} \ili{} \ili{}
\ili{}
\ili{}
\ili{}%\ili{}\input\ili{}{03\ili{}-challenges}\ili{}
\ili{}
\ili{}\section\ili{}{Challenges\ili{} posed\ili{} by\ili{} MWEs}\ili{}
\ili{}\label\ili{}{sec\ili{}:challenges}\ili{}
\ili{}
From\ili{} a\ili{} general\ili{} point\ili{} of\ili{} view\ili{},\ili{} MWEs\ili{} are\ili{} in\ili{} no\ili{} way\ili{} different\ili{} from\ili{} any\ili{} other\ili{} lexical\ili{} object\ili{}:\ili{} they\ili{} can\ili{} be\ili{} encoded\ili{} using\ili{} property\ili{} name\ili{} sets\ili{} as\ili{} in\ili{} \ili{}(\ili{}\ref\ili{}{ex\ili{}:property\ili{}:sets}\ili{})\ili{} or\ili{} using\ili{} the\ili{} table\ili{} format\ili{} from\ili{} Table\ili{}~\ili{}\ref\ili{}{tab\ili{}:table\ili{}-encoding}\ili{}.\ili{} But\ili{} what\ili{} is\ili{} then\ili{} so\ili{} challenging\ili{} about\ili{} MWEs\ili{}?\ili{} On\ili{} the\ili{} one\ili{} hand\ili{},\ili{} it\ili{} is\ili{} the\ili{} peculiarity\ili{} of\ili{} the\ili{} affected\ili{} properties\ili{},\ili{} for\ili{} example\ili{},\ili{} the\ili{} property\ili{} NP\ili{}$_1\ili{}$\ili{}.Det\ili{}.the\ili{} in\ili{} the\ili{} property\ili{} set\ili{} of\ili{} \ili{}\textit\ili{}{kick\ili{} the\ili{} bucket}\ili{}.\ili{} This\ili{} is\ili{} challenging\ili{} with\ili{} respect\ili{} to\ili{} the\ili{} flexibility\ili{} of\ili{} an\ili{} encoding\ili{} format\ili{}.\ili{} On\ili{} the\ili{} other\ili{} hand\ili{},\ili{} the\ili{} interactions\ili{} between\ili{} these\ili{} and\ili{} other\ili{} properties\ili{} pose\ili{} a\ili{} challenge\ili{} to\ili{} the\ili{} power\ili{} of\ili{} an\ili{} encoding\ili{} format\ili{} to\ili{} generalize\ili{}.\ili{} In\ili{} this\ili{} section\ili{},\ili{} we\ili{} will\ili{} go\ili{} through\ili{} some\ili{} of\ili{} these\ili{} challenging\ili{} properties\ili{} and\ili{} interactions\ili{},\ili{} \ili{} confining\ili{} ourselves\ili{} mainly\ili{} to\ili{} syntax\ili{} and\ili{} morphology\ili{}.\ili{}
\ili{}
Let\ili{} us\ili{} first\ili{} examine\ili{} a\ili{} multilingual\ili{} set\ili{} of\ili{} MWE\ili{} examples\ili{}\footnote\ili{}{Each\ili{} example\ili{} is\ili{} preceded\ili{} by\ili{} its\ili{} language\ili{} code\ili{} in\ili{} parentheses\ili{}.\ili{} The\ili{} hash\ili{} \ili{}(\ili{}\\ili{}#\ili{})\ili{} character\ili{} signals\ili{} the\ili{} loss\ili{} of\ili{} the\ili{} idiomatic\ili{} reading\ili{} due\ili{} to\ili{} a\ili{} missing\ili{} property\ili{},\ili{} while\ili{} the\ili{} asterisk\ili{} \ili{}(\ili{}*\ili{})\ili{} means\ili{} ungrammaticality\ili{}.}\ili{} together\ili{} with\ili{} their\ili{} peculiarities\ili{},\ili{} which\ili{} the\ili{} MWE\ili{}-related\ili{} literature\ili{} often\ili{} calls\ili{} irregularities\ili{} or\ili{} idiosyncrasies\ili{}.\ili{} In\ili{} what\ili{} follows\ili{},\ili{} each\ili{} property\ili{} is\ili{} either\ili{} \ili{}\textsc\ili{}{defective}\ili{} or\ili{} \ili{}\textsc\ili{}{restrictive}\ili{}.\ili{} In\ili{} the\ili{} former\ili{} case\ili{},\ili{} it\ili{} excludes\ili{} a\ili{} literal\ili{} interpretation\ili{} of\ili{} a\ili{} given\ili{} object\ili{}.\ili{} In\ili{} the\ili{} latter\ili{},\ili{} it\ili{} reduces\ili{} the\ili{} number\ili{} of\ili{} possible\ili{} surface\ili{} realizations\ili{} of\ili{} a\ili{} given\ili{} object\ili{} with\ili{} respect\ili{} to\ili{} the\ili{} corresponding\ili{} literal\ili{} interpretation\ili{}.\ili{} \ili{}
\ili{}
\ili{}\begin\ili{}{enumerate}\ili{}
\ili{}\item\ili{}\label\ili{}{def\ili{}-agr}\ili{} defective\ili{} agreement\ili{},\ili{} e\ili{}.g\ili{}.\ili{} in\ili{} \ili{}(FR\ili{})\ili{} \ili{}\ilet\ili{}{grands\ili{}-mères}\ili{}{grand\ili{}\\ili{}-mothers}\ili{} the\ili{} adjective\ili{} does\ili{} not\ili{} agree\ili{} with\ili{} the\ili{} noun\ili{} in\ili{} gender\ili{},\ili{} unlike\ili{} most\ili{} regular\ili{} adjectival\ili{} modifiers\ili{};\ili{}
\ili{}\item\ili{}\label\ili{}{restr\ili{}-agr}\ili{} restrictive\ili{} agreement\ili{},\ili{} e\ili{}.g\ili{}.\ili{} \ili{}(EN\ili{})\ili{} \ili{}\ile\ili{}{to\ili{} cross\ili{} one\ili{}'s\ili{} fingers}\ili{} imposes\ili{} agreement\ili{} in\ili{} person\ili{},\ili{} number\ili{} and\ili{} gender\ili{} between\ili{} the\ili{} possessive\ili{} pronoun\ili{} and\ili{} the\ili{} subject\ili{}:\ili{} \ili{}\\ili{}#\ili{}\ile\ili{}{I\ili{} cross\ili{} his\ili{} fingers}\ili{}
\ili{}\item\ili{}\label\ili{}{restr\ili{}-par}\ili{} restrictive\ili{} paradigm\ili{},\ili{} e\ili{}.g\ili{}.\ili{} \ili{}(PL\ili{})\ili{} \ili{}\ilelt\ili{}{zjadłbym\ili{} konia\ili{} z\ili{} kopytami}\ili{}{I\ili{} would\ili{} eat\ili{} a\ili{} horse\ili{} with\ili{} its\ili{} hooves}\ili{}{I\ili{} am\ili{} very\ili{} hungry}\ili{} can\ili{} only\ili{} occur\ili{} in\ili{} conditional\ili{} mood\ili{}:\ili{} \ili{}\\ili{}#\ili{}\ilet\ili{}{zjem\ili{} konia\ili{} z\ili{} kopytami}\ili{}{I\ili{} will\ili{} eat\ili{} a\ili{} horse\ili{} with\ili{} its\ili{} hooves}\ili{};\ili{}
\ili{}\item\ili{}\label\ili{}{def\ili{}-subcat}\ili{} defective\ili{} \ili{}\isi\ili{}{subcategorization}\ili{},\ili{} i\ili{}.e\ili{}.\ili{} imposing\ili{} a\ili{} \ili{}\isi\ili{}{subcategorization}\ili{} frame\ili{} which\ili{} the\ili{} MWE\ili{} headword\ili{} does\ili{} not\ili{} admit\ili{} outside\ili{} MWEs\ili{},\ili{} e\ili{}.g\ili{}.\ili{} \ili{}(PL\ili{})\ili{} \ili{}\ilelt\ili{}{dobrze\ili{} mu\ili{} z\ili{} oczy\ili{} patrzy}\ili{}{well\ili{} him\ili{} looks\ili{} from\ili{} eyes}\ili{}{he\ili{} looks\ili{} like\ili{} a\ili{} good\ili{} person}\ili{} prohibits\ili{} a\ili{} subject\ili{}:\ili{} \ili{}*\ili{}\ilel\ili{}{uczciwość\ili{} dobrze\ili{} mu\ili{} z\ili{} oczy\ili{} patrzy}\ili{}{honesty\ili{} well\ili{} him\ili{} looks\ili{} from\ili{} eyes}\ili{},\ili{} while\ili{} \ili{}\ilet\ili{}{patrzy}\ili{}{looks}\ili{} as\ili{} a\ili{} standalone\ili{} verb\ili{} always\ili{} requires\ili{} one\ili{};\ili{} \ili{}
\ili{}\item\ili{}\label\ili{}{restr\ili{}-dia}\ili{} restrictive\ili{} diathesis\ili{},\ili{} e\ili{}.g\ili{}.\ili{} \ili{}(EN\ili{})\ili{} \ili{}\ile\ili{}{to\ili{} kick\ili{} the\ili{} bucket}\ili{} does\ili{} not\ili{} allow\ili{} \ili{}\isi\ili{}{passivization}\ili{}:\ili{} \ili{}\ile\ili{}{\ili{}\\ili{}#the\ili{} bucket\ili{} was\ili{} kicked}\ili{},\ili{} while\ili{} \ili{}(FR\ili{})\ili{} \ili{}\ilelt\ili{}{les\ili{} carottes\ili{} sont\ili{} cuites}\ili{}{the\ili{} carrots\ili{} are\ili{} cooked}\ili{}{the\ili{} situation\ili{} is\ili{} hopeless}\ili{} only\ili{} allows\ili{} passive\ili{} voice\ili{}:\ili{} \ili{}\\ili{}#\ili{}\ilelt\ili{}{on\ili{} cuit\ili{} les\ili{} carottes}\ili{}{one\ili{} cooks\ili{} the\ili{} carrots}\ili{};\ili{}
\ili{}\item\ili{}\label\ili{}{restr\ili{}-det\ili{}-mod}\ili{} restrictive\ili{} choice\ili{} of\ili{} determiners\ili{} and\ili{} modifiers\ili{},\ili{} e\ili{}.g\ili{}.\ili{} \ili{}(FR\ili{})\ili{} \ili{}\ilelt\ili{}{avoir\ili{} raison}\ili{}{to\ili{} have\ili{} reason}\ili{}{to\ili{} be\ili{} right}\ili{} allows\ili{} neither\ili{} a\ili{} determiner\ili{} nor\ili{} a\ili{} \ili{} modifier\ili{} of\ili{} the\ili{} nominal\ili{} component\ili{}:\ili{} \ili{}\ilet\ili{}{\ili{}\\ili{}#avoir\ili{} \ili{}(une\ili{})\ili{} raison\ili{} évidente}\ili{}{to\ili{} have\ili{} an\ili{} obvious\ili{} reason}\ili{};\ili{}
\ili{}\item\ili{}\label\ili{}{restr\ili{}-det\ili{}-mod\ili{}-comb}\ili{} restrictive\ili{} dependencies\ili{} \ili{} between\ili{} determiners\ili{} and\ili{} modifiers\ili{}:\ili{} \ili{}(FR\ili{})\ili{} \ili{}\ilelt\ili{}{avoir\ili{} envie}\ili{}{to\ili{} have\ili{} desire}\ili{}{to\ili{} feel\ili{} like}\ili{} admits\ili{} no\ili{} determiner\ili{} for\ili{} the\ili{} predicative\ili{} noun\ili{} \ili{}\ilet\ili{}{envie}\ili{}{desire}\ili{},\ili{} if\ili{} it\ili{} takes\ili{} no\ili{} argument\ili{} or\ili{} modifier\ili{},\ili{} or\ili{} if\ili{} it\ili{} takes\ili{} an\ili{} infinitival\ili{} argument\ili{} governed\ili{} by\ili{} the\ili{} preposition\ili{} \ili{}\ilet\ili{}{de}\ili{}{of}\ili{}:\ili{} \ili{}\ilelt\ili{}{j\ili{}'ai\ili{} envie\ili{} de\ili{} le\ili{} faire}\ili{}{I\ili{} have\ili{} desire\ili{} of\ili{} to\ili{} do\ili{} it}\ili{}{I\ili{} feel\ili{} like\ili{} doing\ili{} it}\ili{};\ili{} but\ili{} if\ili{} the\ili{} noun\ili{} is\ili{} modified\ili{} by\ili{} an\ili{} adjective\ili{},\ili{} the\ili{} determiner\ili{} is\ili{} compulsory\ili{}:\ili{} \ili{}\ilelt\ili{}{j\ili{}'ai\ili{} \ili{}\underline\ili{}{une}\ili{} envie\ili{} folle\ili{} de\ili{} le\ili{} faire}\ili{}{I\ili{} have\ili{} a\ili{} crazy\ili{} desire\ili{} of\ili{} to\ili{} do\ili{} it}\ili{}{I\ili{} feel\ili{} a\ili{} lot\ili{} like\ili{} doing\ili{} it}\ili{};\ili{}
\ili{}\item\ili{}\label\ili{}{restr\ili{}-modif}\ili{} restrictive\ili{} modification\ili{},\ili{} e\ili{}.g\ili{}.\ili{} \ili{}(FR\ili{})\ili{} \ili{}\ilet\ili{}{mener\ili{} une\ili{} vie\ili{} \ili{}(de\ili{} riche\ili{})}\ili{}{to\ili{} live\ili{} a\ili{} life\ili{} \ili{}(of\ili{} a\ili{} rich\ili{})}\ili{} imposes\ili{} an\ili{} adjectival\ili{} or\ili{} a\ili{} prepositional\ili{} modifier\ili{} on\ili{} the\ili{} nominal\ili{}:\ili{} \ili{}\\ili{}#\ili{}\ilet\ili{}{il\ili{} mène\ili{} une\ili{} vie}\ili{}{he\ili{} leads\ili{} a\ili{} life}\ili{};\ili{}
\ili{}\item\ili{}\label\ili{}{restr\ili{}-linear}\ili{} restrictive\ili{} linearization\ili{},\ili{} e\ili{}.g\ili{}.\ili{} \ili{}(EN\ili{})\ili{} \ili{}\ile\ili{}{drink\ili{} and\ili{} drive}\ili{} requires\ili{} the\ili{} strict\ili{} order\ili{} of\ili{} its\ili{} coordinated\ili{} verbs\ili{},\ili{} violating\ili{} this\ili{} constraint\ili{} leads\ili{} to\ili{} the\ili{} loss\ili{} of\ili{} the\ili{} idiomatic\ili{} reading\ili{}:\ili{} \ili{}\\ili{}#\ili{}\ile\ili{}{drive\ili{} and\ili{} drink}\ili{};\ili{}
\ili{}\item\ili{}\label\ili{}{restr\ili{}-lex\ili{}-select}\ili{} restrictive\ili{} lexical\ili{} selection\ili{},\ili{} i\ili{}.e\ili{}.\ili{} imposing\ili{} particular\ili{} lexical\ili{} realizations\ili{} of\ili{} certain\ili{} syntactic\ili{} arguments\ili{},\ili{} e\ili{}.g\ili{}.\ili{} \ili{}(EN\ili{})\ili{} \ili{}\ile\ili{}{to\ili{} pull\ili{} someone\ili{}'s\ili{} leg}\ili{} requires\ili{} the\ili{} head\ili{} verb\ili{} \ili{}\ile\ili{}{pull}\ili{} with\ili{} a\ili{} direct\ili{} object\ili{} headed\ili{} by\ili{} \ili{}\ile\ili{}{leg}\ili{}:\ili{} \ili{}\\ili{}#\ili{}\ile\ili{}{to\ili{} pull\ili{} one\ili{}'s\ili{} arm\ili{}/member}\ili{}.\ili{} \ili{}
\ili{}\end\ili{}{enumerate}\ili{}
\ili{}
Note\ili{} that\ili{} while\ili{} the\ili{} above\ili{} properties\ili{} are\ili{} perceived\ili{} as\ili{} unexpected\ili{} or\ili{} unpredictable\ili{},\ili{} they\ili{} are\ili{} most\ili{} often\ili{} shared\ili{} with\ili{} other\ili{} MWEs\ili{},\ili{} therefore\ili{},\ili{} in\ili{} our\ili{} understanding\ili{} \ili{}(cf\ili{}.\ili{} Section\ili{}~\ili{}\ref\ili{}{sec\ili{}:notion\ili{}-regularity}\ili{})\ili{},\ili{} they\ili{} are\ili{} regular\ili{}.\ili{} To\ili{} make\ili{} this\ili{} more\ili{} precise\ili{},\ili{} recall\ili{} that\ili{} regularity\ili{} of\ili{} a\ili{} property\ili{} is\ili{} not\ili{} absolute\ili{} but\ili{} relative\ili{} to\ili{} a\ili{} given\ili{} set\ili{} of\ili{} objects\ili{} \ili{}$E\ili{}$\ili{}.\ili{} In\ili{} linguistic\ili{} modeling\ili{},\ili{} we\ili{} tend\ili{} to\ili{} group\ili{} objects\ili{} into\ili{} sets\ili{} based\ili{} on\ili{} their\ili{} similarities\ili{} rather\ili{} than\ili{} their\ili{} discrepancies\ili{}.\ili{} For\ili{} instance\ili{},\ili{} in\ili{} valence\ili{}-oriented\ili{} modeling\ili{} \ili{}(such\ili{} as\ili{} Walenty\ili{} or\ili{} PART\ili{}-II\ili{} described\ili{} in\ili{} Sections\ili{}~\ili{}\ref\ili{}{sec\ili{}:walenty}\ili{} and\ili{} \ili{}\ref\ili{}{sec\ili{}:patr\ili{}-datr}\ili{},\ili{} respectively\ili{},\ili{} or\ili{} IDION\ili{} and\ili{} the\ili{} MWE\ili{} lexicon\ili{} of\ili{} NorGram\ili{} discussed\ili{} in\ili{} chapters\ili{} MARKANTONATOU\ili{} and\ili{} DYVIK\ili{} of\ili{} this\ili{} volume\ili{})\ili{},\ili{} verbal\ili{} constructions\ili{} are\ili{} grouped\ili{} according\ili{} to\ili{} the\ili{} lemma\ili{} of\ili{} their\ili{} head\ili{} verb\ili{},\ili{} whereas\ili{} in\ili{} more\ili{} constructionist\ili{} approaches\ili{} \ili{}(like\ili{} DUELME\ili{} and\ili{} XMG\ili{},\ili{} introduced\ili{} in\ili{} Sections\ili{}~\ili{}\ref\ili{}{sec\ili{}:duelme}\ili{} and\ili{} \ili{}\ref\ili{}{sec\ili{}:xmg}\ili{})\ili{},\ili{} they\ili{} are\ili{} grouped\ili{} by\ili{} the\ili{} \ili{}\isi\ili{}{syntactic\ili{} structure}\ili{} of\ili{} their\ili{} \ili{}\isi\ili{}{subcategorization}\ili{} frames\ili{}.\ili{} Such\ili{} properties\ili{} used\ili{} to\ili{} group\ili{} objects\ili{} become\ili{} trivially\ili{} regular\ili{} properties\ili{} of\ili{} these\ili{} groups\ili{} \ili{}(since\ili{} they\ili{} are\ili{} shared\ili{} by\ili{} all\ili{} objects\ili{} of\ili{} a\ili{} group\ili{})\ili{}.\ili{} Most\ili{} other\ili{} properties\ili{} have\ili{} a\ili{} varying\ili{} degree\ili{} of\ili{} regularity\ili{} and\ili{} are\ili{} only\ili{} rarely\ili{} truly\ili{} idiosyncratic\ili{}.\ili{} \ili{}
\ili{}
As\ili{} an\ili{} example\ili{},\ili{} let\ili{} us\ili{} consider\ili{} a\ili{} set\ili{} of\ili{} \ili{}\ili\ili{}{English}\ili{} verbal\ili{} expressions\ili{},\ili{} each\ili{} of\ili{} which\ili{} is\ili{} headed\ili{} by\ili{} a\ili{} verb\ili{},\ili{} taking\ili{} a\ili{} subject\ili{} and\ili{} a\ili{} direct\ili{} object\ili{},\ili{} and\ili{} admitting\ili{} modifiers\ili{},\ili{} e\ili{}.g\ili{}.\ili{} \ili{}(EN\ili{})\ili{} \ili{}\ile\ili{}{John\ili{} pulled\ili{} the\ili{} heavy\ili{} door}\ili{}.\ili{} In\ili{} this\ili{} set\ili{},\ili{} the\ili{} property\ili{} of\ili{} allowing\ili{} any\ili{} head\ili{} verb\ili{} with\ili{} the\ili{} proper\ili{} \ili{}\isi\ili{}{subcategorization}\ili{} frame\ili{},\ili{} is\ili{} much\ili{} more\ili{} regular\ili{} than\ili{} restricting\ili{} it\ili{} to\ili{} the\ili{} verb\ili{} \ili{}\ile\ili{}{kick}\ili{}.\ili{} Furthermore\ili{},\ili{} the\ili{} property\ili{} of\ili{} allowing\ili{} \ili{}\isi\ili{}{passivization}\ili{} is\ili{} more\ili{} regular\ili{} than\ili{} prohibiting\ili{} passive\ili{} voice\ili{},\ili{} like\ili{} in\ili{} \ili{}\ilet\ili{}{John\ili{} kicked\ili{} the\ili{} bucket}\ili{}{John\ili{} died}\ili{}.\ili{} Also\ili{},\ili{} allowing\ili{} a\ili{} possessive\ili{} determiner\ili{} of\ili{} the\ili{} object\ili{},\ili{} as\ili{} in\ili{} \ili{}\ile\ili{}{John\ili{} pushed\ili{} the\ili{}/my\ili{} door}\ili{} is\ili{} more\ili{} regular\ili{} than\ili{} imposing\ili{} it\ili{},\ili{} as\ili{} in\ili{} \ili{}\ilet\ili{}{John\ili{} broke\ili{} his\ili{}/her\ili{}/our\ili{} fall}\ili{}{John\ili{} made\ili{} his\ili{}/her\ili{}/our\ili{} fall\ili{} less\ili{} forceful}\ili{},\ili{} which\ili{} itself\ili{} is\ili{} more\ili{} regular\ili{} than\ili{} imposing\ili{} a\ili{} possessive\ili{} which\ili{} agrees\ili{} with\ili{} the\ili{} subject\ili{},\ili{} as\ili{} in\ili{} \ili{}\ile\ili{}{John\ili{} crossed\ili{} his\ili{} fingers}\ili{}.\ili{} This\ili{} last\ili{} property\ili{} is\ili{},\ili{} however\ili{},\ili{} still\ili{} regular\ili{}.\ili{} In\ili{} order\ili{} for\ili{} it\ili{} to\ili{} be\ili{} idiosyncratic\ili{},\ili{} \ili{}\ilet\ili{}{John\ili{} crossed\ili{} his\ili{} fingers}\ili{}{John\ili{} wishes\ili{} luck}\ili{} and\ili{} \ili{}\ilet\ili{}{John\ili{} held\ili{} his\ili{} tongue}\ili{}{John\ili{} refrained\ili{} from\ili{} expressing\ili{} his\ili{} view}\ili{} could\ili{} not\ili{} co\ili{}-occur\ili{} in\ili{} the\ili{} same\ili{} object\ili{} set\ili{},\ili{} which\ili{} would\ili{} hinder\ili{} the\ili{} usability\ili{} of\ili{} such\ili{} a\ili{} set\ili{} for\ili{} linguistic\ili{} modeling\ili{}.\ili{} Without\ili{} resorting\ili{} to\ili{} such\ili{} artificial\ili{} choice\ili{} of\ili{} object\ili{} sets\ili{},\ili{} Property\ili{}~\ili{}\ref\ili{}{restr\ili{}-lex\ili{}-select}\ili{} is\ili{} one\ili{} of\ili{} the\ili{} rare\ili{} truly\ili{} idiosyncratic\ili{} properties\ili{},\ili{} since\ili{} it\ili{} is\ili{} usually\ili{} specific\ili{} to\ili{} one\ili{} MWE\ili{} only\ili{},\ili{} except\ili{} in\ili{} case\ili{} of\ili{} truly\ili{} ambiguous\ili{} MWEs\ili{} like\ili{} \ili{}\ilet\ili{}{to\ili{} go\ili{} on}\ili{}{to\ili{} continue\ili{},\ili{} to\ili{} happen}\ili{}.\ili{}
\ili{}
Note\ili{} finally\ili{} that\ili{} one\ili{} MWE\ili{} usually\ili{} exhibits\ili{} different\ili{} properties\ili{} of\ili{} varying\ili{} degrees\ili{} of\ili{} regularity\ili{}.\ili{} For\ili{} instance\ili{},\ili{} while\ili{} the\ili{} components\ili{} of\ili{} \ili{}(FR\ili{})\ili{} \ili{}\ilet\ili{}{grands\ili{}-mères}\ili{}{grandmothers}\ili{} do\ili{} not\ili{} agree\ili{} in\ili{} gender\ili{},\ili{} they\ili{} do\ili{} agree\ili{} in\ili{} number\ili{}.\ili{} While\ili{} \ili{}(PL\ili{})\ili{} \ili{}\ilelt\ili{}{zjadłbym\ili{} konia\ili{} z\ili{} kopytami}\ili{}{I\ili{} would\ili{} eat\ili{} a\ili{} horse\ili{} with\ili{} its\ili{} hooves}\ili{}{I\ili{} am\ili{} very\ili{} hungry}\ili{} requires\ili{} conditional\ili{} mood\ili{},\ili{} it\ili{} has\ili{} a\ili{} highly\ili{} regular\ili{} inflection\ili{} for\ili{} person\ili{} and\ili{} number\ili{}.\ili{} While\ili{} the\ili{} object\ili{} in\ili{} \ili{}(EN\ili{})\ili{} \ili{}\ile\ili{}{to\ili{} pull\ili{} someone\ili{}'s\ili{} leg}\ili{} is\ili{} partly\ili{} lexicalized\ili{},\ili{} the\ili{} subject\ili{} is\ili{} not\ili{}.\ili{} While\ili{} \ili{}(EN\ili{})\ili{} \ili{}\ile\ili{}{to\ili{} kick\ili{} the\ili{} bucket}\ili{} cannot\ili{} be\ili{} passivized\ili{},\ili{} it\ili{} does\ili{} admit\ili{} a\ili{} restricted\ili{} number\ili{} of\ili{} internal\ili{} modifiers\ili{} as\ili{} in\ili{} \ili{}\ile\ili{}{to\ili{} kick\ili{} the\ili{} proverbial\ili{} bucket}\ili{},\ili{} etc\ili{}.\ili{}
\ili{}
As\ili{} a\ili{} conclusion\ili{},\ili{} the\ili{} challenging\ili{} nature\ili{} of\ili{} MWE\ili{} is\ili{} manyfold\ili{}:\ili{} \ili{}(i\ili{})\ili{} regularity\ili{} of\ili{} properties\ili{} of\ili{} MWEs\ili{} is\ili{} scale\ili{}-wise\ili{},\ili{} \ili{}(ii\ili{})\ili{} properties\ili{} of\ili{} different\ili{} degrees\ili{} of\ili{} regularity\ili{} co\ili{}-occur\ili{} in\ili{} each\ili{} MWE\ili{},\ili{} \ili{}(iii\ili{})\ili{} truly\ili{} idiosyncratic\ili{} properties\ili{} are\ili{} rare\ili{} \ili{}(under\ili{} the\ili{} usual\ili{} similarity\ili{}-oriented\ili{} grouping\ili{} strategies\ili{})\ili{},\ili{} \ili{}(iv\ili{})\ili{} shared\ili{} properties\ili{} can\ili{} be\ili{} unforeseen\ili{} \ili{}(cf\ili{}.\ili{} Property\ili{}~\ili{}\ref\ili{}{restr\ili{}-det\ili{}-mod\ili{}-comb}\ili{})\ili{},\ili{} so\ili{} listing\ili{} them\ili{} all\ili{} in\ili{} advance\ili{} is\ili{} hard\ili{}.\ili{} A\ili{} general\ili{}-purpose\ili{} encoding\ili{} format\ili{} should\ili{} possibly\ili{} face\ili{} all\ili{} these\ili{} challenges\ili{} simultaneously\ili{}.\ili{} Note\ili{} also\ili{} the\ili{} similarity\ili{} of\ili{} observations\ili{} \ili{}(i\ili{})\ili{} and\ili{} \ili{}(ii\ili{})\ili{} with\ili{} the\ili{} notion\ili{} of\ili{} a\ili{} \ili{}\emph\ili{}{flexibility\ili{} continuum}\ili{} in\ili{} idioms\ili{},\ili{} discussed\ili{} in\ili{} chapter\ili{} SHEINFUX\ili{} of\ili{} this\ili{} volume\ili{}.\ili{}
\ili{}
\ili{}%\ili{}\input\ili{}{04\ili{}-formats}\ili{}
\ili{}\section\ili{}{Fixed\ili{} MWE\ili{} encoding\ili{} formats}\ili{}
\ili{}\label\ili{}{sec\ili{}:fixed}\ili{}
\ili{}
\ili{}
While\ili{} lexical\ili{} approaches\ili{} dedicated\ili{} to\ili{} a\ili{} large\ili{} variety\ili{} of\ili{} MWEs\ili{} have\ili{} a\ili{} relatively\ili{} long\ili{} linguistic\ili{} tradition\ili{},\ili{} notably\ili{} with\ili{} \ili{}\cite\ili{}{Gross\ili{}:86}\ili{} and\ili{} \ili{}\cite\ili{}{Melcuketal\ili{}:88}\ili{},\ili{} NLP\ili{}-oriented\ili{} work\ili{} on\ili{} lexical\ili{} encoding\ili{} of\ili{} MWEs\ili{} has\ili{} mainly\ili{} dealt\ili{} with\ili{} continuous\ili{} instances\ili{} \ili{}\citep\ili{}{Savary\ili{}:08}\ili{}.\ili{} More\ili{} recently\ili{},\ili{} proposals\ili{} have\ili{} been\ili{} put\ili{} forward\ili{} which\ili{} also\ili{} take\ili{} \ili{}\isi\ili{}{verbal\ili{} MWEs}\ili{} into\ili{} account\ili{} whose\ili{} components\ili{} are\ili{} discontinuously\ili{} linearized\ili{}.\ili{} Here\ili{} we\ili{} study\ili{} two\ili{} instances\ili{} of\ili{} such\ili{} approaches\ili{} tailored\ili{} to\ili{} specific\ili{} languages\ili{}:\ili{} DuELME\ili{} \ili{}\citep\ili{}{gregoire\ili{}:10}\ili{} for\ili{} \ili{}\ili\ili{}{Dutch}\ili{} and\ili{} Walenty\ili{} \ili{}\citep\ili{}{prz\ili{}:etal\ili{}:14b}\ili{} for\ili{} Polish\ili{}.\ili{} They\ili{} stand\ili{} out\ili{} as\ili{}:\ili{} \ili{}(i\ili{})\ili{} having\ili{} been\ili{} designed\ili{} with\ili{} a\ili{} \ili{}(relative\ili{})\ili{} theory\ili{}-neutrality\ili{} in\ili{} mind\ili{},\ili{} \ili{}(ii\ili{})\ili{} having\ili{} resulted\ili{} in\ili{} MWE\ili{} lexicons\ili{} of\ili{} several\ili{} thousands\ili{} of\ili{} entries\ili{},\ili{} \ili{}(iii\ili{})\ili{} having\ili{} been\ili{} coupled\ili{} with\ili{} real\ili{}-size\ili{} grammars\ili{},\ili{} so\ili{} as\ili{} to\ili{} test\ili{} their\ili{} usability\ili{} for\ili{} parsing\ili{}.\ili{} At\ili{} the\ili{} same\ili{} time\ili{},\ili{} DueLME\ili{} and\ili{} Walenty\ili{} can\ili{} be\ili{} characterized\ili{} as\ili{} fixed\ili{} encoding\ili{} formats\ili{} in\ili{} the\ili{} sense\ili{} that\ili{} their\ili{} encoding\ili{} language\ili{} \ili{}(basically\ili{} the\ili{} set\ili{} of\ili{} property\ili{} names\ili{} and\ili{} their\ili{} interpretation\ili{})\ili{} cannot\ili{} be\ili{} freely\ili{} chosen\ili{} or\ili{} extended\ili{}.\ili{}
\ili{}
\ili{}\subsection\ili{}{DuELME}\ili{}
\ili{}\label\ili{}{sec\ili{}:duelme}\ili{}
\ili{}
DuELME\ili{} \ili{}(\ili{}\ili\ili{}{Dutch}\ili{} Electronic\ili{} Lexicon\ili{} of\ili{} \ili{}\isi\ili{}{Multiword\ili{} Expressions}\ili{},\ili{} \ili{}\citealt\ili{}{gregoire\ili{}:10}\ili{})\ili{} is\ili{} an\ili{} electronic\ili{} lexicon\ili{} comprising\ili{} roughly\ili{} 5\ili{},000\ili{} \ili{}\ili\ili{}{Dutch}\ili{} multiword\ili{} expressions\ili{}.\ili{}\footnote\ili{}{\ili{}\url\ili{}{http\ili{}:\ili{}/\ili{}/duelme\ili{}.clarin\ili{}.inl\ili{}.nl\ili{}/}}\ili{} It\ili{} distinguishes\ili{} two\ili{} sorts\ili{} of\ili{} descriptions\ili{},\ili{} pattern\ili{} descriptions\ili{} and\ili{} MWE\ili{} descriptions\ili{},\ili{} which\ili{} are\ili{} composed\ili{} of\ili{} non\ili{}-intersecting\ili{} sets\ili{} of\ili{} predefined\ili{} fields\ili{}.\ili{} Patterns\ili{},\ili{} also\ili{} called\ili{} \ili{}\textit\ili{}{parameterized\ili{} equivalence\ili{} classes}\ili{},\ili{} represent\ili{} mainly\ili{} the\ili{} syntactic\ili{} structures\ili{} of\ili{} MWEs\ili{} and\ili{} the\ili{} part\ili{}-of\ili{}-speech\ili{} tags\ili{} of\ili{} their\ili{} leaves\ili{}.\ili{} MWE\ili{} descriptions\ili{} express\ili{} MWE\ili{}-specific\ili{} lexical\ili{} and\ili{} morpho\ili{}-syntactic\ili{} constraints\ili{}.\ili{}
\ili{}
Figure\ili{}~\ili{}\ref\ili{}{fig\ili{}:duelme}\ili{} shows\ili{} a\ili{} sample\ili{} pattern\ili{} \ili{}(Lines\ili{} 1\ili{}-\ili{}-5\ili{})\ili{},\ili{} called\ili{} \ili{}\texttt\ili{}{ec1}\ili{},\ili{} and\ili{} a\ili{} MWE\ili{} entry\ili{} \ili{}(Lines\ili{} 7\ili{}-\ili{}-11\ili{})\ili{} assigned\ili{} to\ili{} it\ili{}:\ili{} \ili{}(NL\ili{})\ili{} \ili{}\ilelt\ili{}{zijn\ili{} kansen\ili{} waarnemen}\ili{}{one\ili{}'s\ili{} chances\ili{} perceive}\ili{}{to\ili{} seize\ili{} the\ili{} opportunity}\ili{}.\ili{}
\ili{}\begin\ili{}{figure}\ili{}[h\ili{}]\ili{}
\ili{}\begin\ili{}{duelme}\ili{}
\ili{}%\ili{} Pattern\ili{} description\ili{}
PATTERN\ili{}|\ili{}%\ili{}\_\ili{}%\ili{}|NAME\ili{} ec1\ili{}
POS\ili{} d\ili{} n\ili{} v\ili{}
PATTERN\ili{} \ili{} \ili{}[\ili{}.VP\ili{} \ili{}[\ili{}.obj1\ili{}:NP\ili{} \ili{}[\ili{}.det\ili{}:D\ili{} \ili{}(1\ili{})\ili{} \ili{}]\ili{} \ili{}[\ili{}.hd\ili{}:N1\ili{} \ili{}(2\ili{})\ili{} \ili{}]\ili{}]\ili{} \ili{}[\ili{}.hd\ili{}:V\ili{} \ili{}(3\ili{})\ili{} \ili{}]\ili{}]\ili{}
DESCRIPTION\ili{} Expressions\ili{} headed\ili{} by\ili{} a\ili{} verb\ili{},\ili{} taking\ili{} a\ili{} direct\ili{} object\ili{} consisting\ili{} of\ili{} a\ili{} fixed\ili{} determiner\ili{} and\ili{} a\ili{} modifiable\ili{} noun\ili{}.\ili{}
\ili{}
\ili{}%\ili{} \ili{}|\ili{}%\ili{}\textrm\ili{}{\ili{}\texttt\ili{}{MWE}}\ili{}%\ili{}|\ili{} description\ili{}
EXPRESSION\ili{} zijn\ili{} kansen\ili{} waarnemen\ili{}
CL\ili{} zijn\ili{} kans\ili{}[pl\ili{}]\ili{} waar_nemen\ili{}[part\ili{}]\ili{}
PATTERN\ili{}|\ili{}%\ili{}\_\ili{}%\ili{}|NAME\ili{} ec1\ili{}
EXAMPLE\ili{} hij\ili{} heeft\ili{} zijn\ili{} kansen\ili{} waargenomen\ili{}
\ili{}\end\ili{}{duelme}\ili{}
\ili{} \ili{} \ili{}\caption\ili{}{DuELME\ili{} pattern\ili{} description\ili{} ec1\ili{} \ili{}(from\ili{} \ili{}\citealt\ili{}{gregoire\ili{}:07}\ili{})\ili{} and\ili{} MWE\ili{} description\ili{} of\ili{} \ili{}(NL\ili{})\ili{} \ili{}\ilelt\ili{}{zijn\ili{} kansen\ili{} waarnemen}\ili{}{one\ili{}'s\ili{} chances\ili{} perceive}\ili{}{to\ili{} seize\ili{} the\ili{} opportunity}\ili{} \ili{}(from\ili{} \ili{}\citealt\ili{}{gregoire\ili{}:10}\ili{})}\ili{}
\ili{} \ili{} \ili{}\label\ili{}{fig\ili{}:duelme}\ili{} \ili{} \ili{} \ili{}
\ili{}\end\ili{}{figure}\ili{}
The\ili{} pattern\ili{} describes\ili{} expressions\ili{} headed\ili{} by\ili{} a\ili{} verb\ili{},\ili{} taking\ili{} a\ili{} direct\ili{} object\ili{} consisting\ili{} of\ili{} a\ili{} fixed\ili{} determiner\ili{} and\ili{} a\ili{} modifiable\ili{} noun\ili{}.\ili{} The\ili{} POS\ili{}-entitled\ili{} Line\ili{} 3\ili{} lists\ili{} the\ili{} parts\ili{} of\ili{} speech\ili{} of\ili{} MWE\ili{} components\ili{}.\ili{} The\ili{} PATTERN\ili{}-entitled\ili{} Line\ili{} 4\ili{} shows\ili{} the\ili{} \ili{}\isi\ili{}{syntactic\ili{} structure}\ili{},\ili{} roughly\ili{},\ili{} as\ili{} a\ili{} dependency\ili{} tree\ili{} where\ili{} syntactic\ili{} categories\ili{} \ili{}(\ili{}\texttt\ili{}{VP}\ili{},\ili{} \ili{}\texttt\ili{}{NP}\ili{},\ili{} \ili{}\texttt\ili{}{D}\ili{},\ili{} \ili{}\texttt\ili{}{N1}\ili{}\footnote\ili{}{The\ili{} \ili{}\texttt\ili{}{N1}\ili{} category\ili{} denotes\ili{} an\ili{} NP\ili{} of\ili{} which\ili{} some\ili{} elements\ili{} are\ili{} lexically\ili{} fixed\ili{},\ili{} but\ili{} which\ili{} is\ili{} still\ili{} subject\ili{} to\ili{} standard\ili{} grammar\ili{} rules}\ili{},\ili{} \ili{}\texttt\ili{}{V}\ili{})\ili{} and\ili{} dependency\ili{} labels\ili{} \ili{}(\ili{}\texttt\ili{}{obj1}\ili{},\ili{} \ili{}\texttt\ili{}{det}\ili{},\ili{} \ili{}\texttt\ili{}{hd}\ili{})\ili{} are\ili{} marked\ili{} explicitly\ili{},\ili{} and\ili{} some\ili{} of\ili{} the\ili{} leaves\ili{} are\ili{} indexed\ili{} \ili{}(\ili{}\texttt\ili{}{1}\ili{},\ili{} \ili{}\texttt\ili{}{2}\ili{},\ili{} \ili{}\texttt\ili{}{3}\ili{})\ili{} so\ili{} as\ili{} to\ili{} be\ili{} matched\ili{} with\ili{} components\ili{} of\ili{} a\ili{} particular\ili{} MWE\ili{}.\ili{} Thus\ili{},\ili{} the\ili{} components\ili{} \ili{}\ilet\ili{}{zijn}\ili{}{one\ili{}'s}\ili{},\ili{} \ili{}\ilet\ili{}{kansen}\ili{}{chances}\ili{} and\ili{} \ili{}\ilet\ili{}{waarnemen}\ili{}{perceive}\ili{} of\ili{} the\ili{} MWE\ili{} in\ili{} Lines\ili{} 8\ili{}-\ili{}-9\ili{} are\ili{} implicitly\ili{} co\ili{}-indexed\ili{} with\ili{} the\ili{} \ili{}\texttt\ili{}{det\ili{}:D}\ili{},\ili{} \ili{}\texttt\ili{}{hd\ili{}:N1}\ili{} and\ili{} \ili{}\texttt\ili{}{hd\ili{}:V}\ili{} nodes\ili{} in\ili{} the\ili{} \ili{}\texttt\ili{}{ec1}\ili{} pattern\ili{}.\ili{} Moreover\ili{},\ili{} the\ili{} component\ili{} list\ili{} \ili{}(CL\ili{})\ili{} in\ili{} Line\ili{} 9\ili{} lists\ili{} the\ili{} MWE\ili{}-specific\ili{} values\ili{} of\ili{} the\ili{} \ili{}`\ili{}`parameters\ili{}'\ili{}'\ili{} for\ili{} the\ili{} pattern\ili{},\ili{} i\ili{}.e\ili{}.\ili{} the\ili{} lemmas\ili{} of\ili{} all\ili{} components\ili{},\ili{} as\ili{} well\ili{} as\ili{} some\ili{} morphosyntactic\ili{} constraints\ili{},\ili{} here\ili{}:\ili{} \ili{}\ilet\ili{}{kans}\ili{}{chance}\ili{} must\ili{} be\ili{} in\ili{} plural\ili{} \ili{}(\ili{}\texttt\ili{}{pl}\ili{})\ili{},\ili{} and\ili{} \ili{}\ilet\ili{}{waarnemen}\ili{}{perceive}\ili{} is\ili{} a\ili{} separable\ili{} particle\ili{} verb\ili{} \ili{}(\ili{}\texttt\ili{}{part}\ili{})\ili{}.\ili{} \ili{}
\ili{}
This\ili{} approach\ili{} is\ili{} constructionist\ili{} in\ili{} the\ili{} sense\ili{} that\ili{} MWEs\ili{} are\ili{} grouped\ili{} into\ili{} sets\ili{} based\ili{} on\ili{} their\ili{} structure\ili{} \ili{}(rather\ili{} than\ili{} their\ili{} headword\ili{})\ili{}.\ili{} While\ili{} the\ili{} syntax\ili{} of\ili{} patterns\ili{} seems\ili{} theory\ili{}-specific\ili{},\ili{} they\ili{} might\ili{} be\ili{} seen\ili{} rather\ili{} as\ili{} identifiers\ili{} of\ili{} equivalence\ili{} classes\ili{},\ili{} allowing\ili{} to\ili{} group\ili{} MWEs\ili{} of\ili{} the\ili{} same\ili{} structure\ili{},\ili{} whatever\ili{} the\ili{} syntactic\ili{} formalism\ili{} used\ili{} to\ili{} express\ili{} this\ili{} structure\ili{}.\ili{}\footnote\ili{}{Jan\ili{} Odijk\ili{},\ili{} personal\ili{} communication\ili{} 21\ili{} September\ili{} 2015\ili{}.}\ili{} DuELME\ili{}'s\ili{} view\ili{} of\ili{} the\ili{} regularity\ili{} is\ili{} binary\ili{},\ili{} which\ili{} is\ili{} reflected\ili{} by\ili{} its\ili{} two\ili{}-level\ili{} description\ili{} paradigm\ili{}.\ili{} Namely\ili{},\ili{} it\ili{} is\ili{} assumed\ili{} that\ili{} each\ili{} type\ili{} of\ili{} a\ili{} \ili{}\isi\ili{}{syntactic\ili{} structure}\ili{} has\ili{} some\ili{} \ili{}`\ili{}`generally\ili{} regular\ili{}'\ili{}'\ili{} properties\ili{} covered\ili{} by\ili{} general\ili{} grammar\ili{} rules\ili{}.\ili{} These\ili{} properties\ili{} are\ili{} not\ili{} described\ili{} in\ili{} the\ili{} lexicon\ili{} but\ili{} symbolized\ili{} by\ili{} patterns\ili{}.\ili{} Conversely\ili{},\ili{} the\ili{} MWE\ili{}-specific\ili{} properties\ili{} are\ili{} described\ili{} in\ili{} MWE\ili{} entries\ili{}.\ili{} For\ili{} instance\ili{},\ili{} while\ili{} the\ili{} number\ili{} of\ili{} \ili{}\ilet\ili{}{kans}\ili{}{chance}\ili{} is\ili{} restricted\ili{} to\ili{} plural\ili{} in\ili{} Line\ili{} 9\ili{},\ili{} its\ili{} other\ili{} grammatical\ili{} features\ili{} are\ili{} not\ili{} specified\ili{} since\ili{} they\ili{} are\ili{} supposed\ili{} governed\ili{} by\ili{} grammar\ili{} rules\ili{}.\ili{} This\ili{} principle\ili{} avoids\ili{} some\ili{} grammar\ili{} vs\ili{}.\ili{} lexicon\ili{} redundancy\ili{}.\ili{} Note\ili{},\ili{} however\ili{},\ili{} that\ili{} the\ili{} choice\ili{} of\ili{} properties\ili{} to\ili{} be\ili{} included\ili{} in\ili{} patterns\ili{} is\ili{} rather\ili{} arbitrary\ili{} and\ili{} in\ili{} most\ili{} cases\ili{} leads\ili{} to\ili{} partly\ili{} redundant\ili{} descriptions\ili{}.\ili{} For\ili{} instance\ili{},\ili{} the\ili{} \ili{}\texttt\ili{}{part}\ili{} property\ili{} in\ili{} Line\ili{} 9\ili{} is\ili{} shared\ili{} with\ili{} other\ili{} MWEs\ili{} containing\ili{} separable\ili{} particle\ili{} verbs\ili{},\ili{} and\ili{} has\ili{} to\ili{} be\ili{} specified\ili{} for\ili{} each\ili{} of\ili{} them\ili{}.\ili{} This\ili{} redundancy\ili{} at\ili{} the\ili{} level\ili{} of\ili{} MWE\ili{} descriptions\ili{} could\ili{} be\ili{} avoided\ili{},\ili{} if\ili{} the\ili{} \ili{}\texttt\ili{}{ec1}\ili{} pattern\ili{} were\ili{} restricted\ili{} to\ili{} \ili{}\texttt\ili{}{d\ili{}-n\ili{}-v}\ili{} constructions\ili{} containing\ili{} separable\ili{} particle\ili{} verbs\ili{} only\ili{}.\ili{} This\ili{} would\ili{},\ili{} however\ili{},\ili{} require\ili{} a\ili{} new\ili{} pattern\ili{} with\ili{} the\ili{} same\ili{} structure\ili{} but\ili{} a\ili{} different\ili{} verb\ili{} type\ili{} selection\ili{},\ili{} in\ili{} order\ili{} to\ili{} cover\ili{} e\ili{}.g\ili{}.\ili{} \ili{}(NL\ili{})\ili{} \ili{}\ilel\ili{}{zijn\ili{} debuut\ili{} maken}\ili{}{to\ili{} make\ili{} one\ili{}'s\ili{} debut}\ili{},\ili{} which\ili{} would\ili{} lead\ili{} to\ili{} redundancy\ili{} at\ili{} the\ili{} level\ili{} of\ili{} patterns\ili{}.\ili{} Since\ili{} there\ili{} is\ili{} no\ili{} notion\ili{} of\ili{} reference\ili{},\ili{} or\ili{} reuse\ili{},\ili{} among\ili{} the\ili{} 141\ili{} pattern\ili{} descriptions\ili{} that\ili{} DuELME\ili{} comprises\ili{} \ili{}\citep\ili{}{gregoire\ili{}:07}\ili{},\ili{} such\ili{} redundancy\ili{} could\ili{} not\ili{} be\ili{} avoided\ili{}.\ili{}
\ili{}
As\ili{} a\ili{} conclusion\ili{},\ili{} the\ili{} distinction\ili{} between\ili{} patterns\ili{} and\ili{} MWE\ili{} descriptions\ili{} introduces\ili{} a\ili{} limited\ili{} degree\ili{} of\ili{} factorization\ili{}.\ili{} While\ili{} some\ili{} syntactic\ili{} constraints\ili{},\ili{} e\ili{}.g\ili{}.\ili{} dependencies\ili{},\ili{} are\ili{} mentioned\ili{} more\ili{} or\ili{} less\ili{} explicitly\ili{} in\ili{} patterns\ili{},\ili{} some\ili{} other\ili{} syntactic\ili{} properties\ili{} are\ili{} implicit\ili{} \ili{}(supposed\ili{} to\ili{} be\ili{} covered\ili{} by\ili{} the\ili{} grammar\ili{} and\ili{} known\ili{} to\ili{} the\ili{} NLP\ili{} system\ili{})\ili{}.\ili{} Some\ili{} specific\ili{} constraints\ili{},\ili{} e\ili{}.g\ili{}.\ili{} restrictive\ili{} agreement\ili{},\ili{} diathesis\ili{},\ili{} determination\ili{},\ili{} modification\ili{} and\ili{} linearization\ili{},\ili{} discussed\ili{} in\ili{} Points\ili{}~\ili{}\ref\ili{}{restr\ili{}-agr}\ili{} and\ili{} \ili{}\ref\ili{}{restr\ili{}-dia}\ili{}-\ili{}-\ili{}\ref\ili{}{restr\ili{}-linear}\ili{} in\ili{} Section\ili{}~\ili{}\ref\ili{}{sec\ili{}:challenges}\ili{},\ili{} seem\ili{} not\ili{} possible\ili{} to\ili{} express\ili{}.\ili{} The\ili{} interpretation\ili{} of\ili{} the\ili{} encoding\ili{} is\ili{} led\ili{} partly\ili{} by\ili{} the\ili{} syntax\ili{} of\ili{} patterns\ili{} and\ili{} entries\ili{},\ili{} and\ili{} partly\ili{} by\ili{} textual\ili{} documentation\ili{} \ili{}\citep\ili{}{gregoire\ili{}:07c}\ili{},\ili{} where\ili{} it\ili{} is\ili{} sometimes\ili{} hard\ili{} to\ili{} distinguish\ili{} formal\ili{} properties\ili{} and\ili{} inference\ili{} rules\ili{} from\ili{} methodological\ili{} strategies\ili{} and\ili{} recommendations\ili{},\ili{} i\ili{}.e\ili{}.\ili{} the\ili{} \ili{}\isi\ili{}{transparency}\ili{} level\ili{} of\ili{} the\ili{} format\ili{} is\ili{} relatively\ili{} low\ili{}.\ili{} Last\ili{} but\ili{} not\ili{} least\ili{},\ili{} the\ili{} format\ili{} is\ili{} not\ili{} flexible\ili{},\ili{} i\ili{}.e\ili{}.\ili{} extending\ili{} the\ili{} set\ili{} of\ili{} describable\ili{} properties\ili{} can\ili{} only\ili{} be\ili{} done\ili{} ad\ili{} hoc\ili{} rather\ili{} than\ili{} within\ili{} an\ili{} established\ili{} framework\ili{} with\ili{} a\ili{} clear\ili{} denotational\ili{} semantic\ili{}.\ili{}
\ili{}
It\ili{} is\ili{} worth\ili{} noting\ili{} that\ili{} DuELME\ili{} benefits\ili{} from\ili{} a\ili{} standard\ili{} LMF\ili{} format\ili{} \ili{}\citep\ili{}{odijk\ili{}:13}\ili{},\ili{} which\ili{} makes\ili{} it\ili{} more\ili{} electronically\ili{} versatile\ili{},\ili{} even\ili{} if\ili{} it\ili{} does\ili{} not\ili{} seem\ili{} implementation\ili{} friendly\ili{} in\ili{} the\ili{} sense\ili{} that\ili{} tools\ili{} supporting\ili{} lexicographic\ili{} encoding\ili{} in\ili{} this\ili{} format\ili{} do\ili{} not\ili{} seem\ili{} publicly\ili{} available\ili{}.\ili{}
\ili{}
\ili{}\subsection\ili{}{Walenty}\ili{}
\ili{}\label\ili{}{sec\ili{}:walenty}\ili{}
\ili{}
A\ili{} quite\ili{} different\ili{} encoding\ili{} style\ili{} is\ili{} found\ili{} in\ili{} Walenty\ili{},\ili{} a\ili{} Polish\ili{} large\ili{}-scale\ili{} valence\ili{} dictionary\ili{} that\ili{} includes\ili{} an\ili{} elaborate\ili{} phraseological\ili{} component\ili{} \ili{}\citep\ili{}{prz\ili{}:etal\ili{}:14b\ili{},prz\ili{}:etal\ili{}:16}\ili{}.\ili{} It\ili{} contains\ili{} over\ili{} 100\ili{},000\ili{} syntactic\ili{} frames\ili{},\ili{} 14\ili{},000\ili{} of\ili{} which\ili{} are\ili{} verbal\ili{} frames\ili{} with\ili{} lexicalized\ili{} arguments\ili{},\ili{} i\ili{}.e\ili{}.\ili{} \ili{}\isi\ili{}{verbal\ili{} MWEs}\ili{}.\ili{} An\ili{} entry\ili{} in\ili{} Walenty\ili{} contains\ili{} a\ili{} headword\ili{} \ili{}(here\ili{} a\ili{} verb\ili{})\ili{},\ili{} followed\ili{} by\ili{} a\ili{} list\ili{} of\ili{} argument\ili{} descriptions\ili{} \ili{}(separated\ili{} by\ili{} \ili{}\texttt\ili{}{\ili{}+}\ili{})\ili{}.\ili{}
\ili{}
\ili{}\begin\ili{}{figure}\ili{}[th\ili{}]\ili{}
\ili{}\begin\ili{}{duelme}\ili{}
patrze\ili{}|\ili{}%\ili{}\\ili{}'c\ili{}%\ili{}|\ili{}:\ili{} np\ili{}(dat\ili{})\ili{}+advp\ili{}(misc\ili{})\ili{}+lex\ili{}(prepnp\ili{}(z\ili{},gen\ili{})\ili{},pl\ili{},\ili{}'oko\ili{}'\ili{},natr\ili{})\ili{}
\ili{}\end\ili{}{duelme}\ili{}
\ili{} \ili{} \ili{}\caption\ili{}{Description\ili{} of\ili{} \ili{}\ilelt\ili{}{dobrze\ili{} \ili{}[KOMUŚ\ili{}]\ili{} z\ili{} oczu\ili{} patrzy}\ili{}{well\ili{} someone\ili{}\textsc\ili{}{\ili{}.dat}\ili{} from\ili{} eyes\ili{} looks}\ili{}{someone\ili{} looks\ili{} like\ili{} a\ili{} good\ili{} person}\ili{} in\ili{} Walenty}\ili{}
\ili{} \ili{} \ili{}\label\ili{}{fig\ili{}:patrzy\ili{}:walenty}\ili{}
\ili{}\end\ili{}{figure}\ili{}
\ili{}
Figure\ili{}~\ili{}\ref\ili{}{fig\ili{}:patrzy\ili{}:walenty}\ili{} shows\ili{} a\ili{} \ili{}(slightly\ili{} simplified\ili{})\ili{} sample\ili{} MWE\ili{} entry\ili{} of\ili{} \ili{}(PL\ili{})\ili{} \ili{}\ilelt\ili{}{dobrze\ili{} \ili{}[KOMUŚ\ili{}]\ili{} z\ili{} oczu\ili{} patrzy}\ili{}{well\ili{} someone\ili{}\textsc\ili{}{\ili{}.dat}\ili{} from\ili{} eyes\ili{} looks}\ili{}{someone\ili{} looks\ili{} like\ili{} a\ili{} good\ili{} person}\ili{},\ili{} which\ili{} exhibits\ili{} several\ili{} interesting\ili{} constraints\ili{}.\ili{} Firstly\ili{},\ili{} the\ili{} syntactic\ili{} subject\ili{} is\ili{} prohibited\ili{} here\ili{},\ili{} which\ili{} is\ili{} expressed\ili{} simply\ili{} by\ili{} omitting\ili{} the\ili{} \ili{}\texttt\ili{}{subj}\ili{} argument\ili{} in\ili{} the\ili{} valence\ili{} frame\ili{}.\ili{} Secondly\ili{},\ili{} the\ili{} indirect\ili{} object\ili{} in\ili{} dative\ili{} is\ili{} compulsory\ili{} \ili{}(\ili{}\texttt\ili{}{np\ili{}(dat\ili{})}\ili{})\ili{}.\ili{} These\ili{} two\ili{} properties\ili{} are\ili{} unusual\ili{},\ili{} since\ili{} \ili{}\ilet\ili{}{patrzeć}\ili{}{look}\ili{},\ili{} as\ili{} a\ili{} stand\ili{}-alone\ili{} verb\ili{},\ili{} does\ili{} take\ili{} a\ili{} subject\ili{} and\ili{} it\ili{} only\ili{} admits\ili{} an\ili{} indirect\ili{} object\ili{} with\ili{} prepositional\ili{} complements\ili{} headed\ili{} by\ili{} \ili{}\ilet\ili{}{na}\ili{}{on}\ili{} and\ili{} \ili{}\ilet\ili{}{w}\ili{}{in}\ili{}.\ili{} Thirdly\ili{},\ili{} the\ili{} adverb\ili{} \ili{}\ilet\ili{}{dobrze}\ili{}{well}\ili{} can\ili{} have\ili{} some\ili{} \ili{}\isi\ili{}{variations}\ili{},\ili{} e\ili{}.g\ili{}.\ili{} \ili{}\ilelt\ili{}{źle\ili{} \ili{}[KOMUŚ\ili{}]\ili{} z\ili{} oczu\ili{} patrzy}\ili{}{evilly\ili{} someone\ili{}\textsc\ili{}{\ili{}.dat}\ili{} from\ili{} eyes\ili{} looks}\ili{}{someone\ili{} looks\ili{} like\ili{} an\ili{} evil\ili{} person}\ili{},\ili{} therefore\ili{} it\ili{} is\ili{} encoded\ili{} by\ili{} a\ili{} more\ili{} generic\ili{},\ili{} non\ili{} lexicalized\ili{},\ili{} \ili{}\texttt\ili{}{advp\ili{}(misc\ili{})}\ili{} requirement\ili{} of\ili{} a\ili{} \ili{}`\ili{}`true\ili{}'\ili{}'\ili{} adverbial\ili{} clause\ili{}.\ili{}\footnote\ili{}{A\ili{} \ili{}`\ili{}`true\ili{}'\ili{}'\ili{} adverbial\ili{} clause\ili{} cannot\ili{} be\ili{} realized\ili{} by\ili{} a\ili{} prepositional\ili{} nominal\ili{} group\ili{}.}\ili{} \ili{}
Finally\ili{},\ili{} within\ili{} the\ili{} lexicalized\ili{} prepositional\ili{} group\ili{} \ili{}(\ili{}\texttt\ili{}{lex\ili{}(prepnp\ili{}(\ili{}\ldots\ili{})}\ili{})\ili{},\ili{} which\ili{} does\ili{} not\ili{} admit\ili{} modification\ili{} \ili{}(\ili{}\texttt\ili{}{natr}\ili{})\ili{},\ili{} the\ili{} preposition\ili{} \ili{}\ilet\ili{}{z}\ili{}{from}\ili{} governing\ili{} the\ili{} genitive\ili{} case\ili{} \ili{}(\ili{}\texttt\ili{}{\ili{}(z\ili{},gen\ili{})}\ili{})\ili{} requires\ili{} its\ili{} nominal\ili{} complement\ili{} to\ili{} be\ili{} a\ili{} plural\ili{} form\ili{} of\ili{} the\ili{} lemma\ili{} \ili{}\ilet\ili{}{oko}\ili{}{eye}\ili{} \ili{}(\ili{}\texttt\ili{}{pl\ili{},\ili{}'oko\ili{}'}\ili{})\ili{}.\ili{}
\ili{}
This\ili{} approach\ili{} is\ili{} valence\ili{}-based\ili{},\ili{} i\ili{}.e\ili{}.\ili{} MWEs\ili{} are\ili{} seen\ili{} as\ili{} particular\ili{} syntactic\ili{} frames\ili{} of\ili{} their\ili{} head\ili{} verbs\ili{},\ili{} in\ili{} which\ili{} some\ili{} arguments\ili{} happen\ili{} to\ili{} be\ili{} \ili{}(at\ili{} least\ili{} partly\ili{})\ili{} lexicalized\ili{}.\ili{} Regularity\ili{} is\ili{} implicit\ili{}:\ili{} \ili{}`\ili{}`generally\ili{} regular\ili{}'\ili{}'\ili{} properties\ili{} are\ili{} supposed\ili{} to\ili{} be\ili{} covered\ili{} by\ili{} grammar\ili{} rules\ili{} and\ili{} only\ili{} MWE\ili{}-specific\ili{} properties\ili{} are\ili{} expressed\ili{} in\ili{} lexicon\ili{} entries\ili{}.\ili{} E\ili{}.g\ili{}.\ili{},\ili{} while\ili{} the\ili{} plural\ili{} number\ili{} of\ili{} \ili{}\ilet\ili{}{oko}\ili{}{eye}\ili{} is\ili{} specified\ili{},\ili{} its\ili{} case\ili{} is\ili{} not\ili{},\ili{} since\ili{} it\ili{} is\ili{} supposed\ili{} to\ili{} regularly\ili{} agree\ili{} with\ili{} its\ili{} governing\ili{} preposition\ili{} \ili{}(which\ili{} requires\ili{} genitive\ili{} case\ili{})\ili{}.\ili{} This\ili{} principle\ili{} is\ili{} similar\ili{} to\ili{} the\ili{} one\ili{} admitted\ili{} in\ili{} DuELME\ili{} \ili{}(cf\ili{}.\ili{} Section\ili{}~\ili{}\ref\ili{}{sec\ili{}:duelme}\ili{})\ili{},\ili{} here\ili{} however\ili{} no\ili{} equivalence\ili{} classes\ili{} are\ili{} used\ili{},\ili{} so\ili{} the\ili{} \ili{}\isi\ili{}{syntactic\ili{} structure}\ili{},\ili{} understood\ili{} as\ili{} the\ili{} list\ili{} of\ili{} arguments\ili{} \ili{}(possibly\ili{} structured\ili{} themselves\ili{})\ili{} required\ili{} by\ili{} the\ili{} head\ili{} verb\ili{},\ili{} is\ili{} encoded\ili{} in\ili{} each\ili{} entry\ili{} \ili{}(similarly\ili{} to\ili{} the\ili{} IDON\ili{} lexicon\ili{} discussed\ili{} in\ili{} chapter\ili{} MARKANTONATOU\ili{} of\ili{} this\ili{} volume\ili{})\ili{},\ili{} which\ili{} leads\ili{} to\ili{} redundancy\ili{} in\ili{} the\ili{} lexicon\ili{}.\ili{} For\ili{} instance\ili{} entries\ili{} for\ili{} all\ili{} MWEs\ili{} taking\ili{} a\ili{} non\ili{}-lexicalized\ili{} subject\ili{},\ili{} direct\ili{} object\ili{} and\ili{} indirect\ili{} object\ili{},\ili{} and\ili{} a\ili{} partly\ili{} lexicalized\ili{} prepositional\ili{} complement\ili{},\ili{} contain\ili{} the\ili{} same\ili{} sequence\ili{}:\ili{} \ili{}\texttt\ili{}{subj\ili{}\\ili{}{np\ili{}(str\ili{})\ili{}\}\ili{} \ili{}+\ili{} obj\ili{}\\ili{}{np\ili{}(str\ili{})\ili{}\}\ili{} \ili{}+\ili{} \ili{}\\ili{}{np\ili{}(inst\ili{})\ili{}\}\ili{} \ili{}+\ili{} \ili{}\\ili{}{lex\ili{}(prepnp\ili{}(\ili{}\ldots\ili{})\ili{}\}}\ili{}\footnote\ili{}{The\ili{} \ili{}\texttt\ili{}{str}\ili{} feature\ili{} stands\ili{} for\ili{} a\ili{} structural\ili{} case\ili{}.\ili{} For\ili{} the\ili{} subject\ili{} it\ili{} is\ili{} usually\ili{} nominative\ili{},\ili{} but\ili{} it\ili{} turns\ili{} to\ili{} genitive\ili{} when\ili{} the\ili{} expression\ili{} is\ili{} nominalized\ili{}.\ili{} For\ili{} the\ili{} direct\ili{} object\ili{},\ili{} it\ili{} is\ili{} accusative\ili{} but\ili{} it\ili{} turns\ili{} to\ili{} genitive\ili{} when\ili{} it\ili{} occurs\ili{} under\ili{} the\ili{} scope\ili{} of\ili{} negation\ili{}.}\ili{}.\ili{} Some\ili{} redundancy\ili{} can\ili{} however\ili{} be\ili{} avoided\ili{} due\ili{} to\ili{} macros\ili{} which\ili{} encode\ili{} some\ili{} repetitive\ili{} substructures\ili{}.\ili{} For\ili{} instance\ili{},\ili{} the\ili{} \ili{}\texttt\ili{}{possp}\ili{} macro\ili{} encodes\ili{} all\ili{} possible\ili{} realization\ili{} of\ili{} a\ili{} possessive\ili{} \ili{}\isi\ili{}{phrase}\ili{},\ili{} including\ili{} nominal\ili{} phrases\ili{} in\ili{} genitive\ili{} and\ili{} possessive\ili{} determiners\ili{} like\ili{} \ili{}\ilet\ili{}{mój\ili{},\ili{} czyjś\ili{},\ili{} własny\ili{},\ili{} \ili{}\ldots}\ili{}{my\ili{},\ili{} one\ili{}'s\ili{},\ili{} one\ili{}'s\ili{} own\ili{},\ili{} \ili{}\ldots}\ili{}.\ili{}
\ili{}
Some\ili{} additional\ili{} syntactic\ili{} properties\ili{} can\ili{} be\ili{} expressed\ili{} on\ili{} the\ili{} level\ili{} of\ili{} the\ili{} whole\ili{} MWE\ili{},\ili{} e\ili{}.g\ili{}.\ili{} the\ili{} fact\ili{} that\ili{} the\ili{} head\ili{} verb\ili{} is\ili{} perfective\ili{} or\ili{} imperfective\ili{},\ili{} that\ili{} the\ili{} MWE\ili{} must\ili{} always\ili{} contain\ili{} negation\ili{},\ili{} or\ili{} that\ili{} it\ili{} can\ili{} or\ili{} can\ili{} not\ili{} be\ili{} passivized\ili{}.\ili{} Some\ili{} other\ili{} types\ili{} of\ili{} constraints\ili{},\ili{} e\ili{}.g\ili{}.\ili{} restrictive\ili{} agreement\ili{},\ili{} paradigm\ili{},\ili{} determination\ili{},\ili{} or\ili{} linearization\ili{} \ili{}(cf\ili{}.\ili{} Points\ili{} \ili{}\ref\ili{}{restr\ili{}-agr}\ili{}-\ili{}-\ili{}\ref\ili{}{restr\ili{}-par}\ili{},\ili{} \ili{}\ref\ili{}{restr\ili{}-det\ili{}-mod}\ili{}-\ili{}-\ili{}\ref\ili{}{restr\ili{}-det\ili{}-mod\ili{}-comb}\ili{} and\ili{} \ili{}\ref\ili{}{restr\ili{}-linear}\ili{} in\ili{} Section\ili{}~\ili{}\ref\ili{}{sec\ili{}:challenges}\ili{})\ili{},\ili{} exceed\ili{} Walenty\ili{}'s\ili{} expressive\ili{} power\ili{}.\ili{} Therefore\ili{},\ili{} one\ili{} cannot\ili{} express\ili{} the\ili{} fact\ili{} that\ili{},\ili{} in\ili{} \ili{}(PL\ili{})\ili{} \ili{}\ilelt\ili{}{dobrze\ili{} \ili{}[KOMUŚ\ili{}]\ili{} z\ili{} oczu\ili{} patrzy}\ili{}{well\ili{} someone\ili{}\textsc\ili{}{\ili{}.dat}\ili{} from\ili{} eyes\ili{} looks}\ili{}{someone\ili{} looks\ili{} like\ili{} a\ili{} good\ili{} person}\ili{},\ili{} the\ili{} head\ili{} verb\ili{} \ili{}\ilet\ili{}{patrzeć}\ili{}{look}\ili{} is\ili{} always\ili{} in\ili{} the\ili{} 3rd\ili{} person\ili{} singular\ili{} \ili{}(any\ili{} tense\ili{} or\ili{} mood\ili{})\ili{},\ili{} although\ili{} it\ili{} has\ili{} a\ili{} complete\ili{} inflection\ili{} paradigm\ili{} as\ili{} a\ili{} stand\ili{}-alone\ili{} verb\ili{}.\ili{}\footnote\ili{}{Impersonal\ili{} \ili{}(i\ili{}.e\ili{}.\ili{} allowing\ili{} no\ili{} subject\ili{})\ili{} finite\ili{} verbs\ili{} typically\ili{} occur\ili{} in\ili{} the\ili{} 3rd\ili{} person\ili{} singular\ili{} in\ili{} Polish\ili{},\ili{} so\ili{} the\ili{} expression\ili{} of\ili{} this\ili{} fact\ili{} is\ili{} probably\ili{} left\ili{} to\ili{} the\ili{} grammar\ili{}.\ili{} If\ili{} so\ili{},\ili{} then\ili{} this\ili{} fact\ili{} seems\ili{} implicit\ili{}.}\ili{} \ili{}
Also\ili{},\ili{} there\ili{} is\ili{} no\ili{} means\ili{} to\ili{} specify\ili{} that\ili{} the\ili{} adverb\ili{} \ili{}\ilet\ili{}{dobrze}\ili{}{well}\ili{} should\ili{} usually\ili{} precede\ili{} the\ili{} prepositional\ili{} complement\ili{} and\ili{} the\ili{} verb\ili{}.\ili{}\footnote\ili{}{A\ili{} different\ili{} word\ili{} order\ili{} would\ili{} be\ili{} considered\ili{} as\ili{} marked\ili{}.}\ili{} \ili{}
Note\ili{},\ili{} however\ili{},\ili{} that\ili{} a\ili{} conservative\ili{} extension\ili{} of\ili{} the\ili{} formalism\ili{} to\ili{} include\ili{} some\ili{} of\ili{} these\ili{} constraints\ili{} was\ili{} proposed\ili{} by\ili{} \ili{}\cite\ili{}{prz\ili{}:etal\ili{}:16}\ili{}.\ili{}
\ili{}
The\ili{} interpretation\ili{} of\ili{} the\ili{} encoding\ili{} is\ili{} led\ili{} partly\ili{} by\ili{} the\ili{} syntax\ili{} of\ili{} entries\ili{} and\ili{} explicit\ili{} macro\ili{} extensions\ili{},\ili{} and\ili{} partly\ili{} by\ili{} the\ili{} accompanying\ili{} textual\ili{} documentation\ili{}.\ili{} Some\ili{} inferences\ili{} remain\ili{} unclear\ili{},\ili{} e\ili{}.g\ili{}.\ili{} some\ili{} macros\ili{} contain\ili{} non\ili{}-documented\ili{} shortcuts\ili{},\ili{} and\ili{} some\ili{} codes\ili{} have\ili{} no\ili{} clear\ili{} denotational\ili{} semantics\ili{}.\ili{} The\ili{} format\ili{} is\ili{} rather\ili{} inflexible\ili{},\ili{} that\ili{} is\ili{},\ili{} extending\ili{} the\ili{} set\ili{} of\ili{} describable\ili{} properties\ili{} can\ili{} only\ili{} be\ili{} done\ili{} ad\ili{} hoc\ili{}.\ili{} Walenty\ili{} does\ili{} benefit\ili{} from\ili{} a\ili{} standard\ili{} interchange\ili{} XML\ili{} metaformat\ili{},\ili{} namely\ili{} TEI\ili{}\footnote\ili{}{Text\ili{} Encoding\ili{} Initiative\ili{}:\ili{} \ili{}\url\ili{}{http\ili{}:\ili{}/\ili{}/www\ili{}.tei\ili{}-c\ili{}.org\ili{}/Guidelines\ili{}/P5\ili{}/}}\ili{},\ili{} but\ili{} does\ili{} not\ili{} provide\ili{} its\ili{} precise\ili{} instantiation\ili{} in\ili{} terms\ili{} of\ili{} a\ili{} DTD\ili{},\ili{} RelaxNG\ili{} or\ili{} XML\ili{} schema\ili{}.\ili{} Finally\ili{},\ili{} it\ili{} has\ili{} a\ili{} rather\ili{} elaborate\ili{} lexicographical\ili{} support\ili{},\ili{} with\ili{} several\ili{} user\ili{} roles\ili{},\ili{} where\ili{} the\ili{} existing\ili{} entries\ili{} can\ili{} be\ili{} browsed\ili{} together\ili{} with\ili{} their\ili{} corpus\ili{} examples\ili{},\ili{} and\ili{} new\ili{} entries\ili{} can\ili{} be\ili{} added\ili{},\ili{} corrected\ili{},\ili{} compared\ili{},\ili{} assigned\ili{} to\ili{} users\ili{},\ili{} etc\ili{}.\ili{} \ili{}\citep\ili{}{nit\ili{}:etal\ili{}:16}\ili{}.\ili{} Recent\ili{} developments\ili{} couple\ili{} Walenty\ili{} with\ili{} a\ili{} Polish\ili{} wordnet\ili{} so\ili{} as\ili{} to\ili{} enrich\ili{} valency\ili{} data\ili{} with\ili{} semantic\ili{} frames\ili{}.\ili{} \ili{}
\ili{}
\ili{}%\ili{}\input\ili{}{05\ili{}-patr}\ili{}
\ili{}\section\ili{}{Fully\ili{} flexible\ili{} encoding\ili{} formats}\ili{}
\ili{}\label\ili{}{sec\ili{}:fullyflexible}\ili{}
\ili{}
What\ili{} we\ili{} mean\ili{} by\ili{} fully\ili{} flexible\ili{} is\ili{} that\ili{} properties\ili{},\ili{} property\ili{} names\ili{} and\ili{} inference\ili{} rules\ili{} \ili{}(or\ili{} macros\ili{})\ili{} can\ili{}
be\ili{} freely\ili{} chosen\ili{} \ili{}-\ili{}-\ili{} one\ili{} consequence\ili{} being\ili{} that\ili{} there\ili{} are\ili{} usually\ili{} many\ili{} ways\ili{} to\ili{} implement\ili{} an\ili{} object\ili{} within\ili{} such\ili{} an\ili{} encoding\ili{} format\ili{}.\ili{} In\ili{} this\ili{} section\ili{},\ili{} we\ili{} will\ili{} show\ili{} two\ili{} exemplars\ili{} of\ili{} fully\ili{} flexible\ili{} encoding\ili{} formats\ili{}:\ili{} the\ili{} venerable\ili{} PATR\ili{}-II\ili{} and\ili{} \ili{} the\ili{} more\ili{} recent\ili{} XMG\ili{}.\ili{} The\ili{} motivation\ili{} for\ili{} choosing\ili{} these\ili{} two\ili{} encoding\ili{} formats\ili{} is\ili{} twofold\ili{}.\ili{} On\ili{} the\ili{} one\ili{} hand\ili{},\ili{} both\ili{} engage\ili{} different\ili{} notational\ili{} means\ili{} with\ili{} a\ili{} different\ili{} denotational\ili{} semantics\ili{};\ili{} on\ili{} the\ili{} other\ili{} hand\ili{},\ili{} two\ili{} extremes\ili{} of\ili{} modeling\ili{} argument\ili{} structure\ili{} can\ili{} be\ili{} covered\ili{} that\ili{} were\ili{} the\ili{} focus\ili{} of\ili{} some\ili{} debate\ili{} recently\ili{},\ili{} namely\ili{} the\ili{} lexical\ili{} versus\ili{} the\ili{} phrasal\ili{} approach\ili{} \ili{}\citep\ili{}{Mueller\ili{}:Wechsler\ili{}:14}\ili{}.\ili{} In\ili{} doing\ili{} so\ili{},\ili{} we\ili{} will\ili{} again\ili{},\ili{} as\ili{} in\ili{} the\ili{} preceding\ili{} section\ili{},\ili{} restrict\ili{} ourselves\ili{} to\ili{} the\ili{} tentative\ili{} encoding\ili{} of\ili{} \ili{}(NL\ili{})\ili{} \ili{}\ilet\ili{}{zijn\ili{} kansen\ili{} waarnemen}\ili{}{to\ili{} seize\ili{} the\ili{} opportunity}\ili{} and\ili{} \ili{}(PL\ili{})\ili{} \ili{}\ilet\ili{}{dobrze\ili{} \ili{}[KOMUŚ\ili{}]\ili{} z\ili{} oczu\ili{} patrzy}\ili{}{someone\ili{} looks\ili{} like\ili{} a\ili{} good\ili{} person}\ili{}.\ili{} The\ili{} presentation\ili{} will\ili{},\ili{} we\ili{} think\ili{},\ili{} strengthen\ili{} the\ili{} view\ili{} that\ili{} MWEs\ili{} should\ili{} be\ili{} better\ili{} encoded\ili{} with\ili{} fully\ili{} flexible\ili{} encoding\ili{} formats\ili{} in\ili{} order\ili{} to\ili{} obtain\ili{} and\ili{} maintain\ili{} the\ili{} virtues\ili{} mentioned\ili{} in\ili{} Section\ili{}~\ili{}\ref\ili{}{sec\ili{}:general\ili{}-virtues}\ili{}.\ili{} \ili{} \ili{} \ili{}
\ili{}
\ili{}\subsection\ili{}{PATR\ili{}-II}\ili{}
\ili{}\label\ili{}{sec\ili{}:patr\ili{}-datr}\ili{}
\ili{}
A\ili{} true\ili{} classic\ili{},\ili{} PATR\ili{}-II\ili{} \ili{}\citep\ili{}{Shieber\ili{}:84\ili{},Shieber\ili{}:86}\ili{} dates\ili{} back\ili{} to\ili{} the\ili{} early\ili{} 80s\ili{} and\ili{} has\ili{} greatly\ili{} influenced\ili{} the\ili{} development\ili{} of\ili{} later\ili{} encoding\ili{} formats\ili{},\ili{} for\ili{} example\ili{} LKB\ili{} \ili{}\citep\ili{}[6\ili{}]\ili{}{lkb\ili{}-book}\ili{},\ili{} thanks\ili{} to\ili{} its\ili{} notational\ili{} \ili{}\isi\ili{}{transparency}\ili{} and\ili{} conceptual\ili{} rigor\ili{}.\ili{}\footnote\ili{}{A\ili{} superficially\ili{} similar\ili{} encoding\ili{} framework\ili{} is\ili{} DATR\ili{} \ili{}\citep\ili{}{Evans\ili{}:Gazdar\ili{}:96}\ili{}.\ili{} See\ili{} \ili{}\cite\ili{}{Kilbury\ili{}:etal\ili{}:91}\ili{} for\ili{} a\ili{} comparison\ili{} with\ili{} PATR\ili{}-II\ili{} that\ili{} also\ili{} highlights\ili{} the\ili{} considerable\ili{} differences\ili{} between\ili{} the\ili{} two\ili{}.}\ili{} The\ili{} basic\ili{} idea\ili{} is\ili{} simple\ili{}:\ili{} to\ili{} enhance\ili{} CFG\ili{} rules\ili{} with\ili{} descriptions\ili{} of\ili{} untyped\ili{} feature\ili{} structures\ili{},\ili{} which\ili{} are\ili{} then\ili{} unified\ili{} during\ili{} \ili{}\isi\ili{}{rule}\ili{} applications\ili{}.\ili{} Hence\ili{},\ili{} the\ili{} models\ili{} of\ili{} PATR\ili{}-II\ili{} descriptions\ili{} are\ili{} just\ili{} directed\ili{} acyclic\ili{} graphs\ili{} with\ili{} labeled\ili{} nodes\ili{} and\ili{} edges\ili{}.\ili{} \ili{} But\ili{} the\ili{} means\ili{} of\ili{} description\ili{} are\ili{} more\ili{} elaborate\ili{} and\ili{} do\ili{} also\ili{} include\ili{} templates\ili{},\ili{} lexical\ili{} rules\ili{} and\ili{} sometimes\ili{} \ili{}-\ili{}-\ili{} depending\ili{} on\ili{} the\ili{} PATR\ili{}-II\ili{} implementation\ili{} \ili{}-\ili{}-\ili{} default\ili{} inheritance\ili{}.\ili{}\footnote\ili{}{Default\ili{} inheritance\ili{} is\ili{} available\ili{},\ili{} for\ili{} example\ili{},\ili{} in\ili{} PC\ili{}-PATR\ili{} \ili{}\citep\ili{}{PC\ili{}-PATR\ili{}:manual\ili{}:97}\ili{},\ili{} which\ili{} is\ili{} a\ili{} parser\ili{} for\ili{} PATR\ili{}-II\ili{} grammars\ili{} developed\ili{} at\ili{} the\ili{} Summer\ili{} Institute\ili{} of\ili{} Linguistics\ili{} \ili{}(SIL\ili{})\ili{}.}\ili{} The\ili{} encoding\ili{} examples\ili{} that\ili{} we\ili{} will\ili{} give\ili{} do\ili{} not\ili{},\ili{} however\ili{},\ili{} make\ili{} use\ili{} of\ili{} the\ili{} full\ili{} non\ili{}-monotonic\ili{} power\ili{} of\ili{} PATR\ili{}-II\ili{},\ili{} as\ili{} lexical\ili{} rules\ili{} and\ili{} default\ili{} inheritance\ili{} will\ili{} be\ili{} left\ili{} out\ili{}.\ili{} On\ili{} the\ili{} other\ili{} hand\ili{},\ili{} we\ili{} will\ili{} follow\ili{} the\ili{} head\ili{}-driven\ili{} perspective\ili{} of\ili{} PATR\ili{}-II\ili{} in\ili{} that\ili{} MWEs\ili{} will\ili{} be\ili{} encoded\ili{} in\ili{} their\ili{} head\ili{} only\ili{},\ili{} that\ili{} is\ili{},\ili{} MWEs\ili{} headed\ili{} by\ili{} a\ili{} verb\ili{} will\ili{} essentially\ili{} emerge\ili{} from\ili{} the\ili{} encoding\ili{} of\ili{} their\ili{} verbal\ili{} component\ili{}.\ili{}\footnote\ili{}{The\ili{} only\ili{} previous\ili{} work\ili{} on\ili{} encoding\ili{} MWEs\ili{} with\ili{} PATR\ili{}-II\ili{} that\ili{} we\ili{} are\ili{} aware\ili{} of\ili{} is\ili{} found\ili{} in\ili{} \ili{}\cite\ili{}{Habert\ili{}:Jaquemin\ili{}:95}\ili{}.\ili{} There\ili{} the\ili{} focus\ili{} is\ili{} on\ili{} \ili{}\ili\ili{}{French}\ili{} nominal\ili{} compunds\ili{} like\ili{} \ili{}\textit\ili{}{verre\ili{} à\ili{} vin}\ili{}
\ili{} \ili{}(\ili{}`wineglass\ili{}'\ili{})\ili{}.}\ili{} \ili{}
\ili{}
All\ili{} this\ili{} is\ili{} exemplified\ili{} for\ili{} \ili{}(NL\ili{})\ili{} \ili{}\ile\ili{}{zijn\ili{} kansen\ili{} waarnemen}\ili{} in\ili{} Figure\ili{}~\ili{}\ref\ili{}{fig\ili{}:patr\ili{}:dutch}\ili{}.\ili{}
\ili{}\begin\ili{}{figure}\ili{}[t\ili{}]\ili{}
\ili{} \ili{} \ili{}\begin\ili{}{patr\ili{}-listing}\ili{}
Define\ili{} Verb\ili{} as\ili{}
\ili{} \ili{} \ili{} \ili{} \ili{}[cat\ili{}:\ili{} v\ili{}]\ili{}
\ili{}
Define\ili{} Subject\ili{} as\ili{}
\ili{} \ili{} \ili{} \ili{} \ili{}[subject\ili{}:\ili{} \ili{}[cat\ili{}:\ili{} np\ili{}]\ili{}]\ili{}
\ili{}
Define\ili{} Object\ili{} as\ili{}
\ili{} \ili{} \ili{} \ili{} \ili{}[object\ili{}:\ili{} \ili{}[cat\ili{}:\ili{} np\ili{}]\ili{}]\ili{}
\ili{} \ili{} \ili{} \ili{} \ili{}
Define\ili{} Intransitive\ili{} as\ili{}
\ili{} \ili{} \ili{} \ili{} Verb\ili{}
\ili{} \ili{} \ili{} \ili{} Subject\ili{}
\ili{} \ili{} \ili{} \ili{} \ili{}
Define\ili{} Transitive\ili{} as\ili{}
\ili{} \ili{} \ili{} \ili{} Intransitive\ili{}
\ili{} \ili{} \ili{} \ili{} Object\ili{}
\ili{}
Define\ili{} SubjectPossObjectAgreement\ili{} as\ili{}
\ili{} \ili{} \ili{} \ili{} \ili{}[subject\ili{}:\ili{} \ili{}[agr\ili{}:\ili{} \ili{}$1\ili{}]\ili{}
\ili{} \ili{} \ili{} \ili{} object\ili{}:\ili{} \ili{}[poss\ili{}:\ili{} \ili{}[agr\ili{}:\ili{} \ili{}$1\ili{}]\ili{}]\ili{}]\ili{}
\ili{}
Define\ili{} ZijnKansenWaarnemen\ili{} as\ili{}
\ili{} \ili{} \ili{} \ili{} Transitive\ili{}
\ili{} \ili{} \ili{} \ili{} SubjectPossObjectAgreement\ili{}
\ili{} \ili{} \ili{} \ili{} \ili{}[lex\ili{}:\ili{} waarnemen\ili{}
\ili{} \ili{} \ili{} \ili{} \ili{} object\ili{}:\ili{} \ili{}[lex\ili{}:\ili{} kans\ili{}
\ili{} \ili{} \ili{} \ili{} \ili{} \ili{} \ili{} \ili{} \ili{} \ili{} \ili{} \ili{} \ili{} \ili{} agr\ili{}:\ili{} \ili{}[num\ili{}:\ili{} pl\ili{}]\ili{}
\ili{} \ili{} \ili{} \ili{} \ili{} \ili{} \ili{} \ili{} \ili{} \ili{} \ili{} \ili{} \ili{} \ili{} modifiable\ili{}:\ili{} \ili{}-\ili{}]\ili{}
\ili{} \ili{} \ili{} \ili{} sem\ili{}:\ili{} \ili{}[paraphrase\ili{}:\ili{} seize_the_opportunity\ili{}]\ili{}]\ili{}
\ili{}
Word\ili{} waarnemen\ili{}:\ili{}
\ili{} \ili{} \ili{} \ili{} Verb\ili{}
\ili{} \ili{} \ili{} \ili{} \ili{}{\ili{}[WaarnemenLiteral\ili{}]\ili{} \ili{}[ZijnKansenWaarnemen\ili{}]}\ili{}
\ili{} \ili{} \ili{} \ili{} \ili{}[lex\ili{}:\ili{} waarnemen\ili{}]\ili{}
\ili{} \ili{} \ili{}\end\ili{}{patr\ili{}-listing}\ili{}
\ili{} \ili{} \ili{}\caption\ili{}{PATR\ili{}-II\ili{} description\ili{} \ili{}(with\ili{} PC\ili{}-PATR\ili{} notation\ili{})\ili{} of\ili{} \ili{}(NL\ili{})\ili{} \ili{}\ilet\ili{}{zijn\ili{} kansen\ili{} waarnemen}\ili{}{to\ili{} seize\ili{} the\ili{} opportunity}}\ili{}
\ili{} \ili{} \ili{}\label\ili{}{fig\ili{}:patr\ili{}:dutch}\ili{}
\ili{}\end\ili{}{figure}\ili{}
Templates\ili{} are\ili{} headed\ili{} by\ili{} \ili{}\texttt\ili{}{Define\ili{}-as}\ili{} constructs\ili{}.\ili{} The\ili{} body\ili{} of\ili{} a\ili{} \ili{}\isi\ili{}{template}\ili{} may\ili{} either\ili{} contain\ili{} \ili{}\isi\ili{}{template}\ili{} names\ili{} \ili{}(or\ili{} disjunctions\ili{} thereof\ili{} as\ili{} in\ili{} Line\ili{} 33\ili{})\ili{},\ili{} from\ili{} which\ili{} the\ili{} \ili{}\isi\ili{}{template}\ili{} inherits\ili{},\ili{} or\ili{} feature\ili{} structure\ili{} descriptions\ili{}.\ili{} Word\ili{} entries\ili{} such\ili{} as\ili{} the\ili{} one\ili{} of\ili{} \ili{}\textit\ili{}{waarnemen}\ili{} at\ili{} the\ili{} bottom\ili{} are\ili{} similiar\ili{} to\ili{} templates\ili{} but\ili{} define\ili{} the\ili{} terminals\ili{} of\ili{} CFG\ili{} rules\ili{}.\ili{} Keep\ili{} in\ili{} mind\ili{} that\ili{} \ili{}\textit\ili{}{waarnemen}\ili{} acts\ili{} as\ili{} the\ili{} verbal\ili{} head\ili{} of\ili{} the\ili{} MWE\ili{},\ili{} hence\ili{} the\ili{} templates\ili{} in\ili{} this\ili{} example\ili{} all\ili{} describe\ili{} the\ili{} feature\ili{} structure\ili{} of\ili{} \ili{}\textit\ili{}{waarnemen}\ili{} only\ili{}.\ili{} Also\ili{} note\ili{} that\ili{} the\ili{} features\ili{} are\ili{} chosen\ili{} to\ili{} keep\ili{} the\ili{} example\ili{} as\ili{} simple\ili{} as\ili{} possible\ili{} \ili{}-\ili{}-\ili{} typically\ili{} one\ili{} would\ili{} find\ili{} \ili{}\isi\ili{}{subcategorization}\ili{} lists\ili{} in\ili{} PATR\ili{}-II\ili{} implementations\ili{}.\ili{}
\ili{}
In\ili{} Figure\ili{}~\ili{}\ref\ili{}{fig\ili{}:patr\ili{}:dutch}\ili{},\ili{} the\ili{} first\ili{} five\ili{} templates\ili{} \ili{}(\ili{}\texttt\ili{}{Verb}\ili{},\ili{} \ili{}\texttt\ili{}{Subject}\ili{},\ili{} \ili{}\texttt\ili{}{Object}\ili{},\ili{} \ili{}\texttt\ili{}{Intransitive}\ili{},\ili{} and\ili{} \ili{}\texttt\ili{}{Transitive}\ili{})\ili{} just\ili{} act\ili{} as\ili{} an\ili{} example\ili{} of\ili{} how\ili{} general\ili{} properties\ili{},\ili{} like\ili{} being\ili{} a\ili{} transitive\ili{} verb\ili{},\ili{} \ili{}\textit\ili{}{could}\ili{} be\ili{} factorized\ili{} into\ili{} even\ili{} more\ili{} general\ili{} properties\ili{}.\ili{} Finally\ili{},\ili{} the\ili{} sixth\ili{} \ili{}\isi\ili{}{template}\ili{},\ili{} \ili{}\texttt\ili{}{SubjectPossObjectAgreement}\ili{},\ili{} is\ili{} more\ili{} immediately\ili{} relevant\ili{} to\ili{} the\ili{} MWE\ili{} \ili{}(NL\ili{})\ili{} \ili{}\ile\ili{}{zijn\ili{} kansen\ili{} waarnemen}\ili{} since\ili{} it\ili{} captures\ili{} the\ili{} agreement\ili{} of\ili{} the\ili{} subject\ili{} with\ili{} the\ili{} possessive\ili{} pronoun\ili{} at\ili{} the\ili{} object\ili{}.\ili{} This\ili{} is\ili{} achieved\ili{} by\ili{} using\ili{} the\ili{} shared\ili{} variable\ili{} \ili{}\texttt\ili{}{\ili{}\\ili{}$1}\ili{}.\ili{} Crucially\ili{},\ili{} this\ili{} \ili{}\isi\ili{}{template}\ili{} could\ili{} be\ili{} reused\ili{} in\ili{} many\ili{} other\ili{} MWEs\ili{} such\ili{} as\ili{} \ili{}(EN\ili{})\ili{} \ili{}\ile\ili{}{to\ili{} do\ili{} one\ili{}'s\ili{} best}\ili{}.\ili{} Again\ili{},\ili{} this\ili{} is\ili{} not\ili{} to\ili{} say\ili{} that\ili{} this\ili{} sort\ili{} of\ili{} agreement\ili{} should\ili{} be\ili{} treated\ili{} in\ili{} this\ili{} way\ili{},\ili{} but\ili{} that\ili{} it\ili{} is\ili{} \ili{}\textit\ili{}{possible}\ili{} to\ili{} do\ili{} so\ili{},\ili{} choosing\ili{} here\ili{} just\ili{} one\ili{} of\ili{} the\ili{} many\ili{} available\ili{} options\ili{}.\ili{} In\ili{} other\ili{} words\ili{},\ili{} the\ili{} \ili{}\isi\ili{}{template}\ili{} \ili{}\texttt\ili{}{SubjectPossObjectAgreement}\ili{} is\ili{} an\ili{} instance\ili{} of\ili{} one\ili{} such\ili{} MWE\ili{}-specific\ili{} regularity\ili{} that\ili{} PATR\ili{}-II\ili{} is\ili{} flexible\ili{} enough\ili{} to\ili{} encode\ili{} directly\ili{}.\ili{} Finally\ili{},\ili{} in\ili{} Figure\ili{}~\ili{}\ref\ili{}{fig\ili{}:patr\ili{}:dutch}\ili{},\ili{} the\ili{} \ili{}\isi\ili{}{template}\ili{} \ili{}\texttt\ili{}{ZijnKansenWaarnemen}\ili{} inherits\ili{} from\ili{} the\ili{} templates\ili{} \ili{}\texttt\ili{}{Transitive}\ili{} and\ili{} \ili{}\texttt\ili{}{SubjectPossObjectAgreement}\ili{},\ili{} and\ili{} it\ili{} adds\ili{} further\ili{} information\ili{} on\ili{} the\ili{} shape\ili{} and\ili{} modifiability\ili{} of\ili{} the\ili{} object\ili{} and\ili{} on\ili{} the\ili{} idiomatic\ili{} semantics\ili{} of\ili{} the\ili{} whole\ili{} MWE\ili{}.\ili{}
\ili{}
Comparing\ili{} the\ili{} PATR\ili{}-II\ili{} encoding\ili{} with\ili{} the\ili{} DuELME\ili{} encoding\ili{} from\ili{} Figure\ili{}~\ili{}\ref\ili{}{fig\ili{}:duelme}\ili{},\ili{} it\ili{} gets\ili{} evident\ili{} that\ili{} PATR\ili{}-II\ili{} is\ili{} more\ili{} flexible\ili{} at\ili{} defining\ili{} properties\ili{} or\ili{} factorizing\ili{} what\ili{} are\ili{} called\ili{} \ili{}`\ili{}`patterns\ili{}'\ili{}'\ili{} in\ili{} DuELME\ili{}.\ili{} The\ili{} reason\ili{} for\ili{} this\ili{} divergence\ili{} of\ili{} flexibility\ili{} also\ili{} lies\ili{} in\ili{} the\ili{} fact\ili{} that\ili{} PATR\ili{}-II\ili{} descriptions\ili{} come\ili{} with\ili{} a\ili{} clear\ili{} denotational\ili{} semantics\ili{},\ili{} which\ili{} does\ili{} not\ili{} seem\ili{} to\ili{} be\ili{} fixed\ili{} for\ili{} DuELME\ili{} encodings\ili{}.\ili{} In\ili{} fact\ili{},\ili{} one\ili{} could\ili{} see\ili{} this\ili{} as\ili{} an\ili{} advantage\ili{} of\ili{} DuELME\ili{},\ili{} taking\ili{} it\ili{} as\ili{} a\ili{} sign\ili{} of\ili{} desired\ili{} neutrality\ili{}.\ili{} But\ili{} then\ili{} one\ili{} must\ili{} also\ili{} accept\ili{} in\ili{}\isi\ili{}{transparency}\ili{} and\ili{} inflexibility\ili{},\ili{} at\ili{} least\ili{} to\ili{} some\ili{} degree\ili{}.\ili{}
\ili{}
A\ili{} tentative\ili{} PATR\ili{}-II\ili{} encoding\ili{} of\ili{} \ili{}(PL\ili{})\ili{} \ili{}\ile\ili{}{dobrze\ili{} \ili{}[KOMUŚ\ili{}]\ili{} z\ili{} oczu\ili{} patrzy}\ili{} is\ili{} presented\ili{} in\ili{} Figure\ili{}~\ili{}\ref\ili{}{fig\ili{}:patr\ili{}:polish}\ili{}.\ili{}
\ili{}\begin\ili{}{figure}\ili{}[htp\ili{}]\ili{}
\ili{} \ili{} \ili{}\begin\ili{}{patr\ili{}-listing}\ili{}
Define\ili{} ImpersIntransitive\ili{} as\ili{}
\ili{} \ili{} \ili{} \ili{} \ili{}[cat\ili{}:\ili{} v\ili{}
\ili{} \ili{} \ili{} \ili{} \ili{} pers\ili{}:\ili{} 3\ili{} \ili{}
\ili{} \ili{} \ili{} \ili{} \ili{} num\ili{}:\ili{} sg\ili{}
\ili{} \ili{} \ili{} \ili{} \ili{} subject\ili{}:\ili{} \ili{}-\ili{}
\ili{} \ili{} \ili{} \ili{} \ili{} object\ili{}:\ili{} \ili{}-\ili{}]\ili{}
\ili{}
Define\ili{} IndirectObject\ili{} as\ili{}
\ili{} \ili{} \ili{} \ili{} \ili{}[iobject\ili{}:\ili{} \ili{}[cat\ili{}:\ili{} np\ili{}
\ili{} \ili{} \ili{} \ili{} \ili{} \ili{} \ili{} \ili{} \ili{} \ili{} \ili{} \ili{} \ili{} \ili{} \ili{} case\ili{}:\ili{} dat\ili{}]\ili{}]\ili{}
\ili{}
Define\ili{} PrepositionalObject\ili{} as\ili{}
\ili{} \ili{} \ili{} \ili{} \ili{}[pobject\ili{}:\ili{} \ili{}[cat\ili{}:\ili{} pp\ili{}]\ili{}]\ili{}
\ili{} \ili{} \ili{} \ili{} \ili{}
Define\ili{} DobrzeZOczuPatrzy\ili{} as\ili{}
\ili{} \ili{} \ili{} \ili{} ImpersIntransitive\ili{}
\ili{} \ili{} \ili{} \ili{} IndirectObject\ili{}
\ili{} \ili{} \ili{} \ili{} PrepositionalObject\ili{}
\ili{} \ili{} \ili{} \ili{} Adverb\ili{}
\ili{} \ili{} \ili{} \ili{} \ili{}[pobject\ili{}:\ili{} \ili{}[lex\ili{}:\ili{} z\ili{}
\ili{} \ili{} \ili{} \ili{} \ili{} \ili{} \ili{} \ili{} \ili{} \ili{} \ili{} \ili{} \ili{} \ili{} \ili{} object\ili{}:\ili{} \ili{}[cat\ili{}:np\ili{}
\ili{} \ili{} \ili{} \ili{} \ili{} \ili{} \ili{} \ili{} \ili{} \ili{} \ili{} \ili{} \ili{} \ili{} \ili{} \ili{} \ili{} \ili{} \ili{} \ili{} \ili{} \ili{} \ili{} \ili{} case\ili{}:\ili{} gen\ili{}
\ili{} \ili{} \ili{} \ili{} \ili{} \ili{} \ili{} \ili{} \ili{} \ili{} \ili{} \ili{} \ili{} \ili{} \ili{} \ili{} \ili{} \ili{} \ili{} \ili{} \ili{} \ili{} \ili{} \ili{} num\ili{}:\ili{} pl\ili{}
\ili{} \ili{} \ili{} \ili{} \ili{} \ili{} \ili{} \ili{} \ili{} \ili{} \ili{} \ili{} \ili{} \ili{} \ili{} \ili{} \ili{} \ili{} \ili{} \ili{} \ili{} \ili{} \ili{} \ili{} lex\ili{}:\ili{} oko\ili{}
\ili{} \ili{} \ili{} \ili{} \ili{} \ili{} \ili{} \ili{} \ili{} \ili{} \ili{} \ili{} \ili{} \ili{} \ili{} \ili{} \ili{} \ili{} \ili{} \ili{} \ili{} \ili{} \ili{} \ili{} modifiable\ili{}:\ili{} \ili{}-\ili{}]\ili{}]\ili{}
\ili{} \ili{} \ili{} \ili{} \ili{} adverb\ili{}:\ili{} \ili{}[word\ili{}:\ili{} dobrze\ili{}
\ili{} \ili{} \ili{} \ili{} \ili{} \ili{} \ili{} \ili{} \ili{} \ili{} \ili{} \ili{} \ili{} \ili{} position\ili{}:\ili{} initial\ili{}]\ili{}]\ili{}
\ili{} \ili{} \ili{} \ili{} \ili{} sem\ili{}:\ili{} \ili{}[paraphrase\ili{}:\ili{} someone_looks_like_a_good_person\ili{}]\ili{}
\ili{} \ili{} \ili{} \ili{} \ili{}
Word\ili{} patrzy\ili{}:\ili{}
\ili{} \ili{} \ili{} \ili{} Verb\ili{}
\ili{} \ili{} \ili{} \ili{} \ili{}{\ili{}[PatrzecLiteral\ili{}]\ili{} \ili{}[DobrzeZOczuPatrzy\ili{}]}\ili{}
\ili{} \ili{} \ili{} \ili{} \ili{}[lex\ili{}:\ili{} patrze\ili{}|\ili{}%\ili{}\\ili{}'\ili{}{c}\ili{}%\ili{}|\ili{}]\ili{}
\ili{} \ili{} \ili{}\end\ili{}{patr\ili{}-listing}\ili{}
\ili{} \ili{} \ili{}\caption\ili{}{PATR\ili{}-II\ili{} description\ili{} \ili{}(with\ili{} PC\ili{}-PATR\ili{} notation\ili{})\ili{} of\ili{} \ili{}(PL\ili{})\ili{} \ili{}\ilet\ili{}{dobrze\ili{} \ili{}[KOMUŚ\ili{}]\ili{} z\ili{} oczu\ili{} patrzy}\ili{}{someone\ili{} looks\ili{} like\ili{} a\ili{} good\ili{} person}}\ili{}
\ili{} \ili{} \ili{}\label\ili{}{fig\ili{}:patr\ili{}:polish}\ili{}
\ili{}\end\ili{}{figure}\ili{}
As\ili{} explained\ili{} in\ili{} Section\ili{}~\ili{}\ref\ili{}{sec\ili{}:challenges}\ili{},\ili{} the\ili{} challenge\ili{} with\ili{} this\ili{} MWE\ili{} is\ili{} a\ili{} mixture\ili{} of\ili{} particular\ili{} constraints\ili{} regarding\ili{} the\ili{} \ili{}\isi\ili{}{subcategorization}\ili{} frame\ili{} of\ili{} the\ili{} verb\ili{} \ili{}(\ili{}\ilet\ili{}{patrzy}\ili{}{looks}\ili{} is\ili{} used\ili{} as\ili{} an\ili{} impersonal\ili{} transitive\ili{})\ili{} and\ili{} the\ili{} sentence\ili{} initial\ili{} linearization\ili{} of\ili{} the\ili{} adverb\ili{}.\ili{} The\ili{} encoding\ili{} example\ili{} in\ili{} Figure\ili{}~\ili{}\ref\ili{}{fig\ili{}:patr\ili{}:polish}\ili{} takes\ili{} care\ili{} of\ili{} this\ili{} by\ili{} stipulating\ili{} special\ili{} features\ili{} that\ili{} would\ili{} trigger\ili{} the\ili{} right\ili{} CFG\ili{} rules\ili{} at\ili{} the\ili{} right\ili{} time\ili{}.\ili{} Remember\ili{} that\ili{} the\ili{} constraints\ili{} on\ili{} the\ili{} occurrence\ili{} of\ili{} certain\ili{} arguments\ili{} can\ili{} be\ili{} encoded\ili{} by\ili{} using\ili{} \ili{}\isi\ili{}{subcategorization}\ili{} lists\ili{} in\ili{} the\ili{} usual\ili{} way\ili{}.\ili{} This\ili{} is\ili{} left\ili{} out\ili{} in\ili{} the\ili{} example\ili{}.\ili{} Now\ili{},\ili{} compared\ili{} to\ili{} the\ili{} \ili{} Walenty\ili{} encoding\ili{} in\ili{} Figure\ili{}~\ili{}\ref\ili{}{fig\ili{}:patrzy\ili{}:walenty}\ili{},\ili{} the\ili{} corresponding\ili{} PART\ili{}-II\ili{} \ili{}\isi\ili{}{template}\ili{} \ili{}\texttt\ili{}{DobrzeZOczuPatrzy}\ili{} is\ili{} much\ili{} more\ili{} verbose\ili{},\ili{} not\ili{} only\ili{} because\ili{} it\ili{} contains\ili{} more\ili{} information\ili{}.\ili{} But\ili{} this\ili{} should\ili{} not\ili{} be\ili{} taken\ili{} as\ili{} an\ili{} general\ili{} disadvantage\ili{},\ili{} as\ili{} it\ili{} can\ili{} help\ili{} to\ili{} promote\ili{} \ili{}\isi\ili{}{transparency}\ili{}.\ili{} \ili{} \ili{}
\ili{}
Summing\ili{} up\ili{},\ili{} the\ili{} examples\ili{} provided\ili{} here\ili{} demonstrate\ili{} that\ili{} PATR\ili{}-II\ili{} does\ili{} many\ili{} important\ili{} things\ili{} right\ili{}:\ili{} it\ili{} makes\ili{} available\ili{} a\ili{} transparent\ili{},\ili{} flexible\ili{} enough\ili{} encoding\ili{} language\ili{};\ili{} it\ili{} has\ili{} a\ili{} well\ili{}-defined\ili{} denotational\ili{} semantics\ili{};\ili{} it\ili{} includes\ili{} means\ili{} to\ili{} arbitrarily\ili{} factorize\ili{} properties\ili{} and\ili{} to\ili{} express\ili{} generalizations\ili{} even\ili{} beyond\ili{} strict\ili{} monotonicity\ili{}.\ili{} In\ili{} our\ili{} view\ili{},\ili{} this\ili{} makes\ili{} PATR\ili{}-II\ili{} better\ili{} suited\ili{} to\ili{} encode\ili{} MWEs\ili{} than\ili{} DuELME\ili{} and\ili{} Walenty\ili{} \ili{}\textit\ili{}{in\ili{} the\ili{} long\ili{} run}\ili{},\ili{} since\ili{} it\ili{} can\ili{} integrate\ili{} unforeseeable\ili{} properties\ili{},\ili{} regularities\ili{} or\ili{} encoding\ili{} styles\ili{} much\ili{} easier\ili{}.\ili{}
\ili{}
Yet\ili{} at\ili{} the\ili{} same\ili{} time\ili{},\ili{} encoding\ili{} with\ili{} PATR\ili{}-II\ili{} is\ili{} subject\ili{} to\ili{} some\ili{} severe\ili{} restrictions\ili{}:\ili{}
\ili{}\begin\ili{}{itemize}\ili{}
\ili{}\item\ili{} PATR\ili{}-II\ili{} does\ili{} not\ili{} seem\ili{} to\ili{} allow\ili{} for\ili{} templates\ili{} to\ili{} be\ili{} embedded\ili{}.\ili{} Hence\ili{},\ili{} templates\ili{} can\ili{} only\ili{} be\ili{} applied\ili{} to\ili{} the\ili{} root\ili{} of\ili{} a\ili{} feature\ili{} structure\ili{} description\ili{}.\ili{} \ili{}
\ili{}\item\ili{} Feature\ili{} structures\ili{} are\ili{} untyped\ili{} in\ili{} PATR\ili{}-II\ili{} which\ili{} makes\ili{} them\ili{} harder\ili{} to\ili{} be\ili{} checked\ili{} for\ili{} consistency\ili{} or\ili{} to\ili{} encode\ili{} representations\ili{} that\ili{} rely\ili{} on\ili{} types\ili{}.\ili{} \ili{}
\ili{}\item\ili{} PATR\ili{}-II\ili{} allows\ili{} one\ili{} to\ili{} describe\ili{} full\ili{} word\ili{} forms\ili{} as\ili{} terminals\ili{} of\ili{} CFG\ili{} rules\ili{},\ili{} but\ili{} it\ili{} is\ili{} not\ili{} possible\ili{} to\ili{} analyze\ili{} them\ili{} further\ili{},\ili{} that\ili{} is\ili{},\ili{} describe\ili{} the\ili{} underlying\ili{} morphemes\ili{} and\ili{} how\ili{} they\ili{} combine\ili{}.\ili{} Consequently\ili{},\ili{} it\ili{} is\ili{} at\ili{} least\ili{} tedious\ili{} to\ili{} describe\ili{} morphological\ili{} paradigms\ili{}.\ili{} This\ili{} is\ili{} something\ili{} that\ili{},\ili{} for\ili{} example\ili{},\ili{} DATR\ili{} \ili{}\citep\ili{}{Evans\ili{}:Gazdar\ili{}:96}\ili{} is\ili{} better\ili{} suited\ili{} for\ili{}.\ili{}
\ili{}\item\ili{} In\ili{} PATR\ili{}-II\ili{},\ili{} word\ili{} order\ili{} constraints\ili{} are\ili{} accounted\ili{} for\ili{} by\ili{} filtering\ili{} CFG\ili{} rules\ili{} via\ili{} features\ili{}.\ili{} Thus\ili{} it\ili{} is\ili{} not\ili{} possible\ili{} to\ili{} state\ili{} these\ili{} constraints\ili{} in\ili{} just\ili{} one\ili{} place\ili{},\ili{} but\ili{} one\ili{} has\ili{} to\ili{} think\ili{} of\ili{} which\ili{} features\ili{} prohibit\ili{} or\ili{} trigger\ili{} the\ili{} application\ili{} of\ili{} which\ili{} CFG\ili{} rules\ili{} in\ili{} which\ili{} situation\ili{} of\ili{} a\ili{} derivation\ili{}.\ili{}
\ili{}\end\ili{}{itemize}\ili{}
Furthermore\ili{},\ili{} as\ili{} we\ili{} said\ili{} before\ili{},\ili{} PATR\ili{}-II\ili{} chooses\ili{} a\ili{} lexical\ili{} approach\ili{} to\ili{} argument\ili{} structure\ili{} in\ili{} the\ili{} sense\ili{} of\ili{} \ili{}\cite\ili{}{Mueller\ili{}:Wechsler\ili{}:14}\ili{} where\ili{} the\ili{} argument\ili{} structure\ili{} emerges\ili{} from\ili{} lexical\ili{} units\ili{} and\ili{} crucially\ili{} determines\ili{} the\ili{} syntax\ili{}.\ili{} The\ili{} other\ili{} extreme\ili{},\ili{} namely\ili{} the\ili{} phrasal\ili{} approach\ili{} to\ili{} argument\ili{} structure\ili{},\ili{} rather\ili{} puts\ili{} emphasis\ili{} on\ili{} the\ili{} syntactic\ili{} side\ili{},\ili{} assuming\ili{} phrasal\ili{} representations\ili{} of\ili{} argument\ili{} structure\ili{} that\ili{} exist\ili{} independently\ili{} of\ili{} lexical\ili{} anchors\ili{}.\ili{} This\ili{} latter\ili{} approach\ili{} better\ili{} fits\ili{} into\ili{} the\ili{} encoding\ili{} format\ili{} of\ili{} XMG\ili{},\ili{} which\ili{} will\ili{} be\ili{} presented\ili{} next\ili{}.\ili{} \ili{}
\ili{}
\ili{}
\ili{}%\ili{}\input\ili{}{05\ili{}-xmg}\ili{}
\ili{}\subsection\ili{}{eXtensible\ili{} MetaGrammar}\ili{}
\ili{}\label\ili{}{sec\ili{}:xmg}\ili{}
\ili{}
The\ili{} framework\ili{} of\ili{} eXtensible\ili{} MetaGrammar\ili{} \ili{}(XMG\ili{},\ili{} \ili{}\citealt\ili{}{Crabbe\ili{}:etal\ili{}:13}\ili{} and\ili{} XMG2\ili{},\ili{} \ili{}\citealt\ili{}{Petitjean2016}\ili{})\ili{} most\ili{} obviously\ili{} differs\ili{} from\ili{} the\ili{} ones\ili{} of\ili{} PATR\ili{}-II\ili{},\ili{} DuELME\ili{} and\ili{} Walenty\ili{} in\ili{} that\ili{} it\ili{} can\ili{} be\ili{} used\ili{} to\ili{} generate\ili{} a\ili{} wide\ili{} range\ili{} of\ili{} linguistic\ili{} resources\ili{}.\ili{} The\ili{} variety\ili{} of\ili{} these\ili{} resources\ili{} is\ili{} made\ili{} possible\ili{} by\ili{} XMG\ili{}'s\ili{} modularity\ili{} and\ili{} extensibility\ili{},\ili{} allowing\ili{} to\ili{} create\ili{} new\ili{} dedicated\ili{} compilers\ili{} using\ili{} adapted\ili{} description\ili{} languages\ili{}.\ili{} XMG\ili{} is\ili{} a\ili{} multi\ili{} paradigm\ili{} language\ili{},\ili{} as\ili{} it\ili{} manipulates\ili{} programs\ili{} \ili{}(metagrammars\ili{})\ili{} which\ili{} make\ili{} intensive\ili{} use\ili{} of\ili{} logic\ili{} \ili{}(as\ili{} Prolog\ili{} programs\ili{})\ili{} and\ili{} constraints\ili{}.\ili{} XMG\ili{} also\ili{} borrows\ili{} some\ili{} aspects\ili{} from\ili{} object\ili{}-oriented\ili{} programming\ili{},\ili{} whose\ili{} advantages\ili{} in\ili{} the\ili{} context\ili{} of\ili{} linguistic\ili{} knowledge\ili{} description\ili{} are\ili{} discussed\ili{} in\ili{} \ili{}\cite\ili{}{Daelemans\ili{}:DeSmedt\ili{}:94}\ili{}.\ili{} The\ili{} most\ili{} obvious\ili{} example\ili{} of\ili{} such\ili{} an\ili{} aspect\ili{} is\ili{} that\ili{} XMG\ili{} descriptions\ili{} are\ili{} organized\ili{} into\ili{} \ili{}\textsc\ili{}{classes}\ili{},\ili{} which\ili{} have\ili{} encapsulated\ili{} name\ili{} spaces\ili{}.\ili{} Inheritance\ili{} relations\ili{} may\ili{} hold\ili{} between\ili{} classes\ili{},\ili{} and\ili{} the\ili{} scope\ili{} of\ili{} the\ili{} identifiers\ili{} is\ili{} explicitly\ili{} controlled\ili{} thanks\ili{} to\ili{} \ili{}\texttt\ili{}{export}\ili{} statements\ili{}.\ili{} The\ili{} crucial\ili{} elements\ili{} of\ili{} a\ili{} class\ili{} are\ili{} \ili{}\textsc\ili{}{dimensions}\ili{}.\ili{} Each\ili{} of\ili{} them\ili{} is\ili{} equipped\ili{} with\ili{} a\ili{} description\ili{} language\ili{},\ili{} which\ili{} is\ili{} specifically\ili{} adapted\ili{} to\ili{} the\ili{} kind\ili{} of\ili{} structures\ili{} needed\ili{} in\ili{} the\ili{} dimension\ili{} \ili{}(trees\ili{},\ili{} predicates\ili{},\ili{} \ili{}\ldots\ili{})\ili{}.\ili{} Dimensions\ili{} are\ili{} compiled\ili{} independently\ili{},\ili{} thereby\ili{} enabling\ili{} the\ili{} grammar\ili{} writer\ili{} to\ili{} treat\ili{} the\ili{} levels\ili{} of\ili{} linguistic\ili{} information\ili{} separately\ili{}.\ili{} In\ili{} the\ili{} following\ili{} we\ili{} will\ili{} be\ili{} using\ili{} the\ili{} dimension\ili{} \ili{}\texttt\ili{}{\ili{}<syn\ili{}>}\ili{} for\ili{} the\ili{} syntax\ili{} and\ili{} the\ili{} more\ili{} recent\ili{} \ili{}\texttt\ili{}{\ili{}<frame\ili{}>}\ili{} dimension\ili{} for\ili{} frame\ili{}-semantic\ili{} descriptions\ili{},\ili{} skipping\ili{} over\ili{} other\ili{} available\ili{} dimensions\ili{}.\ili{} Note\ili{} that\ili{} \ili{}\texttt\ili{}{\ili{}<syn\ili{}>}\ili{} contains\ili{} tree\ili{} descriptions\ili{} where\ili{} nodes\ili{} may\ili{} carry\ili{} untyped\ili{} feature\ili{} structures\ili{},\ili{} while\ili{} \ili{}\texttt\ili{}{\ili{}<frame\ili{}>}\ili{} comprises\ili{} \ili{}\textit\ili{}{typed}\ili{} feature\ili{} structure\ili{} descriptions\ili{} \ili{}\citep\ili{}{Lichte\ili{}:Petitjean\ili{}:15}\ili{}.\ili{}
\ili{}
Figure\ili{}~\ili{}\ref\ili{}{fig\ili{}:waarnemen\ili{}:xmg}\ili{} shows\ili{} a\ili{} part\ili{} of\ili{} a\ili{} tentative\ili{} XMG\ili{} encoding\ili{} of\ili{} \ili{}(NL\ili{})\ili{} \ili{}\ile\ili{}{zijn\ili{} kansen\ili{} waarnemen}\ili{}.\ili{}%\ili{},\ili{} namely\ili{} the\ili{} idiosyncratic\ili{} part\ili{}.\ili{} \ili{}
\ili{}\begin\ili{}{figure}\ili{}[h\ili{}]\ili{}
\ili{}\begin\ili{}{xmg}\ili{}
class\ili{} intransitive\ili{}
import\ili{} subject\ili{}[\ili{}]\ili{} verb\ili{}[\ili{}]\ili{}
\ili{}{\ili{} \ili{}<syn\ili{}>\ili{} \ili{}{\ili{} \ili{}?Subj\ili{} \ili{}>\ili{}>\ili{}+\ili{} \ili{}?V\ili{} }}\ili{}
\ili{}
class\ili{} transitive\ili{}
import\ili{} intransitive\ili{}[\ili{}]\ili{} object\ili{}[\ili{}]\ili{}
\ili{}{\ili{} \ili{}<syn\ili{}>\ili{} \ili{}{\ili{} \ili{}?Subj\ili{} \ili{}>\ili{}>\ili{}+\ili{} \ili{}?Obj\ili{};\ili{}
\ili{} \ili{} \ili{} \ili{} \ili{} \ili{} \ili{} \ili{} \ili{} \ili{} \ili{}?Obj\ili{} \ili{}>\ili{}>\ili{}+\ili{} \ili{}?V\ili{} }\ili{} }\ili{}
\ili{}
class\ili{} subject_poss_object_agreement\ili{}
declare\ili{} \ili{}?Subj\ili{} \ili{}?Obj\ili{} \ili{}?NUM\ili{} \ili{}?PERS\ili{} \ili{}?GEND\ili{}
export\ili{} \ili{}?Subj\ili{} \ili{}?Obj\ili{}
\ili{}{\ili{} \ili{}<syn\ili{}>\ili{} \ili{}{\ili{}
\ili{} \ili{} \ili{} \ili{} \ili{}?Subj\ili{}[num\ili{}=\ili{}?NUM\ili{},pers\ili{}=\ili{}?PERS\ili{},gend\ili{}=\ili{}?GEND\ili{}]\ili{};\ili{}
\ili{} \ili{} \ili{} \ili{} \ili{}?Obj\ili{} \ili{}[\ili{}]\ili{} \ili{}{\ili{}
\ili{} \ili{} \ili{} \ili{} \ili{} \ili{} \ili{}[cat\ili{}=d\ili{},num\ili{}=pl\ili{},possnum\ili{}=\ili{}?NUM\ili{},pers\ili{}=\ili{}?PERS\ili{},gend\ili{}=\ili{}?GEND\ili{}]\ili{} \ili{}"zijn\ili{}"}}}\ili{}
\ili{}
class\ili{} zijn_kansen_waarnemen\ili{} \ili{}
import\ili{} transitive\ili{}[\ili{}]\ili{} subject_poss_object_agreement\ili{}[\ili{}]\ili{}
declare\ili{} \ili{}?I\ili{}
\ili{}{\ili{} \ili{}<syn\ili{}>\ili{} \ili{}{\ili{}
\ili{} \ili{} \ili{} \ili{} \ili{}?Subj\ili{}[i\ili{}=\ili{}?I\ili{}]\ili{};\ili{}
\ili{} \ili{} \ili{} \ili{} \ili{}?Obj\ili{} \ili{}[\ili{}]\ili{} \ili{}{\ili{}
\ili{} \ili{} \ili{} \ili{} \ili{} \ili{} \ili{}[cat\ili{}=n\ili{},modifiable\ili{}=\ili{}-\ili{},num\ili{}=pl\ili{}]\ili{} \ili{}"kans\ili{}"}\ili{};\ili{}
\ili{} \ili{} \ili{} \ili{} \ili{}?V\ili{}[\ili{}]\ili{} \ili{}"waar_nehmen\ili{}"\ili{} }\ili{};\ili{}
\ili{} \ili{} \ili{}<frame\ili{}>\ili{} \ili{}{\ili{}
\ili{} \ili{} \ili{} \ili{} \ili{} \ili{}[using\ili{}-event\ili{},\ili{}
\ili{} \ili{} \ili{} \ili{} \ili{} \ili{} actor\ili{}:\ili{}?I\ili{},\ili{}
\ili{} \ili{} \ili{} \ili{} \ili{} \ili{} theme\ili{}:chance\ili{}]}}\ili{}
\ili{}\end\ili{}{xmg}\ili{}
\ili{} \ili{} \ili{}\caption\ili{}{XMG\ili{} encoding\ili{} of\ili{} \ili{}\textit\ili{}{zijn\ili{} kansen\ili{} waarnemen}\ili{} \ili{}(\ili{}`to\ili{} seize\ili{} the\ili{} opportunity\ili{}'\ili{})}\ili{}
\ili{} \ili{} \ili{}\label\ili{}{fig\ili{}:waarnemen\ili{}:xmg}\ili{}
\ili{}\end\ili{}{figure}\ili{}
The\ili{} first\ili{} thing\ili{} to\ili{} notice\ili{} when\ili{} comparing\ili{} the\ili{} XMG\ili{} description\ili{} to\ili{} the\ili{} DuELME\ili{} counterpart\ili{} in\ili{} Figure\ili{}~\ili{}\ref\ili{}{fig\ili{}:duelme}\ili{} is\ili{} that\ili{} there\ili{} is\ili{} no\ili{} principled\ili{} distinction\ili{} between\ili{} \ili{}`\ili{}`patterns\ili{}'\ili{}'\ili{} and\ili{} \ili{}`\ili{}`MWE\ili{} descriptions\ili{}'\ili{}'\ili{} \ili{}(similarly\ili{} to\ili{} the\ili{} PATR\ili{}-II\ili{} encoding\ili{} in\ili{} Figure\ili{}~\ili{}\ref\ili{}{fig\ili{}:patr\ili{}:dutch}\ili{})\ili{}.\ili{} Rather\ili{} they\ili{} are\ili{} equally\ili{} represented\ili{} as\ili{} classes\ili{},\ili{} yet\ili{} of\ili{} varying\ili{} specificity\ili{}.\ili{} Crucially\ili{},\ili{} the\ili{} classes\ili{} stand\ili{} in\ili{} inheritance\ili{} relations\ili{},\ili{} here\ili{} marked\ili{} with\ili{} the\ili{} import\ili{} statement\ili{}.\ili{} For\ili{} example\ili{},\ili{} the\ili{} most\ili{} basic\ili{} class\ili{} shown\ili{} in\ili{} Figure\ili{}~\ili{}\ref\ili{}{fig\ili{}:waarnemen\ili{}:xmg}\ili{},\ili{} \ili{}\texttt\ili{}{intransitive\ili{}[\ili{}]}\ili{},\ili{} imports\ili{} two\ili{} other\ili{} classes\ili{},\ili{} \ili{}\texttt\ili{}{subject\ili{}[\ili{}]}\ili{} and\ili{} \ili{}\texttt\ili{}{verb\ili{}[\ili{}]}\ili{} \ili{}(cf\ili{}.\ili{} Line\ili{} 2\ili{})\ili{}.\ili{} On\ili{} the\ili{} other\ili{} hand\ili{},\ili{} \ili{}\texttt\ili{}{intransitive\ili{}[\ili{}]}\ili{} is\ili{} further\ili{} handed\ili{} down\ili{} to\ili{} \ili{}\texttt\ili{}{transitive\ili{}[\ili{}]}\ili{},\ili{} just\ili{} adding\ili{} \ili{}\texttt\ili{}{object\ili{}[\ili{}]}\ili{}.\ili{} Finally\ili{},\ili{} \ili{}\texttt\ili{}{transitive\ili{}[\ili{}]}\ili{} gets\ili{} imported\ili{} into\ili{} \ili{}\texttt\ili{}{subject\ili{}\_poss\ili{}\_object\ili{}\_agreement\ili{}[\ili{}]}\ili{} to\ili{} add\ili{} the\ili{} compulsory\ili{} agreement\ili{} between\ili{} the\ili{} subject\ili{} and\ili{} the\ili{} possessive\ili{} pronoun\ili{} of\ili{} the\ili{} object\ili{},\ili{} and\ili{},\ili{} in\ili{} turn\ili{},\ili{} this\ili{} class\ili{} is\ili{} further\ili{} imported\ili{} into\ili{} \ili{}\texttt\ili{}{zijn\ili{}\_kansen\ili{}\_waarnemen\ili{}[\ili{}]}\ili{},\ili{} which\ili{} is\ili{} the\ili{} class\ili{} of\ili{} the\ili{} MWE\ili{} proper\ili{}.\ili{} Hence\ili{},\ili{} \ili{}\texttt\ili{}{subject\ili{}\_poss\ili{}\_object\ili{}\_agreement\ili{}[\ili{}]}\ili{} contains\ili{} the\ili{} more\ili{} regular\ili{} properties\ili{} of\ili{} the\ili{} MWE\ili{},\ili{} and\ili{} \ili{}\texttt\ili{}{zijn\ili{}\_kansen\ili{}\_waarnemen\ili{}[\ili{}]}\ili{} the\ili{} less\ili{} regular\ili{} ones\ili{}.\ili{} The\ili{} corresponding\ili{} inheritance\ili{} hierarchy\ili{} of\ili{} the\ili{} used\ili{} classes\ili{} is\ili{} shown\ili{} in\ili{} Figure\ili{}~\ili{}\ref\ili{}{fig\ili{}:waarnemen\ili{}:xmg\ili{}:hierarchy}\ili{},\ili{} in\ili{} which\ili{} the\ili{} MWE\ili{} shows\ili{} up\ili{} as\ili{} leaf\ili{},\ili{} i\ili{}.e\ili{}.\ili{} as\ili{} the\ili{} most\ili{} specific\ili{} class\ili{}.\ili{} Note\ili{} that\ili{} this\ili{} inheritance\ili{} hierarchy\ili{} mirrors\ili{} the\ili{} one\ili{} of\ili{} the\ili{} PATR\ili{}-II\ili{} encoding\ili{} in\ili{} Figure\ili{}~\ili{}\ref\ili{}{fig\ili{}:patr\ili{}:dutch}\ili{}.\ili{} \ili{}
\ili{}\begin\ili{}{figure}\ili{}
\ili{} \ili{} \ili{}\centering\ili{}
\ili{} \ili{} \ili{}%\ili{}%\ili{}\scalebox\ili{}{0\ili{}.76}\ili{}{\ili{}
\ili{} \ili{} \ili{} \ili{} \ili{}\Forest\ili{}{\ili{}
\ili{} \ili{} \ili{} \ili{} \ili{} \ili{} for\ili{} tree\ili{}=\ili{}{\ili{}
\ili{} \ili{} \ili{} \ili{} \ili{} \ili{} \ili{} \ili{} grow\ili{}=north\ili{},\ili{}
\ili{} \ili{} \ili{} \ili{} \ili{} \ili{} \ili{} \ili{} parent\ili{} anchor\ili{}=north\ili{},\ili{} \ili{}
\ili{} \ili{} \ili{} \ili{} \ili{} \ili{} \ili{} \ili{} child\ili{} anchor\ili{}=south\ili{},\ili{}
\ili{} \ili{} \ili{} \ili{} \ili{} \ili{} \ili{} \ili{} \ili{}%\ili{} fit\ili{}=rectangle\ili{},\ili{}
\ili{} \ili{} \ili{} \ili{} \ili{} \ili{} \ili{} \ili{} l\ili{} sep\ili{}=5ex\ili{},\ili{}
\ili{} \ili{} \ili{} \ili{} \ili{} \ili{} \ili{} \ili{} font\ili{}=\ili{}\ttfamily\ili{},\ili{}
\ili{} \ili{} \ili{} \ili{} \ili{} \ili{} }\ili{}
\ili{} \ili{} \ili{} \ili{} \ili{} \ili{} \ili{}[\ili{}{zijn\ili{}\_kansen\ili{}\_waarnemen\ili{}[\ili{}]}\ili{}
\ili{} \ili{} \ili{} \ili{} \ili{} \ili{} \ili{}[\ili{}{subject\ili{}\_poss\ili{}\_object\ili{}\_agreement\ili{}[\ili{}]}\ili{}
\ili{} \ili{} \ili{} \ili{} \ili{} \ili{} \ili{}[\ili{}{transitive\ili{}[\ili{}]}\ili{}
\ili{} \ili{} \ili{} \ili{} \ili{} \ili{} \ili{}[\ili{}{object\ili{}[\ili{}]}\ili{}]\ili{}
\ili{} \ili{} \ili{} \ili{} \ili{} \ili{} \ili{}[\ili{}{intransitive\ili{}[\ili{}]}\ili{}
\ili{} \ili{} \ili{} \ili{} \ili{} \ili{} \ili{}[\ili{}{verb\ili{}[\ili{}]}\ili{}]\ili{}
\ili{} \ili{} \ili{} \ili{} \ili{} \ili{} \ili{}[\ili{}{subject\ili{}[\ili{}]}\ili{}]\ili{}]\ili{}]\ili{}]\ili{}
\ili{} \ili{} \ili{} \ili{} \ili{} \ili{} \ili{}]\ili{}
\ili{} \ili{} \ili{} \ili{} }\ili{}%\ili{}%}\ili{}
\ili{} \ili{} \ili{}\caption\ili{}{Inheritance\ili{} hierarchy\ili{} of\ili{} XMG\ili{} classes\ili{} according\ili{} to\ili{} the\ili{} code\ili{} in\ili{} Figure\ili{}~\ili{}\ref\ili{}{fig\ili{}:waarnemen\ili{}:xmg}}\ili{}
\ili{} \ili{} \ili{}\label\ili{}{fig\ili{}:waarnemen\ili{}:xmg\ili{}:hierarchy}\ili{}
\ili{}\end\ili{}{figure}\ili{} \ili{}
\ili{}
In\ili{} general\ili{},\ili{} classes\ili{} that\ili{} correspond\ili{} to\ili{} irregular\ili{} or\ili{} weakly\ili{} regular\ili{} properties\ili{} of\ili{} lexical\ili{} entries\ili{} appear\ili{} as\ili{} leaves\ili{},\ili{} whereas\ili{} more\ili{} regular\ili{} aspects\ili{} are\ili{} assigned\ili{} to\ili{} dominating\ili{} classes\ili{}.\ili{} Hence\ili{},\ili{} \ili{}`\ili{}`patterns\ili{}'\ili{}'\ili{} can\ili{} be\ili{} arbitrarily\ili{} factorized\ili{},\ili{} which\ili{} is\ili{} in\ili{} sharp\ili{} contrast\ili{} to\ili{} the\ili{} DuELME\ili{} encoding\ili{} format\ili{}.\ili{} Another\ili{} difference\ili{} is\ili{} the\ili{} general\ili{} availability\ili{} of\ili{} variables\ili{} in\ili{} XMG\ili{},\ili{} which\ili{} are\ili{} commonly\ili{} prefixed\ili{} with\ili{} a\ili{} question\ili{} mark\ili{}.\ili{} This\ili{} is\ili{} exploited\ili{} in\ili{} \ili{}\texttt\ili{}{subject\ili{}\_poss\ili{}\_object\ili{}\_agreement\ili{}[\ili{}]}\ili{} when\ili{} expressing\ili{} agreement\ili{} between\ili{} the\ili{} subject\ili{} and\ili{} the\ili{} possessive\ili{} determiner\ili{} using\ili{} the\ili{} variables\ili{} \ili{}\texttt\ili{}{\ili{}?NUM}\ili{},\ili{} \ili{}\texttt\ili{}{\ili{}?PERS}\ili{},\ili{} and\ili{} \ili{}\texttt\ili{}{\ili{}?GEND}\ili{} \ili{}(cf\ili{}.\ili{} Lines\ili{} 14\ili{} and\ili{} 16\ili{})\ili{}.\ili{} Variables\ili{} are\ili{} also\ili{} used\ili{} for\ili{} sharing\ili{} information\ili{} between\ili{} dimensions\ili{},\ili{} for\ili{} example\ili{} between\ili{} \ili{}\texttt\ili{}{\ili{}<syn\ili{}>}\ili{} and\ili{} \ili{}\texttt\ili{}{\ili{}<frame\ili{}>}\ili{},\ili{} which\ili{} holds\ili{} the\ili{} idiomatic\ili{} meaning\ili{} of\ili{} the\ili{} MWE\ili{},\ili{} in\ili{} class\ili{} \ili{}\texttt\ili{}{zijn\ili{}\_kansen\ili{}\_waarnemen\ili{}[\ili{}]}\ili{}:\ili{} the\ili{} unification\ili{} variable\ili{} \ili{}\texttt\ili{}{\ili{}?I}\ili{} here\ili{} is\ili{} the\ili{} frame\ili{} referent\ili{} of\ili{} the\ili{} subject\ili{},\ili{} and\ili{} consequently\ili{} appears\ili{} both\ili{} in\ili{} the\ili{} syntactic\ili{} node\ili{} \ili{}\texttt\ili{}{\ili{}?Subj}\ili{} and\ili{} as\ili{} the\ili{} value\ili{} of\ili{} the\ili{} feature\ili{} \ili{}\texttt\ili{}{actor}\ili{} in\ili{} the\ili{} semantic\ili{} frame\ili{}.\ili{} Finally\ili{},\ili{} features\ili{} and\ili{} variables\ili{} can\ili{} be\ili{} freely\ili{} added\ili{} to\ili{} XMG\ili{},\ili{} for\ili{} example\ili{} features\ili{} to\ili{} indicate\ili{} constraints\ili{} on\ili{} modification\ili{} \ili{}(\ili{}\texttt\ili{}{modifiable}\ili{})\ili{} or\ili{} \ili{}\isi\ili{}{passivization}\ili{}.\ili{}
\ili{}
Remember\ili{} that\ili{} the\ili{} descriptions\ili{} in\ili{} \ili{}\texttt\ili{}{\ili{}<syn\ili{}>}\ili{} are\ili{} tree\ili{} descriptions\ili{},\ili{} which\ili{} are\ili{} able\ili{} to\ili{} express\ili{} the\ili{} usual\ili{},\ili{} potentially\ili{} underspecified\ili{} node\ili{} relations\ili{} regarding\ili{} dominance\ili{} and\ili{} precedence\ili{}.\ili{} For\ili{} example\ili{},\ili{} \ili{}\texttt\ili{}{\ili{}>\ili{}{}\ili{}>\ili{}+}\ili{} \ili{}(cf\ili{}.\ili{} Lines\ili{} 3\ili{},\ili{} 7\ili{} and\ili{} 8\ili{} in\ili{} Figure\ili{}~\ili{}\ref\ili{}{fig\ili{}:waarnemen\ili{}:xmg}\ili{})\ili{} expresses\ili{} the\ili{} transitive\ili{},\ili{} non\ili{}-reflexive\ili{} precedence\ili{} relation\ili{} between\ili{} two\ili{} nodes\ili{} of\ili{} a\ili{} tree\ili{}.\ili{} As\ili{} the\ili{} tree\ili{} descriptions\ili{} can\ili{} be\ili{} underspecified\ili{} in\ili{} this\ili{} way\ili{},\ili{} the\ili{} denotation\ili{} can\ili{} be\ili{} a\ili{} set\ili{} of\ili{} trees\ili{}.\ili{} XMG\ili{} comes\ili{} with\ili{} a\ili{} solver\ili{} for\ili{} these\ili{} descriptions\ili{},\ili{} and\ili{} a\ili{} viewer\ili{},\ili{} which\ili{} are\ili{} both\ili{} available\ili{} online\ili{}.\ili{}\footnote\ili{}{\ili{}\url\ili{}{http\ili{}:\ili{}/\ili{}/xmg\ili{}.phil\ili{}.hhu\ili{}.de\ili{}/}}\ili{} Hence\ili{} the\ili{} solutions\ili{} can\ili{} be\ili{} inspected\ili{} independently\ili{} of\ili{} a\ili{} specific\ili{} application\ili{} belonging\ili{} to\ili{} some\ili{} specific\ili{} framework\ili{}.\ili{}
\ili{}
The\ili{} preliminary\ili{} XMG\ili{} encoding\ili{} of\ili{} \ili{}(PL\ili{})\ili{} \ili{}\ile\ili{}{dobrze\ili{} \ili{}[KOMUS\ili{}]\ili{} z\ili{} oczu\ili{} patrzy}\ili{} is\ili{} presented\ili{} in\ili{} Figure\ili{}~\ili{}\ref\ili{}{fig\ili{}:dobrze\ili{}:xmg}\ili{}.\ili{}\footnote\ili{}{\ili{} We\ili{} owe\ili{} the\ili{} frame\ili{} semantic\ili{} representation\ili{} in\ili{} Figure\ili{}~\ili{}\ref\ili{}{fig\ili{}:dobrze\ili{}:xmg}\ili{} to\ili{} Rainer\ili{} Osswald\ili{}.}\ili{}
\ili{}\begin\ili{}{figure}\ili{}[ht\ili{}]\ili{}
\ili{}\begin\ili{}{xmg}\ili{}
class\ili{} impers_intransitive\ili{}
export\ili{} \ili{}?VP\ili{} \ili{}?V\ili{}
declare\ili{} \ili{}?VP\ili{} \ili{}?V\ili{}
\ili{}{\ili{} \ili{}<syn\ili{}>\ili{}{\ili{}
\ili{} \ili{} \ili{} \ili{} \ili{}?VP\ili{} \ili{}[cat\ili{}=vp\ili{}]\ili{} \ili{}{\ili{} \ili{}?V\ili{} \ili{}[cat\ili{}=v\ili{},pers\ili{}=3\ili{},num\ili{}=sg\ili{}]\ili{} }}}\ili{}
\ili{}
class\ili{} dobrze_z_oczu_patrzy\ili{}
declare\ili{} \ili{}?I\ili{} \ili{}?A\ili{} \ili{}?P\ili{}
import\ili{} impers_intransitive\ili{}[\ili{}]\ili{} ind_object\ili{}[\ili{}]\ili{} pp_object\ili{}[\ili{}]\ili{} adverb\ili{}[\ili{}]\ili{} \ili{}
\ili{}{\ili{} \ili{}<syn\ili{}>\ili{} \ili{}{\ili{}
\ili{} \ili{} \ili{} \ili{} \ili{}?IndObj\ili{} \ili{}[i\ili{}=\ili{}?I\ili{}]\ili{};\ili{}
\ili{} \ili{} \ili{} \ili{} \ili{}?AdvP\ili{} \ili{}[\ili{}]\ili{} \ili{}{\ili{} \ili{}?A\ili{} \ili{}[\ili{}]\ili{} \ili{}"dobrze\ili{}"}\ili{};\ili{}
\ili{} \ili{} \ili{} \ili{} \ili{}?PP\ili{} \ili{}[\ili{}]\ili{} \ili{}{\ili{} \ili{}[case\ili{}=gen\ili{}]\ili{} \ili{}"z\ili{}"\ili{}
\ili{} \ili{} \ili{} \ili{} \ili{} \ili{} \ili{}[\ili{}]\ili{} \ili{}{\ili{} \ili{}
\ili{} \ili{} \ili{} \ili{} \ili{} \ili{} \ili{} \ili{} \ili{}[num\ili{}=pl\ili{},modifiable\ili{}=\ili{}-\ili{}]\ili{} \ili{}"oko\ili{}"}}\ili{};\ili{} \ili{} \ili{} \ili{}
\ili{} \ili{} \ili{} \ili{} \ili{}?V\ili{} \ili{}"patrze\ili{}|\ili{}%\ili{}\\ili{}'\ili{}{c}\ili{}%\ili{}|\ili{}"\ili{};\ili{}
\ili{} \ili{} \ili{} \ili{} \ili{}?VP\ili{} \ili{}-\ili{}>\ili{} \ili{}?PP\ili{};\ili{}
\ili{} \ili{} \ili{} \ili{} \ili{}?VP\ili{} \ili{}-\ili{}>\ili{} \ili{}?IndObj\ili{};\ili{}
\ili{} \ili{} \ili{} \ili{} \ili{}?AdvP\ili{} \ili{}>\ili{}>\ili{}+\ili{} \ili{}?PP\ili{};\ili{}
\ili{} \ili{} \ili{} \ili{} \ili{}?AdvP\ili{} \ili{}>\ili{}>\ili{}+\ili{} \ili{}?V\ili{} }\ili{};\ili{}
\ili{} \ili{} \ili{}<frame\ili{}>\ili{} \ili{}{\ili{}
\ili{} \ili{} \ili{} \ili{} \ili{}[impression\ili{}-about\ili{},\ili{}
\ili{} \ili{} \ili{} \ili{} \ili{} perceiver\ili{}:\ili{} \ili{}?P\ili{},\ili{}
\ili{} \ili{} \ili{} \ili{} \ili{} theme\ili{}:\ili{} \ili{}?I\ili{},\ili{}
\ili{} \ili{} \ili{} \ili{} \ili{} content\ili{}:\ili{}[has\ili{}-prop\ili{},\ili{}
\ili{} \ili{} \ili{} \ili{} \ili{} \ili{} \ili{} \ili{} \ili{} \ili{} \ili{} \ili{} \ili{} \ili{} theme\ili{}:\ili{} \ili{}?I\ili{},\ili{}
\ili{} \ili{} \ili{} \ili{} \ili{} \ili{} \ili{} \ili{} \ili{} \ili{} \ili{} \ili{} \ili{} \ili{} prop\ili{}:\ili{} good\ili{}]\ili{}
\ili{} \ili{} \ili{} \ili{} \ili{}]}\ili{} \ili{}
}\ili{}
\ili{}\end\ili{}{xmg}\ili{}
\ili{} \ili{} \ili{}\caption\ili{}{XMG\ili{} encoding\ili{} of\ili{} \ili{}\textit\ili{}{dobrze}\ili{} \ili{}[KOMUŚ\ili{}]\ili{} \ili{}\textit\ili{}{z\ili{} oczu\ili{} patrzy}\ili{} \ili{}(\ili{}`someone\ili{} looks\ili{} like\ili{} a\ili{} good\ili{} person\ili{}'\ili{})\ili{} }\ili{}
\ili{} \ili{} \ili{}\label\ili{}{fig\ili{}:dobrze\ili{}:xmg}\ili{} \ili{}
\ili{}\end\ili{}{figure}\ili{}
Again\ili{},\ili{} the\ili{} class\ili{} that\ili{} corresponds\ili{} to\ili{} the\ili{} MWE\ili{},\ili{} \ili{}\texttt\ili{}{dobrze\ili{}\_z\ili{}\_oczu\ili{}\_patrzy\ili{}[\ili{}]}\ili{},\ili{} inherits\ili{} from\ili{} more\ili{} abstract\ili{} \ili{}(and\ili{} \ili{}`\ili{}`regular\ili{}'\ili{}'\ili{})\ili{} classes\ili{},\ili{} which\ili{} can\ili{} be\ili{} also\ili{} seen\ili{} from\ili{} the\ili{} inheritance\ili{} hierarchy\ili{} in\ili{} Figure\ili{}~\ili{}\ref\ili{}{fig\ili{}:dobrze\ili{}:xmg\ili{}:hierarchy}\ili{}.\ili{}
\ili{}\begin\ili{}{figure}\ili{}[ht\ili{}]\ili{}
\ili{} \ili{} \ili{}\centering\ili{}
\ili{} \ili{} \ili{}\setlength\ili{}{\ili{}\tabcolsep}\ili{}{2mm}\ili{}
\ili{} \ili{} \ili{} \ili{} \ili{} \ili{} \ili{}\Forest\ili{}{\ili{}
\ili{} \ili{} \ili{} \ili{} \ili{} \ili{} for\ili{} tree\ili{}=\ili{}{\ili{}
\ili{} \ili{} \ili{} \ili{} \ili{} \ili{} grow\ili{}=north\ili{},\ili{}
\ili{} \ili{} \ili{} \ili{} \ili{} \ili{} parent\ili{} anchor\ili{}=north\ili{},\ili{} \ili{}
\ili{} \ili{} \ili{} \ili{} \ili{} \ili{} child\ili{} anchor\ili{}=south\ili{},\ili{}
\ili{} \ili{} \ili{} \ili{} \ili{} \ili{} \ili{}%\ili{} fit\ili{}=rectangle\ili{},\ili{}
\ili{} \ili{} \ili{} \ili{} \ili{} \ili{} l\ili{} sep\ili{}=5ex\ili{},\ili{}
\ili{} \ili{} \ili{} \ili{} \ili{} \ili{} font\ili{}=\ili{}\ttfamily\ili{},\ili{}
\ili{} \ili{} \ili{} \ili{} \ili{} \ili{} }\ili{}
\ili{} \ili{} \ili{} \ili{} \ili{} \ili{} \ili{}[\ili{}{dobrze\ili{}\_z\ili{}\_oczu\ili{}\_patrzy\ili{}[\ili{}]}\ili{}
\ili{} \ili{} \ili{} \ili{} \ili{} \ili{} \ili{}[\ili{}{adverb\ili{}[\ili{}]}\ili{}]\ili{}
\ili{} \ili{} \ili{} \ili{} \ili{} \ili{} \ili{}[\ili{}{pp\ili{}\_object\ili{}[\ili{}]}\ili{}]\ili{}
\ili{} \ili{} \ili{} \ili{} \ili{} \ili{} \ili{}[\ili{}{ind\ili{}\_object\ili{}[\ili{}]}\ili{}]\ili{}
\ili{} \ili{} \ili{} \ili{} \ili{} \ili{} \ili{}[\ili{}{impers\ili{}\_intransitive\ili{}[\ili{}]}\ili{}]\ili{}
\ili{} \ili{} \ili{} \ili{} \ili{} \ili{} \ili{}]\ili{}
\ili{} \ili{} \ili{} \ili{} \ili{} \ili{} }\ili{}
\ili{} \ili{} \ili{}\caption\ili{}{Inheritance\ili{} hierarchy\ili{} of\ili{} XMG\ili{} classes\ili{} according\ili{} to\ili{} the\ili{} code\ili{} in\ili{} Figure\ili{}~\ili{}\ref\ili{}{fig\ili{}:dobrze\ili{}:xmg}}\ili{}
\ili{} \ili{} \ili{}\label\ili{}{fig\ili{}:dobrze\ili{}:xmg\ili{}:hierarchy}\ili{}
\ili{}\end\ili{}{figure}\ili{}
Here\ili{},\ili{} the\ili{} \ili{}\texttt\ili{}{impers\ili{}\_intransitive\ili{}[\ili{}]}\ili{} class\ili{} encodes\ili{} the\ili{} fact\ili{} that\ili{} the\ili{} subject\ili{} is\ili{} absent\ili{} \ili{}(as\ili{} only\ili{} the\ili{} verb\ili{} \ili{}\isi\ili{}{phrase}\ili{} and\ili{} its\ili{} subordinate\ili{} verb\ili{} are\ili{} listed\ili{})\ili{},\ili{} and\ili{} that\ili{} the\ili{} \ili{}(impersonal\ili{})\ili{} verb\ili{} must\ili{} occur\ili{} in\ili{} the\ili{} third\ili{} person\ili{} singular\ili{}.\ili{} Finally\ili{},\ili{} the\ili{} \ili{}\texttt\ili{}{dobrze\ili{}\_z\ili{}\_oczu\ili{}\_patrzy\ili{}[\ili{}]}\ili{} class\ili{} reuses\ili{} the\ili{} previous\ili{} class\ili{} and\ili{} adds\ili{} the\ili{} compulsory\ili{} adverb\ili{}.\ili{} Moreover\ili{},\ili{} certain\ili{} \ili{} nodes\ili{},\ili{} identified\ili{} by\ili{} shared\ili{} variables\ili{},\ili{} are\ili{} further\ili{} specified\ili{} for\ili{} lemmas\ili{} \ili{}(in\ili{} double\ili{} quotes\ili{})\ili{} and\ili{} all\ili{} weakly\ili{} regular\ili{} morphological\ili{} constraints\ili{} are\ili{} listed\ili{}.\ili{} Notably\ili{},\ili{} the\ili{} noun\ili{} governed\ili{} by\ili{} the\ili{} preposition\ili{} \ili{}\textit\ili{}{z}\ili{} \ili{}`from\ili{}'\ili{} is\ili{} restricted\ili{} to\ili{} the\ili{} lemma\ili{} \ili{}\texttt\ili{}{oko}\ili{} \ili{}`eye\ili{}'\ili{} and\ili{} to\ili{} plural\ili{},\ili{} and\ili{} its\ili{} modification\ili{} is\ili{} prohibited\ili{}.\ili{} Note\ili{} that\ili{} the\ili{} genitive\ili{} case\ili{} of\ili{} \ili{}\textit\ili{}{oko}\ili{} is\ili{} not\ili{} specified\ili{} in\ili{} this\ili{} class\ili{},\ili{} as\ili{} it\ili{} is\ili{} already\ili{} part\ili{} of\ili{} the\ili{} agreement\ili{} rules\ili{} which\ili{} were\ili{} inherited\ili{} from\ili{} the\ili{} \ili{}\texttt\ili{}{pp\ili{}\_object\ili{}[\ili{}]}\ili{} class\ili{}.\ili{} Linearization\ili{} constraints\ili{} on\ili{} the\ili{} adverb\ili{} appear\ili{} in\ili{} Lines\ili{} 19\ili{}-\ili{}-20\ili{}.\ili{} The\ili{} example\ili{} also\ili{} includes\ili{} dominance\ili{} constraints\ili{} in\ili{} Lines\ili{} 17\ili{}-\ili{}-18\ili{} that\ili{} use\ili{} \ili{}\texttt\ili{}{\ili{}-\ili{}>}\ili{} to\ili{} describe\ili{} an\ili{} immediate\ili{} dominance\ili{} relation\ili{}.\ili{} Finally\ili{},\ili{} we\ili{} use\ili{} unification\ili{} variable\ili{} once\ili{} again\ili{} to\ili{} express\ili{} the\ili{} fact\ili{} that\ili{} the\ili{} semantic\ili{} referent\ili{} of\ili{} the\ili{} syntactic\ili{} subject\ili{} \ili{}(\ili{}\texttt\ili{}{\ili{}?I}\ili{})\ili{} is\ili{} the\ili{} theme\ili{} of\ili{} the\ili{} semantic\ili{} frame\ili{} of\ili{} the\ili{} MWE\ili{}.\ili{} This\ili{} frame\ili{} can\ili{} be\ili{} read\ili{} as\ili{} follows\ili{}:\ili{} a\ili{} perceiver\ili{} \ili{}\texttt\ili{}{\ili{}?P}\ili{},\ili{} left\ili{} unspecified\ili{},\ili{} has\ili{} an\ili{} impression\ili{} about\ili{} \ili{}\texttt\ili{}{\ili{}?I}\ili{},\ili{} and\ili{} this\ili{} impression\ili{} is\ili{} that\ili{} \ili{}\texttt\ili{}{\ili{}?I}\ili{} has\ili{} the\ili{} property\ili{} of\ili{} being\ili{} a\ili{} good\ili{} person\ili{}.\ili{} Thus\ili{},\ili{} all\ili{} the\ili{} necessary\ili{} constraints\ili{} imposed\ili{} on\ili{} this\ili{} MWE\ili{} can\ili{} be\ili{} covered\ili{} at\ili{} various\ili{} abstraction\ili{} levels\ili{},\ili{} while\ili{} factorizing\ili{} information\ili{} in\ili{} such\ili{} a\ili{} way\ili{} that\ili{} the\ili{} \ili{}\texttt\ili{}{dobrze\ili{}\_z\ili{}\_oczu\ili{}\_patrzy\ili{}[\ili{}]}\ili{} class\ili{} only\ili{} contains\ili{} the\ili{} constraints\ili{} which\ili{} are\ili{} specific\ili{} to\ili{} the\ili{} MWE\ili{} or\ili{} at\ili{} least\ili{} weakly\ili{} regular\ili{}.\ili{}
\ili{}
By\ili{} way\ili{} of\ili{} conclusion\ili{},\ili{} lets\ili{} compare\ili{} the\ili{} presented\ili{} encoding\ili{} examples\ili{} for\ili{} PA\ili{}\\ili{}-TR\ili{}-II\ili{} and\ili{} XMG\ili{} in\ili{} more\ili{} detail\ili{}.\ili{} Despite\ili{} their\ili{} large\ili{} \ili{} commonalities\ili{} when\ili{} contrasting\ili{} them\ili{} with\ili{} fixed\ili{} encoding\ili{} formats\ili{} such\ili{} as\ili{} DuELME\ili{} and\ili{} Walenty\ili{},\ili{} PATR\ili{}-II\ili{} and\ili{} XMG\ili{} can\ili{} differ\ili{} considerably\ili{} in\ili{} some\ili{} of\ili{} their\ili{} properties\ili{}.\ili{}
\ili{}\begin\ili{}{itemize}\ili{}
\ili{}\item\ili{} In\ili{} the\ili{} given\ili{} examples\ili{},\ili{} XMG\ili{} is\ili{} constructionist\ili{} in\ili{} the\ili{} sense\ili{} that\ili{} it\ili{} models\ili{} phrasal\ili{} units\ili{},\ili{} whereas\ili{} PATR\ili{}-II\ili{} assumes\ili{} a\ili{} head\ili{}-driven\ili{} \ili{}(or\ili{} \ili{}`\ili{}`lexicalist\ili{}'\ili{}'\ili{},\ili{} \ili{}\citealt\ili{}{Mueller\ili{}:Wechsler\ili{}:14}\ili{})\ili{} approach\ili{} to\ili{} representing\ili{} argument\ili{} structure\ili{}.\ili{} However\ili{},\ili{} this\ili{} is\ili{} not\ili{} to\ili{} say\ili{} that\ili{} XMG\ili{} cannot\ili{} be\ili{} also\ili{} used\ili{} in\ili{} a\ili{} head\ili{}-driven\ili{} way\ili{}.\ili{} \ili{}
\ili{}\item\ili{} XMG\ili{} supports\ili{} type\ili{} inferences\ili{},\ili{} hence\ili{} the\ili{} unification\ili{} of\ili{} typed\ili{} feature\ili{} structures\ili{}.\ili{} In\ili{} PATR\ili{}-II\ili{},\ili{} feature\ili{} structures\ili{} are\ili{} strictly\ili{} untyped\ili{}.\ili{}
\ili{}\item\ili{} XMG\ili{} comes\ili{} with\ili{} different\ili{} description\ili{} languages\ili{} as\ili{} well\ili{} as\ili{} different\ili{} types\ili{} of\ili{} models\ili{},\ili{} namely\ili{} trees\ili{},\ili{} typed\ili{} feature\ili{} structures\ili{},\ili{} expressions\ili{} of\ili{} predicate\ili{} logic\ili{} and\ili{} even\ili{} strings\ili{}.\ili{} PATR\ili{}-II\ili{} is\ili{} restricted\ili{} to\ili{} the\ili{} description\ili{} of\ili{} feature\ili{} structures\ili{} and\ili{} CFG\ili{} rules\ili{}.\ili{} \ili{} \ili{} \ili{}
\ili{}\item\ili{} XMG\ili{} allows\ili{} for\ili{} directed\ili{} inheritance\ili{} in\ili{} the\ili{} sense\ili{} that\ili{} inherited\ili{} descriptions\ili{} can\ili{} be\ili{} added\ili{} to\ili{} any\ili{} part\ili{} of\ili{} the\ili{} description\ili{},\ili{} not\ili{} just\ili{} the\ili{} root\ili{} part\ili{} as\ili{} with\ili{} PATR\ili{}-II\ili{}.\ili{}
\ili{}\item\ili{} XMG\ili{} is\ili{} more\ili{} verbose\ili{} than\ili{} PATR\ili{}-II\ili{} because\ili{} it\ili{} is\ili{} designed\ili{} to\ili{} implement\ili{} a\ili{} truly\ili{} object\ili{}-oriented\ili{} programming\ili{} style\ili{} with\ili{} encapsulated\ili{} namespaces\ili{} etc\ili{}.\ili{} When\ili{} considering\ili{} just\ili{} toy\ili{} examples\ili{},\ili{} it\ili{} is\ili{} admittedly\ili{} just\ili{} a\ili{} matter\ili{} of\ili{} taste\ili{},\ili{} whether\ili{} this\ili{} is\ili{} something\ili{} worthwhile\ili{}.\ili{} In\ili{} large\ili{}-scale\ili{} grammars\ili{} and\ili{} lexicons\ili{},\ili{} however\ili{},\ili{} the\ili{} advantage\ili{} can\ili{} be\ili{} more\ili{} substantial\ili{} by\ili{} helping\ili{} to\ili{} ensure\ili{} consistency\ili{} due\ili{} to\ili{} the\ili{} extra\ili{} checking\ili{} done\ili{} by\ili{} the\ili{} solver\ili{}.\ili{}
\ili{}\end\ili{}{itemize}\ili{}
In\ili{} sum\ili{},\ili{} XMG\ili{} seems\ili{} to\ili{} be\ili{} generally\ili{} more\ili{} powerful\ili{} than\ili{} PATR\ili{}-II\ili{},\ili{} but\ili{} also\ili{} more\ili{} cumbersome\ili{} in\ili{} the\ili{} way\ili{} of\ili{} encoding\ili{}.\ili{} \ili{}
\ili{}
\ili{}
\ili{}\section\ili{}{Summary}\ili{}
\ili{}%\ili{}\input\ili{}{06\ili{}-summary}\ili{}
\ili{}\label\ili{}{sec\ili{}:summary}\ili{}
\ili{}
Table\ili{} \ili{}\ref\ili{}{tab\ili{}:comparison}\ili{} shows\ili{} a\ili{} comparison\ili{} of\ili{} the\ili{} encoding\ili{} formalisms\ili{} presented\ili{} in\ili{} Sections\ili{}~\ili{}\ref\ili{}{sec\ili{}:fixed}\ili{} and\ili{}~\ili{}\ref\ili{}{sec\ili{}:fullyflexible}\ili{} with\ili{} respect\ili{} to\ili{} the\ili{} encoding\ili{} virtues\ili{} described\ili{} in\ili{} Sections\ili{}~\ili{}\ref\ili{}{sec\ili{}:virtues\ili{}-human}\ili{} and\ili{}~\ili{}\ref\ili{}{sec\ili{}:virtues\ili{}-resource}\ili{}.\ili{} We\ili{} omit\ili{} the\ili{} encoding\ili{} virtues\ili{} with\ili{} respect\ili{} to\ili{} a\ili{} lexical\ili{} object\ili{} \ili{}(cf\ili{}.\ili{} Section\ili{}~\ili{}\ref\ili{}{sec\ili{}:virtues\ili{}-object}\ili{})\ili{}.\ili{} They\ili{} are\ili{} mostly\ili{} related\ili{} to\ili{} a\ili{} particular\ili{} lexical\ili{} encoding\ili{} and\ili{} not\ili{} to\ili{} the\ili{} underlying\ili{} formalism\ili{}.\ili{}
\ili{}
\ili{}\begin\ili{}{table}\ili{}[ht\ili{}]\ili{}\scriptsize\ili{} \ili{}
\ili{} \ili{} \ili{}\begin\ili{}{center}\ili{}
\ili{} \ili{} \ili{} \ili{} \ili{}%\ili{} \ili{}\setlength\ili{}{\ili{}\tabcolsep}\ili{}{2ex}\ili{}
\ili{} \ili{} \ili{} \ili{} \ili{}\fittable\ili{}{\ili{}%\ili{}
\ili{} \ili{} \ili{} \ili{} \ili{}\begin\ili{}{tabular}\ili{}{lccccc}\ili{}
\ili{} \ili{} \ili{} \ili{} \ili{} \ili{} \ili{}%\ili{} \ili{}\hline\ili{}
\ili{} \ili{} \ili{} \ili{} \ili{} \ili{} \ili{}&\ili{} \ili{}\multicolumn\ili{}{4}\ili{}{c}\ili{}{Human\ili{} user\ili{} oriented}\ili{}
\ili{} \ili{} \ili{} \ili{} \ili{} \ili{} \ili{}&\ili{} \ili{}\multicolumn\ili{}{1}\ili{}{c}\ili{}{Lexical\ili{} resource\ili{} oriented}\ili{}\\ili{}\\ili{} \ili{}
\ili{} \ili{} \ili{} \ili{} \ili{} \ili{} \ili{}\cline\ili{}{2\ili{}-6}\ili{}
\ili{}
\ili{} \ili{} \ili{} \ili{} \ili{} \ili{} \ili{}&\ili{} \ili{}\textsc\ili{}{transparency}\ili{}
\ili{} \ili{} \ili{} \ili{} \ili{} \ili{} \ili{}&\ili{} \ili{}\textsc\ili{}{\ili{}\isi\ili{}{flexibility}}\ili{}
\ili{} \ili{} \ili{} \ili{} \ili{} \ili{} \ili{}&\ili{} \ili{}\textsc\ili{}{power\ili{} to}\ili{}
\ili{} \ili{} \ili{} \ili{} \ili{} \ili{} \ili{}&\ili{} \ili{}\textsc\ili{}{implementation}\ili{}
\ili{} \ili{} \ili{} \ili{} \ili{} \ili{} \ili{}&\ili{} \ili{}\textsc\ili{}{electronic}\ili{} \ili{}\\ili{}\\ili{}
\ili{} \ili{} \ili{} \ili{} \ili{} \ili{} \ili{}&\ili{} \ili{}&\ili{}
\ili{} \ili{} \ili{} \ili{} \ili{} \ili{} \ili{}&\ili{} \ili{}\multicolumn\ili{}{1}\ili{}{c}\ili{}{\ili{}\textsc\ili{}{generalize}}\ili{}
\ili{} \ili{} \ili{} \ili{} \ili{} \ili{} \ili{}&\ili{} \ili{}\multicolumn\ili{}{1}\ili{}{c}\ili{}{\ili{}\textsc\ili{}{friendliness}}\ili{}
\ili{} \ili{} \ili{} \ili{} \ili{} \ili{} \ili{}&\ili{} \ili{}\multicolumn\ili{}{1}\ili{}{c}\ili{}{\ili{}\textsc\ili{}{versatility}}\ili{} \ili{}\\ili{}\\ili{}
\ili{} \ili{} \ili{} \ili{} \ili{} \ili{} \ili{}\cline\ili{}{2\ili{}-6}\ili{}
\ili{} \ili{} \ili{} \ili{} \ili{} \ili{} \ili{}\hline\ili{}
\ili{}
\ili{} \ili{} \ili{} \ili{} \ili{} \ili{} \ili{}\multicolumn\ili{}{1}\ili{}{r}\ili{}{DuELME}\ili{}
\ili{} \ili{} \ili{} \ili{} \ili{} \ili{} \ili{}&\ili{} 4\ili{} \ili{}&\ili{} 4\ili{} \ili{}&\ili{} 3\ili{} \ili{}&\ili{} 2\ili{} \ili{}&\ili{} 1\ili{} \ili{} \ili{}\\ili{}\\ili{} \ili{}
\ili{} \ili{} \ili{} \ili{} \ili{} \ili{} \ili{}\multicolumn\ili{}{1}\ili{}{r}\ili{}{Walenty}\ili{}
\ili{} \ili{} \ili{} \ili{} \ili{} \ili{} \ili{}&\ili{} 3\ili{} \ili{}&\ili{} 3\ili{} \ili{}&\ili{} 3\ili{} \ili{}&\ili{} 1\ili{} \ili{}&\ili{} 1\ili{} \ili{}\\ili{}\\ili{} \ili{}
\ili{} \ili{} \ili{} \ili{} \ili{} \ili{} \ili{}\multicolumn\ili{}{1}\ili{}{r}\ili{}{PATR\ili{}-II}\ili{}
\ili{} \ili{} \ili{} \ili{} \ili{} \ili{} \ili{}&\ili{} 1\ili{} \ili{}&\ili{} 2\ili{} \ili{}&\ili{} 2\ili{} \ili{}&\ili{} 4\ili{} \ili{}&\ili{} 4\ili{} \ili{}\\ili{}\\ili{} \ili{}
\ili{} \ili{} \ili{} \ili{} \ili{} \ili{} \ili{}\multicolumn\ili{}{1}\ili{}{r}\ili{}{XMG}\ili{}
\ili{} \ili{} \ili{} \ili{} \ili{} \ili{} \ili{}&\ili{} 1\ili{} \ili{}&\ili{} 1\ili{} \ili{}&\ili{} 1\ili{} \ili{}&\ili{} 3\ili{} \ili{}&\ili{} 3\ili{} \ili{} \ili{}\\ili{}\\ili{}\hline\ili{}
\ili{} \ili{} \ili{} \ili{} \ili{}\end\ili{}{tabular}\ili{}
\ili{} \ili{} \ili{} \ili{} }\ili{}
\ili{} \ili{} \ili{} \ili{} \ili{}\caption\ili{}{Ranking\ili{} of\ili{} encoding\ili{} formats\ili{} in\ili{} different\ili{} categories\ili{} \ili{}-\ili{}-\ili{} lexical\ili{}
\ili{} \ili{} \ili{} \ili{} \ili{} \ili{} encoding\ili{} virtues\ili{} \ili{}-\ili{}-\ili{} with\ili{} special\ili{} focus\ili{} on\ili{} MWEs\ili{}.\ili{}
\ili{} \ili{} \ili{} \ili{} \ili{} \ili{} The\ili{} range\ili{} of\ili{} values\ili{} is\ili{} from\ili{} \ili{}$1\ili{}$\ili{} to\ili{} \ili{}$4\ili{}$\ili{},\ili{} where\ili{} \ili{}$1\ili{}$\ili{} means\ili{} that\ili{} we\ili{} judge\ili{} the\ili{}
\ili{} \ili{} \ili{} \ili{} \ili{} \ili{} corresponding\ili{} format\ili{} as\ili{} relatively\ili{} the\ili{} best\ili{} in\ili{} the\ili{} given\ili{} category\ili{}.\ili{}
\ili{} \ili{} \ili{} \ili{} }\ili{}
\ili{} \ili{} \ili{} \ili{} \ili{}\label\ili{}{tab\ili{}:comparison}\ili{}
\ili{} \ili{} \ili{}\end\ili{}{center}\ili{}
\ili{}\end\ili{}{table}\ili{}
\ili{}
\ili{}
Descriptions\ili{} in\ili{} PATR\ili{}-II\ili{} and\ili{} XMG\ili{} come\ili{} with\ili{} clear\ili{} denotational\ili{} semantics\ili{},\ili{} which\ili{} makes\ili{} these\ili{} two\ili{} formalisms\ili{} stand\ili{} out\ili{} as\ili{} highly\ili{} transparent\ili{} in\ili{} comparison\ili{} with\ili{} their\ili{} less\ili{} flexible\ili{} counterparts\ili{}.\ili{} Transparency\ili{} of\ili{} the\ili{} Walenty\ili{}'s\ili{} encoding\ili{} format\ili{} is\ili{} relatively\ili{} high\ili{}.\ili{} Due\ili{} to\ili{} its\ili{} conciseness\ili{},\ili{} it\ili{} is\ili{} possible\ili{} to\ili{} read\ili{},\ili{} analyze\ili{} and\ili{} write\ili{} new\ili{} entries\ili{} relatively\ili{} quickly\ili{}.\ili{} However\ili{},\ili{} this\ili{} requires\ili{} some\ili{} experience\ili{},\ili{} since\ili{} interpretation\ili{} of\ili{} certain\ili{} syntactic\ili{} constructions\ili{} \ili{}(e\ili{}.g\ili{}.\ili{},\ili{} positions\ili{} in\ili{} lexically\ili{} restricted\ili{} \ili{}\isi\ili{}{phrase}\ili{} descriptions\ili{})\ili{} is\ili{} implicit\ili{}.\ili{} More\ili{} importantly\ili{},\ili{} interpretation\ili{} of\ili{} the\ili{} meaning\ili{} of\ili{} symbols\ili{} used\ili{} in\ili{} Walenty\ili{} descriptions\ili{} is\ili{} often\ili{} implicit\ili{} as\ili{} well\ili{}.\ili{} Certain\ili{} patterns\ili{} \ili{}-\ili{}-\ili{} for\ili{} instance\ili{},\ili{} a\ili{} prepositional\ili{} noun\ili{} \ili{}\isi\ili{}{phrase}\ili{} \ili{}(\ili{}\textsc\ili{}{prepnp}\ili{})\ili{} \ili{}-\ili{}-\ili{} are\ili{} defined\ili{} as\ili{} atomic\ili{} constructions\ili{},\ili{} and\ili{} the\ili{} recommended\ili{} way\ili{} to\ili{} model\ili{} new\ili{} phenomena\ili{} \ili{}-\ili{}-\ili{} for\ili{} instance\ili{},\ili{} agreement\ili{} between\ili{} the\ili{} subject\ili{} and\ili{} the\ili{} possessive\ili{} determiner\ili{} of\ili{} the\ili{} direct\ili{} object\ili{} \ili{}-\ili{}-\ili{} is\ili{} to\ili{} add\ili{} new\ili{} symbols\ili{} to\ili{} the\ili{} alphabet\ili{} of\ili{} the\ili{} formalism\ili{}.\ili{}\footnote\ili{}{In\ili{} fully\ili{} flexible\ili{} formalisms\ili{} such\ili{} new\ili{} syntactic\ili{} phenomena\ili{} can\ili{} be\ili{} factored\ili{} through\ili{} the\ili{} use\ili{} of\ili{} dedicated\ili{} classes\ili{} whose\ili{} semantics\ili{} remains\ili{} explicit\ili{}.}\ili{} This\ili{} can\ili{} be\ili{} seen\ili{} as\ili{} a\ili{} flexible\ili{} solution\ili{},\ili{} but\ili{} it\ili{} may\ili{} also\ili{} lead\ili{} to\ili{} proliferation\ili{} of\ili{} atomic\ili{} symbols\ili{} with\ili{} encoding\ili{}-specific\ili{} semantics\ili{},\ili{} not\ili{} defined\ili{} within\ili{} the\ili{} formalism\ili{} itself\ili{}.\ili{} This\ili{} in\ili{} turn\ili{} may\ili{} harm\ili{} \ili{}\isi\ili{}{transparency}\ili{} of\ili{} the\ili{} individual\ili{} Walenty\ili{}-based\ili{} encodings\ili{} and\ili{} decrease\ili{} its\ili{} overall\ili{} electronic\ili{} versatility\ili{}.\ili{} Finally\ili{},\ili{} there\ili{} seems\ili{} to\ili{} be\ili{} no\ili{} clear\ili{} denotational\ili{} semantics\ili{} defined\ili{} for\ili{} DuELME\ili{} descriptions\ili{} \ili{}(except\ili{},\ili{} maybe\ili{},\ili{} in\ili{} its\ili{} LMF\ili{} standard\ili{} export\ili{} format\ili{})\ili{}.\ili{} Their\ili{} interpretation\ili{} is\ili{} based\ili{} partially\ili{} on\ili{} formal\ili{} properties\ili{} and\ili{} inference\ili{} rules\ili{},\ili{} partially\ili{} on\ili{} methodological\ili{} recommendations\ili{},\ili{} and\ili{} the\ili{} borderline\ili{} between\ili{} the\ili{} two\ili{} is\ili{} hard\ili{} to\ili{} determine\ili{},\ili{} which\ili{} severely\ili{} harms\ili{} the\ili{} clarity\ili{} of\ili{} the\ili{} format\ili{}.\ili{} \ili{}
\ili{}
Not\ili{} very\ili{} surprisingly\ili{},\ili{} XMG\ili{} and\ili{} PATR\ili{}-II\ili{} are\ili{} also\ili{} more\ili{} flexible\ili{} than\ili{} Walenty\ili{} or\ili{} DuELME\ili{}.\ili{} In\ili{} comparison\ili{} to\ili{} XMG\ili{},\ili{} PATR\ili{}-II\ili{} exhibits\ili{} certain\ili{} restrictions\ili{} \ili{}(see\ili{} Section\ili{}~\ili{}\ref\ili{}{sec\ili{}:patr\ili{}-datr}\ili{} for\ili{} details\ili{})\ili{} which\ili{} limit\ili{},\ili{} among\ili{} others\ili{},\ili{} its\ili{} power\ili{} to\ili{} express\ili{} word\ili{} order\ili{} constraints\ili{}.\ili{}\footnote\ili{}{Note\ili{},\ili{} however\ili{},\ili{} that\ili{} while\ili{} word\ili{} order\ili{} constraints\ili{} are\ili{} supposed\ili{} to\ili{} be\ili{} expressed\ili{} in\ili{} PATR\ili{}-II\ili{} through\ili{} filtering\ili{} CFG\ili{} rules\ili{} via\ili{} features\ili{},\ili{} these\ili{} constraints\ili{} could\ili{} be\ili{} also\ili{} expressed\ili{} directly\ili{} as\ili{} feature\ili{} structure\ili{} values\ili{}.}\ili{} Walenty\ili{} is\ili{} flexible\ili{} enough\ili{} to\ili{} account\ili{} for\ili{} most\ili{} of\ili{} the\ili{} MWE\ili{}-related\ili{} properties\ili{}.\ili{} Yet\ili{},\ili{} the\ili{} need\ili{} to\ili{} introduce\ili{} new\ili{} symbols\ili{} to\ili{} express\ili{} previously\ili{} unforeseen\ili{} phenomena\ili{} \ili{}(already\ili{} mentioned\ili{} w\ili{}.r\ili{}.t\ili{}.\ili{} the\ili{} virtue\ili{} of\ili{} \ili{}\isi\ili{}{transparency}\ili{})\ili{} may\ili{} stem\ili{} from\ili{} the\ili{} insufficient\ili{} flexibility\ili{} of\ili{} the\ili{} formalism\ili{}.\ili{} As\ili{} for\ili{} DuELME\ili{},\ili{} we\ili{} see\ili{} its\ili{} relatively\ili{} low\ili{} \ili{}\isi\ili{}{transparency}\ili{} as\ili{} the\ili{} main\ili{} cause\ili{} of\ili{} its\ili{} relatively\ili{} low\ili{} flexibility\ili{} \ili{}-\ili{}-\ili{} it\ili{} is\ili{} hard\ili{} to\ili{} define\ili{} complex\ili{} constructions\ili{} when\ili{} clear\ili{} foundations\ili{} are\ili{} not\ili{} established\ili{}.\ili{}
\ili{}
The\ili{} restrictions\ili{} enforced\ili{} by\ili{} PATR\ili{}-II\ili{} diminish\ili{} also\ili{} its\ili{} power\ili{} to\ili{} express\ili{} certain\ili{} factorizations\ili{} \ili{}-\ili{}-\ili{} notably\ili{},\ili{} by\ili{} not\ili{} allowing\ili{} templates\ili{} to\ili{} apply\ili{} to\ili{} feature\ili{} structure\ili{} nodes\ili{} other\ili{} than\ili{} roots\ili{}.\ili{} Due\ili{} to\ili{} the\ili{} untyped\ili{} nature\ili{} of\ili{} feature\ili{} structures\ili{},\ili{} representation\ili{} of\ili{} certain\ili{} properties\ili{} based\ili{} on\ili{} types\ili{} \ili{}-\ili{}-\ili{} and\ili{},\ili{} therefore\ili{},\ili{} the\ili{} related\ili{} generalizations\ili{} \ili{}-\ili{}-\ili{} may\ili{} be\ili{} hindered\ili{} as\ili{} well\ili{}.\ili{} The\ili{} power\ili{} to\ili{} generalize\ili{} of\ili{} DuELME\ili{} is\ili{} limited\ili{} by\ili{} the\ili{} distinction\ili{} between\ili{} patterns\ili{} and\ili{} MWE\ili{} descriptions\ili{}.\ili{} Moreover\ili{},\ili{} DuELME\ili{} provides\ili{} no\ili{} way\ili{} to\ili{} express\ili{} any\ili{} kind\ili{} of\ili{} sharing\ili{} between\ili{} the\ili{} individual\ili{} patterns\ili{}.\ili{} As\ili{} to\ili{} Walenty\ili{},\ili{} a\ili{} hierarchy\ili{} of\ili{} macros\ili{} \ili{}(in\ili{} the\ili{} sense\ili{} that\ili{} a\ili{} macro\ili{} can\ili{} refer\ili{} to\ili{} other\ili{} macros\ili{})\ili{} can\ili{} be\ili{} used\ili{} to\ili{} account\ili{} for\ili{} repeating\ili{} patterns\ili{}.\ili{} However\ili{},\ili{} it\ili{} is\ili{} not\ili{} clear\ili{} to\ili{} what\ili{} extent\ili{} macros\ili{} constitute\ili{} a\ili{} part\ili{} of\ili{} the\ili{} formalism\ili{} itself\ili{} and\ili{} it\ili{} seems\ili{} that\ili{} the\ili{} mechanism\ili{} of\ili{} macros\ili{} is\ili{} too\ili{} simple\ili{} to\ili{} account\ili{} for\ili{} more\ili{} complex\ili{} patterns\ili{} \ili{}(for\ili{} example\ili{},\ili{} the\ili{} abovementioned\ili{} subject\ili{}/possessive\ili{} agreement\ili{} restriction\ili{})\ili{}.\ili{}
\ili{} \ili{}
Both\ili{} DuELME\ili{} and\ili{} Walenty\ili{} seem\ili{} to\ili{} be\ili{} more\ili{} electronically\ili{} versatile\ili{} than\ili{} XMG\ili{}.\ili{} DuELME\ili{} supports\ili{} the\ili{} standard\ili{} LMF\ili{} format\ili{},\ili{} while\ili{} one\ili{} of\ili{} the\ili{} formats\ili{} supported\ili{} by\ili{} Walenty\ili{} is\ili{} TEI\ili{} \ili{}-\ili{}-\ili{} based\ili{} on\ili{} XML\ili{},\ili{} less\ili{} concise\ili{} than\ili{} the\ili{} default\ili{} Walenty\ili{}'s\ili{} format\ili{} but\ili{} more\ili{} explicit\ili{} and\ili{} application\ili{}-friendly\ili{}.\ili{} While\ili{} XMG\ili{} encodings\ili{} can\ili{} be\ili{} compiled\ili{} and\ili{} stored\ili{} in\ili{} an\ili{} XML\ili{} format\ili{} which\ili{} directly\ili{} represents\ili{} all\ili{} the\ili{} resolved\ili{} property\ili{} names\ili{},\ili{} it\ili{} does\ili{} not\ili{} necessarily\ili{} contain\ili{} all\ili{} the\ili{} underlying\ili{} generalizations\ili{} \ili{}(i\ili{}.e\ili{}.\ili{},\ili{} those\ili{} encoded\ili{} in\ili{} the\ili{} class\ili{} inheritance\ili{} hierarchy\ili{})\ili{}.\ili{} One\ili{} could\ili{} imagine\ili{} parsing\ili{} and\ili{} interpreting\ili{} XMG\ili{} descriptions\ili{} themselves\ili{},\ili{} and\ili{} not\ili{} the\ili{} resulting\ili{} compiled\ili{} encodings\ili{},\ili{} as\ili{} a\ili{} first\ili{} step\ili{} of\ili{} converting\ili{} XMG\ili{} descriptions\ili{} to\ili{} a\ili{} particular\ili{} lexical\ili{} resource\ili{}.\ili{} However\ili{},\ili{} this\ili{} solution\ili{} would\ili{} require\ili{} certain\ili{} knowledge\ili{} about\ili{} the\ili{} formal\ili{} principles\ili{} and\ili{} mechanisms\ili{} underlying\ili{} XMG\ili{}.\ili{} Thus\ili{} the\ili{} additional\ili{} flexibility\ili{} and\ili{} power\ili{} to\ili{} generalize\ili{} of\ili{} XMG\ili{} come\ili{} with\ili{} additional\ili{} cost\ili{} in\ili{} terms\ili{} of\ili{} the\ili{} preprocessing\ili{} work\ili{} that\ili{} needs\ili{} to\ili{} be\ili{} done\ili{} to\ili{} obtain\ili{} a\ili{} particular\ili{} resource\ili{} from\ili{} XMG\ili{} descriptions\ili{}.\ili{} As\ili{} to\ili{} PATR\ili{}-II\ili{},\ili{} there\ili{} seem\ili{} to\ili{} be\ili{} very\ili{} few\ili{} actively\ili{} maintained\ili{} software\ili{} tools\ili{} for\ili{} it\ili{}.\ili{} While\ili{} a\ili{} parser\ili{} of\ili{} this\ili{} formalism\ili{} can\ili{} still\ili{} be\ili{} downloaded\ili{},\ili{} its\ili{} further\ili{} development\ili{} has\ili{} been\ili{} discontinued\ili{} as\ili{} of\ili{} 2006\ili{}.\ili{}\footnote\ili{}{\ili{}\url\ili{}{http\ili{}:\ili{}/\ili{}/software\ili{}.sil\ili{}.org\ili{}/pc\ili{}-patr\ili{}/}}\ili{} We\ili{} therefore\ili{} estimate\ili{} the\ili{} electronic\ili{} versatility\ili{} of\ili{} PATR\ili{}-II\ili{} as\ili{} being\ili{} rather\ili{} low\ili{} due\ili{} to\ili{} the\ili{} current\ili{} unavailability\ili{} of\ili{} dedicated\ili{} software\ili{} tools\ili{}.\ili{} \ili{}
\ili{}
Implementation\ili{} friendliness\ili{} of\ili{} DuELME\ili{} and\ili{} Walenty\ili{} has\ili{} been\ili{} already\ili{} confirmed\ili{} in\ili{} practice\ili{}.\ili{} DuELME\ili{} has\ili{} been\ili{} used\ili{} to\ili{} encode\ili{} a\ili{} lexicon\ili{} of\ili{} 5\ili{},000\ili{} \ili{}\ili\ili{}{Dutch}\ili{} MWEs\ili{},\ili{} while\ili{} Walenty\ili{} underlies\ili{} The\ili{} Polish\ili{} Valence\ili{} Dictionary\ili{} which\ili{},\ili{} in\ili{} particular\ili{},\ili{} contains\ili{} around\ili{} 8\ili{},000\ili{} MWE\ili{} entries\ili{}.\ili{} Moreover\ili{},\ili{} a\ili{} dedicated\ili{} tool\ili{} Slowal\ili{} \ili{}(\ili{}\url\ili{}{http\ili{}:\ili{}/\ili{}/zil\ili{}.ipipan\ili{}.waw\ili{}.pl\ili{}/Slowal}\ili{})\ili{} has\ili{} been\ili{} designed\ili{} for\ili{} creating\ili{},\ili{} editing\ili{} and\ili{} browsing\ili{} Walenty\ili{} dictionaries\ili{}.\ili{} Thus\ili{},\ili{} Walenty\ili{} comes\ili{} with\ili{} an\ili{} implementation\ili{} friendly\ili{} environment\ili{},\ili{} editing\ili{} tools\ili{} and\ili{},\ili{} on\ili{} top\ili{} of\ili{} that\ili{},\ili{} provides\ili{} conversion\ili{} between\ili{} several\ili{} dictionary\ili{} formats\ili{} adapted\ili{} for\ili{} different\ili{} needs\ili{}.\ili{} In\ili{} XMG\ili{},\ili{} MWEs\ili{} are\ili{} defined\ili{} as\ili{} terminal\ili{} classes\ili{} and\ili{} are\ili{} encoded\ili{} directly\ili{} in\ili{} the\ili{} source\ili{} code\ili{}.\ili{} At\ili{} the\ili{} moment\ili{},\ili{} there\ili{} is\ili{} no\ili{} dedicated\ili{} tool\ili{} which\ili{} would\ili{} assist\ili{} a\ili{} human\ili{} user\ili{} with\ili{} encoding\ili{} large\ili{} sets\ili{} of\ili{} MWEs\ili{}.\ili{} At\ili{} the\ili{} same\ili{} time\ili{},\ili{} encoding\ili{} MWEs\ili{} directly\ili{} in\ili{} the\ili{} source\ili{} code\ili{} can\ili{} be\ili{} seen\ili{} as\ili{} a\ili{} flexible\ili{} solution\ili{} which\ili{} allows\ili{} the\ili{} user\ili{} to\ili{} adopt\ili{} his\ili{} or\ili{} her\ili{} own\ili{} organization\ili{} of\ili{} MWE\ili{}-related\ili{} classes\ili{}.\ili{} High\ili{} factorization\ili{} capabilities\ili{} of\ili{} XMG\ili{} should\ili{} also\ili{} facilitate\ili{} handling\ili{} large\ili{} sets\ili{} of\ili{} lexical\ili{} objects\ili{},\ili{} heterogeneous\ili{} yet\ili{} often\ili{} showing\ili{} common\ili{} patterns\ili{}.\ili{} On\ili{} top\ili{} of\ili{} that\ili{},\ili{} the\ili{} process\ili{} of\ili{} compiling\ili{} XMG\ili{} descriptions\ili{} provides\ili{} a\ili{} verification\ili{} mechanism\ili{} which\ili{} allows\ili{} to\ili{} check\ili{} the\ili{} correctness\ili{} of\ili{} the\ili{} individual\ili{} XMG\ili{}-based\ili{} lexical\ili{} entries\ili{}.\ili{} For\ili{} PART\ili{}-II\ili{},\ili{} again\ili{},\ili{} we\ili{} found\ili{} no\ili{} readily\ili{} available\ili{} software\ili{} tool\ili{} that\ili{} is\ili{} designed\ili{} to\ili{} support\ili{} the\ili{} implementation\ili{} process\ili{}.\ili{} \ili{}
\ili{}
As\ili{} a\ili{} general\ili{} conclusion\ili{},\ili{} lexical\ili{} encoding\ili{} of\ili{} MWEs\ili{} is\ili{} a\ili{} highly\ili{} challenging\ili{} task\ili{},\ili{} as\ili{} also\ili{} stressed\ili{} in\ili{} chapters\ili{} MARKANTONATOU\ili{} and\ili{} DYVIK\ili{} of\ili{} this\ili{} volume\ili{},\ili{} due\ili{} to\ili{} the\ili{} complexity\ili{} and\ili{} versatility\ili{} of\ili{} the\ili{} regular\ili{} and\ili{} idiosyncratic\ili{} phenomena\ili{} exhibited\ili{} by\ili{} the\ili{} linguistic\ili{} objects\ili{}.\ili{} The\ili{} four\ili{} encoding\ili{} formats\ili{} examined\ili{} here\ili{} show\ili{} complementary\ili{} strengths\ili{} and\ili{} weaknesses\ili{}.\ili{} We\ili{} believe\ili{} that\ili{} \ili{}\isi\ili{}{transparency}\ili{},\ili{} flexibility\ili{} and\ili{} the\ili{} power\ili{} to\ili{} generalize\ili{}\footnote\ili{}{We\ili{} believe\ili{} that\ili{} the\ili{} latter\ili{} property\ili{} \ili{}-\ili{}-\ili{} the\ili{} power\ili{} to\ili{} generalize\ili{} \ili{}-\ili{}-\ili{} should\ili{} be\ili{} particularly\ili{} helpful\ili{} in\ili{} modeling\ili{} the\ili{} varying\ili{} degrees\ili{} of\ili{} flexibility\ili{} exhibited\ili{} by\ili{} MWEs\ili{},\ili{} discussed\ili{} in\ili{} chapter\ili{} SHEINFUX\ili{} of\ili{} this\ili{} volume\ili{}.}\ili{} are\ili{} the\ili{} fundamental\ili{} virtues\ili{} to\ili{} promote\ili{} in\ili{} lexical\ili{} encoding\ili{} of\ili{} MWEs\ili{},\ili{} and\ili{} in\ili{} this\ili{} respect\ili{} XMG\ili{} seems\ili{} to\ili{} stand\ili{} out\ili{} as\ili{} a\ili{} particularly\ili{} appropriate\ili{} framework\ili{}.\ili{} These\ili{} qualities\ili{} have\ili{} to\ili{} be\ili{} confirmed\ili{},\ili{} however\ili{},\ili{} in\ili{} large\ili{}-scale\ili{} lexicographic\ili{} efforts\ili{},\ili{} which\ili{} call\ili{} for\ili{} enhancing\ili{} its\ili{} implementation\ili{} friendliness\ili{} via\ili{} developing\ili{} a\ili{} lexicographic\ili{} framework\ili{} to\ili{} automate\ili{} the\ili{} encoding\ili{} and\ili{} validation\ili{} process\ili{}.\ili{} Note\ili{} finally\ili{} that\ili{} relatively\ili{} few\ili{} considerations\ili{} have\ili{} been\ili{} made\ili{} here\ili{} on\ili{} semantic\ili{} properties\ili{} of\ili{} MWEs\ili{}.\ili{} Maybe\ili{} the\ili{} most\ili{} outstanding\ili{} feature\ili{} of\ili{} many\ili{} MWEs\ili{} is\ili{} their\ili{} semantic\ili{} non\ili{}-compositionality\ili{},\ili{} and\ili{} addressing\ili{} it\ili{} in\ili{} a\ili{} lexical\ili{} encoding\ili{} framework\ili{} remains\ili{} one\ili{} of\ili{} the\ili{} most\ili{} challenging\ili{} perspectives\ili{}.\ili{}
\ili{} \ili{}
\ili{}\section\ili{}*\ili{}{Acknowledgements}\ili{}
This\ili{} work\ili{} has\ili{} been\ili{} supported\ili{} by\ili{} the\ili{} IC1207\ili{} PARSEME\ili{} COST\ili{} action\ili{},\ili{} by\ili{} the\ili{} Deutsche\ili{} Forschungsgemeinschaft\ili{}
\ili{}(DFG\ili{})\ili{} within\ili{} the\ili{} CRC\ili{} 991\ili{} \ili{}`\ili{}`The\ili{} Structure\ili{} of\ili{} Representations\ili{} in\ili{} Language\ili{},\ili{} Cognition\ili{},\ili{} and\ili{} Science\ili{}'\ili{}'\ili{},\ili{} and\ili{} by\ili{} a\ili{} doctoral\ili{} grant\ili{} from\ili{} the\ili{} \ili{}\ili\ili{}{French}\ili{} Ministry\ili{} of\ili{} Higher\ili{} Education\ili{} and\ili{} Research\ili{}.\ili{}
\ili{} \ili{}
\ili{}\printbibliography\ili{}[heading\ili{}=subbibliography\ili{},notkeyword\ili{}=this\ili{}]\ili{}
\ili{}
\ili{}\end\ili{}{document}\ili{}
\ili{}
\ili{}%\ili{}%\ili{}%\ili{} Local\ili{} Variables\ili{}:\ili{}
\ili{}%\ili{}%\ili{}%\ili{} mode\ili{}:\ili{} latex\ili{}
\ili{}%\ili{}%\ili{}%\ili{} TeX\ili{}-engine\ili{}:\ili{} xetex\ili{}
\ili{}%\ili{}%\ili{}%\ili{} TeX\ili{}-master\ili{}:\ili{} \ili{}"\ili{}.\ili{}.\ili{}/main\ili{}"\ili{}
\ili{}%\ili{}%\ili{}%\ili{} End\ili{}:\ili{}
\ili{}